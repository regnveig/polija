\documentclass[twoside,a5paper,12pt,fleqn,openany]{extbook}
\usepackage{polyglossia}
\usepackage{xcolor}

\setdefaultlanguage{latvian}
\setotherlanguages{english,russian,polish,german,french,lithuanian,spanish,latin,ukrainian,belarusian,greek,hebrew}
\defaultfontfeatures{Ligatures=TeX,Mapping=tex-text}

\newcommand{\ml}[3]{#2}
% \newcommand{\coverfrontfilename}{cover_front_ru.pdf}
\newcommand{\publisherlogofilename}{publisher_logo_ru.pdf}

\usepackage{tabularx}
\usepackage{ltablex}

% ------ METADATA ------
\newcommand{\bookauthor}{Vitālijs~Šalda}
\newcommand{\booktitle}{POLIJA XIX un XX gadsimtā}
\newcommand{\bookstarted}{...}
\newcommand{\bookfinished}{2023}
% ------ METADATA ------

% ----- XELATEX SYMBOL -----

% ----- XELATEX SYMBOL -----

% ----- HYPHENATION -----
\usepackage{hyphenat}
% ----- HYPHENATION -----

% ----- GEOMETRY -----
\usepackage[left=1.5cm,right=1.5cm,top=2cm,bottom=2cm,bindingoffset=0.5cm]{geometry}
% ----- GEOMETRY -----

% ----- INCLUDE PDF AS PAGES -----
\usepackage{pdfpages}
% ----- INCLUDE PDF AS PAGES -----

% ----- DROPPING CAP -----
\usepackage{type1cm,lettrine}
% ----- DROPPING CAP -----

% ----- FONTS -----
\renewcommand{\baselinestretch}{1.2}
\setmainfont{Linux Libertine}
% ----- FONTS -----

% ------ HYPERLINKS ------
\usepackage{hyperref}
\definecolor{LinkColor}{HTML}{0969DA}
\hypersetup{colorlinks=true, linkcolor=LinkColor, citecolor=LinkColor, filecolor=LinkColor, urlcolor=LinkColor}
% ------ HYPERLINKS ------

% ------ FANCY PAGE STYLE ------
\setlength{\headheight}{15pt}
\usepackage{fancyhdr}
\pagestyle{fancy}
\fancyhead[LE,RO]{\thepage}
\fancyhead[LO]{{\small{Polija XIX un XX gadsimtā}}}
\fancyhead[RE]{{\small{\bookauthor}}}
\fancyfoot{}
    \fancypagestyle{plain}{
    \renewcommand{\headrulewidth}{0mm}
    \fancyhead{}
    \fancyfoot{}
}
% ------ FANCY PAGE STYLE ------

% ------ ELEMENTS ------
\newcommand{\asterism}{\vspace{1em}{\centering\Large\bfseries$\ast~\ast~\ast$\par}\vspace{1em}}
\newcommand{\pltxti}[1]{\textit{\textpolish{#1}}}
\newcommand{\rutxti}[1]{\textit{\textrussian{#1}}}
\newcommand{\detxti}[1]{\textit{\textgerman{#1}}}
\newcommand{\frtxti}[1]{\textit{\textfrench{#1}}}
\newcommand{\entxti}[1]{\textit{\textenglish{#1}}}
\newcommand{\lttxti}[1]{\textit{\textlithuanian{#1}}}
\newcommand{\estxti}[1]{\textit{\textspanish{#1}}}
\newcommand{\latxti}[1]{\textit{\textlatin{#1}}}
\newcommand{\betxti}[1]{\textit{\textbelarusian{#1}}}
\newcommand{\uktxti}[1]{\textit{\textukrainian{#1}}}
\newcommand{\eltxti}[1]{\textgreek{#1}}
\newcommand{\hetxti}[1]{\texthebrew{#1}}
\newcommand{\mntxti}[1]{\textit{#1}}
\newcommand{\tttxti}[1]{\textit{#1}}
\newcommand{\zhtxti}[1]{#1}

\newcommand{\citespace}{<\dots{}>}

% ------ ELEMENTS ------

% ------ EPIGRAPH ------
\usepackage{epigraph}
\renewcommand{\epigraphsize}{\footnotesize}
\epigraphrule=0pt
\epigraphwidth=8cm

% ------ EPIGRAPH ------

\addto\captionslatvian{\renewcommand{\contentsname}{Satura rādītājs}}

\begin{document}

% ----- FRONT COVER -----
% \includepdf[pages={1}]{\coverfrontfilename}
% ----- FRONT COVER -----

% ----- EMPTY PAGE -----
\newpage\thispagestyle{plain}~
% ----- EMPTY PAGE -----

% ------ TITLE PAGE ------
\begin{titlepage}
{
\centering
{~\par}
\vspace{0.25\textheight}
{\LARGE\bookauthor\par}
\vspace{1.3cm}
{\Huge\textbf{POLIJA}\par}
{\LARGE\textbf{XIX un XX gadsimtā}\par}
\vfill
{\includegraphics[width=6em]{\publisherlogofilename}\par}
}
\end{titlepage}
% ------ TITLE PAGE ------

\epigraph
{Pagātne nav mirusi, tā nav pat pagājusi.}
{Viljams Folkners (\entxti{William Cuthbert Faulkner})}

\epigraph
{Meklējiet patiesību faktos.}
{Dens Sjaopins (\zhtxti{鄧小平})}

\epigraph
{Vēsturnieki nevar būt pilnīgi objektīvi.
Vēsture pēc savas būtības rada emocijas \citespace{}
Vēsture, izņemot neapšaubāmus faktus, pirmkārt balstās uz interpretāciju un zināmā mērā pieļauj daudzu patiesību esamību.}
{Žaks Le Goffs (\frtxti{Jacques Le Goff})}

\epigraph
{Vēsture nav skolotāja, bet uzraudze, magistra vitae: tā neko nemāca, bet tikai soda par mācībvielas nezināšanu.}
{Vasilijs Kļučevskis (\rutxti{Василий Осипович Ключевский})}

\epigraph
{Pagātne, saglabāta atmiņā, ir daļa tagadnes.}
{Tadeušs Kotarbiņskis (\pltxti{Tadeusz Marian Kotarbiński})}

\epigraph
{Bez īstas mīlestības pret cilvēci nav īstas dzimtenes mīlestības.}
{Anatols Franss (\frtxti{Anatole France})}

\newpage

\epigraph
{Var atrast svētus cilvēkus, bet nav svētu politiķu, valdību, partiju, politisku un sabiedrisku kustību.}
{Voicehs Jaruzeļskis (\pltxti{Wojciech Witold Jaruzelski})}

\epigraph
{Nav jēgas rēķināties ar kaut kādu pasaules taisnīgumu visiem~--- viss ir atkarīgs no spēku samēra.}
{Aleksandrs Zinovjevs (\rutxti{Александр Зиновьев})}

\epigraph
{Ne viss vēsturē ir pelnījis, lai ar to lepotos un ar entuziasmu uz to atsauktos.}
{Voicehs Jaruzeļskis (\pltxti{Wojciech Witold Jaruzelski})}

\epigraph
{Katras valsts vēsturē atrodamas gan varonīgas, gan kaunpilnas lappuses~--- taču pats kaunpilnākais ir noklusēt vai lielīties ar to, no kā ir jākaunas.}
{Natans Eidelmans (\rutxti{Натан Яковлевич Эйдельман})}

\epigraph
{Tautas vēsture ir tautas raksturojums.}
{Mihails Vellers (\rutxti{Михаил Иосифович Веллер})}

\epigraph
{Vēsturiskā atmiņa nekādā gadījumā nav objektīva.}
{Ivars Austers}

\newpage

\epigraph
{Patiesība ir subjektīva un vēstures izpratne tāpat.
Tā ir objektīva tikai tajā ziņā, ka vispār pastāv mūsu apziņā kā sen pagājušu notikumu pēdas.
Tāpēc veltas ir cerības uz jel kādu ,,objektīvu vēsturi” vai ,,vēstures tiesu”, kas nāks kā atpestīšana no mūžīgiem maldiem un noliks patiesību vispārējai apskatei.
Vēsture ir mūsu apziņa šeit šajā brīdī.
Rīt tā jau būs cita.}
{Āris Puriņš}

\epigraph
{Vēstures mūza ir lēnprātīga, zinoša un nepretencioza, bet, kad jūtas atstāta un pamesta, viņa cenšas atriebties un apžilbina tos, kas viņu atstāj novārtā.}
{Lešeks Kolakovskis (\pltxti{Leszek Kołakowski})}

\epigraph
{Vēsturnieku, kurš sagroza savu vēsturi, jāiznīcina tāpat kā naudas viltotāju tāpēc, ka viņi abi kaitē savai valstij.}
{Dons Migels de Servantess Saavedra\\(\estxti{Don Miguel de Cervantes Saavedra})}

\epigraph
{Tos, kuri saka taisnību, nemīl.
Tos, kuri melo~--- nicina.
Izvēle nav liela\ldots{}}
{Jurijs Poļakovs (\rutxti{Юрий Михайлович Поляков})}

\tableofcontents

\chapter*{Priekšvārds}
\addcontentsline{toc}{chapter}{Priekšvārds}

Strādājot Daugavpils Universitātes Vēstures katedrā, starp citiem studiju kursiem šī darba grāmatas autoram nācās lasīt arī lekciju kursu par Vidus un Austrumeiropas valstu vēsturi pēc Otrā pasaules kara.
Acīmredzot tāpēc, mācību programmām pilnveidojoties, Vēstures katedras toreizējais vadītājs asociētais profesors H.~Soms viņam uzdeva sagatavot arī Polijas XX gadsimta vēstures kursu.
Darba gaitā autoram izveidojās uzskats, ka jaunākos politiskos strīdus var saprast tikai tad, ja tiek ievērotas to vēsturiskās saknes; lai dziļāk izprastu XX~gadsimta Polijas vēstures notikumus, ir jādod plašāks ieskats par tās vēsturi XIX~gadsimtā nekā tas tiek sniegts vispārējā jauno laiku vēstures kursā.

Ievērojot to, ka bez profesoru Ē.~Jēkabsona un K.~Poča pētījumiem, kas saistīti ar Polijas vēsturi, latviešu valodā studentu rīcībā tikpat kā nav tai veltītu plašāku darbu, gadu gaitā tapa lasītāju priekšā esošais darbs.
Tas sākotnēji bija iecerēts kā mācību līdzeklis studentiem--vēsturniekiem.
Tomēr, ņemot vērā sabiedrībā pieaugošo interesi par mūsu valsts un tās kaimiņzemju vēsturi, kā arī lielo ar Polijas vēsturi saistīto materiālu apjomu, gala rezultātā ir tapis studentiem domāta mācību līdzekļa un zinātniski populāra darba apvienojums.
Autors cer, ka tādā veidolā tas atradīs lasītājus ne tikai studentu, bet arī citu vēstures interesentu vidū.

Izklāstā ir ietilpināts materiāls kā par jau lielā mērā izpētītiem, tā arī par daudziem vēl neatrisinātiem Polijas vēstures jautājumiem.
Nākotnē otro traktējums, iespējams, būs jāprecizē vai pat būtiski jāpārskata.
Tomēr autors uzskata par nepieciešamu dot priekšstatu arī par šādām problēmām.

Uzskatāmības labad autora tekstam pievienoti vairāku dokumentu tulkojumi un daudzas, galvenokārt INTERNET-ā aizgūtas ilustrācijas.
Lai lasītājiem būtu vieglāk orientēties, tekstā ar \strong{treknu druku} izcelti atsevišķi jēdzieni, problēmas, kuru iztirzājums tālāk seko.
Pieminētajām vēsturiskajām personām, arī pētniekiem un publicistiem darbam pievienotajā viņu sarakstā autors centās norādīt viņu dzīves gadus, kurus gan, diemžēl, ne vienmēr izdevās atrast.
Izklāstu ilustrējoši papildmateriāli nodrukāti citā šriftā un ar nedaudz sīkāku druku.

Bez izmantoto avotu un literatūras sarakstā uzrādītajiem izdevumiem grāmatas tapšanās gaitā samērā plaši izmantoti arī daudzi periodikā un INTERNET-ā atrodami materiāli, kuri nav norādīti bibliogrāfijas sarakstā.
Diemžēl tajos sastopamas daudzas neprecizitātes faktu izklāstā, klaji subjektīvas notikumu interpretācijas, kuras pārbaudīt pēc pirmavotiem ne vienmēr bija iespējams, jo daudzi no tiem autoram nebija pieejami.
Attēli ilustrācijām arī aizgūti galvenokārt INTERNET-ā.

Rakstot šo darbu, autoru ierobežoja arī poļu valodas zināšanu trūkums.
Tā kā poļu dokumentus un vēsturnieku darbus viņam nācās lasīt tulkojumos vācu un krievu valodā vai ar datortulkotāja palīdzību, nav izslēgtas zināmas neprecizitātes to izklāstā, varēja rasties arī kļūdas dažādu viedokļu interpretācijās.
Iespējams, ka faktu atspoguļojumu avotos, dažādu tautību un valstiskās piederības vēsturnieku, publicistu darbos atrodamo secinājumu, notikumu traktējumu, personāžu vērtējumu vidū autoram nav izdevies atrast visprecīzāko, vēstures gaitai atbilstošo atainojumu.
Viņš uzņemas visu atbildību par trūkumiem un kļūdām grāmatas saturā.

Autors arī saprot, ka daudzi viņa vērtējumi izsauks poļu radikālnacionālistu neapmierinātību, iespējams, arī prasību mīkstināt Polijas valdošo slāņu kritiku, bet uz to var viņiem ieteikt atcerēties apustuļu Pētera un Pāvila atbildi uz prasību pārtraukt sludināt Kristus mācību~--- ,,\latxti{Non possumus!}” (,,Mēs nevaram!”), kā arī jau XVI gadsimtā izskanējušos M.~Lutera vārdus: ,,\detxti{Hier stehe ich, ich kann nicht anders}” (,,Šeit es stāvu un citādi nevaru”).
Vēloties būt godīgs vēsturnieks, autors atļaujas sekot nosaukto dižo personību piemēram.

Darba pilnveidē daudz palīdzēja zinātniskā redaktore Dr.~hist., RISEBA vadošā pētniece T.~Bartele.
Tā tehniskā noformēšanā neaizstājama bija dabaszinātņu maģistra, ārsta Ņikitas Skabcova palīdzība.
Viņiem autors izsaka vissirsnīgāko pateicību.
Diemžēl objektīvu iemeslu dēļ darbam bija jāiztiek bez literārā redaktora un korektora.

\chapter*{Grāmatā minēto galveno organizāciju, iestāžu un to saīsinājumu saraksts}
\addcontentsline{toc}{chapter}{Grāmatā minēto galveno organizāciju, iestāžu un to saīsinājumu saraksts}

Abreviatūras (saīsinājumi) tekstā dotas, vadoties no partiju, organizāciju utt. nosaukumiem gan latviešu, gan poļu valodā.
Piemēram: Polijas Apvienotā strādnieku partija (\pltxti{Polska Zjednoczona Partia Robotnicza})~--- saīsinājumā saukta par PASP, nevis PZPR, jo latviešu lasītājs līdz 1990.~gadam izdotajā literatūrā un presē bieži sastapās ar pirmo, latviskoto saīsinājumu.
Taču dažādās valodās apritē jau iegājušie saīsinājumi tieši no poļu valodas atstāti arī šajā darbā.
Piemēram, \pltxti{Armia Krajowa} (burtiski~--- Dzimtenes armija) vācu valodā tiek tulkota kā ,,\detxti{polnische Heimatarmee}'', bet angļu dokumentos tās nosaukums tulkots dažādi: ,,\entxti{Home Army}'', ,,\entxti{Underground Army}'', ,,\entxti{Secret Army}'', kas īsti neatbilst nosaukumam poļu valodā.
Grāmatā lietots saīsinājums AK, nevis DA.
Polijas Sociālistiskā partija (\pltxti{Polska Partia Socjalistyczna}), kuras nosaukums literatūrā parasti tiek saīsināts kā PPS, šādi dots arī šajā grāmatā, nevis kā PSP, kā būtu vajadzējis saīsināt latviešu valodā.
Tāpat īsināti arī daļa citu nosaukumu.
Tā gan ir atteikšanās no vienotas pieejas saīsinājumu darināšanā, taču tā kā minētajiem un arī citiem saīsinājumiem vēstures literatūrā jau ir savs skanējums, to mēģināt mainīt acīmredzot būtu aplami.

Poļu vārds ,,\pltxti{Polska}'' (poļu un Polijas), dažādos nosaukumos tulkots dažādi, galvenokārt atkarībā no organizācijas, kuras nosaukumā tas ietverts, izveides laika.
Ja Polija kā valsts organizācijas izveides laikā nepastāvēja, tad parasti lietots pirmais variants, ja pastāvēja, tad otrais.
Taču ir arī atkāpes no šī principa.
Piemēram, 1892.~gadā dibinātā ,,\pltxti{Polska Partia Socjalistyczna}'' tiek tulkota kā Polijas (nevis Poļu) Sociālistiskā partija, arī 1942.~gada janvārī nodibinātā ,,\pltxti{Polska Partia Robotnicza}'' tulkota kā Polijas Strādnieku Partija, jo šie nosaukumi jau ir iegājuši literatūrā.

Lai atvieglotu darba lasīšanu, dažādas organizācijas, iestādes, politiskās partijas grāmatas teksta blokos vispirms parasti sauktas
pilnos nosaukumus, abreviatūras izmantotas tikai pēc tam.
Retāk pieminētajām organizācijām tās nav lietotas vispār, taču lasītāju ērtības labad to nosaukums un tā tulkojums sarakstā parasti ir ietverts.

\begin{footnotesize}
\noindent
\begin{tabularx}{\linewidth}{|p{3cm}|p{3.5cm}|p{1.4cm}|p{1.6cm}|}
\hline
\strong{Organizācijas vai iestādes nosaukums latviešu valodā} & \strong{Organizācijas vai iestādes nosaukums poļu (angļu vai krievu) valodā} & \strong{Saīsinājums} & \strong{Organizācijas vai iestādes pastāvēšanas gadi} \\
\hline
Apvienotā Zemnieku partija & \pltxti{Zjednoczone Stronnictwo Ludowe} & AZP & 1949--1989 \\
\hline
Apvienoto Nāciju Organizācija & \pltxti{Organizacja Narodów Zjednoczonych} & ANO & 1945--mūsdienas \\
\hline
Apvienoto nāciju palīdzības un atjaunošanas organizācija & \entxti{United Nations Relief ana Rehabilitation Administration} & UNRRA & 1943--1947 \\
\hline
Aktīvās cīņas savienība & \pltxti{Zwiazek Walki Czynnej} & ACS & 1908--1914 \\
\hline
Brīvība un neatkarība (Pilns nosaukums: Pretošanās bez kara un diversijām „Brīvība un neatkarība”) & \pltxti{Wolność i Niezawisłość (Ruch Oporu bez Wojny i Dywersji „Wolność i Niezawisłość”)} & WiN & 1945--1952 \\
\hline
Bruņotās cīņas savienība & \pltxti{Związek Walki Zbrojnej} & BCS & 1939--1942 \\
\hline
Bruņoto spēku delegatūra & \pltxti{Delegatura Sił Zbrojnych na Kraj} & --- & 1945 \\
\hline
Centrisko spēku savienība & \pltxti{Porozumienie Centrum} & PC & 1990--1997 \\
\hline
Civīlmilicija & \pltxti{Milicja Obywatelska} & MO & 1944--1990 \\
\hline
Civīlmilicijas Brīvprātīgā rezerve & \pltxti{Ochotnicza Rezerwa Milicji Obywatelskiej} & ORMO & 1946--1989 \\
\hline
Darba partija & \pltxti{Stronnictwo Pracy} & SP & 1937--1950 \\
\hline
Demokrātiskā partija &\pltxti{Stronnictwo Demokratyczne} & SD & 1939--mūsdienas \\
\hline
Demokrātiskā savienība & \pltxti{Unia Demokratyczna} & UD & 1990--1994 \\
\hline
Demokrātisko kreiso spēku savienība & \pltxti{Sojusz Lewicy Demokratycznej} & SLD & 1991--1999 (partiju savienība), kopš 1999~--- partija \\
\hline
Eiropas Cilvēktiesību tiesa & \entxti{European Court of Human Rights} & ECT & 1998--mūsdienas \\
\hline
Galīcijas un Cešinas Silēzijas poļu sociāldemokrātiskā partija & \pltxti{Polska Partia Socjalno-Demokratyczna Galicji i Śląska Cieszyńskiego} & --- & 1897--1919 \\
\hline
Komunistiskās Internacionāles Izpildu Komiteja & \entxti{The Executive Committee of the Communist International}; \rutxti{Исполнительный комитет Коммунистического Интернационала} & KIIK & 1919--1943 \\
\hline
Krievijas Federācija & \rutxti{Российская Федерация} & KF & 1991--mūsdienas \\
\hline
Krievijas Komunistiskā (boļševiku) partija & \rutxti{Российская Коммунистическая партия (большевиков)} & KK(b)P & 1918--1925 \\
\hline
Krievijas Sociāldemokrātiskā strādnieku partija & \rutxti{Российская социал-демократическая рабочая партия} & KSDSP & 1898--1918 \\
\hline
Krievijas Sociālistiskā Federatīvā Padomju Republika & \rutxti{Российская Социалистическая Федеративная Советская Республика} & KSFPR & 1918--1922 \\
\hline
Kristīgi Nacionālā apvienība & \pltxti{Zjednoczenie Chrześcijańsko-Narodowe} & ZChN & 1989--2010 \\
\hline
Ķirzakas savienība & \pltxti{Związek Jaszczurczy} & --- & 1939--1942 \\
\hline
Liberāli demokrātiskais kongress & \pltxti{Kongres Liberalno-Demokratyczny} & KLD & 1991--1994 \\
\hline
Likums un taisnīgums & \pltxti{Prawo i Sprawiedliwość} & PiS & 2001--mūsdienas \\
\hline
Nacionālā Demokrātija & \pltxti{Narodowa Demokracja~--- Endecja} & ND & 1887--1947 \\
(Tās nosaukumi dažādos laika posmos: & ~ & ~ & ~ \\
Poļu līga & \pltxti{Liga Polska} & --- & 1887--1893 \\
Nacionālā Līga & \pltxti{Liga Narodowa} & --- & 1893--1928 \\
Nacionāli demokrātiskā partija & \pltxti{Stronnictwo Narodowo-Demokratyczne} & --- & 1897--1919 \\
Nacionāli demokrātiskā savienība & \pltxti{Związek Ludowo-Narodowy} & --- & 1919--1928 \\
Nacionālā partija & \pltxti{Stronnictwo Narodowe} & SN & 1928--1947 \\
\hline
Nacionālās atmiņas institūts & \pltxti{Instytut Pamięci Narodowej} & IPN & 2000--mūsdienas \\
\hline
Nacionālā strādnieku partija & \pltxti{Narodowa Partia Robotnicza} & NPR & 1920--1937 \\
\hline
Nacionālās vienotības padome & \pltxti{Rada Jednosci Narodowej} & --- & 1944--1945 \\
\hline
Nacionālās vienotības pagaidu valdība & \pltxti{Tymczasowy Rząd Jedności Narodowej} & NVPV & 1945--1947 \\
\hline
Nacionālie bruņotie spēki & \pltxti{Narodowe Siły Zbrojne} & NBS & 1942--1947 \\
\hline
Nacionālie bruņotie spēki~--- Ķirzakas savienība & \pltxti{Narodowe Siły Zbrojne~--- Związek Jaszczurczy} & --- & 1944--1945 \\
\hline
Padomju Krievijas / PSRS Tautas komisāru padome & \rutxti{Совет народных комиссаров} & TKP & 1917--1946 \\
\hline
Padomju Savienības Komunistiskā partija & \rutxti{Коммунистическая партия Советского Союза} & PSKP & 1952--1991 \\
\hline
Padomju Savienības Telegrāfa aģentūra & \rutxti{Телеграфное агентство Советского Союза} & TASS & 1925--1991 \\
\hline
Pilsoniskā kustība~--- demokrātiskā darbība & \pltxti{Ruch Obywatelski Akcja Demokratyczna} & --- & 1990--1991 \\
\hline
Polijas Apvienotā strādnieku partija & \pltxti{Polska Zjednoczona Partia Robotnicza} & PASP & 1948--1990 \\
\hline
Polijas armija & \pltxti{Wojsko Poliskie} & --- & 1918--1989 \\
\hline
Polijas Jaunatnes savienība & \pltxti{Związek Młodzieży Polskiej} & PJS & 1948--1957 \\
\hline
Polijas karalistes un Lietuvas Sociāldemokrātiskā partija & \pltxti{Socjaldemokracja Królestwa Polskiego i Litwy} & PKunLSD & 1900--1918 \\
\hline
Polijas karalistes Sociāldemokrātija & \pltxti{Socjaldemokracja Królestwa Polskiego} & PKSD & 1893--1900 \\
\hline
Polijas Komunistiskā strādnieku partija & \pltxti{Komunistyczna Partia Robotnicza Polski} & PKSP & 1918--1925 \\
\hline
Polijas Komunistiskā partija & \pltxti{Komunistyczna Partia Polski} & PKP & 1925--1938 \\
\hline
Polijas Kristīgo demokrātu partija & \pltxti{Polskie Stronnictwo Chrześcijańskiej Demokracji}; \pltxti{Chrześcijańska Demokracja} & --- & 1919--1937 \\
\hline
Polijas Nacionālās atbrīvošanas komiteja & \pltxti{Polski Komitet Wyzwolenia Narodowego} & PNAK & 1944 \\
\hline
Polijas Republikas Nacionālā Padome & \pltxti{Rada Narodowa Rzeczypospolitej Polskiej} & --- & 1939--1941, 1942--1945 \\
\hline
Polijas Republikas pagaidu valdība & \pltxti{Rząd Tymczasowy Rzeczypospolitej Polskiej} & PRPV & 1945 \\
\hline
Polijas Republikas pilsoniskā platforma & \pltxti{Platforma Obywatelska Rzeczypospolitej Polskiej} & PO & 2001--mūsdienas \\
\hline
Polijas Republikas Sociāldemokrātija & \pltxti{Socjaldemokracja Rzeczypospolitej Polskiej} & PRSD & 1991--1999 \\
\hline
Polijas Sociālistiskā partija & \pltxti{Polska Partia Socjalistyczna} & PPS & 1892--1948 \\
\hline
Polijas Sociālistiskā partija~--- ļevica & \pltxti{Polska Partia Socjalistyczna~--- Ļevica} & PPS-ļevica & 1906--1918, 1926--1931 \\
\hline
Polijas Sociālistiskā partija~--- revolucionārā frakcija & \pltxti{Polska Partia Socjalistyczna~--- Frakcja Rewolucyjna} & PPS-frakcija & 1906--1919 \\
\hline
Polijas Strādnieku Partija & \pltxti{Polska Partia Robotnicza} & PSP & 1942--1948 \\
\hline
Polijas Tautas armija & \pltxti{Polska Armia Ludowa} & PAL & 1943--1944 \\
\hline
Polijas Tautas republika & \pltxti{Polska Rzeczpospolita Ludowa} & PTR & 1944--1989 \\
\hline
Polijas uzvaras dienests & \pltxti{Służba Zwycięstwu Polski} & --- & 1939 \\
\hline
Polijas valdība emigrācijā & \pltxti{Rząd Rzeczypospolitej Polskiej na uchodźstwie} & PVE & 1939--1945 (formāli~1990) \\
\hline
Polijas Zemnieku partija & \pltxti{Polskie Stronnictwo Ludowe} & PSL & 1945--1949, 1990--mūsdienas \\
\hline
Polijas Zemnieku partija~--- Jaunā atbrīvošanās & \pltxti{Polskie Stronnictwo Ludowe~--- Nowe Wyzwolenie} & PSL-NW & 1946--1947 \\
\hline
Polijas Zemnieku partija~--- kreisā frakcija & \pltxti{Polskie Stronnictwo Ludowe~--- ļewica} & PSL-levica & 1914--1924 \\
\hline
Politiskā konsultatīvā komiteja & \pltxti{Polityczny Komitet Porozumiewawczy} & --- & 1940--1943 \\
\hline
Poļu Kara organizācija & \pltxti{Polska Organizacja Wojskowa} & POW & 1914--1918 \\
\hline
Poļu Komunistu Centrālais Birojs & \pltxti{Centralne Biuro Komunistów Polskich} & PKCB & 1944 \\
\hline
Poļu patriotu savienība & \pltxti{Związek Patriotów Polskich} & PPtS & 1943--1946 \\
\hline
Poļu Sociālistiskā partija~--- Brīvība, Vienlīdzība, Neatkarība & \pltxti{Polska Partia Socjalistyczna~--- Wolność, Równość, Niepodległość} & PPS-WRN & 1939--1944 \\
\hline
Poļu sociālistu strādnieku partija & \pltxti{Rabotnicza Partia Polskich Socjalistów} & PSSP & 1943--1944 \\
\hline
Poļu sociālistiskā partija Prūsijā & \pltxti{Polska Partia Socjalistyczna Zaboru Pruskiego} & --- & 1893--1919 \\
\hline
Poļu Zemnieku partija „Atbrīvošanās” & \pltxti{Polskie Stronnitctwo Ludowe~--- „Wyzwolenie”} & PSL „\pltxti{Wyzwolenie}” & 1915--1931 \\
\hline
Poļu Zemnieku partija~--- Pjasts & \pltxti{Polskie Stronnictwo Ludowe~--- Piast} & PSL-Piast & 1913--1931 \\
\hline
PSRS Iekšlietu tautas komisariāts & \rutxti{Народный комиссариат внутренних дел СССР} & IeTK & 1934--1946 \\
\hline
PSRS Valsts Aizsardzības Komiteja & \rutxti{Государственный комитет обороны СССР} & PSRS VAK & 1941--1945 \\
\hline
Rietumbaltkrievijas Komunistiskā partija & baltkr.: \betxti{Камуністычная партыя Заходняй Беларусі} & RBKP & 1923--1938 \\
\hline
Rietumukrainas Komunistiskā partija & ukr.: \uktxti{Комуністична партія Західної України} & RUKP & 1919--1938 \\
\hline
Sabiedriskās drošības ministrija & \pltxti{Ministerstwo Bezpieczeństwa Publicznego} & SDM & 1945-1954 \\
\hline
Savstarpējās Ekonomiskās Palīdzības Padome & \rutxti{Совет экономической взаимопомощи} & SEPP & 1949--1991 \\
\hline
Starptautiskā Sarkanā Krusta komiteja & \entxti{International Committee of the Red Cross} & SSKK & 1863--mūsdienas \\
\hline
Strādnieku aizsardzības komiteja & \pltxti{Komitet Obrony Robotników} & KOR & 1976--1977 \\
\hline
Strādnieku aizsardzības komitejas Sociālās pašaizsardzības komiteja & \pltxti{Komitet Samoobrony Społecznej KOR} & KOS-KOR & 1977--1981 \\
\hline
Strādnieku Zemnieku Sarkanā armija & \rutxti{Рабоче-крестьянская Красная армия} & SZSA & 1918--1946 \\
\hline
Tautas armija & \pltxti{Armia Ludowa} & AL & 1944 \\
\hline
Tautas gvarde & \pltxti{Gwardia Ludowa} & GL & 1942--1943 \\
\hline
Ukraiņu nacionālistu organizācija & ukr.: \uktxti{Організація Українських Націоналістів} & OUN & 1929--mūsdienas \\
\hline
Ukrainas Sacelšanās armija & ukr.: \uktxti{Українська повстанська армія} & UPA & 1942--1953 \\
\hline
Ukrainas Tautas Republika & ukr.: \uktxti{Українська Народна Республіка} & UTR & 1917--1920 \\
\hline
Vācijas Demokrātiskā Republika & vācu: \detxti{Deutsche Demokratische Republik} & VDR & 1949--1990 \\
\hline
Vācijas Federatīvā Republika & vācu: \detxti{Bundesrepublik Deutschland} & VFR & 1949--mūsdienas \\
\hline
Valsts Nacionālā padome & \pltxti{Krajowa Rada Narodowa} & KRN & 1944--1947 \\
\hline
Valsts Politiskā pārstāvniecība & \pltxti{Krajową Reprezentację Polityczną} & --- & 1943--1944 \\
\hline
Vēlēšanu akcija Solidaritāte & \pltxti{Akcja Wyborcza Solidarność} & AWS & 1996--2003 \\
\hline
Viskrievijas Centrālā izpildu komiteja & \rutxti{Всероссийский Центральный Исполнительный Комитет} & VCIK & 1917--1938 \\
\hline
Vispolijas arodbiedrību savienība & \pltxti{Ogólnopolskie Porozumienie Związków Zawodowych} & VAS & 1984--mūsdienas \\
\hline
Vissavienības Komunistiskā (boļševiku) partija & \rutxti{Всесоюзная коммунистическая партия большевиков} & VK(b)P & 1925--1952 \\
\hline
Zemes armija & \pltxti{Armia Krajowa} & AK & 1942--1945 \\
\hline
Zemes politiskā pārstāvniecība & \pltxti{Krajowa Reprezentacja Polityczna} & --- & 1943--1944 \\
\hline
Zemnieku bataljoni & \pltxti{Bataliony Chłopskie} & BCh & 1940--1945 \\
\hline
Zemnieku partija & \pltxti{Stronnictwo Chłopskie} & SCh & 1926--1931 \\
\hline
Zemnieku partija „Roch” & \pltxti{Stronnictwo Ludowe „Roch”} & SL Roch & 1940--1945 \\
\hline
Zemnieku partija & \pltxti{Stronnictwo Ludowe} & SL & 1931--1939, 1944--1949 \\
\hline
Zemnieku partija „Tautas griba” & \pltxti{Stronnictwo Ludowe „Wola Ludu”} & SL-WL & 1943--1944 \\
\hline
Zemnieku pašpalīdzības savienība & \pltxti{Związek Samopomocy Chłopskiej} & ZPS & 1944--1957 \\
\hline
\end{tabularx}
\end{footnotesize}

\chapter*{Ievads}
\addcontentsline{toc}{chapter}{Ievads}

\epigraph
{Īsts patriots ir tas, kurš saka patiesību pat savai zemei.}
{Žans Žoress (\frtxti{Jean Léon Jaurès})}

\epigraph
{Tēvzemes mīlestība ir ļoti laba lieta, bet ir kas vēl augstāks pār to~--- patiesības mīlestība.}
{Pēteris Čaadajevs (\rutxti{Пётр Яковлевич Чаадаев})}

\epigraph
{Ir labi būt patiesam vienmēr, arī tad, ja tas skar dzimteni. Katra pilsoņa pienākums ir, ja vajadzīgs, mirt par savu dzimteni, bet nevienu nedrīkst piespiest melot dzimtenes vārdā.}
{Šarls Luijs de Monteskjē (\frtxti{Charles-Louis de Secondat, Baron de La Brède et de Montesquieu})}

\epigraph
{Uzbrukumi tautas trūkumiem un netikumiem nav noziegums, bet nopelns, īstens patriotisms.}
{Visarions Beļinskis\\(\rutxti{Виссарион Григорьевич Белинский})}

\newpage

\epigraph
{Īsts patriotisms~--- tas pirmkārt ir vēlēšanās zināt un runāt patiesību par savu Tēvzemi.}
{Vjačeslavs Kostikovs (\rutxti{Вячеслав Васильевич Костиков})}

\epigraph
{Visas nelaimes pasaulē ceļas tādēļ, ka cilvēki lieto divējādas mērauklas: vienu sev un savai dzimtai, citu~--- pārējiem.}
{Zenta Mauriņa}

\epigraph
{Es dodu priekšroku savas dzimtenes šaustīšanai, tās sarūgtināšanai, tikai lai nekrāptu to.}
{Pēteris Čaadajevs (\rutxti{Пётр Я́ковлевич Чаада́ев})}

\epigraph
{Jebkurš nacionālisms ir akls, jo meklē savu nelaimju cēloņus citās tautās, tā sākas neiecietība, netaisnība. Nacionālisms nodara ļaunumu vispirms tā paudējiem, jo nenovēršami izraisa pretreakciju no citu tautu puses.}
{Ilga Apine}

\epigraph
{Visa mūsu vēsture~--- tas ir izdomājums, kuram visi piekrīt.}
{Voltērs (\frtxti{Voltaire, Fransuā Marī Aruē})}

\newpage

\epigraph
{Kas gan ir vēsture, ja ne meli, ar kuriem visi ir vienisprātis?}
{Napoleons Bonaparts (\frtxti{Napoléon Bonaparte})}

\epigraph
{Gala rezultātā mēs pētām vēsturi lai apmierinātu savas intereses un, pēc iespējas, saprastu pie tam savas paša problēmas. Taču nevienu no šiem diviem mērķiem mēs nesasniegsim, ja, atrodoties neauglīgās zinātniskās objektivitātes idejas iespaidā, neuzdrošināsimies zinātniskās problēmas izskatīt no sava skatu punkta.}
{Kārlis Poppers (\entxti{Karl Raimund Popper})}

\epigraph
{Nevar būt vēstures „pagātnei, kāda tā patiesi pastāvēja”, ir iespējamas tikai vēsturiskas interpretācijas, un ne viena no tām nav galīga.}
{Kārlis Poppers (\entxti{Karl Raimund Popper})}

\epigraph
{Vieni no lielākajiem maldiem ir domāt, ka visi jūt, redz un domā tāpat, kā mēs.}
{Pjērs Buasts (\frtxti{Pierre Boiste})}

\epigraph
{Lai kā pieklājas rakstītu vēsturi, ir jāaizmirst par savu ticību, savu tēvzemi, savu partiju.}
{Pjērs Buasts (\frtxti{Pierre Boiste})}

\newpage

\epigraph
{Mūsu ienaidnieku spriedumi par mums ir tuvāk patiesībai, nekā mūsu pašu.}
{Fransuā de Larošfuko (\frtxti{François de La Rochefoucauld})}

\epigraph
{Ticība tam, ka ir tikai viena, tikai tev zināma patiesība, ir visa ļaunuma sakne.}
{Makss Borns (\entxti{Max Born})}

\epigraph
{Patiesība pasaulē nedara tik daudz laba, cik šķietamas patiesības~--- ļauna. Tieksme izskaistināt vēsturi ir instinktīva tieksme paaugstināt grupas pašnovērtējumu.}
{Mihails Vellers (\rutxti{Михаил Иосифович Веллер})}

\epigraph
{Nobriedis prāts ienīst savas valsts netikumus~--- nepilnīgs noliedz tos. Godprātīgs vēsturnieks redz pienākumu izskaust savas zemes netikumus~--- negodprātīgs grib tikai iznīcināt tās ienaidniekus.}
{Mihails Vellers (\rutxti{Михаил Иосифович Веллер})}

\epigraph
{Poļu tautas rakstura varonīgās īpašības nedrīkst mūs piespiest aizvērt aci uz tās neprātīgumu un nepateicību, kas gadsimtu garumā tai radīja neizmērojamas ciešanas.}
{Vinstons Čērčils (\entxti{Winston Leonard Spencer-Churchill})}

\newpage

\epigraph
{Katru nacionālistu vajā ideja, ka pagātni ne vien var izmainīt, bet tā ir jāizmaina.}
{Džordžs Orvels (\entxti{George Orwell})}

\epigraph
{Tas, ko mēs saucam par cinismu, bieži vien ir atklāti pateikta patiesība.}
{Jezups Laganovskis}

\epigraph
{Ciniķis ir cilvēks, kurš nebaidās pateikt to, ko citi domā.}
{Aleksandrs Gordons (\rutxti{Александр Гордон})}

\epigraph
{Polijas nelaime ir tā, ka tās vēsture ir tik melīga, kā nekas cits pasaulē.}
{Ježi Gedroics (\pltxti{Jerzy Władysław Giedroyc})}

\epigraph
{Atcerēsimies, ka Polija citu tautu vidū drīzāk ir liela pele, nekā mazs zilonis.}
{Tadeušs Kotarbiņskis (\pltxti{Tadeusz Marian Kotarbiński})}

\epigraph
{Kāda Centrāleiropa? Rietumāzija!}
{Josifs Brodskis (\rutxti{Иосиф Александрович Бродский})}

Polijai vienmēr ir bijusi nozīmīga vieta Eiropas vēsturē, ko noteica gan tās atrašanās pašā Eiropas centrā, gan iespaidīgais iedzīvotāju skaits, gan arī aktīvā politika. Daudzi svarīgi notikumi ir risinājušies Polijas teritorijā, poļu tauta ir bijusi un joprojām ir iesaistīta ne tikai Eiropas, bet arī visas pasaules mēroga procesos. Poļu sabiedrisko kustību, Polijas valsts politikas iedarbība bija un ir jūtama tālu aiz tās robežām, bet īpaši tajās valstīs, kuras atrodas tai kaimiņos, tai skaitā arī Latvijā.

\strong{Poļi} (pašnosaukums \pltxti{Polacy}) ir skaitliski lielākā rietumslāvu tauta. Pēc dažādiem vērtējumiem pasaulē dzīvo no 44 līdz 60 miljonu poļu, no tiem Polijas Republikā~--- 38 miljoni. Nosaukums radies no rietumslāvu cilts (maztautas) poļānu (\pltxti{polanie}) vārda, kas jau no VIII gadsimta dzīvoja Vartas (poļu \pltxti{Warta}) upes baseinā rajonā ap mūsdienu Poznaņu (poļu \pltxti{Poznań}, vācu \detxti{Posen}). Vārda \pltxti{polanie} tulkojums ir vienkārši „lauku (tīrumu) iedzīvotāji” (\pltxti{pole}~--- lauks, tīrums). Tieši poļānu vidū konsolidējās poļu tautas kodols, šeit IX gadsimtā radās pirmā Polijas valsts Pjastu (\pltxti{Piastowie}) dinastijas (10.--14.gs.) vadībā, un šeit atradās pirmās Polijas galvaspilsētas~--- Gņezno (\pltxti{Gniezno}), Poznaņa (\pltxti{Poznań}). Konsolidējot ap sevi citas rietumslāvu ciltis Vislas baseinā, pamazām veidojās kopīga poļu identitāte.

16.~gadsimta vidū poļi kopā ar lietuviešiem un slāvu ciltīm~--- mūsdienu ukraiņu un baltkrievu priekštečiem~--- radīja Polijas-Lietuvas apvienotu valsti jeb Abu Tautu Republiku (1569--1795). To parasti sauca par Žečpospolitu (poļu \pltxti{Rzeczpospolita}, no latīņu valodas: \latxti{Res publica}~--- republika. Pilns nosaukums: poļu: \pltxti{Rzeczpospolita Obojga Narodów}, lietuv.~--- \lttxti{Žečpospolita/Abiejų Tautų Respublika}, baltkr.~--- \betxti{Рэч Паспалітая Абодвух Народаў}). Tā bija muižnieku republika ar vēlētu monarhu priekšgalā, kur poļi bija valdošā tautība.

Mūsdienu poļu historiogrāfijā poļu valstiskuma dažādiem modeļiem pieņemts dot kārtas numurus: 1569.--1795.~gadu periods tiek saukts par I Žečpospolitu, no 1918. līdz 1939.~gadam~--- par II Žečpospolitu, no 1989.~gada~--- par III Žečpospolitu. Polijas Tautas Republikas pastāvēšanas periodu (1944--1989) šajā numerācijā neievēro vai arī lieto terminu \pltxti{Rzeczpospolita Ludowa} (Tautas republika), lai uzsvērtu tās atšķirību no citām poļu republikām.

Sāka veidoties poļu nācija. Taču tās izveide noslēdzās XVIII~gadsimta otrajā pusē un XIX~gadsimtā jau smagākos apstākļos, kad poļi cīnījās par atbrīvošanos no nacionālās atkarības.

Pakļautais Polijas stāvoklis attiecībās ar citām valstīm iezīmējās jau XVIII~gadsimta sākumā. Tā otrajā pusē Žečpospolita ekonomiskā un politiskā iekšējā vājuma, šļahtas (muižniecības) ķīviņu, etnisko nesaskaņu rezultātā tika trīs reizes kaimiņvalstu sadalīta un beidza pastāvēt. XIX gadsimtā tikai atsevišķos periodos Polijā pastāvēja valstiski veidojumi ar dažādu autonomijas pakāpi (Varšavas hercogiste 1807--1814, Polijas karaliste 1815--1830, Krakovas Republika 1815--1846, autonomā Galīcija pēc 1867.~gada). Šo veidojumu esamība ļāva kaut nelielā mērā saglabāt poļu valstiskuma pārmantojamu, tomēr neapmierināja radikālāk noskaņotos poļu patriotus. Viņi nepārtraukti cīnījās par savu neatkarību, vairākkārt sacēlās pret lielvalstu kundzību, taču tas tikai pasliktināja zemes stāvokli. Atbrīvošanās kustība izplatījās visās poļu zemēs, piepildīja augošo nacionālo pašapziņu ar cīņas apofeozi, bruņotas pretošanās, kritušo un nomocīto nacionālo varoņu kultu. Valsts ar 800~gadu vēsturi sadale ienesa arī pastāvīgu nemiera elementu pārējo Eiropas valstu attiecībās.

Īpaši bagāta sarežģījumiem bija Polijas vēsture XX gadsimtā. Polijas valstiskuma atjaunošana (1918) un Versaļas miers (1919) nekļuva par galīgu Polijas jautājuma risinājumu. Jaltas vienošanās (1945) atkal pakļāva Poliju lielvalstu diktātam. Polija ilgstoši atradās t.s. „sociālistiskā” bloka sastāvā, faktiskā atkarībā no PSRS. Tikai 1989.~gadā poļu tauta atkal atguva neatkarību.

Kā raksta Polijas vēsturi daudz pētījušais britu profesors N.~Deiviss: „Polijas nacionālajai kustībai bija vissenākā vēsture, visstiprākais mandāts, vislielākā apņēmība, vissliktākā reputācija un vismazākie panākumi”. Ir saprotams, ka atzīmētās pretrunas ir skaidrojamas gan ar objektīviem, gan subjektīviem, gan starptautiskiem, gan pašu poļu darbības radītiem faktoriem. Bagātās, bet sarežģītās vēstures dažāda uztvere pašu poļu vidū arī pati par sevi ir radījusi problēmas.

Ungāru vēsturnieks I.~Bibo ir atzīmējis īpašu Austrumeiropas tautu kolektīvu psiholoģijas iezīmi~--- sevišķu jūtīgumu, pat bailes no pastāvošām vai tikai šķietamām nacionālās kopības bojā ejas briesmām. Šīm tautām šīs bailes bija saistītas sākotnēji ar turku, tad vācu, dažkārt arī poļu ekspansiju, pēc tam krievu iespiešanos reģionā. Pēdējā gadsimtā šīs bailes galvenokārt saistījās ar Vāciju, Krieviju un PSRS. Lielajām tautām šīs bailes ir mazsaprotamas. Piemēram, krievu tauta ilgstoši ir bijusi apspiesta, bet parasti apspieda viņu savas valsts vara, kuru viņi neuztvēra kā etniski svešu, ja arī imperatoru dzīslās tecēja faktiski vācu asinis. Tāpēc daudzi krievi pagātnē un bieži arī mūsdienās nevar pilnībā saprast tās tautas, kuras baidās galvenokārt no etniski svešiem apspiedējiem, pat ja šo apspiedēju varā dzīvot bija labāk nekā savu valdītāju kundzībā. Tā, nevar strikti apgalvot, ka vairākumam poļu Prūsijas/Vācijas un Austrijas/Austroungārijas kundzībā dzīvot bija sliktāk nekā savu poļu „panu”~--- savas šļahtas varā līdz Žečpospolitas dalīšanai. Taču liela poļu daļa tās pašas šļahtas vadībā XIX~gadsimtā cīnījās pret nacionālo apspiestību, sava valstiskuma atjaunošanu ne tikai pret „aziātisko” Krievijas impēriju, bet arī „eiropeiskajām” Prūsiju/Vāciju un Austriju/Austroungāriju.

\strong{Polijas attiecības ar kaimiņvalstīm} un to tautām vienmēr ir bijušas neviennozīmīgas. Vissarežģītākās attiecības Polijai gandrīz vienmēr veidojās ar kaimiņos esošajām Vāciju un Krieviju.

Vispirms par \strong{Polijas attiecībām ar Vāciju} (\pltxti{Niemcy}). Vācieši ilgstoši ir bijuši vieni no tuvākajiem un arī bīstamākajiem poļu kaimiņiem. Poļi „vācieti” (\pltxti{niemiec}) gadu simteņiem uztvēra kā ienaidnieku.

1217.~gadā pāvests Honorijs III pasludināja krusta karu pret prūšu pagāniem, ar kuriem karoja Mazovijas (poļu \pltxti{Mazowsze, Mazowiecka ziemia}, vācu ~\detxti{Masowien}) hercogs Konrāds. 1225.~gadā hercogs lūdza palīdzību Teitoņu (Vācu) ordenim (latīņu \latxti{Ordo fratrum domus Sanctae Mariae Theutonicorum Ierosolimitanorum}, \latxti{Ordo Teutonicus}, vācu \detxti{Orden der Brüder vom Deutschen Haus St. Mariens in Jerusalem}, saīsināti: \detxti{Deutscher Orden}, \detxti{Deutscher Ritterorden}~--- katoļu reliģiskais bruņinieku ordenis, dibināts XII gadsimtā Palestīnā). Teitoņi ieradās Polijā 1232.~gadā, apmetās Vislas labajā krastā, sāka sagrābt prūšu zemes, pašus tos piespiežot pieņemt kristietību. Iekarotajās zemēs ieradās vācu kolonisti. Ordenis uzurpēja tiesības kristīt pagānus austrumos, noliedzot tādas pat tiesības poļiem. Galvenais ordeņa pretinieks bija Lietuvas lielkņaziste (lietuviešu \lttxti{Lietuvos Didžioji Kunigaikštystė}, poļu \pltxti{Wielkie Księstwo Litewskie}, krievu \rutxti{Великое княжество Литовское, Русское, Жемойтске и иных}; XIII~gs.--1796.). Lai atvairītu teitoņu triecienus, lielkņaziste 1385.~gadā Krēvas pilī (baltkrievu \betxti{Крэўскі замак}, poļu \pltxti{Zamek w Krewie}, lietuviešu \lttxti{Krėvos pilis}, mūsdienu Baltkrievijas teritorijā) noslēdza Polijas un Lietuvas personālo ūniju, kad abām valstīm bija viens valdnieks no Jagelloņu (lietuv.: \lttxti{Jogailaičiai}, poļu: \pltxti{Jagiellonowie}, baltkr.: \betxti{Ягелоны)} dinastijas. Tas mainīja spēku samēru par sliktu teitoņiem. Taču arī pēc Vācu ordeņa novājināšanās, Reformācijas un ordeņa zemju sekularizācijas vācu un poļu valstu attiecības neuzlabojās.

Gan vācieši, gan poļi kopš viduslaikiem netaupīja negatīvās emocijas pret kaimiņiem.

Jau Prūsijas karalis Fridrihs II, apkopojot poļiem naidīgos uzskatus, izteicās, ka sabiedrība, „kuras uzvārdi beidzas ar --ki”, ir „visās nozīmēs nicināma nācija'' (\detxti{Die ganze „Gesellschaft mit dem Namen auf~--- ki” ist „eine in jeder Hinsicht verächtliche Nation”}.) Prūsijas vēsturnieki poļus un citus slāvus apzīmēja par „barbarisma apustuļiem” („\detxti{Apostel der Barbarei}”). Oficiālais Prūsijas valsts historiogrāfs L.~Ranke uzskatīja, ka romāņu-ģermāņu tautām vienmēr ir bijusi noteicošā loma Eiropas vēsturē, bet kas attiecās uz slāviem, to loma bija aizstāvēt „Eiropas civilizāciju” no klejotāju uzbrukumiem. Šis zinātnieks uz slāviem, tai skaitā poļiem, raudzījās no augšas, uzsverot viņu kulturālo atpalicību un Vācijas civilizējošo lomu to vidū. Arī viņa pēctecis amatā H.~Treičke ģermāņu virzību slāvu zemēs nosauca par „kultūras tautu cīņu pret barbariem”. Vācu vēstures grāmatās Polijas vēsture tika reducēta uz anarhijas, iekšējo krīžu un cīņas pret kaimiņiem attēlojumu. Pozitīvie procesi, sasniegumi saimnieciskajā un kultūras jomā nebija pienācīgi novērtēti. (Arī mūsdienās vācu vēsturiskajā literatūrā var atrast šo stereotipu atliekas.) Polijas vēstures materiālam Vācijas valdošo slāņu acīs bija jākalpo vācu imperiālisma ekspansionistisko centienu pamatojumam.

Savukārt poļu vēsturnieki jau no XIX gadsimta oponēja prūsiskajai historiogrāfijai. Šajā pretstāvē piedalījās arī poļu literatūra un māksla, negatīvās krāsās attēlojot vācu pagātni. Viedokļu pretrunīgumu labi ilustrē divi mīti par kauju pie Grīnvaldes~--- Tannenbergas (poļu \pltxti{Bitwa pod Grunwaldem}, vācu \detxti{Schlacht bei Tannenberg}, 1410).

Pēc poļu mīta Teitoņu ordenis falsificēja Romas pāvesta bullu, pēc tam veica karagājienus pret prūšiem, Lietuvu un Poliju, 1308.~gadā ar viltu sagrāba Gdaņsku un Piejūras apgabalu, mēģināja pakļaut Lietuvu, kuru izglāba 1385.~gada personālā ūnija ar Poliju. Kauja pie Grīnvaldes bija mūžīgo Polijas cīņu pret Vāciju kvintesence. Pēc uzvaras Grīnvaldes kaujā Polijā valdījušās Jagelloņu dinastijas kļūda bija atļauja no krustnešu vasaļvalsts radīt mantojamu hercogisti, kura izveidojās par Prūsiju un vēlāk piedalījās Žečpospolitas dalīšanā. Arī Krievijas impērijā ietilpinātajā Polijā 1910.~gadā tika plaši atzīmēta Grīnvaldes kaujas jubileja.

Berlīnes ieņemšana 1945.~gadā bija otra Grīnvalde, Austrumprūsijas pievienošana Polijai uz visiem laikiem pielika punktu teitoņu agresijai. Pēc 1945.~gada Polijas jauniegūtajās rietumu zemēs laikam nebija pilsētas, kurā nebūtu savas Grīnvaldes ielas. Kaut oficiālā Polijas Tautas Republikas propaganda veidoja divus vāciešu tēlus: Poliju apdraudošus, atklāti agresīvus VFR pilsoņus, un sabiedrotos no VDR, jāsaka gan, ka katoliskie poļu iedzīvotāji juta maz simpātiju pret pēdējiem, kurus parasti identificēja kā prūšus.

Arī jau mūsdienās populāro poļu politiķu brāļu Ļeha un Jaroslava Kačinsku pirmsvēlēšanu kampaņa 2005.~gadā tika uzsākta uz J.~Matejko gleznas „Grīnvaldes kauja” fona.

Pilnīgi pretējs ir vācu mīts. Teitoņu ordenis neauglīgajās Eiropas ziemeļaustrumu zemēs, pārvarot pagānu pretestību, ar krustu un zobenu iedibināja kristietību, piesaistīja sava laikmeta modernās tehnoloģijas, nodibināja 96 pilsētas, uzcēla 90 cietokšņus, radīja sava laikmeta paraugvalsti, kuras mantinieces bija Prūsija un Vācijas impērija. (Vēl nacistiskajā t.s. Trešajā reihā darbojās Teitoņu ordeņa kults.) Nedienas sākās tad, kad Polija, jūtot skaudību pret teitoņu valsts panākumiem, noniecināja kristīgo solidaritāti, vienojās ar pagāniem. 1410.~gadā tā sāka karu pret ordeni un diemžēl kaujā pie Tannenbergas, galvenokārt pateicoties lietuviešu spēkiem, sakāva to. Taču vēsturiskais taisnīgums uzvarēja~--- XVII gadsimtā Prūsijas karaļiem izdevās nomest poļu kundzību, 18.~gadsimtā~--- atgūt agrāk zaudētās teritorijas, atbrīvojot tās no poliskās nekārtības.

Vācijā no XVIII gadsimta eksistēja termins „\detxti{polnische Wirtschaft}”~--- poļu saimniecība, kas, vācu acīm raugoties, nozīmēja pilnīgu nesaimnieciskumu. (Tiesa, pēc 1934.~gadā parakstītās Vācijas un Polijas deklarācijas par spēka nepielietošanu 1936.~gadā iznākušajā V.~Noltinga grāmatā bija norādīts, ka tajā laikā minētais termins esot jānodod „grabažu noliktavā”, taču vēl ilgi daudzi vācieši poļu ekonomiskos sasniegumus augstu nevērtēja.) XIX gadsimtā Vācija Prūsijas vadībā kļuva par valstiskuma, ekonomikas un kara mākslas augstāko sasniegumu. 19.gadsimtā otrajā pusē, kad poļi visiem spēkiem cīnījās par nacionālo pastāvēšanu, vācieši kā pirmo un galveno „senseno” ienaidnieku uzlūkoja frančus; tikai valsts austrumu teritorijās, kur dzīvoja daudz poļu, vācieši ar lielākām bažām vērās austrumu virzienā.

Pēc Pirmā pasaules kara bezdibenis starp vācu un poļu vēstures ainām vēl padziļinājās. Kā atzīmējis poļu vēsturnieks M.~Mročko, poļu priekšstatiem par „vācu virzību uz austrumiem” („\detxti{Deutschen Drang nach Osten}”) blakus pastāvēja vācu viedoklis par „slāvu pieplūdumu” („\detxti{Andrang des Slawentums}”) rietumos. Vācu vēstures zinātne Polijas vēsturi un politiku tēloja kā „spiedienu uz Rietumiem” („\detxti{Drang nach Westen”}) un kā „poļu briesmas” („\detxti{polnische Gefahr}”). Ievērojamais vācu vēsturnieks J.~Hallers 1923.~gadā iznākušajā kapitāldarbā „Vācu vēstures laikmeti'' („\detxti{Epochen der deutschen Geschichte}'') rakstīja: „Poļus mūs ar pilnām tiesībām uzskatām par vācu mūžseniem ienaidniekiem Austrumos''. („\detxti{Polen denken wir uns mit Recht als den Erbfeind der Deutschen im Osten}”.)

Arī mūsdienās, kad bundesvēra augstākie virsnieki noliek ziedus Vesterplatē (\detxti{Westerplatte}~--- pussala Baltijas jūrā, kur Otrā pasaules kara sākumā poļi varonīgi pretojās nacistiskajiem iebrucējiem), vāciešu vidū vēl nav izveidojies pamatots un daudz-maz vienots priekšstats par kaimiņtautas vēsturi. Vācu publicistikā arī XX gadsimta 80.~gados varēja lasīt dusmu pilnus vārdus par „nepateicīgajiem” poļiem, kuri gan saņēma no VFR materiālu palīdzību streikotājiem~--- protestētājiem pret „sociālistisko” pārvaldi Polijā, bet tik un tā neatzina pēckara vācu iedzīvotāju deportāciju netaisnīgumu, vietā un „nevietā” atgādināja vācu iebrucēju nodarījumus Polijas teritorijā utt. Kāda pirms Otrā pasaules kara Polijā dzīvojusi un no turienes izraidīta vāciete E.~Losere 1981.~gadā rakstīja: „Kopš kristīšanas [poļu] tauta tika pakļauta stingrai kleriķu uzraudzībai, kura traucēja personības attīstībai. Viņi nespēj izrauties no šiem spaidiem. Viņi tika tā apspiesti, ka uzkrāto agresiju arvien atkal un atkal izlādē bezprecedenta naidā pret brīvākajiem un bagātākajiem vāciešiem. \citespace{} Baznīca triumfē Polijā. Un katoļu baznīca no pašiem sākumiem bija Vācijas valsts niknākais ienaidnieks. Poļi tika un tiks izmantoti kā svira lai nolaistu vācu un Vācijas tautsaimniecības asinis”. Arī XXI~gadsimta sākumā publicistikā var atrast līdzīgus „zinātniskus'' spriedumus par citām tautām.

Kā Hamburgā izdotajā vācu-poļu žurnālā „\detxti{Dialog}” („Dialogs”) rakstījis poļu žurnālists Ā.~Kržeminskis, vāciešiem attieksme pret Poliju un poļiem arī XXI gadsimtā ir atmiņas politikas (\detxti{Gedächtnispolitik}) pārbaudes līdzeklis. Polija taču bija pirmā valsts, kura ar ieročiem cīnījās pret nacistiskās Vācijas agresiju, no pirmās dienas bija antihitleriskās koalīcijas locekle, cieta kara laikā milzīgus cilvēku un materiālos zaudējumus, un, kaut arī ne no uzvarētājām lielvalstīm, ne arī Vācijas puses netika atzīta par ar tām vienlīdzīgu, tomēr kļuva par Vācijai tik sāpīgi zaudēto teritoriju ieguvēju. Taču vāciešiem atmiņās par Otro pasaules karu Polijā ir periferiāla vieta. Kā norādījis ievērojamais vācu vēsturnieks H.U.~Vēlers, Vācijas abiturientiem un studentiem ir zināšanas par Otrā pasaules kara laikā nogalinātajiem sešiem miljoniem ebreju, taču kad viņiem saka, ka tai pat karā dzīvību zaudēja gandrīz katrs piektais polis [precīzāk gan būtu teikt~--- Polijas pavalstnieks~--- V.Š.], un jau kara sākumā no vācu okupētajiem apgabaliem tika padzīti 800~000 iedzīvotāju, nākas sastapties ar nezināšanu un izbrīnu. Sociologs un vēsturnieks Rietumu institūta direktors Poznaņā A.~Saksons raksta, ka vēl daudz ūdens ir jāaizplūst robežupē Oderā līdz vācieši mainīs savus uzskatus par poļiem. XX gadsimta beigās veiktā socioloģiskā aptauja rādīja, ka 50\% vāciešu nav priekšstata par Poliju vai arī viņiem tā neinteresē. 2000.~gadā Polija viņu acīs bija pēdējā vietā no 26 pievilcīgākajām Eiropas valstīm.

[Pamatīgāks materiāls par Polijas vēstures izpēti vācu zinātniskajā literatūrā atrodams G.~Rodes rakstā „\detxti{Der Geschichte Polens in der deutschen Geschitsschreibung}” („Polijas vēsture vācu vēstures literatūrā'') krājumā „\detxti{Nationalgeschichte als Problem der deutschen und polnischen Geschichtsschreibung”}. Braunschweig, 1983.) Par Polijai veltītās vācu historiogrāfijas vērtējumu no pašu poļu viedokļa var izlasīt H.~Olševska rakstā „\detxti{Die deutsche Historiographie über Polen aus polnischen Sicht}” („Vācu historiogrāfija par Poliju no pašu poļu viedokļa”) rakstu krājumā \detxti{Dittmar Dahlmann (Hg)} „\detxti{Hundert Jahre Osteuropäische Geschichte. Vergangenheit, Gegenwart und Zukunft}''. Stuttgart, 2005].

Negatīvi domāja arī poļi par vāciešiem. Ne velti 1939.~gada 1.~septembrī Polijas prezidenta I.~Moscicka runā izskanēja vārdi: „mūsu senais ienaidnieks ir uzsācis karadarbību pret Polijas valsti”. Poļi labi atceras, ka Varšava, faktiskā Polijas galvaspilsēta kopš 1596.~gada, pēdējos divarpus gadsimtos trīsreiz (1795., 1915., 1939.~gg.) krita vācu militāristu rokās. Aušvicas (\detxti{Auschwitz}) jeb Osvencimas, kā arī daudzas citas nacistu ierīkotās koncentrācijas nometnes atradās Polijā un arī šodien atgādina saviem apmeklētājiem par vācu iebrucēju noziegumiem kopš krustnešu laikiem līdz pat Otrajam pasaules karam, kad vārds „vāciešu” kļuva par sinonīmu vārdam „noziedzīgs”.

Protams, vēsturē atrodami daudzi fakti arī par poļu un vāciešu sadarbību. Patiesība ir tā, ka naidīguma izvirdumiem starp vāciešiem un poļiem un, vēl lielākā mērā, starp viņu politiķiem, var nostādīt pretī periodus, kad abas tautas prata sadzīvot draudzīgi. Piemēram, Boļeslavs Drosmīgais sniedza militāru palīdzību Svētās Romas impērijas ķeizaram Ottonam III, pats tika kronēts pateicoties tā atbalstam. Abi personīgi tikās t.s. Gņezno (poļu \pltxti{Gniezno}, vācu \detxti{Gnesen}) kongresā (1~000), kur tika radīta patstāvīga Gņezno arhibīskapija, kas Polijai garantēja baznīcas neatkarību no vācu baznīcas. Mūsdienās pārsvaru gūst uzskats, ka pretēji abām augstāk izklāstītajām leģendām par Grīnvaldes kauju XV~gadsimtā sadursme starp Poliju un krustnešiem nebija konflikts starp nācijām vai pat civilizācijām, bet gan starp valstīm. Pie tam abi konkurējošie bloki: gan Polija un Lietuva, gan Teitoņu ordenis, kuru atbalstīja arī Svētā Romas impērija un Čehijas karaliste, pārstāvēja daudzus etnosus.

Vācu un poļu kultūras ir savstarpēji bagātinājušas viena otru. Lielākoties vācu (un ebreju) amatnieku apdzīvotās Polijas pilsētas pārņēma vācu pilsētu Lībekas, Magdeburgas, Nirnbergas vai Halles tiesības. Neviena valoda, izņemot radniecīgo krievu valodu, nav tik stipri ietekmējusi poļu valodu kā vācu valoda. Vācu „\detxti{Rathaus}'' (rātsnams) poliski ir „\pltxti{ratusz}”, „\detxti{Bürgermeister}” (birģermeistars)~--- „\pltxti{burmistrz}” un „\detxti{Pflug}” (arkls)~--- „\pltxti{plug}”. Poļu valodnieks G.~Korbuts ir atzinis, ka no 100 poļu sarunu valodas vārdiem 16--17 ir vācu izcelsmes. Savukārt vācu valodā tādi vārdi kā „\detxti{Dolmetscher}” (tulks), „\detxti{Droschke}” (važoņa rati), „\detxti{Gurke}” (gurķis), „\detxti{Peitsche}” (pātaga), „\detxti{Quark}” (biezpiens), „\detxti{Zobel}” (sabulis) u.c. ir pārņemti no senās poļu valodas. Polijas nacionālās himnas mūzikas autora ģenerāļa J.~Dombrovska māte bija vāciete. Himnas teksta autors ģenerālis J.~Vibickis cienīja vācu kultūru un izaudzināja savus dēlus Drēzdenē vācu klasikas garā. Tādu piemēru uzskaitījumu varētu turpināt.

Taču, kaut arī šodien „mūžsenā ienaidnieka” tēls abās pusēs ir krietni pabālējis, daži tam raksturīgi priekšstati ir vēl dzīvi. Poļiem ir paruna: „\pltxti{Póki swiat swiatem, Polak Niemcowi nie bedzie bratem}”. („Tik ilgi, kamēr pasaule pastāvēs, polis nekad nebūs vācietim brālis.'') Īpaši negatīvas emocijas poļos izraisa jēdziens ''\pltxti{Prusy}''(prūši). Līdz mūsdienām Varšavā nav Berlīnes vai Prūšu ielas. Toties ir Sakšu pils, Sakšu parks, Sakšu ass (ceļš), Leipcigas iela un Drēzdenes iela. Poļu--sakšu tradicionālās attiecības ir dzīvas vēl šodien. Vārdam „saksis”, atšķirībā no jēdziena „prūsis” poļu ausīs ir pozitīva nozīme. Tomēr, domājot par Vāciju, poļi vispirms atceras Prūsiju un tās iekarojumus Polijā.

Kā atzīmējis populārais poļu publicists A.~Kržeminskis, ciktāl paši poļi skata savas vēstures tumšās lappuses un kritizē daudzus nacionālos mītus, tiktāl viss ir kārtībā, īpašu saasinājumu parasti nav. Taču, ja kritikai pievienojas kāds vācietis, pieminot, piemēram, poļu represijas pret vācu gūstekņiem un nosoda vācu deportācijas no pēckara Polijas, poļu puse nonāk līdz pat publiskiem skandāliem.

Ilgstoša saskarsme poļiem ir bijusi arī ar krieviem (\pltxti{rosjanie}) un Krieviju (\pltxti{Rosja}). Taču divas kaimiņos dzīvojošas lielas radniecīgas tautas vēstures gaita saistīja ar dažādām kristīgās ticības konfesijām, dažādiem kultūras virzieniem, kas saasināja to attīstības nevienmērību. Un tāpat kā daudzos citos gadījumos, kad starp radniecīgām tautām (Izraēlas un arābu valstu iedzīvotājiem, serbiem un horvātiem u.c.), arī starp poļiem un krieviem ilgstoši pastāv it kā nepārvaramas pretrunas. Neatkarīgās Polijas valsts vēsturē grūti atrast periodus, kad tā būtu ilgstoši sadarbojusies ar Krieviju.

Polijas un Krievijas kopīgā vēsture neļauj ne vienu no tām nosaukt tikai par upuri, ne otru tikai par agresoru. Nebūt ne vienmēr poļi ir bijuši austrumu kaimiņa agresijas objekts, kā tas šķiet daudziem Polijas iedzīvotājiem, bieži viņi paši ir tam uzbrukuši. Ne reizi vien vēstures gaitā Polija ir savā labā izmantojusi Krievijai grūtas situācijas.

No XI līdz XVII gadsimtam poļi daudzkārt iebruka Krievzemes robežās, sagrāba milzīgas teritorijas un gadsimtiem ilgi ekspluatēja tās, tai laikā kad krievu karaspēks šai laikā ne reizi nenonāca pašas Polijas teritorijā.

Tā Kijevas Krievzemes laikos poļu karalis Boļeslavs Drosmīgais pēc sava znota bijušā Kijevas kņaza Svjatopolka, kurš bija zaudējis cīņu savam brālim Jaroslavam Gudrajam, aicinājuma devās uz Volīniju, sakāva Jaroslava Gudrā karadraudzi, 1018.~gadā ieņēma Kijevu, bet tā vietā lai atdotu to valdīšanā savas meitas vīram Svjatopolkam, pats sāka tajā valdīt. Taču pilsētnieki, sašutuši par poļu karadraudzes patvaļībām, sacēlās un Boļeslavs bija spiests atstāt pilsētu. Iejaukties Kijevas Krievzemes lietās mēģināja arī Boļeslavs II Drošais, taču pēc sadursmēm ar vietējiem iedzīvotājiem bija spiests savu karadraudzi izvest.

Jau dziļāku poļu un krievu pretrunu rašanās saistāma ar Maskavas Krievzemes jeb Maskavijas ģenēzi. Vēl XIII--XIV gadsimtā Lietuvas valsts atkaroja Zelta Ordai (mongoļu \mntxti{Altan Ord}, tatāru \tttxti{Altın Urda}, mongoļu-tatāru valsts XIII--XV~gs.) lielu daļu no senkrievu zemēm. Tieši Lietuvas valsts toreiz pretendēja uz šo zemju vienotājas lomu, taču pēc jau minētās 1385.~gadā noslēgtās Krēvas ūnijas arī Lietuvā izplatījās katoļu ticība, sākās senkrievu zemju polonizācija. Tāpēc Lietuva senkrievu acīs pakāpeniski zaudēja „savas” valsts tēlu, atbalsta lomu cīņā pret svešzemju jūgu; par tādu kļuva Maskavija, kura palika uzticīga pareizticībai.

1480.~gadā Lietuvas lielkņazs un Polijas karalis Kazimirs IV mēģināja ieņemt Novgorodu un Pleskavu. Protams, arī krievu kņazi veica karagājienus pret poļiem, taču tiem faktiski bija savu zemju aizsardzības un arī atriebības raksturs.

Turpmāk no 1558.~gada (Livonijas kara sākums) līdz 1939.~gadam (oficiālais Otrā pasaules kara sākums) Polija ar Krieviju karoja 13 reizes, pie tam poļi 2 karus uzvarēja un 11 zaudēja.

Pēc 1569.~gadā noslēgtās Ļubļinas ūnijas (poļu \pltxti{Unia Lubelska}, lietuviešu \lttxti{Liublino unija}, baltkrievu \betxti{Люблінская унія}~--- līgums par Polijas karalistes un Lietuvas lielkņazistes apvienošanos konfederatīvā valstī~--- Žečpospolitā ar vēlētu karali) sākās gadsimtiem ilgusī tās un Maskavas valsts sāncensība par austrumslāvu zemju apvienošanu un pakļaušanu.

Var teikt, ka ap 200 gadus XVI --XVII gadsimtā Maskavija izjuta Žečpospolitas spiedienu. Žečpospolita uzskatīja sevi par galveno katoļu ticības izplatītāju austrumos, turpretī Maskavas valsts cīnījās ne tikai par „krievu zemju” vienību, bet arī „visu pareizticīgo” kristiešu interesēm. Maskavija un Polija saskatīja viena otrā antagonistu slāvu zemēs. Polijai kaimiņos Krievijā pastāvēja Maskavas kā „trešās Romas” koncepcija, vadoties no XV~gadsimtā izskanējušās tēzes, ka „Divas Romas [Roma un Bizantija] ir kritušas, trešā [Maskava] stāv, bet ceturtās nekad nebūs” („\rutxti{«Два Рима пали, третий стоит, а четвёртому не бывать}”). XVI gadsimtā Polijas-Lietuvas valsts robeža atradās netālu no Možaiskas. No Livonijas kara (1558--1583) laikiem, kad Maskavija un Žečpospolita cīnījās par ietekmi Baltijā un Stefans Batorijs neveiksmīgi mēģināja ieņemt Pleskavu, abu valstu savstarpējās cīņas gan apdzisa, gan uzliesmoja ar jaunu sparu.

Tās saasinājās it īpaši pēc Polijas iejaukšanās Krievijas lietās t.~s. „Juku laikos” XVII gadsimta sākumā, kad poļu karaspēks okupēja Maskavu: 1604.~gadā kopā ar Viltusdmitriju (\rutxti{Лжедимитрий}, ?--1606) un 1610.--1612.~gadā „cara” Vladislava laikā. XVII gadsimtā poļi un lietuvieši tik bieži un ilgi necīnījās ne ar vienu citu ienaidnieku kā ar Maskavas valsti. XVII gadsimta sākumā Polijā bija populāra atsaukšanās uz spāņu konkiskadoru veiktajiem iekarojumiem Amerikā. Tika spriests: spāņu taču bija ļoti nedaudz salīdzinājumā ar indiāņiem, bet viņi tos uzvarēja. Poļu ir daudz, vai nu viņi netiks galā ar „moskaļiem” [nicīgs krievu nosaukums, lietots poļu, ukraiņu u.c. vidū.~--- V.Š.]? Nabadzīgā poļu šļahta cerībā uz bagātīgu laupījumu devās palīgā Viltusdmitrijam uz Maskavu. Vairākkārt Polija ir bijusi tuvu savam mērķim nostiprināt robežu ar Krieviju pa Dņepras upi.

Nobeļa prēmijas laureāts krievu rakstnieks A.~Solžeņicins, izklāstot abu valstu savstarpējos pāridarījumus, sāka ar Krievijas puses nodarījumiem Polijai, taču pēc tam rakstīja arī par to, ka „iepriekšējos gadsimtos plaukstošā, stiprā, pašpārliecinātā Polija ne īsāku laiku un ne vājāk iekaroja un apspieda mūs [krievus~--- V.Š.]”. Īpaši viņš izcēla „juku laikus” Krievijā XVII gadsimta sākumā, kad „poļi teju neatņēma mums nacionālo neatkarību, šo briesmu dziļums bija ne mazāks kā tatāru iebrukumam, jo poļi apdraudēja arī pareizticību. Pie sevis viņi to sistemātiski apspieda, dzina ūnijā [domāta 1596.~gadā noslēgtā Brestas ūnija~--- V.Š.] \citespace{} Mūsu juku laikos Polijas ekspansiju poļu sabiedrība uztvēra kā normālu un pat pareizu politiku. Paši sevi poļi iedomājās kā Dieva izredzētu tautu, kristietības bastionu, ar uzdevumu izplatīt īstenu kristietību puspāgāniskajos pareizticīgajos, mežonīgajā Maskavijā, būt par universitāšu renesanses kultūras izplatītājiem”.

Ievērojamais krievu vēsturnieks V.~Kļučevskis apgalvoja, ka polis un tatārs bija īstena krievu cilvēka pastāvīgi ienaidnieki arī XVIII gadsimtā. Pēc vēsturnieka un publicista V.~Fiļeviča vārdiem viduslaikos „krievi sātanu iedomājās poļa izskatā” („\rutxti{русские беса представляли в виде ляха}”). Taču katru reizi poļu ekspansijai sekoja prettrieciens.

Pēc Romanovu (\rutxti{Романовы}) dinastijas nodibināšanas (vēlāk Holšteinas-Gotorpas-Romanovu dinastija~--- \rutxti{Гольштейн-Готторп-Романовская династия}), kura valdīja Krievijas caristē, pēc tam Krievijas impērijā laikā no 1613.~gada līdz 1917.~gadam, tā pakāpeniski atkaroja austrumslāvu zemes.

Dažkārt Polija un Krievija gan bija arī sabiedrotās, piemēram, XVII gadsimta beigās pret kopējo ienaidnieku~--- Turciju, Ziemeļu karā (1700--1721) pret Zviedriju. Taču tad atkal priekšplānā izvirzījās nesaskaņas. Pateicoties cara, vēlāk imperatora Pētera I īstenotajai modernizācijai, Krievija izvirzījās priekšā kaimiņvalstij. XVIII~--- XX gadsimtā jau tā diktēja poļiem politiskās spēles noteikumus. Krievijas robežas rietumos atvirzījās aiz Vislas. XVIII~gadsimtā Žečpospolita zaudēja savu valstiskumu. Taču poļu šļahta nevarēja samierināties ne ar savas neatkarības, ne ar citu tautu apdzīvoto, bet līdz tam Žečpospolitas varā esošo teritoriju zaudējumu, meklēja pret Krieviju sabiedrotos Centrālajā un Rietumu Eiropā. 1812.~gadā poļi, karojot nu jau Napoleona I (\frtxti{Napoléon I Bonaparte}, 1769--1821) karaspēka sastāvā, atkal nonāca Maskavā.

Divreiz XIX gadsimtā uzliesmoja varenas poļu sacelšanās pret Krievijas varu. Kaut poļi cīnījās par savu neatkarību pret visām to sadalījušajām valstīm, taču galvenais ienaidnieks bija Krievija, kura bija ieguvusi centrālos Polijas apgabalus. Iekšējās pretrunas gan Krievijā, gan Polijā neļāva tām miermīlīgi izšķirties pēc Pirmā pasaules kara un Oktobra apvērsuma Krievijā. Daudz sarkanarmiešu zaudēja dzīvību 1920.~gadā pie Varšavas. Poļi kā vienu no svarīgākajiem notikumiem Otrā pasaules kara gados atceras 1939.~gada 17.~septembrī notikušo Sarkanās armijas invāziju Polijas teritorijā, pēc kara ar izteiktu neapmierinātību pārcieta PSRS valdošā staļiniskā režīma ietekmes izplatību uz Poliju. Tā tūkstošgadīgajā līdzāspastāvēšanas laikā divu slāvu tautu attiecībās uzkrājās ne mazums notikumu, kurus katra puse traktēja citādi. Uz vēsturisko notikumu subjektīvas uztveres pamata formējās vispārnacionāli mīti. Krievijā~--- mīts par naidīgajiem poļiem-katoļiem, kuri jebkurā situācijā ir gatavi darboties tā, lai ieriebtu „moskaļiem”, Polijā~--- mīts par vienmēr to apdraudošo, barbarisko un impērisko Krieviju.

No dotā darba, kas veltīts Polijas XIX un XX gadsimta vēsturei, autora viedokļa ļoti būtisks ir jautājums par poļu un krievu tautu attiecībām minētajā laikā. Tieši to samežģījumi ir nesuši poļu tautai daudz ciešanu.

Par šīm attiecībām, par daudzo Polijas un Krievijas konfliktu būtību rakstījuši dažādu tautu vēsturnieki un publicisti. Dažu no viņiem skaidrojumi nevar tikt uzskatīti par zinātniskiem.

Piemēram, baltkrievu publicists A.~Tarass, atsaucoties uz nenosauktu zinātnieku ģenētiskajiem pētījumiem, raksta, ka „krievi (no vienas puses), baltkrievi un poļi (no otras puses) ir etniski atšķirīgas tautas, antropoloģiski dažādas rases” un arī „tieši iedzimtas (ģenētiskas) atšķirības izšķiroši veicināja maskaviešu (vēlāk krievu) tradicionālo noraidīšo attieksmi pret Eiropas dzīvesveidu, Eiropas kultūru, katolicismu, uniātismu un protestantismu, viņu naidu pret Lietuvas Lielkņazisti, Kungu Lielo Novgorodu, Livoniju, Polijas Karalisti.” Minētais autors arī apgalvo, ka „tieši poļu un baltkrievu etniskās kopības fakts izskaidro Žečpospolitas izveidi”. Identificējot baltkrievus ar poļiem, A.~Tarass savos darbos vēsturiskos notikumus skata no polonopfīlām un rusofobām pozīcijām. Tā kā autors nenosauc viņa uzskatus apstiprinošo „ģenētiķu” vārdus, hipotēze šai grāmatā netiek komentēta, vienīgi jāsaka, ka šāds vienkāršots sarežģītu procesu izskaidrojums ir tāls no patiesības.

Daļa vēsturnieku uzskata, ka galvenā bija tīri politiska konkurence par dominanci Austrumeiropā.

Citi aizstāv viedokli, ka visu nesaskaņu pamatā bija nesamierināma cīņa starp kristietības austrumu un rietumu atzariem. Reizē līdzās reliģiskajai konfrontācijai XVII gadsimta notikumos tiek izšķirtas arī etniskam konfliktam raksturīgas pazīmes. Tā, pazīstamais poļu izcelsmes amerikāņu vēsturnieks R.~Paips atzīmējis: „Vēsturiskajā perspektīvā poļu--krievu attiecības nekad nav bijušas labas. Starp \citespace{} [šīm] valstīm vienmēr radās sasprindzinājums. \citespace{} Krievija uzskata Poliju par slāvu tradīciju nodevēju, jo \citespace{} [tā] pieņēma katoļu ticību. Savukārt poļi nemīl krievus, jo pārāk daudz ir no viņiem cietuši”.

Nebūt neatbilst patiesībai priekšstats, ka lielas daļas poļu naidīgā izturēšanās pret visu krievisko radās kā sekas trijām Polijas dalīšanām XVIII gadsimtā, poļu sacelšanos apspiešanām XIX gadsimtā, padomiskā „sociālisma” modeļa uzspiešanai, veselai virknei notikumu, kuros cietusī puse bija poļu tauta. Kā parāda poļu vēsturnieces A.~Neviaras 2006.~gadā publicēts pētījums (Niewiara A. Moskwicin—Moskal—Rosjanin w dokumentach prywatnych. Łódź, 2006) „moskaļa-aziāta”, „moskaļa-iebrucēja,” „moskaļa-Kristus nodevēja” tēls parādās poļu rakstītajos avotos vēl sen pirms XVIII~gadsimta, kad vēl nekāda poļu apspiešana no krievu puses nebija iespējama. Līdz Ļubļinas ūnijai (1569) Polijai nebija robežas ar Krieviju, tā bija Lietuvas lielkņazistei. Pēc ūnijas noslēgšanas ar Krieviju konfliktus risināja jau apvienotā Polijas-Lietuvas valsts (Žečpospolita). Ja līdz tam Polijas galvenie ienaidnieki bija rietumos (Vācija) un dienvidos (Turcija), tad tagad tā iesaistījās karos arī austrumos (pret Krieviju). Tad arī itin dabīgi radās poļu šļahtas naids pret „moskaļiem”. Tas bija ne Polijas apspiešanas no Krievijas puses rezultāts (kaut tā to sekmēja), bet jau senāks poļu šļahtas kolektīvo interešu izpausmes veids.

Līdz „sociālisma” padomju modeļa bankrotam, PSRS sabrukumam, Polijas Tautas republikas pārveidei, historiogrāfijā poļu-padomju attiecības tika skatītas no divām pretējām, savstarpēji izslēdzošām pieejām.

Viena bija pārstāvēta PSRS un PTR vēsturnieku darbos, otra Rietumu pētnieku, poļu emigrantu vidū dzīvojošo vēsturnieku, kā arī to poļu pētnieku monogrāfijās un rakstos, kuri, bieži ar pseidonīmiem, publicējās PTR valdības cenzūrai nepakļautajā vēsturiskajā literatūrā.

Pirmā koncepcija sāka veidoties jau Otrā pasaules kara laikā un bāzējās uz padomju vadības oficiālo pozīciju, kura bija pārstāvēta plaši pieejamos politiskajos un diplomātiskajos dokumentos, padomju līderu runās. Autori, kuri pieturējās šai koncepcijai, PSRS realizēto ārpolitiku, arī pret Poliju, vērtēja kā miermīlīgu, progresīvu, internacionālistisku. Turpretī Polijas Republikas starpkaru un Otrā pasaules kara laikā realizētā politika tika skatīta kā Padomju Savienībai atklāti naidīga, nacistiskās Vācijas agresīvos centienus veicinoša. Visspilgtāk šī tendence atklājās padomju vēsturnieku darbos, veltītos Polijas vēstures apkopojumam, Otrajam pasaules karam, ārpolitikas un diplomātijas vēsturei. Analoģisku pozīciju ar nelielām niansēm ieņēma oficiālā Polijas historiogrāfija 1940.~gadu beigās~--- 80.~gados.

Otrās koncepcijas piekritēji par pamatu ņēma uzskatus, gluži pretējus augstāk izklāstītajai pieejai. Viņi viennozīmīgi saskatīja padomju ārpolitikā agresīvo, pretpolisko ievirzi, kā starpkaru, tā arī Otrā pasaules kara gados, īpaši uzsvēra represijas pret poļiem, to deportācijas, poļu virsnieku nošaušanu Katiņā u.~c. Pirmie tādi pētījumi parādījās jau drīz pēc Otrā pasaules kara. Visaptverošāk šī koncepceja izklāstīta V.~Pobog-Maļinovska darbā „Polijas jaunākā politiskā vēsture”. („\pltxti{Najnowsza historia polityczna Polski} 1864--1945”, 1--3, Paryż, Londyn, 1953.--1962.) Jāsaka, ka PSRS vadības nostāja tikai veicināja Rietumos izplatīto uzskatu izplatību arī pašā Polijā. Tā vietā, lai atslepenotu dokumentus, publicētu avotus par Polijas un Krievijas/PSRS attiecību vēsturi, atspēkotu mītus par to, tika vienkārši klusēts. Tas veda pie secinājuma~--- ja jau PSRS klusē, tad to apsūdzošie notikumu izklāsti ir patiesi, bet varbūt patiesība bija vēl briesmīgāka.

Abas koncepcijas pārstāvošajiem darbiem bija raksturīga klaja tendenciozitāte, tieksme padarīt baltu savu pusi un nomelnot pretējo, plaša sabiedriska fona iztrūkums, savstarpēja pretējās puses nacionālo interešu ignorēšana.

Vēstures mantojums arī mūsdienās bremzē labu kaimiņattiecību veidošanos starp Poliju un Krieviju. Pēc Otrā pasaules kara Polijas Tautas Republikas (1952--1989) historiogrāfijā nereti tika uzsvērts, ka neraugoties uz to, ka Krievija un Polija vēsturiski ilgus gadsimtus bija ienaidnieki, abu tautu attiecībās pārsvarā esot bijušas simpātijas un sadarbība. Taču šis apgalvojums nespēja aizsegt to, ka vēsturiskās kolīzijas, valstiskās attiecības nevarēja neietekmēt arī tautu, ierindas cilvēku savstarpējās jūtas. Tā, krievu revolucionārs P.~Lavrovs 1889.~gadā atstāstīja kā gadu pirms Parīzes komūnas (1871) „man nācās piedzīvot, ka poļu mākslas profesors atstāja telpu, dzirdot, ka Dombrovskis griežas pie manis krievu valodā.” Līdzīgi arī XX gadsimta 90.~gados šī darba autoram personīgi nācās piedzīvot, kā pret citām zemēm un tautām, arī Latviju un latviešiem, draudzīgi noskaņots inteliģents pavecāks polis, kurš prata gan krievu, gan vācu valodu, tomēr nevēlējās tajās sarunāties, ar to demonstrējot savu attieksmi pret bijušajiem „okupantiem”.

Tautu savstarpējās attiecības sevišķi traucējoši ietekmēja gan tā saucamā „sociālisma” apstākļos, gan pēc tā sabrukuma skanējušie oficiālie meli.

Kaut XX gadsimta 90.~gados sākās vēstures deideoloģizācija un depolitizācija, tā nebūt nav sekmīgi pabeigta. Kā pamatoti norādījis Daugavpils Universitātes profesors A.~Ivanovs: „\dots{}Dažādas politiskas institūcijas un politiskās elites vēsturi izmantoja un joprojām izmanto kā vienu no politikas veidošanas un īstenošanas instrumentiem, kas palīdz radīt elitei izdevīgus vēsturiskus mītus un nodrošina vienprātību sabiedrībā. \citespace{} vēstures politizēšanas pakāpi nosaka politiskais režīms: antidemokrātiskā režīma apstākļos vēsture kļūst ārkārtīgi politizēta un ideoloģizēta, tieši un atklāti kalpojot valstij un oficiālajai ideoloģijai, turpretim demokrātiskā vara ļauj vēsturniekiem relatīvi brīvi interpretēt vēstures procesu un vērtēt vēstures faktus. Kaut gan arī šī radošā brīvība nav absolūta.” Turpinot šo domu gaitu, var piebilst, ka centienu „nodrošināt vienprātību sabiedrībā” parādīšanās jebkurā valstī ir līdzvērtīga demokrātijas ierobežošanas mēģinājumiem.

Poļu vēsturnieks J.~Duračinskis ir atzīmējis, ka daudzos pēc 1989.~gada publicētajos darbos pamatoti novērsto iepriekšējo melu vietā parādās jauni meli un „gan tie vecie, gan mūsdienu meli izaug uz noteiktu (kardināli pretēju) simpātiju vai arī atsevišķu pētnieku politisko ieskatu pamata. Tas attiecas, galvenokārt uz izšķirošo notikumu un poļu politiskās skatuves galveno „aktieru” vērtējumu”. Vēlme pārskatīt uzkrāto pieredzi diemžēl bieži izpaužas vienkāršotā novecojušo uzskatu caurskatīšanā, agrāko plusu vietā liekot mīnusus un otrādi. Pēc Otrā pasaules kara norisušo sarežģīto iekšējo procesu padziļinātas interpretācijas vietā bieži tiek runāts tikai par tīri vardarbīgu padomju modeļa uzspiešanu Polijai.

[Minētā daudzu pētījumu autora J.~Duračinska historiogrāfiskie darbi var dod lasītājam diezgan pilnīgu priekšstatu par pašu poļu vēsturnieku veikumu savas dzimtenes XIX un XX~gadsimta vēstures izpētē. Daži no tiem ir pieejami arī krievu valodā. Piemēram: \rutxti{Э.~Дурачинский}. „\rutxti{О польской историографии новейшей истории}” („Par jaunāko laiku vēstures poļu historiogrāfiju”) // %http://library.by/portalus/modules/rushistory/readme.php?subaction=showfull&amp;id=1192094160&amp;archive=&amp;start_from=&amp;ucat=19&amp;category=19
% // 2.05.2010.]

Poļu vēsturnieki nereti nevēlas savas vēstures periodizācijā iekļaut arī pašu poļu izraisītās nacionālās sakāves, parasti cenšas uzsvērt, ka Polija vienmēr ir bijusi saistīta ar Rietumu civilizāciju, reizē mazinot sakaru nozīmi ar austrumu kaimiņiem. Acīmredzot to veicinājis arī tas, ka, runājot Polijas valsts darbinieka un rakstnieka S.~Kata-Mackeviča vārdiem, austrumos „mēs vienmēr bijām kungu nācija”, bet rietumos „mēs bijām tikai strādnieku un zemnieku nācija”.

Diemžēl mūsdienu poļu historiogrāfijā, bet īpaši politiskajā publicistikā krievu un poļu pretrunu un savstarpējās nepatikas jūtu cēloņi parasti tiek meklēti galvenokārt jaunāko laiku periodā~--- XVIII--XX gadsimta ietvaros, deviņos gadījumos no desmit uzmanību pievēršot tikai Polijas dalīšanai, uzsverot tās paverdzināšanu, ko veica Krievija, Prūsija/Vācija un Austrija/Austroungārija, arī nežēlīgo 1794., 1830.--1831. un 1863.--1864.~gada poļu sacelšanos apspiešanu, Otrajā pasaules karā Polijai nodarītās patiesās un iedomātās „netaisnības” un padomju modeļa uzspiešanu tai. Mazāk tiek akcentēta Polijas, tikpat cik Krievijas, agresīvā darbība Livonijas karā, poļu īstenotā Pleskavas aplenkšana (1581--1582, 1615), Krievijas un Žečpospolitas cīņa par Smoļensku, Polocku un kreisā krasta Ukrainu, poļu dalība t.s. „juku laikos” Krievijā XVII~gs. sākumā, kad norisa poļu šļahtas atbalstītā Viltusdmitrija I darbība un tam sekojošās divas poļu intervences Krievijā, kā arī krievu-poļu karš 1654.--1667.~gadā, kad Krievija atsaucās Zaporožjes (\uktxti{Запорiзька Січ}) kazaku hetmaņa B.~Hmeļnicka aicinājumam palīdzēt cīņā pret poļu apspiedējiem un pieņemt Zaporožjes kazakus Krievijas pavalstniecībā.

Ja daļa poļu vēsturnieku Polijas rīcību pret Krieviju vērstajos karagājienos piemin arī paškritiski, tad tik un tā vairākumā viņu rakstītajā jūtams zināms lepnums par poļu „varoņdarbiem” šajos iekarošanas karos. Par daudziem citiem, mazāk „spožiem” notikumiem, poļu sakāvēm, arī par poļu iekarotāju nodarītajām ciešanām citu tautu iedzīvotājiem tiek runāts klusināti. Tiesa, ir arī citi piemēri. Jau minētais poļu zinātnieks J.~Tazbirs uz žurnālista jautājumu: „Kurš pirmais sāka nemīlēt otros~--- mēs krievus vai krievi mūs?” atbildēja: „Vienlaikus. Strīdus un sadursmju objektu veidoja Lietuva. Krievzeme bija vāja un Lietuva sagrāba tās zemes, kuras agrāk bija tās (Krievzemes) īpašumā. Sāncensība turpinājās, un uz tās fona noteikti bija jānonāk līdz konfliktam.”

Diemžēl, arī šeit ir vajadzīgs precizējums. Poļu vēsturnieks citētajā frāzē it kā nevilšus visu atbildību par konfliktiem uzkrauj Lietuvai, Poliju pat nepieminot, kaut Lietuvas lielkņazistē ilgu laiku pārsvarā bija slāvu pareizticīgie iedzīvotāji, visai Rietumu Krievzemei Lietuvas lielkņaziste bija dabīgs pretestības centrs kā pret tatāriem, tā Teitoņu ordeni. Par ietekmi Austrumeiropā cīnījās gan poļu karaļi, gan krievu kņazi, gan leišu kunigaiši, taču Lietuvai pakāpeniski nonākot Polijas jaunākā partnera lomā, galveno cīņas smagumu pret Krievzemi uzņēmās tieši Polija.

Žečpospolitas un Krievijas strīds par to, kura no tām valdīs pār Zaporožjes kazakiem, beidzās ar Andrusovas pamiera noslēgšanu 1667.~gadā, ar kuru pirmā zaudēja visus apgabalus Dņepras kreisajā krastā par labu Krievijai. Tas bija ievērojams pagrieziena punkts poļu-krievu attiecībās. No šī laika pusotru gadsimtu līdz 1815.~gadam Krievija izplatīja savu ietekmi uz rietumiem, līdz liela daļa Žečpospolitas nonāca tās sastāvā. Jau 1686.~gadā karalis J.~Sobeskis noslēdza ar Krieviju t.s. Mūžīgo mieru (poļu historiogrāfijā \pltxti{pokój Grzymułtowskiego}), kurā starp citu Žečpospolita apsolīja saviem pavalstniekiem~--- pareizticīgajiem piešķirt ticības brīvību, bet Krievija apņēmās tos aizstāvēt. Tas deva Krievijai ieganstu iejaukties Žečpospolitas lietās.

Mūsdienās savstarpējos poļu~--- krievu pārmetumos parādā poļu nacionālistiski noskaņotajiem vēsturniekiem un publicistiem nepaliek arī liela daļa Krievijas vēsturnieku un publicistu. Uzstājoties vai nu joprojām no padomiskām vai vienkārši impēriskām pozīcijām, viņi nevēlas atzīt Krievijas un PSRS noziegumus pret poļu tautu, visur saskata tikai poļu darbības negatīvās sekas.

[Īsu pārskatu par krievu un padomju vēsturnieku veikumu Polijas vēstures izpētē sniedz S.~Falkovičas raksts: \rutxti{Светлана Фалькович. Польская проблематика в российской историографии} (Poļu problemātika krievu historiogrāfijā) // %http://jazon.hist.uj.edu.pl/zjazd/materialy/falkowicz.pdf
// 24.07.2011.]

Savstarpējos strīdos diemžēl bieži dzimst nevis patiesība, bet jauni aizspriedumi pret kaimiņtautām. Noteiktu liecību par poļu un krievu attiecībām sniedz arī abu valstu oficiālie svētki.

Tagadējās Krievijas Federācijas likumdevēji izvēlējās par Krievijas Tautas vienības dienu noteikt 4.~novembri, t.i.~--- dienu, kad pēc dažiem pieņēmumiem 1612.~gadā krievu zemessardze K.~Miņina un D.~Požarska vadībā atbrīvoja Maskavu no poļu iebrucējiem.

Tiesa, svētku datums tika izraudzīts bez pienācīgām konsultācijām ar speciālistiem. Mūsdienu krievu ekonomists, vēsturnieks un politiskais darbinieks V.~Šeiniss uzsvēris, ka šai dienā poļi tika padzīti tikai no vienas Maskavas pilsētas daļas (\rutxti{Китай-Город}), bet Kremlī viņi vēl palika, un viņu karogi no tā sienām tika nomesti tikai 1612.~gada 7.~novembrī. Plašāku izvērtējumu Krievijas tautas vienības dienai izvēlētajam datumam 1612.~gadā ir devis ievērojamais krievu vēsturnieks V.~Nazarovs. Šai dienā Maskavā nekas ievērojams nenotika, minētā Maskavas daļa (\rutxti{Китай-Город}) tika atbrīvota pēc jaunā stila nevis 4., bet gan 1.novembrī. Nepareizais datējums radās hronoloģiskas kļūdas rezultātā, saskaņojot Pareizticīgās baznīcas un laicīgo kalendāru. 5.novembrī Kremlī ielenktie interventi, kuru rindas bija raibas pēc nacionālā sastāva un poļi, visticamāk, bija mazākumā, parakstīja kapitulāciju, bet nākamajā dienā Kremļa garnizona padevās. Tieši Kremļa atbrīvošana kļuva par ievērojamu notikumu. Tādejādi tagad Krievijā spēkā esošā svētku diena balstās uz kļūdainu datējumu, neprecīzu vēsturisko avotu traktējumu un notikumu interpretāciju, kura neatbilst avotiem.

Lai arī kam būtu taisnība, svētku datuma izvēles politizēto raksturu nevar noliegt.

Polijas Republikā kā Polijas armijas diena (\pltxti{Świeto Wojska Polskiego}) tiek atzīmēts 15.augusts, kad Polijas armija 1920.~gadā uzvarēja Sarkano armiju kaujā pie Varšavas.

Šāda masu vēsturiskās atmiņas politizēta virzība traucē atklātības apstarotu, mūsdienīgu attiecību izveidi gan atsevišķu cilvēku, gan veselu tautu starpā. Savukārt, mūsdienās pastāvošās politiskās problēmas traucē objektīvi izvērtēt Polijas vēsturi.

Nākas ar nožēlu konstatēt, ka vēsturē diemžēl var atrast argumentus par labu gandrīz jebkuram viedoklim. Īpaši tas attiecās uz divu tautu attiecībām, kuras dzīvojušas blakus vairāk nekā tūkstoš gadu un kuras gan vieno, gan šķir tūkstošgadu savstarpējās cīņas vēsture. Šai gadījumā galīgi aplami ir jautājumi „kurš pirmais sāka?” un „kurš ir vairāk vainīgs?”. Jau attiecībā uz XX gadsimta sākuma situāciju krievu filozofs N.~Berdjajevs runāja par to, ka poļiem un krieviem ir jāatbrīvojas no atmiņām par gadsimtiem ilgo pretstāvi, grēkus, atzīstot savu vainu, ir jānožēlo visiem. Šai sakarā dziļu patiesību nes poļu vēsturnieka S.~Keņeviča doma: „Nācijai īsta patiesība, lai kāda arī tā būtu, nevar kaitēt \citespace{} Aprakstot pagājušo gadsimtu konfliktus, ir jāņem vērā ne tikai mūsu labā esošie argumenti, bet arī argumenti, ka nāk par labu mūsu bijušajiem pretiniekiem, lai neaprobežotos tikai ar sašutumu par mums nodarītajām pārestībām, bet parunātu arī par pāridarījumiem, kurus mēs esam nodarījuši citiem, lai, galu galā, nestiprinātu mūsu pārliecību par pārākumu pār citām nācijām, atceroties to, ka analoģiska citu nāciju pārliecība attiecībā par poļiem mūs aizvaino”.

Bijušais Polijas komunistu līderis un Valsts prezidents V.~Jaruzeļskis, ir stāstījis, kā viņš savas vizītes laikā Londonā tikās ar Lielbritānijas premjerministri M.~Tečeri un ieminējās tai, ka viņš ir liels Napoleona I cienītājs. M.~Tečere viņu uzveda savas mājas bēniņos, parādīja plānu ādas mapīti un teica: „Jūs zināt, kam tā piederēja? Napoleonam. Taču francūzi es nekad šurp neatvestu!” V.~Jaruzeļskis to komentēja: „Skatieties, 200~gadu ir pagājis, bet angļi nevar aizmirst [Napoleona karus]. Viņi ar frančiem dzīvo savienībā, divus pasaules karus karoja kopā. Un neraugoties uz to, katrai valstij palicis diametrāli pretējs Napoleona redzējums, sava vēstures versija. Taču tas nedrīkst traucēt sadarbību.”

Kā atzīmējis Polijas Zinātņu akadēmijas zinātniskais līdzstrādnieks, Berlīnes Brīvās universitātes goda profesors R.~Traba, poļiem ir gan vajadzīga Eiropas telpa dialogiem, taču pirmkārt viņiem ir jāatrisina „iekšējie strīdi”, kritiski jāizvērtē sava pagātne, lai „skeleti skapī” netraucētu tikt vaļā no iemantotajiem kompleksiem un politiskā balasta. Var piebilst, ka poļiem, daudz cietušiem no staļiniskajiem PSRS drošības orgāniem, nevajadzētu aizmirst, ka pirmie divi to vadītāji bija poļu šļahtiču kārtai piederīgie F.~Džeržinskis un V.~Menžinskis, tāpat no šīs kārtas nāca arī viena no staļinisma laikmeta drūmākajām figūrām~--- staļiniskais prokurors t.~s. „atklātajos procesos”~--- A.~Višinskis.

Ilgstoši apspiestas tautas nacionālais jūtīgums traucē tai un arī tās vēsturniekiem objektīvi izvērtēt vēsturi.

Izcilais mūsdienu krievu filozofs un sociologs A.~Zinovjevs ir norādījis uz interesantu likumsakarību: pēc lieliem vēsturiskiem notikumiem, tajos sakāvi cietusī puse, jau <em>post faktum</em> analizējot situāciju, apstākļus, kas noveda pie šīs sakāves, nespēj būt objektīva. Raudzīties pagātnē no zaudētāju skatu punkta ir mazproduktīvi, jo viņi taču cieta sakāvi, tātad izrādījās netālredzīgi, nesaprata vai vismaz slikti saprata, kurp ved notikumu gaita, tai skaitā to notikumu, kurus izsauca viņu pašu aktivitātes. To vispirms var teikt par revolūcijās varu zaudējušajiem, taču tāpat arī attiecināt uz ar Otro pasaules karu saistītajos notikumos varu, ietekmi zaudējušajiem Polijas, un ne tikai Polijas, bijušajiem valdošajiem slāņiem.

Poļu zinātnieki par XX gadsimta otrās puses lielāko lūzumu uzskata 1989.~gada notikumus, kas ne tikai mainīja Polijas valsts raksturu, atjaunoja Polijas Republiku (\pltxti{Rzeczpospolita Polska}), bet arī atveseļoja situāciju Eiropā kopumā. Jāuzsver, ka šis lūzums, tāpat kā PSRS sabrukums 1991.~gadā, ir pavēris arī daudz labvēlīgākas iespējas padziļināti izpētīt Polijas vēsturi, tuvināties vēsturiskajai patiesībai. Tomēr tas nenozīmē, ka šī patiesība atklāsies pati no sevis. Pēdējās desmitgadēs pēc Polijas Tautas Republikas likvidācijas, kad oficiāla cenzūra vairs neeksistē, pie varas esošie spēki ir parādījuši, ka arī ar attiecīgu finanšu politiku var panākt, lai daudzi vēsturnieki, publicisti rūpētos ne tik daudz par objektīvās patiesības atklāšanu, cik politiskās elites pasūtītas vēstures interpretācijas izplatību. Daudzos mūsdienu poļu publicistu, arī zinātnieku, tāpat kā citu Austrumeiropas tautu vēsturnieku darbos, dominē politiskās nostādnes, nevis vēsturiskā patiesība.

2004.~gadā grupa poļu vēsturnieku izvirzīja iniciatīvu izstrādāt un realizēt aktīvu vēsturisko politiku (\pltxti{polityka historyczna}). Termins (\detxti{Geschichtspolitik}) bija aizgūts no Vācijas, kur tas radās jau XX gs. 80.~gados, tiesa, ar nedaudz citu saturu. Polijā vēsturiskās politikas realizācijas piekritēji vēsturi un vēsturisko atmiņu lielākoties uzskata par politiskās cīņas arēnu pret ārējo un iekšējo ienaidnieku. Ar to viņi faktiski attaisno atkāpes no profesionālās ētikas, arī domu brīvības ierobežošanu. Apriori tiek uzskatīts, ka „ārējais pretinieks” cenšas nostiprināt savu vēstures notikumu interpretāciju, tāpēc „savu” vēsturnieku pienākums ir solidāri pretoties šādām briesmām, aizstāvēt argumentus, pretējus pretinieku lietotajiem. Pie varas esošie politiskie spēki, izmantojot valsts administratīvos un finanšu resursus, cenšas nostiprināt tiem tīkamas vēsturisko notikumu interpretācijas. Valdošie slāņi cenšas uzurpēt un monopolizēt tiesības interpretēt vēsturi, tagadni uzskatot par mērlīdzekli pagātnes izvērtēšanai. Parasti speciālas vēsturiskas politikas realizēšanas nepieciešamībai kā attaisnojums tiek minēta nepietiekamā patriotisma audzināšana vēstures stundās skolā. Nacionāli noskaņotie radikāļi, lai stiprinātu „nacionālo pašapziņu”, prasa radīt „harmonisku” vēstures ainu, panākt vēsturiskās atmiņas vienveidību.

Tā 2005.--2007.~gadā politizēti publicisti mēģināja iestāstīt Polijas sabiedrībai, ka humanitārās zinātnes pēc 1989.~gada, tai laikā radītās mācību grāmatās esot nepietiekami patriotiskas, bezierunu patriotisma vietā izplatot kritisko patriotismu. Lai īstenotu vēsturisko politiku, tiek radītas speciālas valstiskas un sabiedriskas iestādes un muzeji, iesniegti likumprojekti lai iemūžinātu „vienīgi pareizo” vēsturisko notikumu traktējumu, ieviešot pat kriminālatbildību par tā neievērošanu, kontroli pār izglītības sistēmu. Izmantoti ievērojami finansiālie resursi dažādu politiski tendētu projektu īstenošanai, komisiju dibināšanai, augsti atalgotu amatu radīšanai, ieviešot vēsturisko „varoņu” panteonu, atceres dienas utt. Godājot un nereti arī pārspīlējot savus upurus, kas nesti cīņā pret „svešajiem”, bieži tiek tīšuprāt aizmirsts, ka savi noziegumi un to rezultātā nestie citu tautu upuri nevar tikt attaisnoti pat ar savas tautas interešu aizstāvību. Bieži politisko uzskatu maiņa ved pie tā, ka tiek izcelti vieni un noniecināti citi vēstures personāži. Piemēram, Poznaņā viena no garākajām ielām tagad nes nevis Jaroslava Dombrovska, bet Jana Henrika Dombrovska vārdu, jo pirmais mūsdienu „pilsētas tēviem” bijis pārāk kreiss.

Šāda vēsturiskās politikas īstenošana ved pie diskusiju ierobežošanas savā valstī un konfliktsituāciju radīšanas ar ārpasauli. Kritiskās pieejas trūkums pret savas zemes vēsturi un savu patriotismu traucē demokrātiskas politiskās kultūras izveidei, ved pie atteikšanās no plurālisma vēsturnieku darbos. Vēsture un vēsturiskā atmiņa ir svarīga kolektīvās identitātes sastāvdaļa, taču skatot to tikai caur vēsturiskās politikas prizmu, tiek tikai sarežģīta strīdu un problēmu atrisināšana. Poļu vēsturniece A.~Volfa-Poveska norāda, ka arī mūsdienu Polijas vēsturnieku un žurnālistu vidū netrūkst cilvēku, kuri demonstrē, ka var „taisīt” politiku ar vēstures palīdzību, ka poļu vēsturiskajai politikai ir ģeopolitiskas koordinātes, tā atrodas starp divu savu lielāko kaimiņvalstu vēsturiskajām politikām.

Vērtējot vēsturiskos notikumus, jāņem vērā arī tās pārvērtības, kas notiek ar apspiestām tautām, kad tās atgūst valstiskumu un kļūst par noteicējām zemē, kur dzīvo citas mazākumtautības. Bijušajiem cietējiem ir vajadzīgs patiešām augsts kultūras un demokrātiskās attīstības līmenis, lai no apspiestajiem nepārvērstos par apspiedējiem. Ne velti neatkarīgās Polijas valsts vēsturi XX gadsimta 20--30.~gados tik atšķirīgi vērtē, no vienas puses, poļi un, no otras puses, tajā dzīvojušie ukraiņi, baltkrievi, lietuvieši, vācieši, ebreji un krievi.

Vēstures jautājumiem ir jāpaliek par speciālistu un vēstures entuziastu meklējumu un diskusiju objektu, bet politiķiem jāmeklē ceļi valstu un tautu attiecību uzlabošanai. Kā atzinis Krievijas ZA loceklis A.~Čubarjans: „Attiecības starp cilvēkiem un starp tautām kļūst stiprākas, dziļākas un godīgākas, ja tās pamatojas uz vēsturiskās patiesības atzīšanu un gatavību kritiskai analīzei, arī savas vēstures pārvērtēšanai.” Acīmredzams, ka saprašanās starp kaimiņtautām~--- poļiem un krieviem~--- būs iespējama tikai tad, ja abas puses pilnībā apzināsies tās grūtības, kuras šai ceļā nosaka vēsturiskais mantojums, kā arī ievēros un cienīs pretējās puses tradīcijas un uzskatus. Mūsdienās poļu un krievu zinātnieki no Polijas un Krievijas Zinātņu akadēmijām cenšas radīt objektīvu Polijas un Krievijas sarežģīto attiecību versiju, taču joprojām abu grupu pieejā ir daudz subjektīvisma. Krievu akadēmiķis I.~Kovaļčenko norādījis: „Nacionālā vēsture~--- tā ir autobiogrāfija, bet katra autobiogrāfija ir subjektīva, tāpēc, lai piedotu nacionālajām vēsturēm objektīvu raksturu, to uzrakstīšanai ir jāpiesaista citu valstu vēsturnieki.”

Blakus pašiem poļu pētniekiem, kuri ne vienmēr ir spējīgi būt pietiekami kritiski pret savu valsti, daudz par Polijas vēstures problēmām rakstījuši arī tās kaimiņvalstu vēsturnieki. Tas gan atvieglo, gan sarežģī tās līdzsvarotāku izvērtēšanu. Atvieglo tāpēc, ka ļauj izzināt vienas puses noklusēto, sarežģī~--- jo liek meklēt patiesību bieži diametrāli pretējās faktu interpretācijās. Bieži vien vairāk nekā pašiem poļiem un arī Polijas kaimiņvalstu vēsturniekiem var uzticēties neieinteresēto ārzemju pētnieku, piemēram, šveicieša A.~Kapellera, kanādieša M.~Dž.~Karleija u.c. vērtējumiem.

XX un XXI gs. mijā starp Poliju un Krieviju attīstījās vēsturnieku sadarbība, notika savstarpēja apmaiņa ar dokumentiem un materiāliem. Apritē nāca un joprojām nāk arvien jauni arhīvu materiāli, taču faktu uzkrāšana nevar automātiski atrisināt to interpretācijas problēmu.

Starp citu, vēsturnieku aprindās bieži skan gaušanās, ka pārāk lēnu tiek atvērti vai arī nemaz netiek atvērti Krievijas arhīvi, kas vismaz daļēji atbilst patiesībai. Taču, kā norādījis krievu diplomāts, politiķis un vēsturnieks V.~Faļins, tai pat laikā tiek aizmirsts, ka ASV uz nenoteiktu laiku ir liegusi jebkādu pieeju dokumentiem, kurus amerikāņu karavīri Otrā pasaules kara beigās ieguva nacistiskās Vācijas vadības mītnē Tīringijā. Acīmredzot tie satur pārāk daudz nepatīkama materiāla par amerikāņu politiku Eiropā, varbūt arī Polijā.

Ir vajadzīgs laiks, lai atbrīvotos no aizspriedumiem un liekām emocijām, neizbēgamām īpaši XX gadsimta vētraino notikumu vērtēšanā. Diemžēl, starptautiskā situācija ne vienmēr ir labvēlīga patiesības noskaidrošanai. Tā, Rietumvalstu un Krievijas interešu pretstāve Ukrainas teritorijā, faktiskais NATO valstu un Krievijas hibrīdkarš, īpaši pēc tiešas militāras sadursmes sākuma 2022.~gada februārī, kurš objektīvi veicināja Polijas kā Ukrainas kaimiņvalsts lomas pieaugumu, tikai sekmē arvien jaunu vairāk vai mazāk būtisku pretrunu, kā arī jau minēto aizspriedumu un emociju saasināšanos.

Tikai patiesības noskaidrošana, falsifikāciju atmaskošana spēj radīt pamatu normālām, cieņas pilnām, draudzīgām attiecībām starp tautām. Tas attiecas arī uz poļiem, vāciešiem, krieviem, ukraiņiem, baltkrieviem, lietuviešiem, ebrejiem un arī latviešiem, kuri gan ilgstoši dzīvoja kaimiņos, bet kuru attiecības veidojās dažādu iekšpolitisku un ārpolitisku apstākļu ietekmē. Lai atceramies t.s. ''poļu laikus” Latvijas teritorijā, arī to, ka XX gadsimta divdesmitajos gados par Poliju Latvijā inteliģences aprindās skanēja novērtējums „panu Polija”. Profesors J.~Tazbirs vēl XXI gadsimta sākumā norādīja, ka no vairākuma Polijā dažādos izglītības līmeņos lietojamo vēstures grāmatu grūti uzzināt, ka daudzu gadsimtu gaitā Žečpospolita nacionālā ziņā atgādināja mozaīku. Kā atzinis E.~Duračinskis, arī mūsdienās daudzi poļi bijušās Polijas iedzīvotājus: ukraiņus, baltkrievus un lietuviešus joprojām uzlūko kā „mazākumtautības”, kuras nav pelnījušas neko vairāk kā Otrās Žečpospolitas realizētās politikas pret „nacionālajiem mazākumiem” uzlabotu variantu. Tāpēc arī šajā darbā nacionālajām attiecībām starp dažādajām Polijā un tās kaimiņos dzīvojošajām tautām ierādīta nozīmīga vieta.

Bijušā Polijas prezidenta B.~Komarovska padomnieks vēstures jautājumos T.~Nalenčs intervijā presei uzsvēra, ka valstij un sabiedrībai jākoncentrē uzmanība to vēsturisko motīvu aprites veicināšanai, ar kuriem Polija var lepoties, piemēram, jāpopularizē pasaulē lielākā Malborkas (poļu \pltxti{Malbork}, vācu \detxti{Marienburg}) pils vai Ļ.~Valensas, kurš stāvēja pie „jaunās Polijas” sākumiem, personība. Pēc T.~Nalenča vārdiem, vēsturi nedrīkst izmantot kā avotu negatīvu emociju formēšanai gan valsts iekšienē, gan ārpus tās. Piekrītot tam, ka vēsture jāizmanto pozitīvu ideālu audzināšanai, autoram gan jāpiezīmē, ka ar tās palīdzību nedrīkst arī redzēt un popularizēt tikai savas un tā noniecināt citu tautu sniegumus, kā arī „neredzēt” negatīvo savas valsts vēsturē.

Lasītāja priekšā esošā darba autors centies cik iespējams objektīvi un daudzpusīgi atspoguļot Polijas vēstures XIX un XX gadsimtā sarežģītās problēmas, taču uzreiz jāsaka, ka viņš tomēr piedāvā tikai vienu~--- savu redzējumu. Jāuzsver, ka mūsdienās ir ļoti grūti pētīt, analizēt, vienkārši rakstīt par XX gadsimta notikumiem, pilnībā atsakoties no sava subjektīvā vērtējuma. Cerams, ka pēc gadiem, kad kaislības būs norimušas, objektivitāti sasniegt būs vieglāk. Taču apstākļos, kad uz politiskās skatuves darbojas spēki, kuri sevi ar lielākām vai mazākām tiesībām pozicionē kā to vai citu XX gadsimtā darbojošos partiju, strāvojumu mantiniekus, prasību raudzīties uz XX gadsimta notikumiem „pilnīgi neitrāli”, droši var nosaukt par utopisku.

Godīgi jāpasaka, ka grāmatā visi vēsturiskās izšķiršanās brīži, notikumi, to veicēji vērtēti no iespējami demokrātiskāka, vardarbību izslēdzoša attīstības ceļa sasniegšanas viedokļa. Visdažādākajos nacionālajos strīdos autora personīgās simpātijas pieder apspiestajiem nacionālajiem mazākumiem, bet ne lielvalstiskas, bieži šovinistiskas nacionālās politikas realizētājiem. Tai pašā laikā autoram nav pieņemama to pagātnē (un dažkārt arī mūsdienās) apspiesto tautu nacionālistisko spēku darbība, kuri savu mērķu sasniegšanai nereti gatavi pārkāpt citu tautu, citas tautības cilvēku tādas pat tiesības uz brīvību un vienlīdzību, neapstājoties pat noziegumu priekšā.

Stādīts mērķis veicināt no aizspriedumiem un stereotipiem brīvu un līdzsvarotu izpratni par Polijas un arī visas Eiropas vēsturi Otrā pasaules kara gados, palīdzēt lasītājiem kritiski uztvert un izvērtēt dažādus apzināti un neapzināti tendenciozus sacerējumus par Poliju Otrajā pasaules karā, kuri pēdējās desmitgadēs izplatās Austrumeiropā.

Otrā pasaules kara notikumos autora simpātijas ir antihitleriskās koalīcijas sabiedroto pusē, kuri cīnījās par vispārcilvēciskiem mērķiem. Tāpēc viņš kritiski raugās uz tādiem Austrumeiropas politiķiem, kuri, piesedzoties ar „patriotiskiem” lozungiem, kara laikā faktiski atbalstīja nacistus, bet pēc tam mēģināja uzspiest šī vispasaules nozīmības izšķiršanās punkta vērtējumam savu šauri nacionālistisko skatījumu. Arī mūsdienās darbojas viņu ideju mantinieki.

Runājot par pēckara situāciju pasaulē, autors neuzskata par iespējamu staļinisko (arī pēc J.~Staļina nāves uzlaboto) „sociālisma” modeli identificēt ar to mērķi, kuru stādīja tā daļa cilvēces attīstības ceļu meklētāju, kuri par galveno uzskatīja sociālā taisnīguma ideālu sasniegšanu. Sociālisms kā sabiedriska iekārta, kurā cilvēks neekspluatē citu cilvēku, kur pastāv sabiedrisks, bet ne valdošās elites piesavināts īpašums uz ražošanas līdzekļiem, netika uzcelts ne PSRS, ne citās t.~s. „sociālisma” valstīs. Ražošanas līdzekļu koncentrācija valsts (faktiski~--- partijiski-birokrātiskā aparāta) rokās veda pie virsmonopolizācijas, kas bremzēja ražošanas attīstību. „Sociālisma” valstīs nepastāvēja sociālais taisnīgums. Pastāvēja pat superekspluatācija, kad daļa sabiedrības (staļiniskajās nometnēs ieslodzītā) praktiski nesaņēma nekādu atlīdzību, daļa strādāja par niecīgu atalgojumu, bet sabiedrības virsslānis~--- partijiski-birokrātiskā elite saņēma savam veikumam neadekvāti augstu atalgojumu. Sociālisma princips „no katra pēc spējām, katram pēc viņa darba”, „sociālisma” valstīs netika īstenots, jo nebija jau objektīvas mērauklas, kā šo darbu novērtēt. Tirgus vietā vērtējumu deva kā valsts birokrātijas slānis kopumā, tā atsevišķi tā pārstāvji. Tāpēc, runājot par PSRS, tās satelītvalstīm, arī Polijas Tautas Republiku, par tur it kā uzcelto sabiedrisko iekārtu autors parasti raksta kā par „sociālismu”, t.i.~--- lietojot pēdiņas. Tikai runājot par sociālismu kā mērķi, kuru pieņēma un centās sasniegt daļa uz labāku dzīvi cerošo, bet lielākoties nekritiski domājošo tautas masu, tas lietots bez pēdiņām.

Pēckara notikumos autora visdziļākā cieņa pieder cīnītājiem par demokrātiju un tautas labklājību, kuri sava mērķa sasniegšanai bija gatavi lietot dažādus cīņas līdzekļus, taču atteicās izmantot vardarbību, „šķirisko”, „sociālistisko” vai „nacionālo” interešu vārdā piekopt teroru pret savas un citu tautu piederīgajiem. Ar to viņi krasi norobežojās no saviem pretiniekiem~--- totalitāro, autoritāro, etnokrātisko varas sistēmu aizstāvjiem, kuriem savukārt visi līdzekļi bija derīgi it kā „valstisko”, „nacionālo”, „šķirisko”, „sociālistisko”, arī „demokrātisko” u.tml., bet faktiski~--- savu savtīgo mērķu sasniegšanai. No šādām pozīcijām arī skatīti gan poļu, gan citzemju darbinieki, viņu pārstāvētie sabiedriskie spēki, viņu ietekmētie notikumi.

Patiesam Polijas vēstures redzējumam lasītājs tuvināsies, salīdzinot dažādus pētījumus, kritiski vērtējot kā šo, tā arī citus Polijas vēsturei veltītus darbus.

Ilustrācijas ievadam

\chapter{Polija kaimiņvalstu varā. 1795~--- 1918}

\epigraph
{Karot poļi neprot. Taču dumpoties!}
{Hugo~Kollontajs (\pltxti{Hugo Kołłątaj})}

\epigraph
{Diemžēl mēs [poļi] neprotam strādāt! Kauties, lieliski cīnīties, nomirt, uz to vienmēr esam gatavi; taču uzcītīgi strādāt, ilgstoši, bez trokšņa un uzslavām, strādāt varbūt ne priekš sevis, tas mums ir par daudz.}
{Ludvika Plātere (\lttxti{Ludwika Plater})}

\epigraph
{Polija ir unikāla valsts ar tieksmi pēc impērijas, kuras tai nekad nav bijis.}
{Dmitrijs Kuļikovs (\rutxti{Дмитрий Куликов})}

\newpage

\epigraph
{Nelaimīga ir valsts, kurai nav varoņu.~--- Nē! Nelaimīga ir tā valsts, kurai ir vajadzīgi varoņi.}
{Bertolts Brehts (\detxti{Eugen Bertolt Friedrich Brecht})}

\epigraph
{Lepnums mums izmaksā dārgāk nekā bads, slāpes un aukstums.}
{Tomass Džefersons (\entxti{Thomas Jefferson})}

\epigraph
{Starptautiskajā politikā morāles nav bijis, nav un nebūs.}
{Jakovs Kedmi (\hetxti{יעקב קדמ})}

\epigraph
{Nabags nav tas, kam maz pieder, bet tas, kurš daudz grib.}
{Angļu sakāmvārds}

\epigraph
{Ja visi vainīgi, neviens nav vainīgs.}
{Pēteris Krupņikovs}

\epigraph
{Vistālāk iet tas, kurš nezina, kurp iet.}
{Olivers Kromvels (\entxti{Oliver Cromwell})}

\newpage

\epigraph
{Gandrīz visi dižie līderi savai dzimtenei ir maksājuši asins jūras.}
{Mihails Vellers (\rutxti{Михаил Иосифович Веллер})}

\epigraph
{Nācijai zaudēt savu valsti, savu patstāvību un neatkarību ir liela nelaime. To var salīdzināt ar smagu slimību, kas kropļo nācijas dvēseli.}
{Nikolajs Berdjajevs (\rutxti{Николай Бердяев})}

\epigraph
{Dažas tautas traģēdijas tiek uzvestas bez starpbrīžiem.}
{Staņislavs Ježijs Lecs (\pltxti{Stanisław Jerzy Lec})}

\epigraph
{Kādreiz dedzinātāji ir pārliecināti, ka tautai priekšā nesuši lāpu.}
{Staņislavs Ježijs Lecs (\pltxti{Stanisław Jerzy Lec})}

\epigraph
{Pēc neveiksmīgām revolūcijām vienmēr seko ienīstas un atriebīgas valdības.}
{Pjērs Buasts (\frtxti{Pierre Boiste})}

\epigraph
{Visas revolūcijas beidzas ar reakciju. Tas nav novēršams. Tas ir likums. Un jo negantākas un niknākas bijušas revolūcijas, jo stiprāka bija reakcija. Revolūciju un reakcijas nomaiņā ir kāds maģisks riņķis.}
{Nikolajs Berdjajevs (\rutxti{Николай Бердяев})}

\newpage

\epigraph
{Neprasme pārciest nelaimi ir liela nelaime.}
{Bions no Borisfēnas (\eltxti{Βίων Βορυσθενίτης})}

\section{Poļu zemes XVII gadsimta beigās un XIX gadsimtā}

\epigraph
{Laimīga tā tauta, kurai ir garlaicīga vēsture.}
{Šarls Luijs de Monteskjē (\frtxti{Charles-Louis de Secondat, Baron de La Brède et de Montesquieu})}

\epigraph
{Vēsture ir vislabākais skolotājs, kuram ir paši sliktākie skolnieki.}
{Indira Prijadaršinī Gandija (\entxti{Indira Priyadarshini Gandhi})}

\epigraph
{Nabadzība noved pie revolūcijas, revolūcija pie nabadzības.}
{V.~Igo (\frtxti{Victor Marie Hugo})}

\epigraph
{Tur, kur ir divi poļi, pastāv trīs viedokļi.}
{Poļu paruna}

\epigraph
{Pateicoties vienprātībai aug mazas valstis, ķildu dēļ iet bojā lielvalstis.}
{Henriks Senkevičs (\pltxti{Henryk Adam Aleksander Pius Sienkiewicz})}

\newpage

\epigraph
{Piesien kādu skrandu pie spieķa, turi to augstu, un tu redzēsi, cik daudzi sekos tai kā karogam.}
{Staņislavs Ježijs Lecs (\pltxti{Stanisław Jerzy Lec})}

\epigraph
{Lielā Kartāga veda trīs karus. Pēc pirmā tā joprojām bija spēcīga. Pēc otrā tā vēl bija apdzīvota. Pēc trešās tā vairs nebija atrodama.}
{Bertolts Brehts (\detxti{Bertolt Brecht})}

\epigraph
{Pat visa zeme nav vienas veltīgi izlietas asins lāses vērta.}
{Aleksandrs Suvorovs (\rutxti{Александр Васильевич Суворов})}

\epigraph
{Kad tiek meklēti kontrrevolūcijas panākumu cēloņi, no visām pusēm var saņemt parocīgu atbildi, ka X kungs vai pilsonis Y nodeva tautu. Šī atbilde var būt pareiza un arī nē \dots{} katrā gadījumā tā nekādi nepaskaidro, kā tas notika, ka tauta ļāva sevi nodot.}
{Fridrihs Engelss (\detxti{Friedrich Engels})}

\epigraph
{Naids ir baiļu sekas, mēs vispirms baidāmies un tikai pēc tam ienīstam.}
{Sirils Konolijs (\entxti{Cyril Vernon Connolly})}

\newpage

\epigraph
{Patriotisms~--- tas ir „mīlu savu”, nacionālisms~--- „nīstu svešu”.}
{Šarls de Golls (\frtxti{Charles André Joseph Marie de Gaulle})}

\epigraph
{Patriotisms ir mīlestība pret savējiem, nacionālisms~--- naids pret citiem.}
{Rihards fon Vaiczekers (\detxti{Richard Karl von Weizsäcker})}

\epigraph
{Nacionālisms daudz vairāk asociējas ar naidu pret svešu tautu nekā ar mīlestību pret savējo.}
{Nikolajs Berdjajevs (\rutxti{Николай Александрович Бердяев})}

\epigraph
{Cilvēks, kas ienīst citu tautu, nemīl arī savējo.}
{Nikolajs Dobroļubovs (\rutxti{Николай Александрович Добролюбов})}

\epigraph
{Novests līdz galējam sasprindzinājumam, nacionālisms dzen postā tautu, kas tam ļāvusies, padarot šo tautu par cilvēces ienaidnieci, jo cilvēce vienmēr izrādīsies stiprāka par atsevišķu tautu.}
{Vladimirs Solovjovs (\rutxti{Владимир Сергеевич Соловьёв})}

\subsection{Polijas dalīšanas, Varšavas hercogiste un Vīnes kongress}


Poļu etnosa attīstības un poļu nācijas veidošanās apstākļi līdz XVIII gadsimta beigām bija samērā labvēlīgi. Poļu apdzīvotā teritorija bija kompakta, nesadalīta ar dabīgiem kalnu vai ūdens šķēršļiem. Senseno poļu zemju kodols gadsimtu gaitā ietilpa vienotā valstī. Tiesa, arī aiz šīs valsts robežām atradās vairākas poļu teritorijas, bet pati valsts bija daudzetniska, taču lielākā iedzīvotāju masa tajā runāja vienā~--- poļu valodā, piederēja vienai~--- katoļu ticībai. Mūsdienu poļu vēsturnieks J.~Tazbirs gan raksta, ka pēc aptuveniem aprēķiniem tikai ap 40\% Žečpospolitas (\pltxti{Rzeczpospolita Obojga Narodów}~--- Polijas-Lietuvas apvienotā valsts jeb Abu Tautu Republika) iedzīvotāju bija poļi, pie tam tikai daļai no tiem piemita nacionālā apziņa. (Pēc šī darba autora domām attiecībā uz šo laiku pareizāk būtu runāt par etnisko, nevis nacionālo apziņu.) Sākot ar Ļubļinas ūniju 1569.~gadā, kad tika izveidota Žečpospolita un Lietuvas lielkņaziste zaudēja savu krievisko politisko komponenti, Žečpospolitā sākās Pareizticīgās baznīcas apspiešana, rietumkrievu rakstiskās valodas izraidīšana no darbvedības, visa krieviskā vajāšana. Tā īpaši pastiprinājās pēc Brestas baznīcas ūnijas (1596), kad pastiprinājās katoļu ekspansija (Brestas ūnijas rezultātā daļa Žečpospolitas pareizticīgo garīdznieku pakļāva Romas pāvestam, saglabājot daļu pareizticīgo ritu un savu hierarhiju) tajās pareizticīgo austrumslāvu zemēs, kuras bija Žečpospolita sastāvdaļa.

XVII--XXVIII gadsimtā Polijas sabiedrība dalījās trijās kārtās: muižniecībā jeb šļahtā, sīkpilsoņos un zemniekos.

Šļahta bija karojošu kungu kārta, kura sevi krasi norobežoja no pārējām~--- zemākajām kārtām, saucot to piederīgos par liellopiem (\pltxti{bydło}). Šļahtai piederēja monopols uz varu, zemes īpašumu un sabiedrisko prestižu. XV--XVI gadsimtā izveidojās, bet nākamajos gadsimtos par visas šļahtas ideoloģijas sastāvdaļu kļuva t.s. sarmatisms. Tas balstījās pieņēmumā, ka šļahtas izcelsme ir saistāma ar sarmatiem (sena nomadu tauta, kura runāja indoirāņu valodā), kas antīkajā laikmetā pakļāvuši slāvu ciltis un izveidojuši virsslāni. Sarmatisms it kā pamatoja Žečpospolitas muižniecības tiesības norobežoties no etniski ,,svešajiem''~--- slāvu un lietuviešu zemniekiem. Sarmatisma ideoloģijas galvenie elementi bija~--- šļahtas neierobežota brīvība, nacionālā augstprātība, ticība Polijas vēsturiskajai izredzētībai, ksenofobija (neiecietība pret svešo). Sarmatisms kultivēja īpašas parašas. Tā, ,,īsts šļahtičs'' labāk mira badā, bet neaptraipīja rokas ar tam nepiedienīgo fizisko darbu.

Starp citu, poļu publicists un politiķis A.~Vasiļevskis atzīmējis, ka poļu kultūras vēsturē gadsimtu gaitā izveidojušās divas nacionālo problēmu aprakstīšanas skolas, kur katra no tām vadījās no savas patriotisma izpratnes. Sarmatiskais virziens identificēja patriotismu ar pašslavināšanu, ar nekritisku, visa, kas bija savs, attaisnošanu. Otrs~--- demokrātiski reformistiskais virziens nebaidījās teikt tautai rūgtu patiesību, bet patriotismu identificēja ar atklātu norādīšanu uz nacionālajiem trūkumiem, kaut šādas pārdomas par saviem grēkiem bija arī sāpīgas. Šī darba autors var piebilst, ka arī mūsdienu Polijas vēstures literatūrā šie virzieni turpina pastāvēt.

Žečpospolitas spilgtākā īpatnība bija nekur citur Eiropā neredzēts šļahtas~--- šī priviliģētā slāņa daudzskaitlīgums. XVIII gadsimta vidū gandrīz ik desmitais valsts iedzīvotājs bija šļahtičs. Kaut kārtas uzvedības noteikumi paredzēja visu augstdzimušo šļahtiču pilnīgu vienlīdzību, patiesībā tā bija visai iluzora. Dižciltīgo ģerboņu īpašnieku vidū lielai daļai jau nepiederēja nedz zeme, nedz dzimtcilvēki. Šie nemantīgie šļahtiči parasti pelnīja iztiku un pajumti pie šļahtas bagātās daļas~--- t.s. magnātiem (no latīņu \latxti{magnatus}~--- liels cilvēks, \latxti{magnatis}~--- dižciltīgs cilvēks, parasti zemes lielīpašnieks. Žečpospolitā pastāvēja ap 60 magnātu dzimtu), kuru politiskais svars bija atkarīgs no t.s. ,,klientu''~--- viņu atbalstītāju šļahtas rindās skaita.

Viens no Polijas neveiksmju cēloņiem bija tas, ka jebkura spēku koncentēšana centrālās varas vadībā kādu tālu mērķu sasniegšanas vārdā rada kategoriskus iebildumus Polijas nacionālajā elitē. No vienas puses, XVII gadsimtā poļu aristokrātijai bija pilnīgi nesaprotami, kāpēc saspringt, kad viss ir pieejams: gan galmi, lai tur ,,spīdētu'', gan teritorijas ar līdz pusvergu stāvoklim nospiestiem iedzīvotājiem austrumos, kurus varēja pērt un no kuriem varēja vākt nodevas. No otras puses, centrālās varas nostiprināšanās, pakļaujot sev magnātus un šļahtu, nebija pēdējo interesēs.

Piemēram, atbalsta trūkuma centrālai varai dēļ Polija XVII gadsimtā palika bez kara flotes, kas bija viena no galvenajām militāro panākumu atslēgām (Šī perioda Polijas karaļi mēģināja izveidot reālu floti, taču viņu mēģinājumi atkārtoti izgāzās līdzekļu trūkuma dēļ karaļa kasē, jo Polijas šļahta neredzēja nepieciešamību pēc flotes un atteicās paaugstināt nodokļus tās celtniecībai. Vēl esošie kuģi tika pārdoti 1641.--1643.~gadā, kas nozīmēja Žečpospolitas flotes beigas.)

Nākamajā~--- XVIII gadsimtā šļahta nesaprata, kāpēc tai būtu jāatsakās no \latxti{liberum veto} (no latīņu, brīvais veto jeb brīvais aizliegums~--- Žečpospolitas parlamentārās uzbūves princips, kurš ļāva jebkuram Seima loceklim pārtraukt jautājuma apspriešanu. Deputātam vajadzēja tikai skaļi izsaukties latīniski ,,\latxti{Sisto activitatem!}'' (Es pārtraucu darbības) vai poliski ,,\pltxti{Nie pozwalam!}'' (Es neļauju!) un jādod vienlīdzīgas tiesības visiem valsts iedzīvotājiem. Rezultātā no 1573.~līdz 1763.~gadam, kad sanāca apmēram 150 Seimu, aptuveni trešdaļā no tiem netika pieņemts lēmums. Protams, ar vienu faktora, lai arī tik svarīga kā šļahtas nostāja, nevar visu izskaidrot.

Abas pārejās kārtas bija galvenās nodevu maksātājas, bet bez politiskām tiesībām. Pie sīkpilsoņiem piederēja pilsētu iedzīvotāji, uz kuriem attiecās pilsētu tiesības un kuri guva pastāvīgus ienākumus no tirdzniecības un amatniecības. 1790.~gadā pilsētās dzīvoja ap 16\% iedzīvotāju (bez ebrejiem gan tikai 6\%). Zemniecība sastādīja ap 75\% visu iedzīvotāju, lielākā tās daļa (ap 65\%) dzīvoja uz šļahtai piederošas zemes, pārējie~--- baznīcai un karalim piederošas. Pilsonībai (buržuāzijai)~--- lieltirgotājiem, manufaktūru īpašniekiem, baņķieriem XVIII gadsimta beigās varēja pieskaitīt tikai dažus desmitus cilvēku. Zemnieki (ap 85--90\%) pildīja klaušas, pārējie maksāja renti~--- t.s. činšu (poļu \pltxti{czynsz}, vācu \detxti{Zins}~, no latīņu \latxti{census}~--- procents~--- Polijā, Lietuvā un arī Latgalē tā sauca zemnieku nodevas jeb renti naudas vai produktu veidā). Pēc feodāļu-muižnieku gribas pāreja no činša atpakaļ uz klaušu saimniecību nesastapa grūtības un bija bieža parādība. Uz 100 zemnieku saimniecībām bija 20, kurās tika apsaimniekoti 7 līdz 8 ha zemes, 62 saimniecībās tās bija mazāk, 16 bija bezzemes saimniecības.

Ārpus kārtām stāvēja garīdzniecība, pilsētu nabagi un beztiesiskie un diskrimenētie ebreji. Augstākā un vidējā garīdzniecība izcelsmes ziņā bija tuva šļahtai, bet zemākā (īpaši no XIX gadsimta otrā ceturkšņa)~--- zemniecībai. Tā kā reliģija bija cieši saistītā ar nacionālo pašapziņu, pēc Polijas neatkarības zaudēšanas liela garīdzniecības daļa kļuva par aktīvu nacionālās atbrīvošanās kustības spēku.

Tiesa, vienprātības par garīdzniecības vērtējumu nav. Piemēram, vācu autors R.~Vingendorfs pirms Otrā pasaules kara rakstīja, ka katoļu garīdzniecībai gan esot bijusi mazāk svarīga nacionālā ideja, cik iespēja to izmantot kā cīņas līdzekli pret pareizticīgo Krieviju un protestantisko Prūsiju. Krievu vēsturnieks V.~Djakovs savukārt norādīja, ka sociālie motīvi garīdzniecībai bija mazāk nozīmīgi nekā citiem iedzīvotāju slāņiem, un kritiskos momentos tā parasti pieslējās šļahtai, tikai atsevišķi indivīdi nostājās zemniecības un citu darba cilvēku pusē.

Citi minētie slāņi, kaut piedalījās vēsturiskajos notikumos (sacelšanās u.c.), kādu patstāvīgu sabiedriski-politisku pozīciju neieņēma.

Pārdzīvojusi ,,ziedu laikus'' XVII gadsimtā, Žečpospolita XVIII gadsimtā nonāca ekonomiskā un politiskā panīkumā, cīnījās par savu izdzīvošanu. Žečpospolita soli pa solim tuvojās katastrofai. Konfrontācija ar kaimiņvalstīm un pārmērīgi lielas kara izmaksas izsūca zeltu no valsts kases. Daudzie kari: ar Krieviju par ukraiņu un baltkrievu zemēm, ar Zviedriju~--- par Livoniju, ar Prūsiju~--- par tās austrumu apgabaliem, Žečpospolitu novājināja. Pirmā Žečpospolita izrādījās vāja valsts, kas nespēja aizstāvēt savu neatkarību un teritoriālo integritāti, bet tajā pašā laikā tik ķildīga, lai visos apkārtējos kaimiņos pamodinātu dedzīgu vēlmi atbrīvoties no šāda kaimiņa. Krievijai gan Žečpospolita droši piesedza tās robežas no Eiropas problēmām, un tā nekad nebūtu piekritusi šīs bufervalsts iznīcināšanu, ja tā parādītu zināmu saprātīgumu un necenstos apspiest pareizticīgos un pretoties Krievijas ietekmei.

Nebeidzamie iekšējie kari pret pareizticīgajiem iedzīvotājiem Žečpospolitas austrumu zemēs veicināja panīkumu. Karos lija poļu asinis, graujot valsts cilvēcisko potenciālu.

Valsti nomocīja dažādu iekšējo politisko spēku savstarpējās cīņas, pretrunas starp priviliģētajiem katoļiem un nevienlīdzīgajiem protestantiem, pareizticīgajiem un jūdu ticības piekritējiem, laikmetam neatbilstošā, novecojusī valsts iekārta, kad katrs pārstāvniecības iestādes~--- Seima loceklis varēja izmantot t.s. \latxti{liberum veto} tiesības, nobloķējot vairākuma pieņemta lēmuma izpildi. Poļu paruna pat ar zināmu lepnumu konstatēja, ka valsts pastāv pateicoties nekārtībām: ,,\pltxti{Polska nierzadem stoi}''.

Nebija vienotības poļu un lietuviešu izcelsmes muižniecības vidū. Poļu vēsturnieks H.~Visners ir uzsvēris, ka poļu valdošie slāņi kopš Jagelloņu dinastijas (poļu \pltxti{Jagiellonowie}, lietuv. \lttxti{Jogailaičiai}~--- valdīja Lietuvā un Polijā no XIV līdz XVI gadsimtam) valdīšanas sākuma plānoja vienota Polijas politiskā organisma izveidi ar Lietuvas kunigaitiju. XVIII gadsimtā poļu programma, balstoties uz gadsimtiem ilgo savienības tradīciju, vairs neredzēja vietu suverēnai Lietuvas valstij. Arī vēlāk~--- jau XIX gadsimtā poļi nesaskatīja, ka lietuvieši rada savu etnosociālu un etnopolitisku kopību. Lietuviešu vēsturnieks un politologs A.~Kulakauskas secinājis, ka poļi uz XVIII gadsimta pārmaiņām Žečpospolitas teritrorijā, kuru viņi uzskatīja par savu īpašumu, reaģēja vēsturiski nepareizi [t.i~--- neatbilstoši savām tālākajām interesēm.~--- V.Š.]. Viņi neatzina lietuviešu tautu par pilnvērtīgu, uzskatīja Lietuvu par Polijas provinci un lietuviešus kā nacionālo mazākumu, kuri nedrīkst pat izvirzīt prasību pēc kulturālas autonomijas. Lai aizstāvētu savas tiesības pastāvēt kā patstāvīgai tautai savā valstī lietuvieši \lttxti{volens nolens} (gribot negribot) bija spiesti izvēlēties savu nacionālās attīstības ceļu, norobežojoties no poļiem.

Vēl viena problēma bija augstais ebreju iedzīvotāju īpatsvars Žečpospolitā. Ir aprēķini, ka 1800.~gadā 70\% visu pasaules ebreju dzīvoja Polijā un aptuveni 25\% bijušās Žečpospolitas iedzīvotāju bija ebreji, šeit izveidojās bagāts ebreju kultūras mantojums. Taču autohtonās etniskās grupas, kuras izsenis dzīvoja šai teritorijā, nekādi nevēlējās atzīt ebrejus par līdztiesīgiem.

Etniskās pretrunas savijās ar kārtu pretišķībām. Augstdzimusī ,,poļu tauta'' (\pltxti{narod Polski}), kā viduslaikos sauca poļu šļahtu, centās turēt paklausībā vienkāršo ,,kalpu tautu'' (\pltxti{narod chlopskie}).

Dažādās pretrunas Žečpospolitā labi saskatīja tālaika gaišākie Eiropas prāti. Jau Ž.~Ž.~Russo rakstīja: ,,Lasot poļu valdīšanas vēsturi, ar grūtībām var saprast, kā tik dīvaini uzbūvēta valsts varēja pastāvēt tik ilgi.'' Arī lielais Apgaismības klasiķis Voltērs izteicās: ,,\frtxti{Un Polonais~--- c´est un charmeur; deuz polonais~--- une begarre; trois polonais, eh bien, c´est la question polonaise.}'' (Viens polis ir apburošs cilvēks, divi poļi~--- tracis, trīs poļi, nu jā, tas jau ir Polijas jautājums.). Pagāja nedaudzi gadu desmiti, un virknes karu izpostītā Žečpospolita pat nespēja pati sevi aizstāvēt.

Tālākais Polijas liktenis, valsts sadale saistījās ar Eiropas valstu pūliņiem saglabāt spēku līdzsvaru starptautiskajā arēnā. Kompromisus varēja atrast uz tai laikā vājās Polijas rēķina. Kaimiņvalstis~--- Austrija, Prūsija un Krievija, izmantojot konfesionālās nesaskaņas, t.s. disidentu (no latīņu \latxti{dissidens (disidentis)}~--- tāds, kas nepiekrīt. Polijā, kur valdīja katolicisms, tā sauca kristiešus, kuri neatbalstīja valdošo konfesiju) cīņas pret katoļu privilēģijām, iejaucās valsts iekšējās lietās, traucēja nepieciešamo politisko reformu pieņemšanu, kas varētu stiprināt valsti.

Analizējot Žečpospolitas sadalīšanas iekšējos cēloņus, ievērojamais krievu vēsturnieks S.~Solovjovs pētījumā ''Polijas krišanas cēloņi'', (\_rutxti{Соловьёв С.М. История падения Польши, Москва, 1863}) uzsvēra, ka pirmajā vietā starp galvenajiem tās katastrofas cēloņiem liekami ne kaimiņvalstu agresīvie centieni, bet gan spēcīgā krievu (tai skaitā baltkrievu un ukraiņu) nacionālās atbrīvošanās kustība ''zem reliģiskā karoga'' pret poļu jūgu, par savām tiesībām, par vienlīdzību. Domājams, jāpiekrīt S.~Solovjova uzskatam, ka pašu poļu vaina bija tā, ka Polijas pareizticīgo vidū radās disidentu kustība, kuru galvenokārt atbalstīja Krievija, bet ne tikai. Polijas katoļu vairākums pat negribēja dzirdēt par atteikšanos no savām privilēģijām un tiesību vienlīdzību ar nekatoļiem un neuniātiem. Krievu vēsturnieks norādīja, ka 1653.~gadā, kad Maskavijas cara Alekseja Mihailoviča sūtnis kņazs B.~Repņins no Polijas valdības pieprasīja, lai pareizticīgie krievu cilvēki Žečpospolitā neciestu reliģiskos spaidus, pēdējās valdība nepiekrita šai prasībai, un sekas bija Mazkrievijas atkrišana no Žečpospolitas. Pēc simts gadiem Krievijas ķeizarienes Katrīnas~II vēstnieks, arī kņazs [N.]~Repņins, izteica tādu pašu prasību, bet saņēma atteikumu, un rezultāts bija pirmā Polijas dalīšana. S.~Solovjovs ar to gan vienpusīgi vienkāršoja vēsturisko procesu, taču būtisku tās aspektu atspoguļoja. Aizbildniecība pār t.s. disidentiem, vienlīdzīgu viņu tiesību ar katoļiem kā krievu tautā vispopulārākās lietas aizstāvība, bija īpaši svarīga Katrīnai~II,~--- rakstīja cits krievu vēsturnieks V.~Kļučevskis. Sākotnēji runa pat vairāk bija par, kā mūsdienās teiktu, cilvēktiesību aizstāvības politiku, nevis par Krievijas valsts teritoriālo paplašināšanu un liel-, maz- un baltkrievu apvienošanu Krievijas impērijas robežās.

Krievijai labvēlīgu politiku gan ieturēja 1764.~gadā ar tās atbalstu par Žečpospolitas karali ievēlētais viens no Krievijas imperatores Katrīnas~II favorītiem S.~Poņatovskis. Karalis Staņislavs~II Augusts sāka savu politisko darbību, patiesi ticot, ka tikai ar Krievijas palīdzību Polijā var īstenot viņa iecerētās reformas. Viņš atbalstīja poļu rūpniecības attīstību, tirdzniecības kompāniju dibināšanu, daudz vērības veltīja literatūrai un zinātnei. Reizē viņš mēģināja ierobežot magnātu patvaļu un nostiprināt centrālo varu. Taču drīz pierādījās, ka tāda Polijas attīstība ir nevēlama ne tikai lielai daļai poļu feodāļu, bet arī kaimiņivalstīm. Krievija pieprasīja atrisināt disidentu nevienlīdzības jautājumu. Tikai 1768.~gadā milzīga Krievijas spiediena ietekmē Seims bija spiests atzīt pareizticīgo vienlīdzību ar katoļiem Žečpospolitā. 24.~februārī tas pieņēma jaunu ,,mūžīgu'' miera līgumu ar Krieviju, kurā garantēja disidentiem toleranci un vienlīdzību, bet poļu valdošajam slānim tādas ,,tiesības'' kā brīvas karaļa velēšanas un \latxti{liberum veto}. Tomēr poļu virsslānis šādu vienlīdzību nepieņēma. Tam vienlīdzība tiesībās ar citticībniekiem bija līdzvērtīga visu poļu brīvību zaudēšanai. Ar līgumu neapmierinātā daļa garīdzniecīnas, magnātu un šļahtas Baras cietoksnī izveidoja konfederāciju (\pltxti{Konfederacja barska}, 1768--1772) un uzsāka sacelšanos. V.~Kļučevskis to raksturoja kā ''poļu šļahtiču ''Pugačova dumpi'' \citespace{} apspiedēju laupīšanas gājienu par tiesībām uz apspiešanu''. Izraisījās pilsoņu karš, kas savukārt izsauca kaimiņvalstu intervenci.

1772.~gadā Austrija un Prūsija, kuras baidījās, ka Krievija varētu patstāvīgi sagrābt poļu un lietuviešu zemes, iniciēja \strong{pirmo Žečpospolitas sadali}, kuras rezultātā tā šķirās no vairākām svarīgām teritorijām. Cariskā Krievija sākotnēji iebilda pret sadali, cenšoties panākt visas Žečpospolitas pakļaušanu savai ietekmei, jo pie savām rietumu robežām tai bija izdevīgāk saglabāt vāju Žečpospolitu nekā spēcīgu Prūsiju, kura jau 1648.--1721.~gadā bija pievienojusi sev daļēji poļu apdzīvotās Rietumu Pomorjes jeb Pomerānijas \pltxti{(poļu Pòmòrzé}, vācu \detxti{Pommern}) un 1740.~gadā Silēzijas (poļu \pltxti{Śląsk}, vācu \detxti{Schlesien}) teritorijas.\footnote{Par jau pirms Polijas trijām dalīšanām Prūsijas rokās nonākušajām poļu apdzīvotajām teritorijām šajā darbā ies runa tikai kopsakarā ar citām, XVIII~gs. otrajā pusē Žečpospolitā ietilpstošajām zemēm.} Taču kad starptautiskā situācija (1768.~gadā Turcija pieteica karu Krievijai, kurš ilga līdz 1774.~gadam, draudēja arī Austrijas iesaistīšanās tajā pret Krieviju) virzīja Krieviju uz savienību ar Prūsiju, tā piekrita sadalei. Katrīna II raksturoja Polijas dalīšanas cēloņus no sava skatu punkta: ,,Šīs [poļu] tautas nepastāvības, tās ļaunprātības un naida pret mūsu [tautu], nemitīgas tieksmes uz izvirtību un franču negantībām rezultātā, mēs nekad tajā neatradīsim nedz mierīgu, nedz drošu kaimiņu, vienīgi kā novedot to būtiskā nespēkā un nevarenībā''.

Austrija ieguva Austrumgalīcijas (bieži literatūrā apgabalu sauc arī vienkārši par Galīciju) ar Ļvovu (ukraiņu \uktxti{Львів}, poļu \pltxti{Lwów}, vācu \detxti{Lemberg}) un t.s. Mazpoliju (\pltxti{Małopolska}~--- apgabals Polijas dienvidaustrumos Vislas vidus un augštecē, tiek saukts arī par Rietumgalīciju. Vēsturiski galvenā pilsēta~--- Krakova (Krakow), tikai bez Krakovas, kura palika Žečpospolitā. Mūsdienās Austrumgalīcija atrodas Ukrainas, bet Mazpolija jeb Rietumgalīcija Polijas teritorijā. Lai aizmaskotu Polijas sadali, Austrija atcerējas, ka kādreiz (XIV gadsimtā) Austrijas pakļautā Ungārija valdīja pār Galīciju un tāpēc esot notikusi ,,atkalapvienošanās''. 1774.~gadā Austrijā pat izkala medaļu ar uzrakstu ,,\latxti{Antigua jura Vindicata Galicia et Lodomeria in Fidem recepetis MDCCLXXIII}'' (Senās Galīcijas un Lodomērijas tiesības atgūtas 1873.~gadā.

Prūsija, kura sevišķi bija ieinteresēta apvienot divas līdz tam atsevišķi pastāvējušās savas valsts daļas: Brandenburgu ar Pomerāniju un Austrumprūsju, saņēma Rietumprūsiju (vācu~--- \detxti{Westpreußen}~--- apgabals Vislas upes lejteces abos krastos), gan bez Dancigas (vācu \detxti{Danzig}, poļu \pltxti{Gdansk}) un Toruņas (poļu \pltxti{Toruń}, vācu \detxti{Thorn}), Kujāvijas (poļu \pltxti{Kujawy}, vācu \detxti{Kujawien}) reģiona ziemeļdaļu (Vislas kreisajā krastā) un daļu Lielpolijas (poļu \pltxti{Wielkopolska}~--- apgabals Polijas rietumos Vartas (poļu~--- \pltxti{Warta}, vācu~--- \detxti{Warthe}) upes baseinā, kuru agrāk apdzīvoja visļanu un poļanu ciltis). 1773.~gadā Prūsijas karalis Fridrihs~II vēstulē Voltēram, tēlojot šo provinču šķietamo atpalicību, rakstīja: ,,Poļu provinces nevar salīdzināt ne ar vienu no Eiropas valstīm. Augstākais, tās ir salīdzināmas ar Kanādu. Būs jāpieliek daudz pūļu un laika, lai panāktu to, kas garos gadsimtos sliktas pārvaldes dēļ ir nokavēts.'' Prūsija tagad ar Žečpospolitai atņemto piejūras apgabalu savienoja Austrumprūsiju ar pārējo valsts teritoriju. Ar to Polija zaudēja arī savus svarīgākos ārējās tirdzniecības ceļus, bet Prūsija ieguva iespēju kontrolēt vairāk nekā piektdaļu Polijas ārējās tirdzniecības, gūt no muitas nodevām, ar kurām tika apliktas pa Vislu vestās poļu preces (pie Dancigas muitas nodoklis sastādīja 12\% no preču vērtības), ienākumus, kuri pēc dažiem vērtējumiem bija lielāki par pārpalikušās Polijas ienākumiem.

Krievija savukārt ieguva Latgali ar \lttxti{Dinaburgu} (tagadējo Daugavpili) un Austrumbaltkrieviju ar Polocku, Vitebsku un Mogiļevu un t.s. Melno Krieviju (lietuviešu \lttxti{Juodoji Rusia}~--- agrākās Lietuvas lielkņazistes daļu Daugavas labajā un Berezinas upes kreisajā krastā). Prūsijas karalis Fridrihs~II atzina: ,,\dots{}Krievijai ir daudz tiesību tā rīkoties ar Poliju, ko gan nevar teikt par mums ar Austriju''. Karaļa teiktais prasa komentāru. Zemes, ko sagrāba Prūsija, bija pārsvarā poļu apdzīvotas. Austrijai piešķirtajās zemēs dzīvoja galvenokārt rietumukraiņi un poļi. Vāciski runājošo tajās bija nedaudz. No šī viedokļa Krievijai, pievienojot sev slāvu apdzīvotas zemes, bija uz tām vairāk tiesību nekā tās sabiedrotajām. 1772.~gadā, ko poļi atzīmē kā Polijas pirmo dalīšanu, ievērojama Baltkrievijas daļa tika atbrīvota no poļu jūga, un Krievijai šis notikums bija ne mazāk nozīmīgs kā daļas Mazkrievijas (Ukrainas) atbrīvošanās 1654.~gadā no tāda pat jūga un apvienošanās ar Krieviju. Krievu vēsturnieki to vērtē kā faktisku visu trīs bijušo senkrievu atzaru~--- baltkrievu, lielkrievu un mazkrievu apvienošanu vienas Krievijas valsts ietvaros pēc vairāku gadsimtu šķelšanās. Par iespēju atbrīvot daļu ticības brāļu~--- baltkrievu no katoļu diskriminācijas, par viņu pievienošanu Krievijai gan nācās piešķirt brīvas rokas Prūsijai un Austrijai attiecībā uz citām poļu zemēm.

Taču jau padomju vēsturnieki uzsvēra, ka nedz Prūsija, nedz Austrija neuzdrošinājās sagrābt Žečpospolitas teritorijas, kamēr to nebija sankcionējusi cariskā Krievija. Tāpēc carismam tāpat kā tā sabiedrotajiem jānes pilna atbildība par 1772.~gada Polijas dalīšanu, kas nesa prūšu un austriešu varas nodibināšanu pār to valdījumos nonākušajām poļu un ukraiņu tautas daļām.

Jau tūlīt pēc pirmās Polijas dalīšanas neapmierinātie poļu šļahtiči uzsāka cīņu pret carisko Krieviju. Tā, daži pret Krievijas ietekmi Polijā karojošās un sakautās šļahtiču Baras konfederācijas dalībnieki, kuri tika izsūtīti uz Krievijas iekšieni, J.~Pugačova vadītā dumpja jeb Zemnieku kara (1773--1775) laikā pieslējās tam.

Austrija kopumā saņēma ap 83 tūkstošus km$^{2}$ ar 2,6 miljoniem iedzīvotāju. Prūsija kopā ieguva ap 36 tūkstošus km$^{2}$ teritorijas ar 580~000 iedzīvotāju. Krievijas guvums bija ap 92 tūkstoši km$^{2}$ ar 1,3 miljoniem iedzīvotāju. (Literatūrā sastopami arī nedaudz citādi dati. Tautas skaitīšana, kuru veica Poliju dalījušās valstis līdz 1776.~gadam, rādīja, ka Prūsijas iedzīvotāju skaits pieaudzis par 0,6~miljoniem, Austrijas~--- 2,1~miljonu, Krievijas~--- 1,3~miljonu cilvēku.)

Teritoriālās pārmaiņas apstiprināja Žečpospolitas Seims 1773.~gadā. Tikai Novogrudokas (baltkrievu \betxti{Навагрудак}, poļu \pltxti{Nowogródek}) deputāts T.~Rejtans pret to protestēja. XIX~gadsimtā poļu sabiedriskā doma viņu pacēla nacionāla varoņa augstumos. Izcilais poļu mākslinieks J.~Matejko šim notikumam veltīja vienu no savām gleznām.

Tādejādi pirmajā dalīšanā Žečpospolita šķīrās no vairāk nekā 28\% savas teritorijas un vairāk nekā trešdaļas tās iedzīvotāju. Tomēr apmēram 527~tūkstošu km$^{2}$ plašais Polijas valsts pārpalikums vēl joprojām bija tikpat liels, kā Francija vai Lielbritānija un lielāks nekā 1918.~gadā radītā Polijas valsts. Pēc pirmās dalīšanas lielākā daļa poļu, visas lietuviešu, daļa ukraiņu un baltkrievu zemju joprojām palika Žečpospolitas sastāvā. Pēc poļu vēsturnieka T.~Korzona vērtējuma pārpalikušajā Žečpospolitā dzīvoja 7,4~miljoni iedzīvotāju. (Tādejādi, pirms pirmās dalīšanas Žečpospolitā bija ap 11,4~miljoni iedzīvotāju).

Valsts sadalīšana izsauca poļu sabiedrībā šoku, patriotisma uzplūdus un vēlmi modernizēt valsti. Polijas vēstures pētnieks I.~Balabans gan rakstīja, ka pēc pirmās dalīšanas ,,visi saprata, ka ir nepieciešamas reformas'', taču tad jāmin arī poļu vēsturnieka M.~Bobržinska konstatējums, ka Polijas tautu valsts dalīšana ne pārāk satrauca. Lielāku ietekmi atstāja Apgaismības kustība Eiropā, vēlāk~--- Franču revolūcijas sākums (1789). Norisa reformas izglītības un administratīvās pārvaldes jomā. Tā, Polijā pēc karaļa Staņislava Augusta priekšlikuma tika radīta laicīga Izglītības komisija (\pltxti{Komisja nad Edukacją Młodzi Szlacheckiej Dozór Mająca}, 1773--1794) pirmā Eiropā iestāde, kura pēc savām funkcijām līdzinājās visu valsti aptverošai izglītības ministrijai. Pēc Prūsijas parauga tika reorganizēta armija. Tomēr pietiekami lielas un modernas armijas radīšanu, kura varētu nodrošināt Polijas politisko neatkarību, nepieļāva valsts finansiālais vājums. (Pēc M.~Bobržinska datiem vairāk kā puse Žečpospolitas ienākumu tika tērēta armijas uzturēšanai.)

Savukārt finansiālo vājumu lielā mērā noteica valsts politiskā iekārta. Absolūtie monarhi kaimiņvalstīs daudz brīvāk varēja aplikt savus pavalstniekus ar nodokļiem nekā Polijas Seima deputāti savus vēlētājus. Kā rakstīja poļu vēsturnieks J.~Rutkovskis, galvenais traucēklis enerģiskas finansiālās politikas realizācijai bija Seimu ekonomiskā pozīcija. Seimu deputāti pārstāvēja šļahtas intereses, kura vēlējās savās rokās koncentrēt maksimāli lielu nacionālā ienākuma daļu. Agrārpolitikā šļahta aizstāvēja dzimtbūšanas saglabāšanu. Rūpniecības un tirdzniecības jomā Seimi pirmkārt rūpējās par iespēju pārdot lauksaimniecības ražojumus par iespējami augstākām un iepirkt rūpniecības preces par iespējami zemākām cenām. Tas deva šļahtai iespēju celt savu dzīves līmeni. Zemnieki nekādu labumu no šīs politikas neguva. Amatnieki par savu darbu tika atalgoti arvien mazāk. Tas traucēja amatnieciskās ražošanas pāraugšanu par kapitālistisko ražošanu.

Ap 1780.~gadu sabiedriskā aina Polijā gan bija mainījusies uz labo pusi. Neraugoties uz teritoriālajiem zaudējumiem un tirdzniecības ierobežojumiem, saimniecība Žečpospolitā attīstījās. Ja 1776.~gadā valstī ieveda preces par 48 miljoniem zlotu, bet izveda tikai par 22 miljoniem, tad 1785.~gadā eksports (pamatos lauksaimnieciskā produkcija) pirmoreiz ilgu gadu laikā pārsniedza importu. Pieauga valsts ienākumi. Lauksaimniecības ražotā galvenā prece iekšējā tirgū bija rudzi, ārējā~--- kvieši. Tomēr pakāpeniski graudkopība zaudēja savas pozīcijas. No XVIII gadsimta beigām sākās kartupeļu audzēšana. To izplatība iezīmēja progresu lauksaimniecības preču ražošanā, jo kartupeļus izmantoja pārtikā, lopbarībā un kā izejvielu pārtikas rūpniecībā. No tehniskajām kultūrām plaši tika audzēti lini un kaņepāji. Pastāvēja manufaktūras un rūpali. Paplašinājās saražoto preču sortiments. Sākās pāreja uz jaunu kurināmā veidu~--- akmeņoglēm. XVIII gadsimta 90.~gados darbojas ap 280~uzņēmumu, kuri izmantoja galvenokārt algotu darbaspēku. Dzimtcilvēki raka kanālus, kuri savienoja Baltijas un Melnās jūras baseina upju augšteces, kas ļāva daļu tirdzniecības novirzīt uz Melno jūru. Auga pilsētas kā ārējās tirdzniecības centri~--- Gdaņska (poļu \pltxti{Gdańsk}, vācu \detxti{Dancig}) un Poznaņa (poļu \pltxti{Poznań}, vācu \detxti{Posen}). Pēc jau minētā T.~Korzona aprēķiniem, kopš 1775.~gada iedzīvotāju skaits pieauga par 1,4 miljoniem un sasniedza 8,8 miljonus. Varšavas iedzīvotāju skaits 1791.~gadā sasniedza 120~tūkstošus. Tiesa, iekšējā tirdzniecība bija attīstīta vēl vāji. Izplatītākā tās organizācijas forma bija vietējie un gadatirgi. Galvenais šaurā tirgus cēlonis bija zemnieku saimniecību pusnaturālā rakstura saglabāšanās, brīvā tirgus attiecību vājā iespiešanās laukos. Zemnieki sastādīja 72,7\%, pilsētnieki-kristieši~--- 6,8\%, ebreji, kuri mita galvenokārt pilsētās un miestos,~--- 10,2\%, šļahta~--- 8\%, garīdzniecība~--- 0,6\%, pārējie~--- 1,7\%.

Uzplauka kultūra. Karalis Staņislavs II Augusts (Poņatovskis) bija Apgaismības un rokoko mākslas atbalstītājs. Viņa ciešie kontakti ar Drēzdeni veicināja poļu mūzikas dzīvi. Karalis uzstājās arī kā mākslas mecenāts un veicināja celtniecības darbus Varšavā. Viņa laikā uzbūvētā ievērojamākā celtne ir Karaļa pils jeb ,,Pils uz ūdens'' (\pltxti{Pałac Na Wodzie}) Lazenkos (\pltxti{Łazienki}).

Taču valsts tautsaimniecības un kultūras attīstību traucēja dzimtbūšana, brīva darbaspēka trūkums, arī novecojusī politiskā iekārta.

Demokrātiski-patriotiskais virziens iestājās par mantojamas monarhijas ieviešanu, \latxti{liberum veto} tiesību likvidēšanu. Kad 1787.~gadā sākās jauns Krievijas~--- Turcijas karš, kas saistīja Krievijas spēkus, poļos radās cerība izmantot situāciju, lai likvidētu Polijas atkarību no tās. Krievu vēsturnieks S.~Solovjovs šai sakarā rakstīja: ,,Redzēja [poļi] Krieviju apgrūtinātā stāvoklī~--- un gribēja to izmantot; nespēja to izmantot lai iedvestu jaunus spēkus paralizētajā [Polijas] valsts ķermenī, toties pilnībā izbaudīja prieku iespert lauvam, netikuši skaidrībā, ka lauva ne tikai nav tuvs nāvei, bet nav pat saslimis.'' Kopā ar Krieviju pret Turciju karoja arī Austrija. Savukārt Prūsija baidījās no Austrijas un Krievijas nostiprināšanās.

Krievijas valdniece Katrīna II cerēja iesaistīt arī Žečpospolitu savienībā pret Turciju, 1788.~gadā sanāca konfederatīvs (šeit~--- tāds, kurā lēmumus pieņēma ar vairākumu balsu) Seims (t.s. Četrgadu Seims~--- \pltxti{Sejm Czteroletni}, 1788--1792), lai apspriestu šo jautājumu. Pret Žečpospolitas savienību ar Krieviju iebilda Prūsijas karalis Fridrihs Vilhelms II, kurš baidījās no Krievijas nostiprināšanās kara rezultātā. Seims atteicās noslēgt līgumu ar Krieviju. Apstākļos, kad Eiropā brieda lieli notikumi (1789.~gada 14.~jūlijā Parīzē tika ieņemta Bastīlija), patriotiskie poļu spēki Seimā uzsāka cīņu par valsts politisku un ekonomisku reformēšanu. Viens no patriotu pārstāvjiem filozofs un rakstnieks S.~Stašics 1790.~gadā griezās pie sabiedrības ar uzsaukumu ,,\pltxti{Przestroi dla Polsi}'' (,,Brīdinājums Polijai''), kurā bija vārdi: ,,Cik tālu Polija ir atpalikusi! Polijā ir sācies tikai XV gadsimts, kad pārējā Eiropā beidzas XVIII gadsimts!''

To, ka šī atpalicība tomēr bija visai nosacīta, rādīja \strong{1791.~gada 3.~maijā} Četrgadu Seima pieņemtā jaunā valsts \strong{Konstitūcija}, kura likvidēja Žečpopolitas konfedaratīvo raksturu, Lietuvas lielkņazistes nosacīto patstāvību, ieviesa franču filozofa un rakstnieka Š.~Monteskjē proponēto varas dalīšanas principu, nodalot likumdošanas, izpildvaru un tiesu varu, pasludināja t.s. pilsoniskās brīvības. Polijas Konstitūcija iedibināja arī citus jaunievedumus: kontrasignācijas (no latīņu \latxti{contra} pret + \latxti{signare} parakstīt) principu, kad valdības vadītājs vai atsevišķs ministrs līdzās valsts galvam parakstīja tā izdotus aktus, tā uzņemdamies politisku un juridisku atbildību; parlamentārās (politiskās) atbildības principu un neuzticības votumu (no latīņu \latxti{votum}~--- vēlēšanas, lēmums, domas, kas izteiktas balsojot) ar sekojošu ministru atsaukšanu no amata. Tika noteikta Seima kā divpalātu parlamenta uzbūve (deputātu un senatoru palāta, ar deputātu palātas izšķirošu nozīmi), Seima sēžu kārtība, likumdošanas procedūra un likumprojektu balsošanas kārtība, kvoruma (latīņu \latxti{quorum}~--- nepieciešamais sapulces dalībnieku skaits, lai tā būtu pilntiesīga) princips. Tāda parlamenta struktūra darbojas arī mūsdienās.

Šļahtas un garīdzniecības privilēģijas gan lielā mērā saglabājās, bet, likvidējot \latxti{liberum veto} un nostiprinot Seima tiesības pieņemt lēmumus ar balsu vairākumu, tika mazināta feodālā anarhija, magnātu varenība, paplašinātas augošās pilsonības (buržuāzijas) tiesības. Kā norādīja poļu vēsturnieks un valsts darbinieks H.~Jablonskis, patriotiskajām aprindām, kuras centās glābt Žečposplitu, bija skaidrs, ka, lai saglabātu jau satricināto Polijas neatkarību, pirmkārt bija jācenšas iznīcināt magnātu varu. Tāpēc 3.~Maija Konstitūcijas ieviestajai nemantīgās šļahtas, kura sastādīja magnātu galveno balstu, politisko tiesību ierobežošanai bija progresīva nozīme, kaut vecās iekārtas aizstāvji skaļi vaimanāja par ,,šļahtiskās demokrātijas'' iznīcināšanu.

Polijas 1791.~gada 3.~Maija Konstitūcija bija neapšaubāms solis uz priekšu, tā bija otrā rakstītā satversme pasaulē pēc ASV 1787.~gada Konstitūcijas un pirmā Eiropā (Francijā 1791.~gada Konstitūcija tika pieņemta 3.~septembrī.) Franči Polijas satversmi izmantoja, izstrādājot savas valsts 1791., 1793. un 1795.~gada Konstitūcijas. Neraugoties uz ierobežotību, Polijas Konstitūcijai bija progresīvs raksturs un tās pieņemšanas gadadiena Polijā tiek svinīgi atzīmēta vēl joprojām. Četrgadu Seima veikums un īpaši jaunā Konstitūcija izsauca līdzjūtīgu vērtējumu ,,visā Eiropā''. Tomēr, kaut arī Polijā tika ieviesta konstitucionāla monarhija, Konstitūcija bija novēlota, tā būtiski nemainīja sabiedrisko iekārtu, valstī joprojām dominēja feodālā kārtība, pilsonība palika ļoti vāja. Turpināja pastāvēt dzimtbūšana un ebreju īpašais nelīdztiesīgais stāvoklis.

Pie tam jaunās satversmes pieņemšanas ar karaļa piekrišanu diena bija izraudzīta tad, kad tās pretinieki Seimā vēl atradās brīvdienās un nevarēja piedalīties balsošanā. Pēc vācu vēsturnieka E.~Meijera datiem bija ieradušies tikai ap 100 no 500 Seima deputātiem. Ar to Konstitūcijas likumību varēja apšaubīt un tās pieņemšanu pielīdzināt valsts apvērsumam. Pie tam Konstitūcijas nostiprināšanai tika draudēts lietot arī nekonstitucionālus līdzekļus. Tā, 13.~maijā visās Varšavas ielās bija piestiprināti drukāti nezināmas izcelsmes uzsaukumi, kuri aicināja nokaut katru, kurš runāja, rakstīja, pretojās vai arī gatavojās to darīt pret 3.~maija konstitūciju. Katram tādam slepkavam tika solīta atlīdzība. Policija gan uzsaukumus noplēsa. Tomēr visi vietējie seimeļi (\pltxti{sejmik}) atzina Konstitūciju.

Taču izveidojās Žečpospolitai nelabvēlīga starptautiskā situācija. Ja poļu reformatori cerēja izmantot Austrijas un Prūsijas nesaskaņas, viņi vīlās. Pēc Austrijas ķeizara Jozefa II nāves 1790.~gada 20.~februārī jau 27.~jūlijā Austrija un Prūsija savstarpēji izlīga, bet 1791.~gada augustā noslēdza vienošanos pret Franciju. Mainījās Krievijas nostāja. Vēl 1790.~gada 12.~oktobrī Krievijas imperatore Katrīna II paziņoja ,,Tā kā mēs raugāmies uz Poliju kā uz valsti, kura atrodas starp četrām spēcīgām valstīm [Prūsiju, Krieviju, Austriju, Turciju] un kalpo kā šķērslis to strīdiem, tad šo šķēršļi ir jāsaglabā un jāsargā, lai arī ko tas mums maksātu. Un mēs par to parūpēsimies. Taču tikai līdz brīdim, līdz mūsu ienaidnieku un pašas Polijas ienaidnieku ļaunie nodomi nepiespiedīs mūs mainīt mūsu politiku.'' Taču pēc krievu--turku kara (1787--1791) beigām 1792.~gada maijā Katrīna~II, kura baidījās no Polijas nostiprināšanās un tās mēģinājumiem atjaunot 1772.~gada robežas, kā arī revolucionārā ,,franču mēra'' izplatīšanās, bija gatava iejaukties Polijas iekšējās lietās, aicinot poļus uz pilsonisku nepakļaušanos.

Saprotams, ka, baidoties par savām pozīcijām, magnāti un no tiem atkarīgie ,,klienti'' meklēja Katrīnas~II atbalstu. Ievērojamais vācu vēsturnieks H.f.~Treičke par toreizējo situāciju Polijā rakstīja: ,,Pār malām plūstošs cīņas prieks un gatavība upurēties, dedzīgas runas un brālīgi apskāvieni, vaimanājoši priesteri un augstprātīgas, skaistas sieviete, pie tā klāt punšs un mazurka, cik tik sirds vēlas, bet arī savstarpējais partiju ienaids, nepaklausība, dusmīgas apsūdzības un drošsirdīgu, sajūsminātu vīru viļņošanās vidū nevienas valstiskas galvas, neviena liela rakstura.'' Krievijas piekritēji 1792.~gada maijā organizēja t.s. Targovices konfederāciju (\pltxti{Konfederacja targowicka}, pēc miesta nosaukuma), kura pasludināja jauno Konstitūciju par spēkā neesošu un ar krievu karaspēka palīdzību vērsās pret karali un Seimu. 1792.~gada 18.~maijā Krievijas karaspēks iemaršēja Polijas teritorijā. Atsaucoties uz 1791.~gada 3.~maija poļu ,,revolūciju'', to pašu darīja arī Prūsijas armija. (Interesanti, ka vācu vēsturnieks H.f.~Zitzevics šo rīcību salīdzinājis ar PSRS un tās sabiedroto, arī Polijas, karaspēka ieiešanu Prāgā 1968.~gadā.)

Poļu karaspēku komandēja karaļa brāļa dēls J.~Poņatovskis un Amerikas atbrīvošanās kara dalībnieks, ASV pilsonis ģenerālis \strong{T.~Kostjuško} u.c.

T.~Kostjuško bija dzimis mūsdienu Baltkrievijas senā, bet ne pārāk turīgā augstmaņu dzimtā. Iespējams, ka viņa pirmā valoda bija baltkrievu, un viņš tika kristīts gan pareizticībā, gan katolicismā. 1769.~gadā Kostjuško tika piešķirta karaļa stipendija mācībām Parīzē. Šeit viņš kā eksterns mācījās Parīzes kara akadēmijā. Parīzē pavadītie pieci gadi būtiski ietekmēja T.~Kostjuško vēlākos uzskatus. Piedalījās Amerikas neatkarības karā. 1776.~gada augustā ieradās ASV, dažus mēnešus vēlāk kļuva par Kontinentālās armijas galveno inženieri un vadīja Filadelfijas fortifikācijas darbus, sadraudzējās ar T.~Džefersonu. Pēc septiņiem gadiem armijas dienestā Kongress 1783.~gada 13.~oktobrī T.~Kostjuško piešķīra brigādes ģenerāļa dienesta pakāpi. Viņam tika piešķirta ASV pilsonība, īpašumi. Tomēr 1784.~gadā viņš devās atpakaļ uz Poliju, bet savu īpašumu novēlēja izmantot melno vergu izpirkšanai un izglītošanai. 1784.~gada augustā K.~Kostjuško ieradās Polijā un apmetās savā dzimtas īpašumā. Viņš būtiski uzlaboja savu dzimtcilvēku stāvokli, atvieglodams klaušas, tādejādi izpelnīdamies liberāļa slavu.

Poļi guva panākumus Zeleņces kaujā (\pltxti{Bitwa pod Zieleńcami}). Pēc kaujas karalis Staņislavs II Augusts nodibināja \latxti{Virtuti Militari} (kara nopelnu) ordeni, ar kuru apbalvoja J.~Poņatovski un T.~Kostjuško. Arī mūsdienās šis ordenis ir Polijas augstākais militārais apbalvojums. Tomēr tā kā atbilstoši stratēģiskajai situācijai poļi pārspēka priekšā turpināja atkāpties, krievu avotos kauja skaitījās kā viņu uzvarēta. Turpmāk Krievijas armija vairākās kaujās sakāva poļu un lietuviešu spēkus. 1792.~gada jūlijā arī Polijas karalis Staņislavs~II Augusts pievienojās konfederātiem, izdeva rīkojumu par savas armijas atlaišanu. Krievijas karaspēks ieņēma Varšavu.

1793.~gada janvārī Krievija un Prūsija, kuras vienoja bailes no revolucionārajiem notikumiem Francijā, to ietekmes izplatības Polijā, (Prūsija vēl arī nevēlējās pieļaut Krievijas robežu tālāku izplatību uz rietumiem) parakstīja slepenu vienošanos par jaunu, jau \strong{otro Polijas dalīšanu}. Austrija, kura tajā laikā bija cietusi vairākas sakāves karā ar Franciju, kā arī centās sagrābt Bavāriju, Polijas otrajā dalīšanā nepiedalījās. Tās noteikumi poļiem tika paziņoti 1793.~gada 27.~martā.

Krievija saņēma Rietumbaltkrieviju ar Minsku (baltkr. \betxti{Мінск)}, Labā krasta Ukrainu ar Žitomiras (ukraiņu \uktxti{Житомир}) pilsētu, Volīnijas (ukraiņu \uktxti{Волинь}, poļu \pltxti{Wołyń}) austrumu daļu un daļu Podolijas (poļu \pltxti{Podole}, ukr. \uktxti{Поділля}, \uktxti{Podilla}, krievu \rutxti{Подолье}), ar Braslavas (poļu \pltxti{Wrocław}, vācu \detxti{Breslau}, krievu \rutxti{Бреславль}) pilsētu, Prūsija~--- t.s. Lielpolijas apgabalu ar Gņezno (poļu \pltxti{Gniezno}, vācu. \detxti{Gnesen}), Poznaņas, Gdaņskas un Toruņas pilsētām. Katrīna~II, kura jau 1792.~gada decembrī Krievijas sūtnim Polijā J.~Siversam rakstīja, ka Krievija nolēmusi ,,atbrīvot kādreiz Krievijai piederējušās, tās tautiešiem (\rutxti{единоплеменники}) un vienai ticībai piederīgajiem apdzīvotās zemes un pilsētas no viņus apdraudošās kārdināšanas un apspiešanas'', tagad par godu ukraiņu un baltkrievu apdzīvoto zemju ,,atgriešanai'' Krievijas sastāvā lika izkalt speciālu medaļu. Tās viena pusē bija viņas pašas profils, otrajā~--- Krievijas ērglis, kurš savieno agrāko Krievijas un pievienoto zemju kartes ar uzrakstu ,,\rutxti{Отторженная возвратихъ}'' (Atņemtā atgriešana).

Mūsdienās baltkrievu vēsturnieki lielākoties uzskata, ka tikai pateicoties baltkrievu zemju iekļaušanai Krievijas impērijās sastāvā tika pārtraukta to katolizācija un polonizācija, kas ļāva baltkrieviem saglabāties kā etnosam. Tieši Krievijas impērijas sastāvā noformējās patstāvīga baltkrievu nācija.

Žečpospolitas teritorija samazinājās vēl divas reizes. Tagad tā aptvēra vairs tikai 230~000 km$^{2}$ ar 4,4~miljoniem iedzīvotāju. Īpaši sāpīgs Žečpospolitai bija Gdaņskas pievienošana Prūsijai, jo ar to tika zaudēta arī pieeja jūrai. Valstiski-tiesiskās attiecības starp Krieviju un Žečpospolitu noteica traktāts, ko tās parakstīja 1793.~gada oktobrī. Pēc traktāta noteikumiem abas valstis apņēmās, ka, uzbrukuma gadījumā vienai no tām, otra palīdzēs tai ar visiem spēkiem. Apvienoto karaspēku komandēt tiesības saņemtu valsts, kura piešķirtu vairāk karaspēka. Praktiski tas nozīmēja, ka poļu karaspēks tika pakļauts krievu pavēlniecībai, jo Polijai bija tiesības turēt tikai 15~000 lielu armiju. Krievija saņēma tiesības vajadzības gadījumā ievest savu karaspēku Polijā, tiesības apstiprināt tās ārpolitiskos līgumus. Tika atcelta arī 3.~maija Konstitūcija. Faktiski pēc otrās dalīšanas Žečpospolita zaudēja savu neatkarību. Tā no iespējamā Krievijas pretinieka kļuva par tās vasaļvalsti.

Žečpospolitas sadalīšanu u.c. noteikumus apstiprināja t.s. Grodņas Seims 1793.~gada rudenī. Panākt šī lēmumu pieņemt Seima deputātus Krievijas sūtnis J.~Siverss centās ar kukuļiem, draudiem, tiešu spiedienu (Grodņas pils bija pilna krievu kareivju). Kad Seima maršals prasīja nobalsot par līgumu, deputāti vairākas stundas klusēja. Galu galā atskanēja Krakovas deputāta J.~Ankviča balss: ,,Klusēšana ir piekrišana''. Tad nu Seima maršals paziņoja, ka līgums pieņemts vienbalsīgi. Krievu vēsturnieks S.~Solovjovs uzsvēra, ka šādu notikumu gaitu noteica gadsimtiem ilgā poļu tautas klusēšana, kad trokšņoja vienīgi šļahta seimos.

Taču poļu patriotu cīņas griba vēl nebija sagrauta, cerot uz revolucionārās Francijas palīdzību, viņi slepus gatavoja \strong{sacelšanos}. Par savu vadītāju viņi izvirzīja T.~Kostjuško, kurš sevi jau bija pierādījis par drosmīgu karavadoni Amerikas brīvības cīņās un arī Polijā. 1794.~gada 16.~martā Krakovas iedzīvotāji pasludināja T.~Kostjuško par republikas diktatoru un nacionālo bruņoto spēku virspavēlnieku. T.~Kostjuško komandētie poļu spēki sasniedza līdz 70~000 cilvēku (kopumā, ieskaitot mobilizētos pilsētniekus un zemniekus, tautas miliciju, poļu vienības sasniedza vairākus simtus tūkstošus cilvēku), taču bija ļoti vāji apbruņoti. 1794.~gada 4.~aprīlī T.~Kostjuško vadītajām vienībām izdevās gūt panākumus kaujā pie Raclaviciem (\pltxti{Bitwa pod Racławicami}), kur krievu karaspēks uzbruka ciešā ierindā, bet T.~Kostjuško, izmantojot pieredzi, gūtu ASV Neatkarības karā, poļu strēlniekus izvietoja izklaidus, izmantojot dabīgos aizsegus. Pats T.~Kostjuško izveda ar izkaptīm bruņoto zemnieku vienības krievu daļu aizmugurē, kur tie negaidītā straujā uzbrukumā sagrāba artilēriju un piespieda krievu kājniekus atkāpties. Taču iznīcināt tos neizdevās un kara darbība turpinājās.

Poļu panākumam vēlāk tika veltīta J.~Matejko glezna. 1894.~gadā Lembergā tika atklāta 114 metru gara, 15 metru augsta kaujas panorāma ar 38 metru diametru. 1944.~gadā kara gaitā panorāma tika daļēji bojāta un to sāka atjaunot tikai 1980.~gadā, bet 1985.~gadā atklāja Vroclavā. Nosauktie mākslas darbi joprojām kalpo romantizētam poļu vēstures izklāstam.

Pēc uzvaras pie Raclaviciem sacelšanās sākās arī citviet. Sacēlušos rokās nonāca Varšava un Viļņa. Varšavas sacelšanās gaitā liela daļa krievu garnizona gāja bojā, (viens bataljons~--- ap 500 karavīru atradās baznīcā, protams, neapbruņots. Sacēlušies ielauzās baznīcā un lielāko daļu karavīru nogalināja), otrai izdevās atstāt pilsētu. Kā rakstīja I.~Balabans, pēc krievu garnizona iznīcināšanas ,,asiņainajiem skatiem'' ,,franču revolūcijas piemēra iejūsminātā tauta'' izvilka no mājām arī ,,Tēvzemes nodevējus'' un pakāra uz laternu stabiem. Vēsts par briesmu darbiem Varšavā sasniedza krievu armijas daļas, radot tajās spēcīgas atriebības jūtas.

Savā laikā padomju vēsturnieki V.~Djakovs un I.~Millers PSRS un Polijas sakariem veltītā rakstu krājumā rakstīja, ka ,,humānā un labvēlīgā sacēlušos attieksme pret viņu gūstā nonākušajiem krievu virsniekiem un karavīriem lika tiem iestāties revolucionārās armijas rindās''. Šiem krievu karavīriem poļi nelika karot pret saviem tautiešiem, bet nosūtīja uz divīziju, kura atvairīja prūšu uzbrukumu. Nenoliedzot šādu faktu esamību, domājams, ka visiem revolucionāriem simpatizējošie padomju autori sacēlušos humānismu pārspīlēja un apžēloti kā reiz tika tikai tie, kuri izteica gatavību karot poļu pusē.

T.~Kostjuško stāvokli sarežģīja nepieciešamība rēķināties kā ar kustības rojālistisko, tā revolucionāro spārnu, kurš Varšavā jau organizēja revolucionāros tribunālus un politisko pretinieku pakāršanu. Piemēram, tika sodīts ar nāvi Viļņas bīskaps I.~Masaļskis, Lietuvas hetmanis J.~Zabello u.c. Vācu vēsturnieks H.f.~Zitzevics šo linča tiesu (angļu~--- \entxti{the Lynch law}, aizdomās par noziegumiem turēto cilvēku nogalināšanu bez tiesas un izmeklēšanas) pret īstiem vai iedomātiem sacelšanās pretiniekiem pastāvēšanu, kuras nobiedēja mēreno elementus, pat minējis kā tās sakāves galveno cēloni.

T.~Kostjuško mēģinājumi iesaistīt cīņā arī plašākus zemnieku spēkus izsauca šļahtiču neapmierinātību. 1794.~gada 7.~maijā Poļaņicas (\pltxti{Połaniec}) pilsētas apkārtnē tika pasludināts manifests (\pltxti{Uniwersał połaniecki}, poļu \pltxti{uniwersal}~--- vēstījums visiem), kurš prasīja, lai zemniekiem, kuri bija samaksājuši parādus un izpildījuši virkni citu nosacījumu, tiktu piešķirta personīgā brīvība, lai klaušu dienu skaits tiktu samazināts, lai zemes īpašnieki vai tās pārvaldnieki atbildētu tiesas priekšā par zemnieku apspiešanu kā vainīgi vēlmē pazudināt nacionālās sacelšanās lietu utt. T.~Kostjuško biedēja šļahtu, ka Maskava grib pret to sacelt poļu zemniekus, norādot uz to smago dzīvi un solot Katrīnas II labvēlību. Šļahta bija sašutusi par šādu mēģinājumu graut viņu īpašuma tiesības uz zemniekiem un universālam nebija praktisku seku, bet tikai simboliska nozīme.

Kad T.~Kostjuško nosūtīja savu pārstāvi uz Parīzi lūgt palīdzību Sabiedriskās glābšanas komitejai (\frtxti{Comité de salut publicē}, 1793--1795), tā apšaubīja poļu revolucionaritāti un atteica palīdzību. Komiteja poļu pārstāvim uzdeva jautājumus: ,,Kā izskaidrot, ka jūsu Kostjuško, tautas diktators, cieš sev blakus karali, kuru pie tam, kā būtu jāzin Kostjuško, tronī iesēdināja Krievija? Kā izskaidrot, ka jūsu diktators aiz bailēm aristokrātu priekšā, kuri nevēlas atteikties no ,,darba rokām'', neuzdrošinājās veikt zemnieku masu mobilizāciju? Kā izskaidrot, ka viņa proklamācijas zaudē savu revolucionāro nokrāsu, attālinoties no Krakovas? Kā izskaidrot, ka viņš nekavējoties apspieda karātavām tautas sacelšanos Varšavā? Atbildiet!'' Poļu pārstāvim nācās klusēt.

Toties pret T.~Kostjuško armiju karoja arī Austrijas un Prūsijas bruņotie spēki (visai neaktīvi, kaut daži vācu vēsturnieki apgalvo, ka tieši Prūsijas karaspēka pievienošanās Krievijas armijai noteikusi kara iznākumu).

Sacelšanās apspiešanā piedalījās arī slavenais krievu karavadonis A.~Suvorovs. 28.septembrī poļu galvenie spēki tika smagi sakauti, T.~Kostjuško ievainots un saņemts gūstā.

Pēc nostāstiem, tad viņš izteicis vārdus ,,\latxti{Finis Poloniae}'' (no latīņu: ,,Beigas Polijai'', literatūrā ir arī cits šo vārdu variants: ,,\latxti{Finis regni Poloniae''}~--- ,,Beigas Polijas karaļvalstij.''). Pats T.~Kostjuško vēlāk, 1803.~gadā gan rakstīja, ka viņš bija smagi ievainots (kājā un galvā), gūstā krita jau bez samaņas, ko atguva tikai pēc divām dienām, un nekādus pravietiskus vārdus nespēja pateikt. Viņš arī neuzskatījis sevi par pēdējo poli, ar kura nāvi Poljai pienāktu beigas. Pēc citas versijas, kad Varšavā uzzināja par notikušo, tad gan atskanējuši saucieni; ,,Nav Kostjuško! Beigas tēvzemei!''.

Interesanti, ka T.~Kostjuško vadībā cīnījās arī jauns virsnieks M.~Oginskis, kurš pēc sakāves bija spiests atstāt dzimteni. Pēc vienas mūzikas zinātnieku versijas tieši 1794.~gadā arī radās viņa pasaulslavenā polonēze ,,\pltxti{Pożegnanie Ojczyzny},, (Atvadas no dzimtenes), saukta arī par ,,\pltxti{Polonez Oginsky}'' (Oginska polonēzi). (Pēc citas versijas polonēze tapa pēc 1820.~gada, kad M.~Oginskis jau bija Krievijas imperatora Aleksandra I amnestēts, saņēma atpakaļ konfiscētās muižas, ieņēma senatora amatu, bet, neapmierināts ar imperatora politiku, devās uz Itāliju.)

Pēc tam A.~Suvorova vadībā krievu karaspēks devās uzbrukumā Varšavai 24.~oktobrī, ieņemot tās priekšpilsētu Prāgu (\pltxti{Praga}).

Uzbrukumā piedalījās arī tie pulki, kuri bija izcietuši poļu pēkšņo sacelšanos Varšavā. Cīņas Prāgā bija asiņainas. A.~Suvorovs ziņojumā par kauju rakstīja: ,,Pārvarot visas grūtības un uzveicot pretinieka sīvo aizstāvēšanos trijos nocietinājumos, mūsu karaspēks ielauzās Prāgā. Briesmīga bija asinsizliešana, katrs solis uz ielām bija nosegts ar kritušajiem. Visi laukumu noklāti ķermeņiem, pēdējā un pati briesmīgākā iznīcināšana notika Vislas krastā, ko redzēja Varšavas tauta. Šis briesmīgais skats iedvesa viņiem šausmas…''

Poļu vēstures literatūrā uzsvērts, ka krievu karavīri zvērīgi izrīkojās ar civiliedzīvotājiem. Polijā stingri nostiprinājies viedoklis par mierīgo poļu iedzīvotāju ,,masu slaktiņu''. Poļu vēsturnieki Varšavas piepilsētas civiliedzīvotāju nogalināšanu piemin bieži, neminot gan daudzus līdzīgus gadījumus tā laika karu un sacelšanos vēsturē. Šis sižets ietverts daudzās periodam veltītajās monogrāfijās un arī skolai domātās mācību grāmatās, kaut autoru vidū nav vienprātības par slaktiņa mērogiem. Tiek runāts par ,,pilnīgu'' izkaušanu, ,,ievērojamas daļas'' apslaktēšanu un vienkāršu ,,slaktiņu''. Piemēram, ievērojamais poļu publicists, vēsturnieks, vēlāk arī diplomāts L.~Vasiļevskis rakstīja, ka Krievijas karaspēks ,,apkāva un noslīcināja Vislā vairāk nekā 10 tūkstošu sieviešu un bērnu''. Viedoklis ir pārvērties par aksiomu, kuru Polijā vairs neviens neapstrīd. Jāatzīmē gan, ka mūsdienu krievu vēsturnieks A.~Širokorads atzīmējis, ka neviens no poļu vēsturniekiem nav devis atbildi uz jautājumu, kāpēc, ilgus mēnešus gatavojoties pilsētas aizsardzībai, poļu vadītāji neevakuēja Prāgas iedzīvotājus vismaz pāri Vislai un neizmitināja pārējo varšaviešu namos.

PSRS šī epizode netika pieminēta nedz vēsturnieku darbos, nedz uzziņu literatūrā. No pirmspadomju Krievijas vēsturniekiem par šo kara epizodi visplašāk rakstījis N.~Kostomarovs, kurš secināja, ka poļu stāsti par šo epizodi ,,neiztur kritiku''. Atsaucoties uz krievu ,,tā laika avotiem'', vēsturnieks raksta, ka pavisam gāja bojā ap 12 tūkstošu poļu (karavīru un civiliedzīvotāju), pie tam daudzi, ,,glābjoties no krievu durkļiem'' noslīka Vislā, gūstā krita ap 1~tūkstotis cilvēku. Pēc krievu vēsturnieka D.~Bantiša-Kamenska datiem, Prāgas ieņemšanā tika nogalināti 13,5 tūkstoši poļu karavīru, ap 11,5~tūkstošu tika saņemti gūstā, līdz diviem tūkstošiem noslīka Vislā, cenšoties tai pārkļūt, ap tūkstotim tas izdevās. No 22~tūkstošiem krievu karavīru, kas piedalījās uzbrukumā Prāgai, 580 tika nogalināti un 960 ievainoti. (1894.~gadā Sanktpēterburgā Krievijas ģenerālštāba pulkveža N.~Orlova izdotā darbā par Prāgas ieņemšanu poļu kritušo skaits minēts no 9 līdz 10 tūkstošiem, ievainoto no 11 līdz 13 tūkstošiem, krievu zaudējumi: 300--450 nogalināto un ap 2~000 ievainoto.) Jau poļu-krievu spēku samērs vien liecina, ka, ja poļu karavīri kaut cik drosmīgi cīnījās, krievu karavīriem nebija laika nodarboties ar civiliedzīvotāju ,,slaktēšanu''.

Memuāros stāstīts par briesmīgiem nežēlības gadījumiem, kādi sastopami gandrīz katra cietokšņa ieņemšanas gaitā, gan arī uzvarētāju augstsirdības gadījumiem. Ir atrodamas publicista F.~Bulgarina pierakstītas krievu ģenerāļa I.~fon Klugena atmiņas, kurš stāstīja: ,,Uz mums šāva no ēku logiem un jumtiem, un mūsu karavīri, ielaužoties mājās, nogalināja visus, kas viņiem gadījās ceļā \citespace{} virsnieki vairs nespēja apturēt asins izliešanu \citespace{} Pie tilta atkal sākās slaktiņš. Mūsu karavīri šāva pūļos, neņemot vērā p~--- un spalgie sieviešu kliedzieni, bērnu brēcieni stindzināja dvēseli \citespace{} Saniknotie mūsu karavīri katrā dzīvā radībā redzēja mūsējo slepkavnieku Varšavas sacelšanās laikā. ,,Nav nevienam piedošanas''~--- kliedza mūsu karavīri un nogalināja visus, neievērojot ne gadus, ne dzimumu. \citespace{} Četrās stundās notika briesmīga atriebība par mūsējo apslaktēšanu Varšavā.'' Šeit gan jāņem vērā, ka Prāgas aizstāvju vidū bija vairāki tūkstoši brīvprātīgo vietējo iedzīvotāju, kuri upuru noteikšanas gaitā varēja tikt pieskaitīti civiliedzīvotājiem, kaut karoja ar ieročiem rokās. Prāgas ieņemšanas laikā bija arī vēsturnieku minētā epizode, kad daļa tās aizstāvju izlauzās līdz Vislai un krievu karavīri tos, kuriem neizdevās upi pārpeldēt, iznīcināja kreisā krasta iedzīvotāju acu priekšā. Šie fakti varēja kalpot par pamatu leģendai par civiliedzīvotāju ,,slaktiņu''. Jāatzīst, ka krievu karavīru vidū, kuriem bija izdevies sacelšanās sākumā atstāt Varšavu, dzīvas bija atmiņas par tur zvērīgi nogalinātājiem ieroču biedriem, valdīja atriebības jūtas, kuras acīmredzot atsevišķos gadījumos varēja izpausties arī pret civilistiem, taču tad nu jārunā par savstarpējiem ,,slaktiņiem''.

Prāgas ieņemšana nobiedēja Varšavas iedzīvotājus, daļa metās bēgt projām no pilsētas, citi nosūtīja deputātus pie karaļa, pieprasot kapitulāciju. Viņš atbalstīja lūgumu un pie A.~Suvorova tika nosūtīta Varšavas maģistrāta delegācija. Kad daži poļu virsnieki pirms tam mēģināja ar spēku izvest no pilsētas karali un iepriekš saņemtos krievu gūstekņus, lai cīņu turpinātu, pilsētnieki to nepieļāva. Tā paši Varšavas iedzīvotāji nepieļāva sacelšanās dalībnieku mēģinājumus turpināt cīņu. Kapitulācija tika pieņemta.

Varšavas maģistrāts kopā ar sālsmaizi pasniedza pilsētas atslēgas A.~Suvorovam. 26.~oktobrī viņa armija iegāja pilsētā. Kad A.~Suvorovs novērsa iespējamo pilsētas izlaupīšanu, atbrīvoja no gūsta sākotnēji 6~000 poļu zemessargu, pēc tam vēl arī 500 sagūstīto poļu virsnieku, Varšavas maģistrāts iedzīvotāju vārdā viņam uzdāvināja zelta tabakas dozi ar briljantiem un uzrakstu ,,Varšava~--- savam glābējam''. Imperatorei Katrīnai~II par Varšavas ieņemšanu A.~Suvorovs ziņoja vēstulē, sastāvošā no trim vārdiem: ,,Urā! Varšava mūsu!'' Katrīnas~II atbilde bija tikpat īsa: ,,Urā! Feldmaršals Suvorovs!'' Tātad par savu veikumu Polijā A.~Suvorovs saņēma feldmaršala pakāpi, kas apliecina ne tikai viņa militāro spēju novērtējumu, bet arī to nozīmi, kādu Katrīna~II piešķīra Polijas pakļaušanai.

Daļa poļu patriotu, kam izdevās izglābties no Krievijas karaspēka, gan mēģināja vēl turpināt cīņu, bet drīz tika galīgi sakauti. Var atzīmēt, ka poļu šļahtiči 1795.~gadā neveiksmīgi mēģināja izraisīt sacelšanos ukraiņu un baltkrievu zemnieku vidū ar lozungu ,,Par mūsu un jūsu brīvību'' (\pltxti{Za naszą i waszą wolność).} Daļa poļu bēga uz ārzemēm un no šī laika tur pastāvīgi dzīvoja poļu emigranti. 1796.~gadā Krievijas cars Pāvils I apžēloja T.~Kostjuško un līdz ar viņu arī citus 20~000 poļu politiskos ieslodzītos, kuri bija nometināti Sibīrijā. T.~Kostjuško emigrēja uz ASV.

Kaut cietusi sakāvi, 1794.--1795.~gada sacelšanās ievadīja virkni poļu nacionālās atbrīvošanās kustības dalībnieku sacelšanās vairāku gadu desmitu garumā, nostādot Polijas jautājumu Eiropas valstu politikas dienas kārtībā. Jau minētie padomju vēsturnieki V.~Djakovs un I.~Millers uzskatīja, ka šī sacelšanās iezīmē poļu atbrīvošanās kustības sākumrobežu.

Kā vēsturisku mītu var pieminēt baumas, kas klīda par Katrīnas~II nāvi, (pēc oficiālās versijas viņa mira no asinsizplūduma smadzenēs), kas it kā bija saistīta ar pēc T.~Kostjuško sakāves un trešās Polijas dalīšanas uz Pēterburgu atvesto Pjastu (\pltxti{Piasty}~--- poļu karaļu dinastija, kura valdīja no X līdz XIV gadsimtam) dinastijas troni. Katrīna~II esot personīgi likusi savos apartamentos Ziemas pilī to apvienot ar tādu toreizējo jaunievedumu kā savu personīgo ūdens klozetu. (Poļu garīdznieks un vēsturnieks, 1830--1831~gada sacelšanās dalībnieks V.~Kaļinka raksta par S.~Poņatovska troni, uz kura sēdējuši arī S.~Batorijs, Sigizmungs Vaza un viņa dēls Vladislavs. 1796.~gada 6.(17.) novembra rītā ķeizariene pēc pamošanās devās to apmeklēt, bet pēc kāda brīža galminieki izdzirdēja krītoša ķermeņa troksni. Kādu brīdi sulaiņi šaubījās līdz tomēr iedrošinājās atvērt durvis. Ķeizariene gulēja uz grīdas bez samaņas un noasiņoja. Tūlīt izsauktie ārsti neko vairs nespēja līdzēt, pēc dažām stundām Katrīna~II nomira no vaginālās asiņošanas. Pēterburgas aristokrātiskajos salonos vēl ilgi mēļoja, ka ķeizarienes ūdens klozetā zem Pjastu troņa esot bijis noslēpies kāds poļu fanātiķis, iespējams, punduris, un no apakšas viņu ievainojis ar šķēpu vai zobenu, bet pēc tam, izmantojot apjukumu, aizbēdzis no Ziemas pils. Loti jāšaubās, vai šai mītā ir kaut grans patiesības. Iespējams, baumu izcelsmi veicināja poļu svētuma Pjastu troņa izvešana no Polijas un it kā notikusī zaimošana, kas poļu acīs brēca pēc soda.

T.~Kostjuško vadītās sacelšanās, kura ilga 238~dienas, sakāve kalpoja par pamatojumu \strong{Polijas-Lietuvas valstiskuma likvidācijai}. 1795.~gada 24.~oktobrī valstis, kuras piedalījās Polijas sadalē, noteica savas jaunās robežas. \strong{Tā bija Polijas trešā dalīšana}. Karalis Staņislavs~II Augusts Poņatovskis 1795.~gada 25.novembrī nolika savas pilnvaras. Bijušās Žečpospolitas zemju valstiski-tiesiskā situācija būtiski mainījās: nodibinājās absolūtismam raksturīga pārvalde, poļu šļahta zaudēja politisko varu. Ja līdz Žežpospolitas sadalei jau sāka veidoties kopējs Vispolijas tirgus, tās sadale šo procesu pārtrauca, sarāva daudzus tradicionāli izveidojušos ekonomiskos sakarus starp atsevišķiem Polijas apgabaliem. Nepieciešamība veidot jaunus sakarus un pielāgoties Poliju sadalījušo valstu ekonomiskajām struktūrām nostādīja poļu apgabalus nevienlīdzīgā situācijā ar šo valstu teritorijām. Poļu apgabalu ekonomika nonāca atkarībā no minēto valstu politikas, to valdību centieniem ievērot vai neievērot (mazievērot) jauno pakļauto apgabalu intereses. Taču magnātu pozīcijas tika maz skartas. Jaunajos apstākļos, lielvaru aizsardzībā viņiem vairs nebija vajadzīgs sīkās šļahtas atbalsts. Pēdējā zaudēja iespēju saņemt no saviem labvēļiem līdzekļus par tiem sniegtajiem pakalpojumiem.

Austrijas varā pārgāja Galīcijas ziemeļu daļa līdz Bugas upei, ko sāka saukt par Rietumgalīciju, (atšķirībā no tās jau agrāk iegūtās Austrumgalīcijas ar Krakovu), daļa Mazovijas (\pltxti{Mazowsze}) un daži citi novadi ar 147 tūkstošu km$^{2}$ kopējo teritoriju un 1,2~miljoniem iedzīvotāju.

Prūsija ieguva zemes uz rietumiem no Piļicas (\pltxti{Pilica}), Vislas (\pltxti{Wisła}), Bugas (\pltxti{Boh}) un Nemunas (poļu \pltxti{Niemen}, lietuviešu \lttxti{Nemunas}, vācu \detxti{Memel}) upēm ar Varšavu, no kuras tās dumpīguma dēļ atteicās citas Polijas dalītājvalstis, kā arī Rietumlietuvas (Žemaitijas, poļu \pltxti{Żmudź}) zemes, kas kopumā sastādīja 55 tūkstošus km$^{2}$ ar 1~miljonu iedzīvotāju. Krievu vēsturnieks A.~Pogodins uzsvēra, ka ,,pašas vecākās, īsteni poļu zemes'' 3.~dalīšanā ieguva Prūsija.

Tātad, kad runā par XVIII gadsimtā notikušajām trijām Polijās dalīšanām, faktiski tiek runāts par Žečpospolitas~--- apvienotās Polijas un Lietuvas~--- dalīšanām.

Etnisko poļu apdzīvotās teritorijas faktiski savā starpā sadalīja Austrija ar Prūsiju.

Krievija saņēma lietuviešu (daļa lietuviešu apdzīvoto teritoriju ar Suvalkiem gan ieguva Prūsija), baltkrievu un ukraiņu zemes uz austrumiem no Bugas upes ar kopējo platību 120 tūkstošu km$^{2}$ un 1,2 miljoniem iedzīvotāju.

Tā Žečpospolitas triju dalīšanu rezultātā Krievijas rokās nonāca latviešu (Kurzeme), lietuviešu, baltkrievu (izņemot teritorijas daļu ar Belostoku (\pltxti{Białystok}), kas pārgāja Prūsijas īpašumā) un ukraiņu zemes (izņemot Austrumgalīcijas apgabalu ar Lembergu (Ļvovu), kurš palika Austrijas rokās).

Tiesa, Krievijai bija jāsamaksā smaga cena~--- jāļauj nostiprināties Prūsijai un Austrijai. Jautājums nebija par Polijas likteni. Kāpēc Katrīnai~II vajadzēja ņemt vērā Polijas intereses, ja pēdējā nevēlējās ņemt vērā Krievijas un Polijā dzīvojošo krievu, baltkrievu un ukraiņu intereses? Krievijai svarīgāks bija kas cits~--- pazuda buferis starp Krieviju un vācu valstīm. Austrija un Prūsija tagad atradās tieši pie Krievijas robežām. Bet alternatīva būtu tikai atteikšanās pievienot Krievijai radniecīgos baltkrievus un ukraiņus. Karot pret Prūsiju un Austriju, ko, iespējams, atbalstītu Anglija, lai saglabātu Poliju tās etnogrāfiskajās robežās, nu nekādi neatbilda Krievijas interesēm. Var teikt, ka XVIII gadsimtā Prūsijas un Austrijas agresīvā politika pavēra iespēju Krievijai atrisināt Rietumkrievijas jautājumu bez asiņaina kara ar Eiropas lielvalstīm.

Krievu sabiedriskā doma gan toreiz, gan vēlāk apsveica ieguvumus. Kā savās 1899.~gadā izdotajās lekcijās par Krievijas vēsturi rakstīja krievu vēsturnieks S.~Platonovs, ,,attiecībā pret Poliju Krievijas uzdevums bija atbrīvot pareizticīgos krievu iedzīvotājus no katolisko poļu valdīšanas, t.i.~--- atņemt Polijai vecās krievu zemes un no šīs puses sasniegt krievu tautības etnogrāfiskās robežas.'' Kā redzam, ,,krievu tautībai'' šeit tika pieskaitītas arī citas (baltkrievu un mazkrievu jeb ukraiņu) tautības.

Līdzīgs viedoklis bija arī ievērojamajam angļu vēsturniekam, kulturologam un sociologam A.~Toinbi. Viņš 1947.~gadā rakstīja: ,,Rietumos valda uzskats, ka Krievija ir agresors, \citespace{} XVIII gadsimtā Polijas dalīšanas laikā Krievija sagrāba teritorijas lielāko daļu: XIX gadsimtā tā ir Polijas apspiedējs \citespace{} Novērotājs no malas, ja tāds eksistētu, teiktu, ka krievu uzvaras pār zviedriem un poļiem ir tikai pretuzbrukums, \citespace{} XIV gadsimtā labākā daļa īsteno Krievijas teritoriju~--- gandrīz visa Baltkrievija un Ukraina~--- tika atrauta un pievienota rietumu kristietībai \citespace{} Poļu iekarotās īstenās krievu teritorijas \citespace{} tika atgrieztas Krievijai tikai 1930.--1945.~gada pasaules kara pēdējā fāzē.'' Citējot šo vērtējumu kā centienu panākt objektivitāti apliecinājumu, krievu literatūrzinātnieks un publicists V.~Kožinovs norādījis, ka Rietumu uzbrukums Krievijai sākās vēl agrāk, jau ar Polijas karaļa Boļeslava Drosmīgā iebrukumu Kijevā 1018.~gadā (par to runāts šī darba ievadā). V.~Kožinovs uzsvēris, ka nevar runāt par Krievijas dalību Polijas [kā poļu apdzīvotas zemes~--- V.Š.] sadalē. Patiesībā poļu zemes dalīja Austrija un Prūsija, Krievija sev pievienoja tikai senās austrumslāvu [neredzot lietuviešus un latviešus~--- V.Š.] apdzīvotās zemes, kuras arī mūsdienās ietilpst Baltkrievijas un Ukrainas sastāvā. Tāpēc V.~Kožinovs Krievijas dalību Polijas sadalīšanā nosaucis par ,,liberālu mītu.''

Arī mūsdienu krievu vēsturnieks O.~Ņemenskis 2012.~gada 21.~decembrī konferencē ''Baltkrievijas un Krievijas apvienošanās'', veltītā Žečpospolitas pirmajai sadalīšanai, uzsvēra: ''Krievija visos trijos Žečpospolitas dalīšanas posmos nesaņēma ne pēdas poļu zemes, nepārkāpa Polijas etnogrāfisko robežu. Krievijas līdzdalības ideoloģija [Žečpospolitas] dalīšanās sastāvēja tieši no iepriekš vienotās Krievzemes atkalapvienošanās''. Tādejādi pēc vēsturnieka domām apvainojumi Krievijai par dalību poļu zemju dalīšanā XVIII gadsimtā pēc būtības neatbilst patiesībai.

Taču ievērojamais krievu vēsturnieks S.~Kļučevskis savukārt atzīmējis, ka Krievija, tā vietā lai pievienotu sev Rietumkrieviju, piedalījās Polijas sadalīšanā, tādejādi ,,atrisinājums neatbilda uzdevumam''. Krievija sev pievienoja ne tikai Rietumkrieviju, bet arī Lietuvu un Kurzemi, taču daļu Rietumkrievijas~--- Galīciju atdeva Austrijai. Pēc S.~Kļučevska domām Polija nebija lieks loceklis Ziemeļaustrumu Eiropas valstu ģimenē, kalpojot par vāju starpnieku starp trijiem stipriem kaimiņiem. ,,Atbrīvojusies no to vājinošās Rietumkrievijas un pārveidojusi savu valsts iekārtu, kā centās tās labākie sadales laikmeta ļaudis, tā varētu sniegt lielisku pakalpojumu slāvu lietai un starptautiskajam līdzsvaram, kļūstot par balstu pret ar visiem spēkiem uz austrumiem tiecošos Prūsiju. Pēc Polijas krišanas sadursmes starp nosauktajām trijām valstīm nevājināja vairs nekāds starptautisks buferis un tām bija vissāpīgāk jāatsaucas uz Krieviju, kuras robeža pa Nemunu nekļuva drošāka tāpēc, ka tai kaimiņos atradās prūšu priekšposteņi \citespace{} Bez krievu apgabaliem, savās nacionālajās robežās, pat ar izlabotu valsts iekārtu patstāvīga Polija būtu mums nesalīdzināmi mazāk bīstama, nekā tā pati Polija austriešu un vācu provinču veidā. Visbeidzot, poļu valsts iznīcināšana neatsvabināja mūs no cīņas pret poļu tautu: nepagāja ne 70~gadu kopš trešās Polijas dalīšanas, bet Krievija jau trīs reizes karoja ar poļiem (1812., 1831. un 1863.~gg.). Žečpospolitas rēgs, ceļoties no vēsturiskā kapa, radīja dzīva tautas spēka iespaidu. Varbūt, lai izvairītos no kara pret tautu, vajadzēja saglabāt tās valsti.'' S.~Kļučevska secinājumos faktiski visdziļāk izvērtēti Krievijas politikas rezultāti attiecībā pret Poliju.

Situācija, kādā nonāca poļi, tai laikā nebūt nebija ārkārtēja. Daudzas pasaules tautas gan pirms, gan arī pēc tam nonāca daudznacionālu impēriju sastāvā. Šoreiz šis liktenis bija piemeklējis poļus. Jāpiemetina, ka ar Žečpospolitas sadalīšanu sašķelta tika ne tikai poļu tauta. Arī citu tajā dzīvojošo tautu pārstāvji (tai skaitā ap 800~000 ebreju, kam nebija savas kompaktas teritorijas), nonāca dalītājvalstu varā, kur viņu statuss un dzīves apstākļi arī bija dažādi.

Pretsparu veselas Eiropas tautas nomākšanai, tās teritorijas sadalīšanai trijās valstīs pārējā Eiropa nevēlējās dot. Laikabiedru vairākums attaisnoja notikušo ,,Eiropas kartes racionalizēšanu''.

Daudzi citu tautu vēsturnieki ir pauduši viedokli, ka poļi paši bija vainīgi savā nelaimē.

Viens no marksisma pamatlicējiem F.~Engelss tieši vainoja poļu feodālo aristokrātiju~--- magnātus savienībā ar trim lielvalstīm, kuras dalīja Žečpospolitu, lai tikai izbēgtu no revolūcijas, kā rezultātā tika sadalīta valsts, kas bija tikpat liela cik Francija. Domājams, ka bailes no revolūcijas, ko it kā izjuta magnāti, F.~Engelss pārspīlēja, jo viņi taču joprojām turpināja savstarpējās ķildas. Visi poļu magnāti vēlējās panākt stipras unitāras valsts eksistenci, taču tikai tai gadījumā, ja viņi paši tajā būtu noteicēji. Tādā situācijā visas reformas bija nolemtas neveiksmei.

Arī krievu vēsturnieki parasti uzsvēra Polijas sabrukuma iekšējos cēloņus, bet apskatot ārējo faktoru, galveno atbildību par Žečpospolitas sadali piedēvēja Prūsijai, turpretī Krievijas dalību tajās dažkārt pat raksturoja kā neatbilstošu tās nacionālajām interesēm. Tā, jau minētais krievu vēsturnieks A.~Pogodins rakstīja, ka Polijas valsts gāja bojā, jo tā nepaguva laikus nostāties uz tai laikā vienīgi pareizā monarhiskās varas un reizē ar to militārisma, rūpniecības un tirdzniecības attīstības ceļa.

Kopumā sākotnēji arī poļu vēsturnieku vidū lielākā uzmanība tika pievērsta valsts neatkarības zaudēšanas iekšējiem cēloņiem. Piemēram, autoritatīvais poļu vēsturnieks un valstsvīrs (viņš 1908.--1913.~gadā bija Galīcijas un Lodomērijas vietvaldis (\detxti{Statthalter})) M.~Bobržinskis, atzīmējot, ka nav bijis valsts Eiropā, pret kuru jaunajos laikos tās kaimiņi nebūtu turējuši ļaunus nodomus, uzsvēra, ka pat salīdzinoši mazas nācijas spēja no šīs cīņas iziet kā uzvarētājas, turpretī ,,poļi, kad viņus dalīja kaimiņi, nebija vājāki par katru no tiem ne pēc zemes platības, ne iedzīvotāju skaita, ne materiālās labklājības, ne garīgās attīstības, pat bija pārāki pār tiem tai vai citā jomā, taču tikai vieni poļi krita, pie tam bez cīņas, bez īstas cīņas, uz kādu tie bija spējīgi''. Atsevišķu personību gatavība nest upurus valsts labā ,,nevarēja aizstāt gatavības trūkumu sabiedrībā nest upurus''. Vēsturnieks secināja ,,Ne [neizdevīgās] robežas un ne kaimiņi, bet vienīgi iekšējās nesaskaņas noveda poļus pie politiskās pastāvēšanas zaudēšanas''.

Krievu vēsturnieks un filozofs N.~Karejevs 19.~gadsimta 80.~gadu beigās secināja, ka visi pētnieki ir vienoti~--- Polija gāja bojā iekšējo cēloņu rezultātā. ,,Diagnoze~--- likumdevējas varas bezspēcība un pilnīgs izpildvaras sajukums.''

Tomēr poļu vēsturnieku viedokļi XIX gadsimta beigās~--- XX gadsimta sākumā sadalījās divās skolās. T.s. Krakovas skolas pārstāvji uzskatīja, ka XVIII gadsimta beigās notikušās Žečpospolitas sagrūšanas cēloņi meklējami valsts iekšējā vājumā, otras~--- Varšavas skolas uzskatu aizstāvji tos saskatīja tai apstāklī, ka pēc 1772.~gadā notikušās dalīšanas valsts nonāca intensīvas modernizācijas fāzē, bet krita par upuri savu kaimiņvalstu~--- Krievijas, Prūsijas un Austrijas negausībai. Arī ievērojamais vēsturnieks no Lembergas O.~Balcers, kurš kā pirmais no poļu zinātniekiem 1921.~gadā tika apbalvots ar Baltā ērgļa (\pltxti{Orła Białego}) ordeni, apgalvoja, ka,, \dots{} īstais, izšķirošais mūsu valstiskuma pagrimuma cēlonis, īsts šī notikuma \latxti{causa efficiens} (izraisītājs) bija apvienoto, tātad vareno, kaimiņu alkatība, kuri noslēdza savienību, lai pazudinātu Poliju.''

Neviens no virzieniem tā arī neieguva izšķirošu pārsvaru. Vairākums nonāca pie secinājuma, ka Polijas dalīšanas izsauca kā iekšējo, tā ārējo faktoru mijiedarbība. Ir izteikts diezgan pārliecinošs viedoklis, ka atsevišķi par sevi ne vieni, ne otri nenovestu pie valstiskuma zaudēšanas.

Jau pieminētais vēsturnieks no Poznaņas J.~Rutkovskis rakstīja: ,,Polijas iekšējā iekārta bija pietiekami stipra, lai tā varētu realizēt savu pašas iekšējo politiku, taču bija par vāju, lai varētu aizstāvēties pret ārējo spiedienu.'' Viņš arī uzsvēra Žečpospolitas militāro vājumu, kuru lielā mērā noteica tās politiskais vājums, kad Seims nevēlējās stiprināt armiju, baidoties, ka karalis to izmantos absolūtās monarhijas radīšanai. Poļu vēsturnieks J.~Tazbirs, kurš uzsver, ka Poliju XVIII gadsimtā līdz bojā ejai noveda trūkumi valsts uzbūvē, nevis nacionāli netikumi, ir norādījis, ka poļu tauta pēc pirmās Žečpospolitas sadales ar reformām, 1791.~gada 3.~maija Konstitūcijas pieņemšanu, T.~Kostjuško vadīto sacelšanos pierādīja, ka tā ir pati spējīga labot savu valsti, cīnīties par to, līdz ar to apstrīdot tēzi, ka ,,mēs kritām mūsu trūkumu rezultātā.'' Jāpiekrīt, ka norisa cīņa par pretrunu, valsts trūkumu pārvarēšanu, bet reizē jāatzīmē, ka notikušās Žečpospolitas dalīšanas taču tieši rādīja, ka panākumi šai cīņā bija nepietiekami lai saglabātu valsti. Domājams, ka pārmetumus nav pelnījusi tauta, bet gan galvenokārt tās augšslāņi. Nosacījumus Polijas sadalīšanai radīja tās bezspēcība, tās valdošo slāņu nespēja kopot tautas spēkus vispārnacionālam uzdevumam~--- suverenitātes un teritoriālās integritātes aizstāvēšanai, tā vietā nodarbojoties ar cīņu par savu privilēģiju sargāšanu un paplašināšanu. Vienkāršie zemnieki neko neieguva nedz no karaļu nomaiņām, nedz varas pārdalēm, nedz arī Žečpospolitas sadalīšanas.

Mūsdienu vēsturnieku darbos autoram nav gadījies atrast Polijas valstiskuma iznīcināšanu attaisnojošus uzskatus, bet domstarpības pastāv par tās novērtējumu XVIII gadsimta notikumu kontekstā, par citu valstu tiesībām uz tām vai citām teritorijām.

Prūsija no bijušajām poļu zemēm izveidoja trīs provinces: Rietumprūsiju, Dienvidprūsiju un Jauno Austrumprūsiju. Prūsijas poļu valdījumos sākās poļu asimilācija. Daudzas valsts un baznīcas muižas nonāca prūšu muižnieku rokās. Prūsijai piederošajā Polijas daļā jau no 1776.~gada darbojās likums, kurš šļahtiču zemes atļāva iepirkt ne tikai vācu muižniekiem, bet arī pilsētniekiem. Ar dažādām privilēģijām poļu zemēm tika piesaistīti vācu zemnieki, amatnieki un tirgotāji. Poļu ciemu vidū veidojās vācu zemnieku ciemi. No 1797.~gada visa tiesvedība un administratīvā lietvedība notika vācu valodā, poļi tika izraidīti no valsts dienesta, nodibināja vācu skolas. Poļiem tika uzspiesta visas sabiedriskās dzīves reglamentācija. Poļu vēsturnieks J.~Feldmans uzsvēra, ka Prūsija uz Polijas rēķina palielinājās vairāk nekā divas reizes un kļuva par ,,modru Polijas jautājuma kapraci'' Eiropas politikā, par sīkstāko poliskuma apkarotāju agrākās Žečpospolitas teritorijā.

Zemes, kuras nonāca Austrijā, ieguva nosaukumu Galīcija un Lodomērija (vācu \detxti{Königreich Galizien und Lodomerien}, poļu \pltxti{Królestwo Galicji i Lodomerii}, kur \latxti{Lodomeria} bija latīņu valodā lietots Galīcijas un Vladimiras kņazistes nosaukums XIII--XIV gs.) un tika sadalītas 12 apvidos. 1806.~gadā Austrija no Galīcijas guva 19\% visu valsts ienākumu. Arī Galīcijā sākās poļu asimilācijas mēģinājumi, gan ne tik enerģiski kā Prūsijas zemēs. 1790.~gadā Vīnes augstmaņu slepenā memorandā sakarā ar visai biklu Konstitūcijas projektu, kuru ierosināja provinces seims un kuru noraidīja imperators Leopolds~II, bez aplinkiem bija norādīts, ka Austrijas mērķim jābūt ,,galīciešu pakāpeniskai pārvēršanai par vāciešiem''.

Krievijas rokās nonākušās teritorijas tika sadalītas Kurzemes, Viļņas un Grodņas guberņās. Saglabājās kārtu pašpārvalde, poļu muižnieku kundzība pār zemniekiem nebija satricināta. Uz laiku saglabājās vietējie likumi, arī Lietuvas statuta (\lttxti{Lietuvos Statutas}~--- likumu kodekss bija radīts XVI~gs. un civillietās palika spēkā daļā no bijušās Lietuvas lielkņazistes teritorijas līdz 1840.~gadam) nozīme. Krievijas dalība Polijas sadalē uz diviem turpmākajiem gadsimtiem lielā mērā noteica Krievijas ārējo politiku, jo starptautiskā stabilitāte bija lielā mērā atkarīga no valstu~--- triju dalīšanas dalībnieču attiecībām.

Dažādi Polijas apgabali vēl līdz tās dalīšanai attīstījās nevienmērīgi, tas bija redzams arī tikai etnisko poļu apdzīvotajās zemēs. Taču šis nevienmērīgums, kā uzskatīja poļu ekonomikas vēsturnieks V.~Kula, arī stiprināja valsts vienotību, radot priekšnoteikumus darba dalīšanai starp atsevišķām zemēm. Pēc Polijas sadales ekonomiskās attīstības nevienmērīgums pieauga. Tā, t.s. Lielpolija~--- ekonomiski attīstītākais Žečpospolitas apgabals XVIII gadsimta beigās, kurš acīmredzot saglabātu šo lomu neatkarīgā valstī, pēc tās sadales kļuva par Vācijas impērijas ,,aizmuguri'' (vācu \detxti{Hinterland}~--- zemi, kuru raksturoja mazāks iedzīvotāju blīvums un sliktāka ekonomiskā un infrastruktūras attīstība kā blakus esošajos attīstītākajos apgabalos), bet Austrumgalīcija kļuva par ekonomiski atpalikušāko no visām poļu provincēm. Ar sadali tika sarauti vai vismaz apgrūtināti veidojamie ekonomiskie sakari starp dažādām poļu teritorijām. Tiesa, vēlreiz jāatgādina, ka ne tikai pirms 1795., bet arī pirms 1772.~gada ne visas poļu apdzīvotās teritorijas atradās vienas valsts sastāvā. Ārpus Žečpospolitas palika tādas zemes, uz kurām tā pretendēja, kā jau pieminētās Silēzijas (poļu \pltxti{Śląsk}, vācu \detxti{Schlesien}) un Rietumu Pomorjes jeb Pomerānijas (poļu \pltxti{Pòmòrzé}, vācu \detxti{Pommern}) teritorijas. Reizē jāuzsver, ka dalīšanas, sagraujot vienus sakarus, veicināja citus. Piemēram, sakarus starp Pozenes un Silēzijas apgabaliem. Tie pat stiprināja poļu elementus Silēzijā, kas guva izpausmi jau XIX gadsimta pirmajā pusē.

Par vienu no poļu zemju attīstības nevienmērības rādītājiem var kalpot iedzīvotāju pieauguma dinamika. Tā, 1816.--1856.~gadā iedzīvotāju skaits gandrīz dubultojās Prūsijas rokās esošajā Pomerānijā un Augšsilēzijā (vācu \detxti{Oberschlesien}, poļu \pltxti{Górny Śląsk}), par 73\% pieauga Pozenes hercigistē, par 36\% Polijas karalistē un tikai nedaudz vairāk par 20\% Austrijai piederošajā Galīcijā. V.~Kula uzsvēris, ka Polijas sadale paildzināja feodālisma pastāvēšanu tās zemēs. Vājā poļu feodālā valsts, kāda bija Žečpospolita savas pastāvēšanas pēdējos gados, nespējusi feodāļiem garantēt viņu privilēģijas revolucionāras situācijas priekšā, tāpēc tie bija ieinteresēti iegūt absolūtisko kaimiņvalstu aizbildniecību, kuras tad arī nodrošināja to privilēģiju pastāvēšanu vēl vairāk nekā pusgadsimta garumā. Taču šis spriedums nevar būt viennozīmīgs. Pats V.~Kula bija spiests atzīt, ka Žečpospolitas sadale neapturēja kapitālisma elementu nobriešanu feodālisma iekšienē un ja arī to aizturēja, tad tikai uz īsu laiku, ka kapitālisma attīstība daudzās nozarēs un teritorijās pēc Žečpospolitas sadales pat paātrinājās. Viņš arī norādījis, ka reģionālās ekonomikas iezīmes lielā mērā veidojās daudzu sociāli-ekonomisku faktoru ietekmē, starp kuriem politiskajām robežām nebija galvenā loma. Tā, vēlāk Krievijas atkarībā pastāvošajā Polijas karalistē (1815--1915) esošais Kališas (poļu \pltxti{Kalisz}, vācu \detxti{Kalisch}) rajons ekonomiskā ziņā maz atšķīrās no Pozenes provinces. Dombrovas baseins karalistē faktiski veidoja vienu ekonomisku rajonu ar Augšsilēziju Prūsijā, bet Austrumgalīcijai ar Lembergu bijis mazāk kopīgā ar Rietumgalīciju ap Krakovu, kas abas atradās Austrijas sastāvā, nekā starp Rietumgalīciju un tai Polijas karalistes pusē pieguļošo Mehovas (poļu \pltxti{Miechów}) rajonu. Ar kapitālisma attīstību pieauga reģionālās atšķirības, kas nesakrita ar politiskajām robežām. V.~Kula arī norādījis, ka nevis Žečpospolitas sadale, bet kapitālisma attīstība tās sadalītajās daļās noveda pie Polijas ,,A'' (ekonomiski attīstītāko rajonu) un Polijas ,,B'' (ekonomiski vājāko, atpaliekošo rajonu'') izveides. Pat vienā pašā Polijas karalistē bija novērojams šāds dalījums. Karalistē abos Vislas krastos pastāvēja dažādas Polijas, kur atšķirības bija lielākas nekā starp Pozenes provinci Prūsijā un Kališas rajonu Polijas karalistē.

\asterism


Taču \strong{Žečpospolitas valstiskuma likvidācija} notika apstākļos, kad poļu nācijas konsolidācijas process jau bija izvērsies, tāpēc tā sastapa stipru tautas pretestību, \strong{izsauca poļu nacionālo kustību}. Poļu emigrācija atgādināja pasaulei par Polijas pastāvēšanu. No šī laika starptautiskajās attiecībās sāka pastāvēt ,,poļu jautājums'', kuru ar Polijas dalītājām konkurējošās valstis centās izmantot savās interesēs.

Kaut Polijas valsts pārstāja pastāvēt, daudzi poļi, īpaši šļahtiči, neatmeta cerības atjaunot neatkarību, kaut dažādie valsts atjaunošanas plāni visbiežāk izrādījās nereāli. Tieši \strong{šļahtiči nostājās nacionālās atbrīvošanās kustības priekšgalā}. Polijas īpatnība bija šļahtiču lielais īpatsvars~--- ap 6 līdz pat 10\% no visiem iedzīvotājiem. Tiesa, par šļahtičiem Žežpospolitā par īpašiem nopelniem varēja kļūt arī zemāko kārtu pārstāvji, taču tas notika reti, jo jautājumu izlēma Seims. Tā, ir dati, ka 1788.--1792.~gadā, kad Polijā darbojās jau minētais Četrgadu Seims (\pltxti{Sejm Czteroletni}), par šļahtičiem kļuva ap 400 cilvēku, tikpat cik visos iepriekšējos XVIII gadsimta gados.

Tomēr bija arī ceļi, kā apiet likumu. Viens no līdzekļiem, kā pierādīt savu piederību šļahtiči kārtai, bija 12 liecinieku uzrādīšana. Bagātam plebejam vispirms bija jāaiziet pie bārddziņa un jālūdz iegriezt pāris brūču, kuras nebūtu bīstamas, bet radītu skaidri saskatāmas rētas~--- jo, kas gan par šļahtiču bez kaujās gūtām rētām! Pēc tam bija jānoalgo kāds, kas apvainotu šļahtiča statusa pretendentu, ka viņš nav šļahtičs, un tam pretī jānostāda 12 liecinieki, kuri šo apgalvojumu noliegtu. Tā ar tiesas spriedumu varēja iekļūt priviliģēto kārtā. Bez tam Lietuvas lielkņazistē līdz 1764.~gadam darbojās likums, pēc kura katrs ebrejs, kurš pārgāja kristīgajā ticībā, uzreiz kļuva par šļahtiču. Tas bija apbalvojums par ticības maiņu. Bija arī gadījumi, kad plebeji pārgāja judaismā, lai pēc dažiem gadiem atgrieztos katolicismā un tā iegūtu šļahtiča ģerboni. Tāpēc arī minētais likums galu galā tika atcelts.

Kopējā šļahtiču masā zemes īpašnieki sastādīja mazākumu. Pēc Žečpospolitas pirmās sadales tādu muižnieku bija vairāk nekā 300~000, bet t.s. sīko šļahtiču, kuru īpašumā bija tikai daļa ciema vai vispār nebija sava īpašuma~--- vairāk nekā 400~000. Pēc poļu vēsturnieka T.~Ļepkovska aprēķiniem XVIII~gadsimta beigās bezzemes šļahta sastādīja 55\%, bet 40\% šļahtiču īpašumā nebija dzimtcilvēku. Lielie zemes īpašnieki~--- magnāti, kuru īpašumā bija vairāki simti ciemu, dzīvoja galvenokārt valsts austrumos, tai skaitā zemēs, kur poļi bija etnisks mazākums. Viduspolijā par magnātu jau skaitījās šļahtičs, kura īpašumā bija vairāki desmiti ciemu. Sīko šļahtiču īpašumā parasti bija muiža ar dažiem klaušu zemniekiem. Viņi parasti algoja strādniekus, jo savu klaušinieku nepietika. Šādu šļahtiču dēli bieži nodarbojās ar tirdzniecību (1775.~gadā tika atcelts aizliegums šļahtičiem nodarboties ar amatniecību un tirdzniecību), kļuva par garīdzniekiem, dienēja pie magnātiem, pārejot bezīpašuma šļahtiču statusā. Daļa šļahtiču magnātu dienestā līdz vecumam nopelnīja pensiju, bet vairākums cieta trūkumu, atšķiroties no zemniekiem tikai ar savām iluzorajām pilsoniskajām un politiskajām tiesībām. Nabadzīgākie bezīpašuma šļahtiči rentēja zemes gabalus, tai skaitā arī no zemniekiem, kurus arī paši apstrādāja. Īpaši Mazovijas (poļu \pltxti{Mazowsze}, vēsturisks apgabals Polijas centrā) un Podlases (poļu \pltxti{Роdlаsiе}~--- no poļu ,,\pltxti{Pod lasem}'' (zem meža)~--- vēsturisks apgabals tagadējās Polijas austrumos) novados varēja redzēt cilvēku ar zobenu pie sāniem, kurš ara zemi. Taču šos arājus-šļahtičus nekad neatstāja pārākuma apziņa pār saviem kaimiņiem dzimtcilvēkiem.

Jau XVIII gadsimta otrajā pusē ar magnātu karadraudžu likvidāciju magnātu aizbildniecībā esošā t.s. šļahtiču klientūra strauji mazinājās, tā vairs magnātiem nebija vajadzīga. Ar neatkarības zaudēšanu un ar kapitālistisko attiecību attīstību sīkā šļahta izrādījās vairs nevajadzīga savā agrākajā~--- karotāju veidolā. Pāreja uz činšu (renti) padarīja nevajadzīgus šļahtiču amatus magnātu saimniecībās. Arvien vairāk šļahtiču pārcēlās uz pilsētām. Ja turīgākie pilsētu šļahtiči papildināja augstāko ierēdņu, advokātu un brīvo profesiju pārstāvju rindas, tad nabadzīgākie līga darbā par apsargiem, mājkalpotājiem, reizēm kļūstot arī par klaidoņiem.

Angļu vēsturnieks N.~Deiviss pat uzskata, ka juridiskā ziņā Polijas aristokrātijai pienāca gals, kad pēc Žečpospolitas sadalīšanas tika anulēti likumi, kas noteica tās statusu. Taču N.~Deiviss nav precīzs. Protams, magnātu un šļahtas ietekme citām valstīm pievienotajās poļu teritorijās mazinājās, taču pilnībā neizzuda.

Daudzas magnātu muižas tika pārdotas, lai nomaksātu parādus. To valstu valdības, kuras sadalīja Poliju, pārņēma savā īpašumā vairākumu valsts un baznīcas zemju (tā atņemot iztiku daudziem šļahtičiem~--- karaļa zemju nomniekiem), bet par dalību 1794.~gada sacelšanās konfiscētās zemes sadalīja savas valsts augstmaņiem. Administratīvā pārvalde un tiesa nonāca nepoļu ierēdņu rokās. Prūsijā šļahta zaudēja personas neaizskaramības tiesības. Austrija arī atcēla šļahtiču personīgās brīvības un to īpašuma neaizskaramības tiesības, privilēģiju būt atbrīvotiem no nodokļiem. Kā raksta jau minētais poļu vēsturnieks J.~Tazbirs, XVIII gadsimta pirmajā pusē valdošais valsts aparāta vājums, nespēja iekasēt nodokļus nāca par labu šļahtai, kura iedzīvojās uz valsts rēķina. Kad Polija tika sadalīta, šļahtiči uzreiz izjuta, ka okupanti nodokļus prot ievākt daudz labāk nekā Žečpospolita. Lielā mērā šis apstāklis bijis pamatā vienam no šļahtas kultivētajiem mītiem~--- ka Žečpospolitas valsts iekārta bijusi gandrīz vai ideāla. Īpašu sašutumu šļahtā viesa tas, ka poļu zemes sadalījušo valstu aparāts iejaucās muižnieku un zemnieku attiecībās.

Taču šo triju lielvalstu, kuras pašas bija feodālas kārtu monarhijas, politika nevarēja būt radikāla. Tās poļu zemēs saglabāja feodālās attiecības un kārtu struktūru. Daļai poļu magnātu savu dižciltību izdevās apstiprināt Austrijā un Prūsijā. Īpaši Austrijas imperators labprāt apmaiņā pret likvidējamajiem senatoru, vojevodu, kastelānu (\pltxti{Kasztelan}, no latīņu \latxti{castellum}~--- pils, XVIII gadsimtā tie dienesta hierarhijā ieņēma otro vietu pēc vojevodām) u.c. amatiem piešķīra poļu lielmuižniekiem savulaik Žečpospolitā mazizplatītos grāfu un baronu titulus. Vairākums poļu grāfu un baronu ģimeņu savus titulus ieguva Vīnē un Berlīnē pirmajā desmitgadē pēc Žečpospolitas krišanas. Poļu šļahta zaudēja daudzas savas tiesības, taču saglabājās kā kārta.

Pēc spāņu izcelsmes angļu vēsturnieces I.~De~Madariagas datiem Krievijas muižnieku (krievu \rutxti{дворяне}) kopskaits 1795.~gadā sastādīja ap 111~600. Pēc Polijas sadales tiem pievienojās 250~974 šļahtiču, kas Krievijas impērijā sastādīja 66,22\% no visiem muižniekiem. Tātad Krievijai piederošajā Polijas daļā šļahtiču bija vairāk nekā muižnieku visā pārējā Krievijas impērijā. Līdz 1830.~gadam Polijas karalistē saglabājās bagāto zemes īpašnieku pārsvars sabiedriskajā un ierēdņu hierarhijā. Tika apstiprināti prūšu un austriešu piešķirtie tituli, tika dāvāti arī jauni. Tā kā šļahtiču īpatsvars bija salīdzinoši liels, nav brīnums, ka viņu uzskati, tikumi lielā mērā ietekmēja veidojošos poļu nāciju. Šļahtas psiholoģiskais veidols un tās tradīcijas arī turpmāk atstāja lielu iespaidu poļu kultūras attīstībā, kad no to rindām nākusī inteliģences daļa saistīja sevi ar buržuāzijas ekonomiskajām un sabiedriskajām interesēm.

Nedaudz apsteidzot notikumus, jāsaka, ka 1817.~gadā Krievijai piederošajā Polijas daļā tika izdots likums, kurš, tāpat kā to paredzēja Krievijas impērijā spēkā esošā vēl 1722.~gadā Pētera I ieviestā Rangu tabula, deva iespējas par nopelniem valsts labā iegūt muižnieka (šļahtiča) tiesības. Katrs 10 gadus nokalpojis skolotājs, katrs karavīrs, kurš uzdienēja līdz kapteiņa pakāpei, katrs, kam bija ievērojami nopelni valsts priekšā, ieguva šļahtiča tiesības. Avīze ,,\pltxti{Dziennik Praw}'' (,,Likumu Dienasgrāmata'') publicēja šo cilvēku sarakstus. Tas veda pie šļahtas stāvokļa pakāpeniskas nonivelēšanas, ar ko ,,īstie'' šļahtiči bija neapmierināti. Vēl krasāk stāvoklis mainījās pēc 1830.~gada sacelšanās. 1836.~gadā Polijas karalistē tika izdots dekrēts par muižniecību, kura mērķis bija radīt jaunu, saistītu ar carismu muižnieku kārtu. Mantojamās muižnieku (šļahtas) tiesības bija likumīgi jāapstiprina: dokumentāli jāpierāda, ka kāda no vīriešu kārtas senčiem īpašumā līdz 1775.~gadam bija vismaz viens ciems, vai līdz 1795.~gadam viņš bija Seima deputāts, senators, valsts darbinieks, poļu ordeņa kavalieris. Poļu šļahta saglabāja zināmas privilēģijas (kara dienestā, vidējās un augstākās izglītības ieguvē), taču zaudēja monopolu uz zemes īpašumu, nodokļu atlaidēm, atbrīvošanu no kara klausības. Līdz 1861.~gadam, kad tika beigts izskatīt jautājumu par šļahtiču tiesību apstiprināšanu, kārtas privilēģijas bija zaudējuši apmēram ¼ dzimušo šļahtiču. Polijas karalistē XIX gadsimta vidū tikai ap 5~000 šļahtiču ģimeņu piederēja muižas. Saprotams, ka jau tas vien radīja pamatu cietušo neapmierinātībai.

Tiesa, arī turpmāk Krievijas muižniecība bija daudznacionāla. Pēc 1897.~gada skaitīšanas datiem tikai 53\% dzimtmuižnieku par savu dzimto nosauca krievu valodu. 28,3\% no viņiem sevi uzskatīja par poļiem. Tātad vēl XIX gadsimta beigās vairāk nekā ceturtdaļa Krievijas impērijas muižnieku bija poļi. Tomēr poļu šļahtiču īpatsvars Krievijas muižnieku vidū bija ievērojami mazinājies. Daļa bagātās šļahtas~--- zemes īpašnieki, uzkrājot kapitālu, paplašinot preču ražošanu, izmantojot algotu darbaspēku, kļuva par kapitālistiskiem uzņēmējiem, saglabājot arī sociālo prestižu un sabiedrisko ietekmi. Bezīpašuma šļahta piegādāja zemākā un vidējā līmeņa ierēdņus, tās pārstāvji ieņēma daudz virsnieku posteņu armijā, taču tikai nelielai tās daļai bija vēlēšanu tiesības vietējās pašvaldības iestādēs. Sīko un vidējo muižu īpašnieki, bet vēl jo vairāk zemes īpašumus zaudējušie šļahtiči asi izjuta briesmas, kuras viņu privilēģijām nesa kapitālisma attīstība. Liela daļa no viņiem zaudēja savu mantu, to stāvoklis tuvinājās valsts zemnieku statusam. Daļa bijušo šļahtiču veidoja deklasētu ļaužu grupu. (Piemēram, bija gadījumi, ka nabadzībā nonācis poļu šļahtičs~--- dzimtmuižnieks 19.~gadsimta beigās Rīgā strādāja par amatnieku vai strādnieku.)

Šļahta ,,bija galvenā pretkrieviskā noskaņojuma barotne visa XIX gadsimta laikā'', kā to raksturojis N.~Deiviss. Šļahtas centieni panākt tādu valsts iekārtu, kuras attīstību viņi varētu ietekmēt sev labvēlīgā virzienā, galvenokārt saistījās ar Polijas valsts neatkarības atjaunošanu, pie tam atjaunošanu agrākajās robežas, kad poļu feodāļi ekspluatēja arī citu tautu zemniekus. Šļahtiči un viņu sekotāji vēl XX gadsimtā par katru cenu centās atjaunot Poliju 1772.~gada robežās ,,no jūras līdz jūrai'' (\pltxti{Od morza do morza}), tātad~--- vienā valstī apvienojot gan poļu, gan lietuviešu, gan ukraiņu un baltkrievu zemes. Sociālās priekšrocības, kuras šī kārta izmantoja gadsimtu gaitā (izglītība, prestižs sabiedrībā), ļāva tai ieņemt vadošo vietu poļu nacionālajā kustībā.

Tomēr jāuzsver, ka šļahtiču politiskie uzskati bija daudzveidīgi: dažas grupas ieņēma klerikālas, feodālas pozīcijas, citas, ,,maksājot nodevas'' romantismam un lielvalsts atjaunošanas ideālam, pēc uzskatiem tuvinājās buržuāziskajiem revolucionāriem, izvirzīja arī progresīvas prasības par zemnieku stāvokļa uzlabošanu, to feodālās atkarības likvidēšanu, kārtu privilēģiju likvidēšanu, vēlēšanu tiesību paplašināšanu un citu buržuāzisko politisko brīvību ieviešanu.

Par poļu šļahtičiem dažādos laikmetos ir izteikti visdažādākie viedokļi, no cildinošiem līdz paļājošiem. Piemēram, kritiski pret tiem bija noskaņots krievu publicists un beletrists F.~Bulgarins, no mātes puses polis, viņa tēvs bija karojis poļu pusē pret Krieviju un tāpēc izsūtīts uz Sibīriju. Pats viņš bija dienējis gan Krievijas armijā, gan poļu leģionos Napoleona armijā, bet pēc tam bija ierēdnis un nodarbojās ar literāru darbību Pēterburgā. F.~Bulgarins rakstīja: ,,Polijā no laika gala mēļoja par brīvību un vienlīdzību, kuru īstenībā nebija nevienam, tikai bagātie pani bija pilnīgi neatkarīgi no visām varām, taču tā nebija brīvība, bet patvaļa \citespace{} Sīkā šļahta, nevaldāma un neizglītota, vienmēr atradās pilnīgā atkarībā no katra, kurš to baroja un dzirdināja, un pat stājās viszemākajos amatos pie paniem un bagātās šļahtas, un pacietīgi pacieta kāvienus,~--- ar noteikumu, lai tas notiktu ne uz kailas zemes, bet uz paklāja, taču muļķīgas lepnības dēļ nicināja nodarbošanos ar tirdzniecību un amatiem kā neatbilstošu šļahtiča nosaukumam.''

Nepakļāvīgo šļahtas attieksmi pret iekarotājiem, to monarhiem lielā mērā balstīja tās senās tradīcijas. Pat zaudējuši tiesības uz šļahtiča nosaukumu, daudzi no tiem turpināja cienīt savu ģerboni. Ilgstoši saglabājās šļahtas prestižs sabiedrībā. Pat sīkie šļahtiči, kuriem vidējās izglītības un ierēdņa vietas ieguve pilsētā nozīmēja jau lielu personīgu panākumu, centās norobežoties no nedižciltīgajiem pilsētniekiem. ,,Īsts'' šļahtičs karali uzskatīja par sev līdzvērtīgu, nevis augstākstāvošu personu, aizstāvēja savas \latxti{veto} tiesības, bija gatavs uz nepakļaušanos līdz pat atklātam dumpim (\pltxti{rokosz}) savu tiesību un brīvību aizsardzības vārdā. Politiska opozīcija esošajai varai tika uzskatīta par cienījamu tikumu, pilsonisko pienākumu. Faktiski princips, ka vara valdniekam nav Dieva dota, bet viņš to saņēmis no vēlētājiem, par kuru it kā iestājās šļahta, XIX gadsimta cīņās pret nacionālajiem apspiedējiem, neliecināja par šļahtas progresivitāti, bet par tradicionālismu, jo kā pilntiesīgi valdnieka vēlētāji tika atzīti tikai paši šļahtiči. Tas arī bija viens no galvenajiem šļahtas bezspēcības cēloņiem XIX gadsimta brīvības cīņās. Kā rakstīja ievērojamais krievu vēsturnieks S.~Solovjovs: ,,Valstisko un sabiedrisko aizturu iztrūkums, sava spēka apzināšanās, savas pilntiesības un neatkarības vienreizīguma apzināšanās bija par pamatu galējai personības attīstībai poļu šļahtā, tieksmei pēc neierobežotas brīvības, nemācēšanai ar savu es piekāpties vispārēja labuma priekšā.''

Pēc Polijas sadales daļa muižnieku, labi saprotot, ka ārējā vara nodrošina viņiem pastāvošās sabiedriskās iekārtas un privilēģiju saglabāšanu, piekopa samierniecisku politiku. Taču daudzi magnāti un ievērojama daļa šļahtas piedalījās atbrīvošanās kustībā. Parasti viņus vadīja ne tikai patriotisma jūtas, bet arī personīgās intereses. Tomēr, runājot par atbrīvošanās kustību Polijā, V.~Ļeņins līdz XIX gadsimta 60.~gadiem to sauca par ,,šļahtas atbrīvošanās kustību'', norādot, ka tā ,,ieguva milzīgu, pirmšķirīgu nozīmi ne vien no visas Krievijas, ne vien no visu slāvu, bet arī no visas Eiropas demokrātijas viedokļa.''

Katra jauna šļahtiču paaudze (vismaz daļa tās) mēģināja atbrīvoties no apspiedējiem. Vieni cerēja kooperēties ar Krieviju vai Prūsiju, lai ar to palīdzību atjaunotu Polijas vienotību. Piemēram, poļu ģenerālis J.H.~Dombrovskis, kurš jau bija karojis T.~Kostjuško vadībā, 1796.~gadā piedāvāja Prūsijas karalim Fridriham Vilhelmam II poļu karavīrus nostādīt Prūsijas pavēlniecībā. Taču karalis par to neizrādīja nekādu interesi. Citi poļu šļahtiči vai nu pievienojās to valstu, kuras sadalīja Poliju, pretiniekiem, vai paši rīkoja sacelšanās. Pēc 1794.~gada daudzi poļi devās uz Itāliju un Franciju, viņu apmešanās centri bija Venēcija un Parīze. Īpašas cerības tika saistītas ar Franciju, kura veda nacionālus revolucionārus karus pret kontrrevolucionāro monarhiju koalīciju. Izveidojoties Napoleona I vadītajai Francijas impērijai, tā pakļāva virkni Eiropas nacionālo valstu, veda imperiālistiskus iekarošanas karus, kas savukārt izraisīja pret to nacionālus atbrīvošanās karus. Taču to daudzi poļu patrioti, norūpējušies tikai par savas dzimtenes brīvību, vairs nevēlējās saskatīt.

Vīlies Prūsijā, 1796.~gada oktobrī ģenerālis J.~H.~Dombrovskis piedāvāja Francijas Direktorijai (valdībai, 1795--1799) organizēt \strong{poļu leģionu}. Nākamā gada sākumā tika izveidoti divi tādi Francijas vasaļvalsts Cisalpīnas Republikas (itāļu \latxti{Repubblica Cisalpina}) Napoleona komandētās armijas sastāvā. Tajos galvenokārt dienēja poļi (1797.~gadā~--- ap 6~000), kuri bija dienējuši Austrijas armijā un saņemti gūstā vai arī bija dezertējuši no tās. Poļu leģionāri savu cīņu jēgu saskatīja Polijas valstiskuma atjaunošanā ar Francijas palīdzību. J.~H.~Dombrovskis izteicās: ,,Kā uzvarētāji mēs uzcelsim no jauna mūsu Tēvzemi''. Leģioni kļuva par poļu nacionālisma audzināšanas skolu. Poļu leģionu karavīri nēsāja poļu mundierus ar franču kokardēm. J.~Dombrovskis ierosināja Napoleonam nosūtīt leģionus caur Turcijai pakļautajām teritorijām Balkānos uz Austrijai piederošo Galīciju, kur tika cerēts izraisīt poļu šļahtas sacelšanos. Tiesa, drīz (1797.~gada aprīlī) tika noslēgts Francijas un Austrijas miera līgums, kurš leģionāriem nāca pilnīgi negaidīts. Tas izsauca dziļu vilšanos poļu emigrantu vidū Francijā. (Daži pat devās uz Krieviju un iestājās tās armijā.) Tika kaldināti arī citi plāni kā atjaunot Polijas neatkarību. Tomēr lielākā daļa leģionāru naivi ticēja, ka Francija agri vai vēlu palīdzēs īstenot sapni par Polijas brīvību. Ne bez pamata ir krievu vēsturnieku izteiktā doma, ka poļu emigrācijas vadībai leģioni bija vajadzīgi nevis kā bāze nākamo Polijas bruņoto spēku veidošanai, bet kā ,,avanss'' Francijai, lai pierādītu, cik svarīgi tai būtu atjaunot Poliju kā savu sabiedroto Eiropā.

Starp citu, poļu publicists un sabiedriskais darbinieks J.~Vibickis, lai iedvesmotu leģionārus, 1797.~gadā sacerēja vārdus plaši pazīstamajai ,,Dombrovska mazurkai'' (,,\pltxti{Mazurek Dąbrowskiego}''), sauktai arī par Dombrovska maršu, (Pirmais nosaukums gan bija ,,Poļu leģionāru Itālijā dziesma''~--- ,,\pltxti{Pieśń Legionów Polskich we Włoszech''}) Dziesmas pirmais pants un piedziedājums skanēja:

\vspace{1.5em}

\noindent
\begin{minipage}{0.45\textwidth}
\pltxti{
Jescze Polska nie zginela,\\
kiedu my źyjemu!\\
Co nam obca przemosc wzęeła,\\
szabla odbierzemy.\\
Marsz, marsz, Dąbrowski,\\
z ziemi wloskiej do poliski,\\
za twoim przewodem\\
złązcym się z narodem!}
\end{minipage}
\hspace{2em}
\begin{minipage}{0.45\textwidth}
Polija nav zudusi,\\
kamēr mēs vēl dzīvi!\\
Ko mums sveša vara ņēma,\\
Zobeniem atgūsim.\\
Marš, marš, Dombrovski,\\
no Itālijas uz Poliju,\\
mēs tavā vadībā\\
ar tautu vienosimies!
\end{minipage}

\vspace{1.5em}


Dziesmu izpildīja ar tautas mūziku mazurkas ritmā. Šī dziesma visur pavadīja poļu leģionārus, to viņi dziedāja, kad ar Napoleona armiju atgriezās dzimtenē, vēlāk slepenās sazvērnieku sapulcēs, 1831.~gadā tā kļuva par poļu sacelšanās himnu. Neraugoties uz to, ka poļu nebrīves gados tā tika aizliegta, kā aicinoša uz dumpi, viņi to turpināja dziedāt. 1926.~gadā ,,Dombrovska mazurka'' kļuva par oficiālo Polijas valsts himnu.

Otrā pasaules kara laikā PSRS izveidotās T.~Kostjuško vārdā nosauktās poļu divīzijas karavīri himnu dziedāja bez pantiem, kuros bija rindas ,,vācietis un krievs neizturēs, kad zobenu rokā ņemsim'' un ,,visi vienā balsī runā, pietiks mums šīs nebrīves''. Taču mazurkā bija arī pants par poļu karavadoni S.~Čerņecki, kurš kļuva pazīstams ne tikai ar savu varonību, bet arī ar savu nežēlību, apspiežot pret poļu varu vērstās sacelšanās Ukrainā. Viņš arī apgānīja ukraiņu nacionālā varoņa B.~Hmeļņicka kapu un mirstīgās atliekas.

Pēc Otrā pasaules kara 1950.~gadā Polijas tautas republikā (PTR) kā valsts himna tika publicēti tikai pirmie divi panti, bet 1952. un 1953.~gadā tikai 1.~pants ar piedziedājumu. Izskanēja priekšlikums sarīkot konkursu jaunai himnai, taču tas tā arī netika realizēts. Mūsdienās himna tiek dziedāta bez ,,vācieša un krieva'' pieminēšanas. Toties pants par S.~Čerņecki ir palicis.

Poļu leģionāri demonstrēja izcilu varonību kaujās, taču viņu liktenis arī parādīja, cik nesavienojama ir cīņa par brīvību ar kalpošanu iekarošanas politikai. 1798.~gadā pie diviem esošajiem leģioniem tika pievienots arī trešais, kurš nesa ,,Donavas'' leģiona nosaukumu, kaut tika izvietots pie Reinas. Leģionāri, ciešot ievērojamus zaudējumus, piedalījās kara darbībā pret Napoleona pretiniekiem. Poļu leģionāriem bija visai nozīmīga vieta karos Itālijā, 1798.~gadā tie iegāja Romā, piedaloties Pāvesta valsts (itāļu \latxti{Stati Pontifici}) pagaidu likvidācijā. Kad 1799.~gadā Napoleons kļuva par pirmo konsulu~--- Francijas valsts galvu, viņš leģionus pārformēja, tie arī nomināli kļuva par Francijas armijas sastāvdaļu. Daļa leģionāru palika Itālijā un 1805.--1807.~gadā piedalījās Francijas vestajos karos, daļa (ap 5~000 ,,durkļu'') 1801.~gadā, neraugoties uz pretestību, tika nosūtīta uz Haiti un tika izmantota cīņā pret nēģeru brīvības cīnītājiem F.~Tusēna-Luvertīra vadībā, kaujās un no slimībām zaudējot 2/3 sastāva. Pēc dažām ziņām tikai ap trīs simtu poļu atgriezās Francijā. (Pēc citām ziņām 1814.~gadā jau kā angļu gūstekņi Eiropā atgriezās ap 500 cilvēku.) Kopumā 1797.--1807.~gadā caur poļu leģioniem izgāja ap 35~000 cilvēku. Daļa no viņiem vēlāk veidoja Varšavas hercogistes armijas kodolu, daļa (ap 8~tūkstošiem) palika Francijas dienestā. No šī laika poļu leģiona ideja līdz pat valsts atjaunošanai starptautisku sarežģījumu brīžos atkal un atkal kļuva populāra poļu patriotu aprindās. Piemēram, Krimas kara (angļu \entxti{Crimean War}, franču \frtxti{guerre de Crimée}, krievu \rutxti{Крымская война}, 1853--1856) laikā izcilais poļu dzejnieks Ā.~Mickevičs Turcijas armijā mēģināja izveidot poļu daļas karam pret Krieviju.

Tāpat kā Francija, arī tās pretinieki: Krievija, Prūsija un Austrija, diplomātisku apsvērumu vadītas, dažkārt runāja par nākotnes plāniem atjaunot Polijas valstiskumu. Tādas baumas labprāt uzklausīja zināmas poļu aristokrātijas un arī zemākās šļahtas aprindas. Krievijā šādu plānu iniciators bija kņazs Ā.~Čartorijskis. Būdams Krievijas imperatoram Aleksandram~I tuva persona, viņš 1804--1806.~gadā bija pat Krievijas ārlietu ministrs. Ā.~Čartorijskis izstrādāja Žečpospolitas atjaunošanas plānu tās agrākajās robežās Aleksandra I protektorātā. Ar dažādu kompensāciju palīdzību Austrija būtu jāpierunā atdot Krievijai Galīciju, bet Prūsijai Lielpolija (\pltxti{Wielkopolska}~--- apgabals Polijas rietumos Vartas upes baseinā) būtu jāatņem ar spēku. Aleksandrs~I šo plānu tieši nenoraidīja, bet skatījās uz to kā uz diplomātisku Prūsijas ietekmēšanas līdzekli.

Jāuzsver, ka jau Krievijas imperators Pāvils I, tāpat kā vēlāk viņa dēls Aleksandrs I, mēģināja realizēt politiku, lai Krievija kļūtu it kā par Polijas labo aizbildni uz Austrijas un Prūsijas stingrākās varas fona. Īpaši Aleksandra I valdīšanās laikā Krievijas valsts veica pasākumus, lai radītu sev piekritējus poļu valdošajā slānī. Aleksandra~I liberālie žesti (konfiscēto muižu atdošana to agrākajiem īpašniekiem vai arī zaudējumu atlīdzināšana, nodokļu atvieglinājumi), pat privilēģiju piešķiršana (neraugoties uz vispārējo labības izvešanas aizliegumu, Aleksandrs~I piešķīra poļu muižniekiem atsevišķas atļaujas eksportēt labību, izmantojot Melnās jūras ostas) rada atbalsi poļu sabiedrībā. Nemazsvarīga loma bija ekonomisko sakaru nodibināšanai, agrāko Žečpospolitas apgabalu iekļaušanai Viskrievijas tirgū. Taču tai pat laikā Krievijas valdnieki nevarēja atteikties no bijušās Žečpospolitas baltkrievu un ukraiņu apgabaliem. Polija bija tradicionāls Krievijas ienaidnieks un Krievijas imperatori nevarēja atļauties atjaunot stipru varenu Poliju, kas varētu apdraudēt Krieviju.

\asterism

1806.~gadā, kad Napoleons uzsāka karu pret Prūsiju, Polija viņa plānos ieguva ievērojamu lomu. Napoleonu tā interesēja kā militārs placdarms, izdevīgs līdzeklis diplomātiskajā spēlē. Karā iesaistījās arī Krievija. Jauno kara fāzi Napoleons nosauca par ,,Pirmo Polijas karu'' un uzdeva ģenerālim J.~H.~Dombrovskim organizēt poļu spēkus. J.~H.~Dombrovskis izveidoja vairākas divīzijas (ap 30~000 vīru) un tās sekmīgi karoja pret prūšiem un krieviem. 1806.~gada novembrī J.H.~Dombrovska ietekmē Pozenē notika poļu sacelšanās, poļi paši vairākās pilsētas atbruņoja Prūsijas garnizonus. Tas, protams, atviegloja Napoleona karaspēka darbību. 28.~novembrī franči iegāja Varšavā un 19.~decembrī tās iedzīvotāji uzgavilēja Napoleonam. Taču īstenot poļu cerības un atjaunot Polijas valsti viņš nesteidzās. 1807.~gada janvārī agrākajos Polijas apgabalos, kurus tagad okupēja Francija, Napoleons nodibināja ,,Valdošo komisiju'' (,,\pltxti{Komisja Rządząca''}), kura darbojās franču virsvadībā.

1807.~gada jūlijā, sakāvis Prūsiju un Krieviju, Napoleons par lielu vilšanos poļiem noslēdza Tilzītes mieru, pēc kura no poļu zemēm, kuras līdz tam ietilpa Prūsijas sastāvā (izņemot Rietumprūsiju un Pomerāniju (\pltxti{Pòmòrzé}), kas palika Prūsijā, un Dancigu (Gdaņsku), kura pirmo reizi [vēlāk tas notika pēc Pirmā pasaules kara] tika atzīta par ,,brīvpilsētu'', taču ar franču garnizonu), izveidoja miniatūru \strong{Varšavas lielhercogisti} (\pltxti{Księstwo Warszawskie}, 1807--1815, faktiski~--- 1813), kurā ap 104~000 km$^{2}$ lielā apgabalā dzīvoja 2,6 miljoni iedzīvotāju. Hercogistes nosaukumam nebija nekāda vēsturiska pamata. Tas radās tikai tāpēc, ka abi imperatori: Napoleons I un Aleksandrs I, parakstot Tilzītes miera līgumu, nevēlējās lai jaunās Francijas vasaļvalsts oficiālajā nosaukumā figurētu vārds ,,poļi'', ,,Polija'' vai to atvasinājumi. 1809.~gadā pēc Napoleona uzvaroša kara pret Austriju hercogistei pievienoja teritorijas, kuras pēc trešās Žečpospolitas dalīšanas iegāja Austrijas sastāvā. Radās ap 160~000 km$^{2}$ liela ,,valsts'' ar 4~350~000 iedzīvotāju. Nacionālā ziņā iedzīvotāju sastāvs bija daudz kompaktāks nekā Žečpospolitā. Pēc 1810.~gada statistikas hercogistē dzīvoja: 79\% poļu, ap 7\% ebreju, ap 6\% vāciešu, 8\%~--- lietuviešu un baltkrievu. Varšavas hercogistē administrācijā, tiesās, skolās tika lietota poļu valoda.

Pēc zināmām Napoleona svārstībām par hercogistes nominālo valdnieku kļuva Žečpospolitas priekšpēdējā karaļa Fridriha Augusta mazdēls, Saksijas karalis Fridrihs Augusts, kuru par likumīgu Polijas troņa mantinieku atzina vēl 1791.~gada Konstitūcija. Polijas valsts gan līdz tam jau nepastāvēja, taču poļu apdzīvotās teritorijas ar to nepilnu četrdesmit gadu laikā tika kārtējo reizi sadalītas un, kaut gan vēstures literatūrā to nav pieņemts darīt (Piemēram, Vācijas~--- Polijas attiecību speciālists vācu vēsturnieks V.~Jakobmeijers 1939.~gadā notikušo Polijas dalīšanu uzskata par piekto, pirms tam pieminot 1772., 1793., 1795. un 1815.~gada dalīšanas, bet aizmirstot par 1807.~gadā notikušo Napoleona īstenoto Varšavas hercogistes izveidi, kārtējo (ceturto) reizi pārdalot agrākās Žečpospolitas teritoriju), tomēr pēc autora viedokļa \strong{Napoleona veikto pārdali var kvalificēt par ceturto Polijas dalīšanu}. Pie tam Varšavas hercogiste pastāvēja vairākus gadus, šīs ceturtās dalīšanas rezultātā izveidotās robežas saglabājās ilgāk nekā pēc otrās dalīšanas radītās. Var jau uzskatīt Varšavas hercogistes radīšanu par Polijas atjaunošanas mēģinājumu, taču aiz šī valstiskā veidojuma palika daudzas etniskās poļu zemes.

Faktiski Varšavas hercogiste bija Francijas protektorāts, kurš piegādāja pēdējai pārtiku un ,,lielgabalu gaļu''. Napoleons 1806.~gadā paziņoja: ,,Polija ir pārāk sarežģīts jautājums: [poļi] pieļāva sadalīšanu, pārstāja būt par [vienotu] tautu, zaudēja sabiedrības garu, šļahtai tur ir pārāk liela loma, tautai pārāk maza. Tas ir līķis, kurā vispirms ir jāiedveš dzīvība, pirms es sākšu domāt, ko ar to darīt \citespace{} Es iegūšu no viņiem karavīrus, virsniekus, bet pēc tam paskatīšos [ko darīt tālāk].'' Tomēr pilnībā no Francijas atkarīgās hercogistes radīšanu daudzi poļi uzskatīja par Polijas atdzimšanas sākumu. 1807.~gada 22.~jūlijā Drēzdenē Napoleons izdeva hercogistes ,,Konstitūciju'' (,,\frtxti{Statut konstitutionennel du Duché de Varsovie''}). Napoleons Polijas pārvaldē balstījās uz tās muižniecību, taču visos oficiālos aktos izvairījās no vārdu ,,polis'' vai ,,poļu'' lietošanas, lai neradītu Austrijā un Krievijā bailes par to piesavināto poļu zemju drošību un pārliecinātu tās, ka Varšavas hercogistes radīšana nebūt vēl nenozīmē Polijas valsts atjaunošanu tās vēsturiskajās robežās. Dāvājot Konstitūciju ,,Varšavas un Lielpolijas'' iedzīvotājiem, Napoleons uzņēmās saistības ,,saskaņot to ar kaimiņvalstu mieru''. Konstitūcijā nebija norāžu uz vārda, preses, sapulču, biedrošanās brīvībām, personas un mantas neaizskaramību. Tiesa, atšķirībā no 1791.~gada 3.~maija Konstitūcijas visi iedzīvotāji bez kārtu atšķirībām tika pasludināti vienlīdzīgi likuma priekšā, taču šī vienlīdzība netika attiecināta uz ebrejiem. Seims tika vēlēts, otrā palāta~--- Senāts~--- monarha iecelts. Parlamenta kompetence bija ierobežota, tas varēja apspriest tikai budžeta, civilās un kriminālās likumdošanas jautājumus. 1807.~gadā tika atcelta dzimtbūšana, ieviesta zemnieku personīgā brīvība, taču zeme palika šļahtas rokās kā privātīpašums. Atstājot savu muižnieku, zemniekam bija pienākums ,,atdot muižniekam pēdējā zemes īpašumu, kas sastāv no ēkām, inventāra un sējumiem''. Šļahtiči varēja zemniekus padzīt no viņu saimniecībām. Reformas īstenošana veicināja zemes koncentrāciju šļahtas rokās. 1810.~gadā hercogistē 54\% zemes atradās folvarkos, 2\%~--- baznīcas un 44\%~--- zemnieku rokās. Ap 20\% valsts muižu Napoleons izdāvāja saviem maršaliem un ģenerāļiem.

Interesantu Napoleona un poļu šļahtiču attiecību aspektu atklājis jau pieminētais poļu ekonomikas vēsturnieks V.~Kula. Tā Polijas daļa, kura pēc trim dalīšanām bija Prūsijas varā, bija nonākusi arī tās kredītiestāžu darbībā sfērā. Šīs iestādes bija radītās, lai atvieglotu prūšu muižniekiem (junkuriem) pāreju uz intensīvāku saimniekošanu. Iespēju saņemt valsts kredītus ieguva arī poļu muižnieki. Viņi to arī pilnā mērā izmantoja. Taču pēc Napoleona uzvaras pār Prūsiju viņš ar uzvarētāja tiesībām pasludināja sevi par šo kredītu īpašnieku un pieprasīja tos atmaksāt viņam. Izrādījās, ka atšķirībā no prūšu muižniecības lielākās daļas poļu muižnieki pārsvarā bija izmantojuši kredītus nevis savu muižu attīstībai, bet vienkārši izšķērdējuši. Poļu muižnieciskā historiogrāfija pat radīja tēzi, ka kredīti poļu šļahtičiem tikuši piešķirti ar nolūku tos izputināt, novest līdz bankrotam un tādā veidā iznīdēt arī ,,poļu gara cietoksni''~--- šļahtiču muižas. Kad noskaidrojās, ka poļu šļahta nespēj samaksāt savus parādus Napoleonam, sākās sarunas. Varšavas hercogistes valdības delegācija sekoja Napoleonam pa visu Eiropu, lai vestu ar viņu cinisku kaulēšanos. Napoleons parādus pamazām atlaida~--- kā maksu par katru jaunu poļu pulku, kurš devās karot uz Spāniju. Tikai krietni vēlāk, XIX gadsimta vidū poļu muižnieki iemācījās izmantot naudu lai, atbilstoši tirgus principam, tā nestu jaunu naudu.

Arī hercogistei kopumā nācās ciest no Napoleona pastāvīgajām naudas prasībām. Tās saimniecisko attīstību traucēja viņa ievestā kontinentālā blokāde (1806--1814).

1808.~gadā tika ieviests Napoleona kodekss (\frtxti{Code Napoléon, arī Code Civil des Français}) kā hercogistes civillikums, kurš privātīpašuma principu sabiedriskajā dzīvē padarīja par izšķirošo, radot privātīpašniekiem sabiedriskās augšupejas iespējas neatkarīgi no kārtu piederības. Tas deva pirmo pamatīgo triecienu kārtu iekārtai, ievērojami sekmēja kapitālistisko attiecību attīstību feodālo vietā. Tomēr nedrīkst noklusēt, ka Napoleona kodeksa ieviešanu Varšavas hercogistes Seims pat īsti neapsprieda, uzstājās trīs deputāti, kuri ierosināja to pieņemt, uzreiz notika balsošana un ar 105 pret 2 balsīm kodekss tika pieņemts.

Tradicionālās pašvaldības tika atceltas un hercogisti pārvaldīja pēc Francijas departamentu sistēmas parauga. Hercogiste sākotnēji dalījās sešos departamentos un tie katrs desmit apriņķos. Pēc Napoleona uzvaras 1809.~gadā pār Austriju klāt nāca vēl četri departamenti no bijušajām Austrijas teritorijām. Bez tam Napoleons arī radīja atsevišķu 30~000 cilvēku lielu poļu armiju, no kuras daļa devās Napoleonam palīgā karot Spānijā. 1810.~gadā tās karavīru skaitu palielināja līdz 60~000 un 1812.~gadā līdz 75~000. Kopumā Varšavas hercogistes armijā ilgāku vai īsāku laiku dienēja ap 200~000 poļu. Komandēja armiju hercogistes kara ministrs J.~Poņatovskis, pēdējā Žečpospolitas karaļa Staņislava II Augusta Poņatovska brāļa dēls.

Kritiskāk noskaņotie poļi ironiski runāja par jauno valsti: ,,Hercogiste Varšavas, monētas prūšu, armija poļu, karalis sakšu, bet kodekss franču'', taču apzinoties savu atkarību no Napoleona labvēlības, neuzdrošinājās atklāti kurnēt.

Par Varšavas hercogistes pastāvēšanas laiku eksistē anekdote.

\begin{quote}
Kāds ārzemnieks, nonācis Varšavas hercogistē, vietējiem iedzīvotājiem apjautājās:

,,Kas tā ir par zemi?''

,,Varšavas hercogiste,''~--- skanēja atbilde.

,,Un kas ir tās valdnieks?''

,,Saksijas karalis.''

,,Un kāda viņam ir armija?''

,,Poļu.''

,,Bet kādas tiesības šeit pastāv?''

,,Franču.''

,,Kāda nauda?''

,,Prūšu.''

,,Tad gan jūs dzīvojat kā pie Bābeles torņa,''~--- nošūpoja galvu ārzemnieks.
\end{quote}


Poļu dižciltīgie baidījās par savu stāvokli, taču Napoleons, paredzot tos izmantot pret Krieviju, centās īstenot viņu interesēm atbilstošu politiku.

Lai arī Varšavas hercogiste nebija poļu sapņu piepildījums, tās radīšanai bija pozitīva nozīme. Hercogistē attīstījās nacionālā kultūra, zinātne, izglītība. Jau pašam faktam, ka pastāv valstisks veidojums, kurš turpina poļu valstiskuma tradīcijas, bija liela nozīme laikabiedru acīs. Polijas jautājums atkal tika izvirzīts aktuālās politikas ierindā, ar to nācās rēķināties Vīnes kongresam 1815.~gadā. Daudzi franču ieviestie likumi un institūcijas saglabāja savu iespaidu arī pēc hercogistes likvidācijas līdz pat XX gadsimtam.

Daļa poļu leģionāru arī pēc Varšavas hercogistes izveides palika Francijas armijā un palīdzēja Napoleonam okupēt Spāniju. Spāņi šo karu sauca~--- ,,karš par Spānijas neatkarību'' (\estxti{Guerra de la Independencia Española}). Poļi cerēja, ka tādā ceļā viņi tuvina savas valsts atjaunošanu, kaut mūsdienās ir grūti izprast, kā, apspiežot citu tautu, var cīnīties par savas tautas neatkarību.

1810.--1811.~gadā Napoleons, cenšoties piesaistīt savā pusē poļu šļahtu, solīja tai Polijas atjaunošanu 1772.~gada robežās. Tiesa, ja solījums arī tiktu īstenots, šai valstij bija nolemts palikt Francijas atkarībā. 1810.~gada Francijas un Krievijas konvenciju, kurā Francija solīja nepaplašināt Varšavas hercogistes teritoriju, Napoleons atteicās ratificēt, taču šis solis vairāk vērtējams kā Aleksandra I šantažēšanas līdzeklis, nekā liecība par īstenajiem Napoleona nodomiem. Atbildot uz to, Aleksandrs I inspirēja baumas par autonomas Lietuvas lielkņazistes radīšanu no Krievijas rietumu guberņām, kura varētu kļūt par nākamās poļu valsts kodolu. Taču sarežģīto diplomātisko spēli pārtrauca 1812.~gadā sācies karš.

Neilgi pirms tā sākuma~--- 1812.~gada 10.~februārī poļu ģenerālis M.~Sokolņickis iesniedza Napoleonam detalizētu kara vešanas plānu pret Krieviju ar nosaukumu ,,Par līdzekļiem kā Eiropai atbrīvoties no Krievijas ietekmes, bet pateicoties tam, arī no Anglijas ietekmes''. Tajā bija izstrādāts ne tikai Napoleona armijas virzīšanās maršruts, bet arī Krievijas sadalīšanas un kā neatkarīgas valsts likvidēšanas scenārijs. Francijas imperators varēja būt apmierināts, ka tā iekarošanas plānu realizēšanai atradās tāds iegansts kā Polija. 1812.~gada 22.jūnijā Napoleons parakstīja savu uzsaukumu Francijas ,,Lielajai armijai'' (\frtxti{Grande Armée}): ,,Karavīri, otrais poļu karš ir sācies. Pirmais beidzās Fridlandē [pilsēta~--- tagad Pravdinska, pie kuras notika uzvaroša Napoleona kauja ar Krievijas armiju 1807.~gada 14.~jūnijā] un Tilzītē. \citespace{} Tātad dosimies uz priekšu, pāriesim Nemunu, ienesīsim karu tās [Krievijas] teritorijā. Otrais poļu karš nesīs tādu pat slavu franču ieročiem kā pirmais. Taču miers, kuru mēs noslēgsim, tiks nodrošināts un nesīs beigas tai postošajai ietekmei, kurus Krievija nu jau 50 gadu kā atstāj uz Eiropu''. Sarunās ar poļu darbiniekiem Napoleons tieši norādīja, ka gaida poļu šļahtas uzstāšanās pret Krieviju lietuviešu, baltkrievu un ukraiņu zemēs. Taču 1812.~gada martā noslēgtajā līgumā ar Austriju Napoleons tai garantēja Galīcijas paturēšanu. Kāda varētu kļūt poļu valsts Napoleona uzvaras gadījumā, tā arī nebija skaidrs.

Naktī uz 24.~jūniju sākās pārcelšanās pār Nemunu un 300 poļu huzāru kā pirmie to šķērsoja. Poļu sajūsma bija liela. Ā.~Mickevicš to vēlāk (1834) tēloja poļu nacionālajā eposā ,,\pltxti{Pan Tadeusz}'' (Tadeuša kungs). 26.~jūnijā Varšavas hercogistes Seims pasludināja, ka priekšā stāv Polijas ,,atkalapvienošanās''. Poļu karavīri no Spānijas atgriezās Varšavas hercogistē. Napoleona pusē karojošais poļu korpuss bija visuzticamākais no ,,Lielās armijas'' cittautiešu daļām. Polijas atbrīvošana kļuva par vienu no lozungiem, kas ļāva Napoleona armijā ievilināt daudz poļu. 1812.~gadā pirms karagājienu pret Krieviju Francijas imperatora rīcībā bija ap 85~000 (pēc citiem datiem~--- ne mazāk par 120~000) poļu karavīru. Kara gājienā uz Maskavu Napoleona \frtxti{Grande Armee} (Lielā armija) sastāvā bija ap 100~000 poļu karavīru (Citi dati: pret Krieviju Napoleona armijas sastāvā karoja 70 tūkstošu poļu ar 105 lielgabaliem.) Tādejādi katrs piektais Napoleona karavīrs bija polis. Viņi sastādīja gandrīz vai pusi no Napoleona kavalērijas. Taču visus poļus apvienot vienkopus Napoleons nevēlējās. Daļa no tiem tika izkaisīti starp citām franču armijas vienībām. Tikai ap 30 tūkstošu (citi dati~--- 37 tūkstošu) vīru lielu poļu kontingentu komandēja J.~Poņatovskis. Kopā ar viņu cīnījās arī poļu ģenerāļi J.H.~Dombrovskis, J.~Zaijončeks u.c.

T.~Kostjuško, kurš, atgriezies Eiropā, 1798.~gadā apmetās Parīzes apkaimē un piedalījās poļu politiskajā dzīvē, arī atbalstīja Poļu leģionu izveidi, 1799.~gada nogalē T.~Kostjuško tikās ar Napoleonu, taču, nesaņēmis no viņa solījumu par konstitucionālas Žečpospolitas atjaunošanu bijušajās robežās, uzskatīja par sev pazemojošu akli kalpot Francijas imperatoram. Viņš nepārcēlās arī uz Varšavas hercogisti, kaut 1807.~gadā sarunā ar Francijas policijas ministru J.~Fušē, piesolīja Napoleonam savu palīdzību, ja tas dotu rakstisku solījumu, publicētu avīzēs, ka Polijā tiks izveidota valsts iekārta līdzīga Anglijas iekārtai, ka zemnieki tiks atbrīvoti ar zemi un Polijas robežas sniegsies no Dancigas līdz Ungārijai, iekļaujot arī Galīciju. Atbildei uz to Napoleons rakstīja J.~Fušē: ,,Es nepiešķiru nekādu nozīmi Kostjuško. Viņam savā zemē nav tās ietekmes, kurai viņš pats tic. Vispār, visa viņa uzvedība vieš pārliecību, ka viņš ir vienkārši muļķis. Vajag ļaut viņam darīt, ko vēlas, nepiegriežot viņam nekādu uzmanību''. Par Napoleonu T.~Kostjuško izteicās: ,,Viņš domā tikai par sevi, nicina ikvienu lielu tautību un vēl vairāk neatkarības garu. Viņš ir tirāns.''

Poļu daļas pirmās iegāja krievu karaspēka pamestajā Viļņā. Taču poļi drīz bija vīlušies. Atkarojis Krievijai Lietuvu, Napoleons to nepievienoja Varšavas hercogistei, bet radīja patstāvīgu hercogisti, kura gan ilgi nepastāvēja.

Poļi piedalījās arī Borodinas u.c. kaujās pret krievu karaspēku, ciešot lielus zaudējumus. Pēc tam, kad jau bija pierādījusies krievu tautas pretestība iebrucējiem, Napoleona franču ģenerāļi ieteica viņam pasludināt dzimtbūšanas atcelšanu Krievijā, ko imperators tomēr nedarīja. Vācu vēsturnieks E.~Veiss uzskata, ka tas, iespējams, notika tāpēc, ka viņš negribēja kaitēt savai sabiedrotajai~--- poļu šļahtai. Tomēr jāsaka, ka pēc tam, kad krievu zemnieki bija izjutuši Napoleona karavīru pārestības, kuri atņēma tiem pēdējos mājlopus un pēdējo gabalu maizes, cerēt uz vietējo iedzīvotāju atbalstu iebrucēji diezin vai varēja, pat ja paziņotu par dzimtbūšanas jūga atcelšanu.

1812.~gada beigās, kad Varšavā nonāca ziņas par Napoleona armijas grūto stāvokli Krievijā, tika mēģināts savākt jaunus poļu brīvprātīgos, bet cerēto 30~tūkstošu vietā pieteicās vien ap 400~cilvēku. Atgriezušās Polijā, izretinātās poļu daļas ar J.~Poņatovski priekšgalā gan vēlējās aizstāvēt Varšavu pret Krievijas armiju, taču spēku nepietika un poļu karavīri devās līdzi Napoleona armijai, pametot Poliju.

Napoleons, atkāpjoties no Krievijas, izstrādāja divus variantus nākotnei. Pēc pirmā viņš bija gatavs kompromisa mieram ar Krieviju, solot tai nepieļaut Polijas valsts atjaunošanu. Viņš teica: ,,Ja poļi neizpildīs savu pienākumu [domāta brīvprātīga masu mobilizācija karam pret Krieviju], tad Francijai un visai pasaulei miera jautājums vienkāršosies, jo tad ar Krieviju noslēgt mieru būs viegli.'' Otrais variants paredzēja piekāpšanos Austrijai, piešķirot tai teritorijas Balkānos, bet tai piederošo Galīciju atdodot Varšavas hercogistei un pasludinot Polijas atjaunošanu. Par Polijas karali tad vajadzētu kļūt kādam no Napoleona tuvākajiem līdzgaitniekiem. Izmantojot poļu sajūsmu par valstiskuma atgūšanu, Napoleons vēl cerēja vest uzvarošu karu pret Krieviju. Taču neviens no variantiem neīstenojās, jo Eiropas monarhi nevēlējās slēgt nekādas vienošanās ar Napoleonu.

1813.~gadā Napoleona pusē vēl karoja ap 40~000 poļu. J.~Poņatovskis kļuva par vienīgo ārzemnieku, kurš Leipcigas t.s. ,,Tautu kaujas'' laikā (1813.~gada 16.--19.~oktobrī) no Napoleona rokām saņēma Francijas maršala zizli, taču pēc tam~--- arī pavēli piesegt franču daļu atkāpšanos. Franči priekšlaikus uzspridzināja tiltu un viņš, nevēloties padoties, ievainots mēģināja pārpeldēt Elsteres upi un noslīka.

Karojot Napoleona vadībā, 1812.~gadā poļi atstāja smagas atmiņas Krievijas civiliedzīvotājiem. Galvenie vardarbību un laupīšanu veicēji Krievijā bija nevis Napoleona disciplinētie francūži, bet vācieši un poļi. Par to rakstījuši daudzi krievu autori, sākot no A.~Puškina līdz J.~Tarlem. Kā rakstīja baltkrievu izcelsmes vēsturnieks profesors M.~Kojalovičs: ,,Maskavā tauta neieredzēja un baidījās no tiem francūžiem, kuri saprata krieviski un runāja krieviski. Tie bija poļi.'' Krievijas apmija, sekojot Napoleona armijai un nonākot Polijā, bija gatava to pārvērst par tuksnesi, taču poļus no tā paglāba imperators Aleksandrs I, kurš Rietumeiropā vēlējās ieiet kā atbrīvotājs, nevis atriebējs. Krievu muižnieku aprindās gan klīda baumas, ka imperatora labvēlīgā attieksme pret poļiem ir saistīta ar viņa faktisko otro, neoficiālo sievu, dzimušo polieti A.~Nariškinu, ar kuru viņam bija vairāki bērni. Poļu šļahta kā savu līderi pie Aleksandra I nosūtīja Ā.~Čartorijski, taču viņš tā arī nesaņēma no Krievijas imperatora skaidru atbildi par Polijas nākotni. Līdz jautājuma galīgai starptautiskai izlemšanai Aleksandrs I kā augstāko pārvaldes iestādi lika radīt pagaidu Augstāko padomi (\pltxti{Rada Tumsczasowa Najwyźsza}), kuras sastāvā bija divi krievi, divi poļi un viens prūsis.

1814.~gadā Napoleona pusē vēl karoja ap 4~000 poļu. Pat pēc Napoleona atteikšanās no troņa viņam uzticīgie poļu brīvprātīgie izveidoja vieglās kavalērijas eskadronu, kurš pavadīja bijušo imperatoru uz Elbas salu. Šis eskadrons piedalījās arī Napoleona ,,100 dienu'' epopejā un pilnībā gāja bojā Vaterlo (\entxti{Waterloo}, 1815.~gada 18.~jūnijā) kaujā. Vēl visu XIX gadsimtu poļu sabiedrībā pastāvēja kas līdzīgs Napoleona kultam.

Vācu vēsturniece A.~Šmidte-Roslere uzskata, ka to nosacīja Napoleona mīlas sakars ar poļu muižnieci M.~Vaļevsku un viņu kopējais dēls A.~Vaļevskis, kurš kļuva par Francijas pavalstnieku, bet tomēr piedalījās 1830--1831.~gada poļu sacelšanās norisē, tika apbalvots, atgriezās Francijā, kur Napoleona III laikā kļuva par tās ārlietu ministru. Vēsturniece atsaucas uz B.~Prusa romānu ,,Lelle'' (,,\pltxti{Lalka}''). Domājams, ka tomēr lielāka nozīme bija poļu patriotu utopiskajām cerībām, ka Napoleons uzvaras gadījumā pār Krieviju atjaunos Žežpospolitu, tāpēc simpātijas pret viņu neizdzisa vēl ilgi.

Šai sakarā gan krievu dzejnieks kņazs P.~Vjazemskis, kurš savu karjeru sāka Varšavā, jaunībā bija liberāli noskaņots un 1830.~gadā uzstājās pret krievu karaspēka ievešanu Polijā, savā dienasgrāmatā rakstīja: ,,Lai cik arī poļi būtu truli, taču nevar iedomāties, ka vesela tauta brīvprātīgi ietu nāvē, pretī neglābjamai bojā ejai \dots{} Napoleons viņus sagūstīja ar divām, trijām frāzēm \citespace{} Ko viņš ir izdarījis Polijas labā? Griezies pie tās ar vairākiem madrigāliem savās proklamācijās, izdalījis tai dažus Goda Leģiona krustus, nopirktus ar poļu asins straumēm. Tas arī viss.'' Taču poļu patrioti kā slīcējs pie salmiņa pieķērās jebkurai cerībai, kas tiem solīja valstiskuma atjaunošanu.

Jau 1812.~gada decembrī, ieradies Viļņā, Krievijas imperators Aleksandrs I pasludināja vispārēju amnestiju Krievijas pavalstniekiem~--- poļiem, kuri bija karojuši Napoleona pusē, vēlāk atļāva izdzīvojušajiem poļu karavīriem atgriezties dzimtenē. Tur viņi tika iekļauti jaunveidojamās Polijas karalistes (\pltxti{Królestwo Polskie}) armijā. Ģenerālis J.~H.~Dombrovskis saņēma no Aleksandra~I kavalērijas ģenerāļa pakāpi, piedalījās Krievijas pakļautībā esošas Polijas karalistes jaunveidojamās armijas radīšanā un kļuva par Polijas senatoru. Ģenerālis J.~Zaijončeks kļuva par Aleksandra I vietvaldi (\rutxti{наместник}) Polijas karalistē, saņēma kņaza titulu.

Arī T.~Kostjuško 1814.~gadā griezās pie Aleksandra I ar vēstuli, kura saturēja padomus, kā labiekārtot Poliju. T.~Kostjuško lūdza Aleksandru I pasludināt sevi par Polijas karali, dot Polijai Konstitūciju, līdzīgu tai, kāda pastāvēja Anglijā utml. Sākotnēji Krievijas imperators pret poļu ģenerāli izturējās labvēlīgi, pat piedāvāja viņam vadīt Polijas karalistes administrāciju, taču pēc būtības atbildēja izvairīgi, ka cerot ,,paveikt varonīgās tautas atdzimšanu''. T.~Kostjuško, uzzinot, ka Polija netiks atjaunota 1772.~gada robežās, atteicās sadarboties ar uzvarētājiem karā un devās uz Šveici, kur arī mira. T.~Kostjuško ķermenis tika iebalzamēts un 1818.~gadā pārvests uz Krakovu, apglabāts Vāvelas (\pltxti{Wawel}) pils katedrālē, kur atrodas Polijas karaļu kapenes.

\asterism

% page 69

% ----- INFO PAGE -----
\newpage\thispagestyle{plain}
{\centering

% ----- LICENSE SIGNS BLOCK -----
\includegraphics[width=2.5em]{cc.pdf}~\includegraphics[width=2.5em]{by.pdf}
% ----- LICENSE SIGNS BLOCK -----

\vspace{0.2em}

% ----- LICENSE BLOCK -----
{\ml{$0$}{Данная книга распространяется под лицензией \textbf{CC~BY~4.0}.}{This book is distributed under the \textbf{CC~BY~4.0} license.}\par}
{\ml{$0$}{Подробнее о лицензии:}{Details:} \href{https://creativecommons.org/licenses/by/4.0/deed.ru}{creativecommons.org/licenses/by/4.0}\par}
% ----- LICENSE BLOCK -----

\vfill

% ----- AUTHOR AND TITLE -----
{\large\bookauthor\par}
\vspace{0.5em}
{\Huge\textbf{POLIJA}\par}
{\Large\textbf{XIX un XX gadsimtā}\par}
% ----- AUTHOR AND TITLE -----

\vfill

% ----- BOOK INFO -----
{\ml{$0$}{Начато:}{Started:} \textit{\bookstarted}\par}
{\ml{$0$}{Последняя редакция:}{Latest revision:} \textit{\bookfinished}\par}
{\ml{$0$}{Подробнее о книге:}{The book details:} \href{https://github.com/regnveig/tofa}{github.com/regnveig/tofa}\par}
% ----- BOOK INFO -----

\vfill

% ----- DOCUMENT INFO -----
{\ml{$0$}{Дизайн обложки: \textit{Э.\,Весна}}{The cover design by E.\,Viesn\'a}\par}
{\ml{$0$}{Компьютерная вёрстка: \textit{Э.\,Весна}}{The computer layout by E.\,Viesn\'a}\par}
{\ml{$0$}{Создано с помощью \XeLaTeX}{Created with \XeLaTeX}\par}
% ----- DOCUMENT INFO -----

\vspace{0.5em}

% ----- PUBLISHER INFO -----
{\textbf{\ml{$0$}{Свободное издательство <<Цунами>>}{\textsc{Tsunami}, an independent publisher}}\par}
{\ml{$0$}{Томск 2023}{Tomsk 2023}\par}
% ----- PUBLISHER INFO -----

\vfill

% ----- QR CODES -----
~
% ----- QR CODES -----

% }
% % ----- INFO PAGE -----
%
% % ----- EMPTY PAGE -----
% \newpage\thispagestyle{plain}~
% % ----- EMPTY PAGE -----
%
% % ----- BACK COVER -----
% \includepdf[pages={1}]{cover_back.pdf}
% % ----- BACK COVER -----

\end{document}
