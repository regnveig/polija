\documentclass[twoside,a5paper,12pt,fleqn,openany]{extbook}
\usepackage{polyglossia}
\usepackage{xcolor}

\usepackage{tocloft}

\renewcommand{\cftchapleader}{\cftdotfill{\cftdotsep}}
\setlength{\cftsecindent}{0pt}% Remove indent for \section
\setlength{\cftsubsecindent}{0pt}% Remove indent for \subsection

\setdefaultlanguage{latvian}
\setotherlanguages{english,russian,polish,german,french,lithuanian,spanish,latin,ukrainian,belarusian,greek,hebrew,romanian,italian,czech,slovene,swedish}
\defaultfontfeatures{Ligatures=TeX,Mapping=tex-text}

\newcommand{\ml}[3]{#2}
\newcommand{\coverfrontfilename}{cover_front.pdf}
\newcommand{\publisherlogofilename}{publisher_logo.pdf}

\usepackage{tabularx}
\usepackage{ltablex}

% ------ METADATA ------
\newcommand{\bookauthor}{Vitālijs~Šalda}
\newcommand{\booktitle}{POLIJA XIX un XX gadsimtā}
\newcommand{\bookstarted}{...}
\newcommand{\bookfinished}{2023}
% ------ METADATA ------

% ----- XELATEX SYMBOL -----

% ----- XELATEX SYMBOL -----

% ----- HYPHENATION -----
\usepackage{hyphenat}
% ----- HYPHENATION -----

% ----- GEOMETRY -----
\usepackage[left=1.5cm,right=1.5cm,top=2cm,bottom=2cm,bindingoffset=0.5cm]{geometry}
% ----- GEOMETRY -----

% ----- INCLUDE PDF AS PAGES -----
\usepackage{pdfpages}
% ----- INCLUDE PDF AS PAGES -----

% ----- DROPPING CAP -----
\usepackage{type1cm,lettrine}
% ----- DROPPING CAP -----

% ----- FONTS -----
\renewcommand{\baselinestretch}{1.2}
\setmainfont{Linux Libertine}
% ----- FONTS -----

% ------ HYPERLINKS ------
\usepackage{hyperref}
\definecolor{LinkColor}{HTML}{0969DA}
\hypersetup{colorlinks=true, linkcolor=LinkColor, citecolor=LinkColor, filecolor=LinkColor, urlcolor=LinkColor}
% ------ HYPERLINKS ------

% ------ FANCY PAGE STYLE ------
\setlength{\headheight}{15pt}
\usepackage{fancyhdr}
\pagestyle{fancy}
\fancyhead[LE,RO]{\thepage}
\fancyhead[LO]{{\small{Polija XIX un XX gadsimtā}}}
\fancyhead[RE]{{\small{\bookauthor}}}
\fancyfoot{}
    \fancypagestyle{plain}{
    \renewcommand{\headrulewidth}{0mm}
    \fancyhead{}
    \fancyfoot{}
}
% ------ FANCY PAGE STYLE ------

% ------ ELEMENTS ------
\newcommand{\asterism}{\vspace{1em}{\centering\Large\bfseries$\ast~\ast~\ast$\par}\vspace{1em}}
\newcommand{\pltxti}[1]{\textit{\textpolish{#1}}}
\newcommand{\rutxti}[1]{\textrussian{#1}}
\newcommand{\detxti}[1]{\textit{\textgerman{#1}}}
\newcommand{\frtxti}[1]{\textit{\textfrench{#1}}}
\newcommand{\entxti}[1]{\textit{\textenglish{#1}}}
\newcommand{\lttxti}[1]{\textit{\textlithuanian{#1}}}
\newcommand{\estxti}[1]{\textit{\textspanish{#1}}}
\newcommand{\latxti}[1]{\textit{\textlatin{#1}}}
\newcommand{\betxti}[1]{\textbelarusian{#1}}
\newcommand{\uktxti}[1]{\textukrainian{#1}}
\newcommand{\rotxti}[1]{\textit{\textromanian{#1}}}
\newcommand{\ittxti}[1]{\textit{\textitalian{#1}}}
\newcommand{\cstxti}[1]{\textit{\textczech{#1}}}
\newcommand{\sltxti}[1]{\textit{\textslovene{#1}}}
\newcommand{\svtxti}[1]{\textit{\textswedish{#1}}}
\newcommand{\eltxti}[1]{\textgreek{#1}}
\newcommand{\hetxti}[1]{\texthebrew{#1}}
\newcommand{\mntxti}[1]{\textit{#1}}
\newcommand{\tttxti}[1]{\textit{#1}}
\newcommand{\zhtxti}[1]{#1}

\newcommand{\citespace}{[\dots{}]}

% ------ ELEMENTS ------

% ------ EPIGRAPH ------
\usepackage{epigraph}
\renewcommand{\epigraphsize}{\footnotesize}
\epigraphrule=0pt
\epigraphwidth=8cm
\setlength{\beforeepigraphskip}{.1\baselineskip}
\setlength{\afterepigraphskip}{.2\baselineskip}

% ------ EPIGRAPH ------

\addto\captionslatvian{\renewcommand{\contentsname}{Satura rādītājs}}

\begin{document}

% ----- FRONT COVER -----
\includepdf[pages={1}]{\coverfrontfilename}
% ----- FRONT COVER -----

% ----- EMPTY PAGE -----
\thispagestyle{plain}~
% ----- EMPTY PAGE -----

% ------ TITLE PAGE ------
\begin{titlepage}
{
\centering
{~\par}
\vspace{0.25\textheight}
{\LARGE\bookauthor\par}
\vspace{1.3cm}
{\Huge\textbf{POLIJA}\par}
{\LARGE\textbf{XIX un XX gadsimtā}\par}
\vfill
{\includegraphics[width=6em]{\publisherlogofilename}\par}
}
\end{titlepage}
% ------ TITLE PAGE ------

\epigraph
{Pagātne nav mirusi, tā nav pat pagājusi.}
{Viljams Folkners (\entxti{William Cuthbert Faulkner})}

\epigraph
{Meklējiet patiesību faktos.}
{Dens Sjaopins (\zhtxti{鄧小平})}

\epigraph
{Vēsturnieki nevar būt pilnīgi objektīvi.
Vēsture pēc savas būtības rada emocijas \citespace{}
Vēsture, izņemot neapšaubāmus faktus, pirmkārt balstās uz interpretāciju un zināmā mērā pieļauj daudzu patiesību esamību.}
{Žaks Le Goffs (\frtxti{Jacques Le Goff})}

\epigraph
{Vēsture nav skolotāja, bet uzraudze, magistra vitae: tā neko nemāca, bet tikai soda par mācībvielas nezināšanu.}
{Vasilijs Kļučevskis (\rutxti{Василий Осипович Ключевский})}

\epigraph
{Pagātne, saglabāta atmiņā, ir daļa tagadnes.}
{Tadeušs Kotarbiņskis (\pltxti{Tadeusz Marian Kotarbiński})}

\epigraph
{Bez īstas mīlestības pret cilvēci nav īstas dzimtenes mīlestības.}
{Anatols Franss (\frtxti{Anatole France})}



\epigraph
{Var atrast svētus cilvēkus, bet nav svētu politiķu, valdību, partiju, politisku un sabiedrisku kustību.}
{Voicehs Jaruzeļskis (\pltxti{Wojciech Witold Jaruzelski})}

\epigraph
{Nav jēgas rēķināties ar kaut kādu pasaules taisnīgumu visiem~--- viss ir atkarīgs no spēku samēra.}
{Aleksandrs Zinovjevs (\rutxti{Александр Зиновьев})}

\epigraph
{Ne viss vēsturē ir pelnījis, lai ar to lepotos un ar entuziasmu uz to atsauktos.}
{Voicehs Jaruzeļskis (\pltxti{Wojciech Witold Jaruzelski})}

\epigraph
{Katras valsts vēsturē atrodamas gan varonīgas, gan kaunpilnas lappuses~--- taču pats kaunpilnākais ir noklusēt vai lielīties ar to, no kā ir jākaunas.}
{Natans Eidelmans (\rutxti{Натан Яковлевич Эйдельман})}

\epigraph
{Tautas vēsture ir tautas raksturojums.}
{Mihails Vellers (\rutxti{Михаил Иосифович Веллер})}

\epigraph
{Vēsturiskā atmiņa nekādā gadījumā nav objektīva.}
{Ivars Austers}



\epigraph
{Patiesība ir subjektīva un vēstures izpratne tāpat.
Tā ir objektīva tikai tajā ziņā, ka vispār pastāv mūsu apziņā kā sen pagājušu notikumu pēdas.
Tāpēc veltas ir cerības uz jel kādu ``objektīvu vēsturi” vai ``vēstures tiesu”, kas nāks kā atpestīšana no mūžīgiem maldiem un noliks patiesību vispārējai apskatei.
Vēsture ir mūsu apziņa šeit šajā brīdī.
Rīt tā jau būs cita.}
{Āris Puriņš}

\epigraph
{Vēstures mūza ir lēnprātīga, zinoša un nepretencioza, bet, kad jūtas atstāta un pamesta, viņa cenšas atriebties un apžilbina tos, kas viņu atstāj novārtā.}
{Lešeks Kolakovskis (\pltxti{Leszek Kołakowski})}

\epigraph
{Vēsturnieku, kurš sagroza savu vēsturi, jāiznīcina tāpat kā naudas viltotāju tāpēc, ka viņi abi kaitē savai valstij.}
{Dons Migels de Servantess Saavedra\\(\estxti{Don Miguel de Cervantes Saavedra})}

\epigraph
{Tos, kuri saka taisnību, nemīl.
Tos, kuri melo~--- nicina.
Izvēle nav liela\ldots{}}
{Jurijs Poļakovs (\rutxti{Юрий Михайлович Поляков})}

\tableofcontents

\chapter*{Priekšvārds}
\addcontentsline{toc}{chapter}{Priekšvārds}

Strādājot Daugavpils Universitātes Vēstures katedrā, starp citiem studiju kursiem šī darba grāmatas autoram nācās lasīt arī lekciju kursu par Vidus un Austrumeiropas valstu vēsturi pēc Otrā pasaules kara.
Acīmredzot tāpēc, mācību programmām pilnveidojoties, Vēstures katedras toreizējais vadītājs asociētais profesors H.~Soms viņam uzdeva sagatavot arī Polijas XX gadsimta vēstures kursu.
Darba gaitā autoram izveidojās uzskats, ka jaunākos politiskos strīdus var saprast tikai tad, ja tiek ievērotas to vēsturiskās saknes; lai dziļāk izprastu XX~gadsimta Polijas vēstures notikumus, ir jādod plašāks ieskats par tās vēsturi XIX~gadsimtā nekā tas tiek sniegts vispārējā jauno laiku vēstures kursā.

Ievērojot to, ka bez profesoru Ē.~Jēkabsona un K.~Poča pētījumiem, kas saistīti ar Polijas vēsturi, latviešu valodā studentu rīcībā tikpat kā nav tai veltītu plašāku darbu, gadu gaitā tapa lasītāju priekšā esošais darbs.
Tas sākotnēji bija iecerēts kā mācību līdzeklis studentiem--vēsturniekiem.
Tomēr, ņemot vērā sabiedrībā pieaugošo interesi par mūsu valsts un tās kaimiņzemju vēsturi, kā arī lielo ar Polijas vēsturi saistīto materiālu apjomu, gala rezultātā ir tapis studentiem domāta mācību līdzekļa un zinātniski populāra darba apvienojums.
Autors cer, ka tādā veidolā tas atradīs lasītājus ne tikai studentu, bet arī citu vēstures interesentu vidū.

Izklāstā ir ietilpināts materiāls kā par jau lielā mērā izpētītiem, tā arī par daudziem vēl neatrisinātiem Polijas vēstures jautājumiem.
Nākotnē otro traktējums, iespējams, būs jāprecizē vai pat būtiski jāpārskata.
Tomēr autors uzskata par nepieciešamu dot priekšstatu arī par šādām problēmām.

Uzskatāmības labad autora tekstam pievienoti vairāku dokumentu tulkojumi un daudzas, galvenokārt INTERNET-ā aizgūtas ilustrācijas.
Lai lasītājiem būtu vieglāk orientēties, tekstā ar \strong{treknu druku} izcelti atsevišķi jēdzieni, problēmas, kuru iztirzājums tālāk seko.
Pieminētajām vēsturiskajām personām, arī pētniekiem un publicistiem darbam pievienotajā viņu sarakstā autors centās norādīt viņu dzīves gadus, kurus gan, diemžēl, ne vienmēr izdevās atrast.
Izklāstu ilustrējoši papildmateriāli nodrukāti citā šriftā un ar nedaudz sīkāku druku.

Bez izmantoto avotu un literatūras sarakstā uzrādītajiem izdevumiem grāmatas tapšanās gaitā samērā plaši izmantoti arī daudzi periodikā un INTERNET-ā atrodami materiāli, kuri nav norādīti bibliogrāfijas sarakstā.
Diemžēl tajos sastopamas daudzas neprecizitātes faktu izklāstā, klaji subjektīvas notikumu interpretācijas, kuras pārbaudīt pēc pirmavotiem ne vienmēr bija iespējams, jo daudzi no tiem autoram nebija pieejami.
Attēli ilustrācijām arī aizgūti galvenokārt INTERNET-ā.

Rakstot šo darbu, autoru ierobežoja arī poļu valodas zināšanu trūkums.
Tā kā poļu dokumentus un vēsturnieku darbus viņam nācās lasīt tulkojumos vācu un krievu valodā vai ar datortulkotāja palīdzību, nav izslēgtas zināmas neprecizitātes to izklāstā, varēja rasties arī kļūdas dažādu viedokļu interpretācijās.
Iespējams, ka faktu atspoguļojumu avotos, dažādu tautību un valstiskās piederības vēsturnieku, publicistu darbos atrodamo secinājumu, notikumu traktējumu, personāžu vērtējumu vidū autoram nav izdevies atrast visprecīzāko, vēstures gaitai atbilstošo atainojumu.
Viņš uzņemas visu atbildību par trūkumiem un kļūdām grāmatas saturā.

Autors arī saprot, ka daudzi viņa vērtējumi izsauks poļu radikālnacionālistu neapmierinātību, iespējams, arī prasību mīkstināt Polijas valdošo slāņu kritiku, bet uz to var viņiem ieteikt atcerēties apustuļu Pētera un Pāvila atbildi uz prasību pārtraukt sludināt Kristus mācību~--- ``\latxti{Non possumus!}” (``Mēs nevaram!”), kā arī jau XVI gadsimtā izskanējušos M.~Lutera vārdus: ``\detxti{Hier stehe ich, ich kann nicht anders}” (``Šeit es stāvu un citādi nevaru”).
Vēloties būt godīgs vēsturnieks, autors atļaujas sekot nosaukto dižo personību piemēram.

Darba pilnveidē daudz palīdzēja zinātniskā redaktore Dr.~hist., RISEBA vadošā pētniece T.~Bartele.
Tā tehniskā noformēšanā neaizstājama bija dabaszinātņu maģistra, ārsta Ņikitas Skabcova palīdzība.
Viņiem autors izsaka vissirsnīgāko pateicību.
Diemžēl objektīvu iemeslu dēļ darbam bija jāiztiek bez literārā redaktora un korektora.

\chapter*{Grāmatā minēto galveno organizāciju, iestāžu un to saīsinājumu saraksts}
\addcontentsline{toc}{chapter}{Grāmatā minēto galveno organizāciju, iestāžu un to saīsinājumu saraksts}

Abreviatūras (saīsinājumi) tekstā dotas, vadoties no partiju, organizāciju utt. nosaukumiem gan latviešu, gan poļu valodā.
Piemēram: Polijas Apvienotā strādnieku partija (\pltxti{Polska Zjednoczona Partia Robotnicza})~--- saīsinājumā saukta par PASP, nevis PZPR, jo latviešu lasītājs līdz 1990.~gadam izdotajā literatūrā un presē bieži sastapās ar pirmo, latviskoto saīsinājumu.
Taču dažādās valodās apritē jau iegājušie saīsinājumi tieši no poļu valodas atstāti arī šajā darbā.
Piemēram, \pltxti{Armia Krajowa} (burtiski~--- Dzimtenes armija) vācu valodā tiek tulkota kā ``\detxti{polnische Heimatarmee}'', bet angļu dokumentos tās nosaukums tulkots dažādi: ``\entxti{Home Army}'', ``\entxti{Underground Army}'', ``\entxti{Secret Army}'', kas īsti neatbilst nosaukumam poļu valodā.
Grāmatā lietots saīsinājums AK, nevis DA.
Polijas Sociālistiskā partija (\pltxti{Polska Partia Socjalistyczna}), kuras nosaukums literatūrā parasti tiek saīsināts kā PPS, šādi dots arī šajā grāmatā, nevis kā PSP, kā būtu vajadzējis saīsināt latviešu valodā.
Tāpat īsināti arī daļa citu nosaukumu.
Tā gan ir atteikšanās no vienotas pieejas saīsinājumu darināšanā, taču tā kā minētajiem un arī citiem saīsinājumiem vēstures literatūrā jau ir savs skanējums, to mēģināt mainīt acīmredzot būtu aplami.

Poļu vārds ``\pltxti{Polska}'' (poļu un Polijas), dažādos nosaukumos tulkots dažādi, galvenokārt atkarībā no organizācijas, kuras nosaukumā tas ietverts, izveides laika.
Ja Polija kā valsts organizācijas izveides laikā nepastāvēja, tad parasti lietots pirmais variants, ja pastāvēja, tad otrais.
Taču ir arī atkāpes no šī principa.
Piemēram, 1892.~gadā dibinātā ``\pltxti{Polska Partia Socjalistyczna}'' tiek tulkota kā Polijas (nevis Poļu) Sociālistiskā partija, arī 1942.~gada janvārī nodibinātā ``\pltxti{Polska Partia Robotnicza}'' tulkota kā Polijas Strādnieku Partija, jo šie nosaukumi jau ir iegājuši literatūrā.

Lai atvieglotu darba lasīšanu, dažādas organizācijas, iestādes, politiskās partijas grāmatas teksta blokos vispirms parasti sauktas
pilnos nosaukumus, abreviatūras izmantotas tikai pēc tam.
Retāk pieminētajām organizācijām tās nav lietotas vispār, taču lasītāju ērtības labad to nosaukums un tā tulkojums sarakstā parasti ir ietverts.

\begin{footnotesize}
\noindent
\begin{tabularx}{\linewidth}{|p{3cm}|p{3.5cm}|p{1.4cm}|p{1.6cm}|}
\hline
\strong{Organizācijas vai iestādes nosaukums latviešu valodā} & \strong{Organizācijas vai iestādes nosaukums poļu (angļu vai krievu) valodā} & \strong{Saīsinājums} & \strong{Organizācijas vai iestādes pastāvēšanas gadi} \\
\hline
Apvienotā Zemnieku partija & \pltxti{Zjednoczone Stronnictwo Ludowe} & AZP & 1949--1989 \\
\hline
Apvienoto Nāciju Organizācija & \pltxti{Organizacja Narodów Zjednoczonych} & ANO & 1945--mūsdienas \\
\hline
Apvienoto nāciju palīdzības un atjaunošanas organizācija & \entxti{United Nations Relief ana Rehabilitation Administration} & UNRRA & 1943--1947 \\
\hline
Aktīvās cīņas savienība & \pltxti{Zwiazek Walki Czynnej} & ACS & 1908--1914 \\
\hline
Brīvība un neatkarība (Pilns nosaukums: Pretošanās bez kara un diversijām „Brīvība un neatkarība”) & \pltxti{Wolność i Niezawisłość (Ruch Oporu bez Wojny i Dywersji „Wolność i Niezawisłość”)} & WiN & 1945--1952 \\
\hline
Bruņotās cīņas savienība & \pltxti{Związek Walki Zbrojnej} & BCS & 1939--1942 \\
\hline
Bruņoto spēku delegatūra & \pltxti{Delegatura Sił Zbrojnych na Kraj} & --- & 1945 \\
\hline
Centrisko spēku savienība & \pltxti{Porozumienie Centrum} & PC & 1990--1997 \\
\hline
Civīlmilicija & \pltxti{Milicja Obywatelska} & MO & 1944--1990 \\
\hline
Civīlmilicijas Brīvprātīgā rezerve & \pltxti{Ochotnicza Rezerwa Milicji Obywatelskiej} & ORMO & 1946--1989 \\
\hline
Darba partija & \pltxti{Stronnictwo Pracy} & SP & 1937--1950 \\
\hline
Demokrātiskā partija &\pltxti{Stronnictwo Demokratyczne} & SD & 1939--mūsdienas \\
\hline
Demokrātiskā savienība & \pltxti{Unia Demokratyczna} & UD & 1990--1994 \\
\hline
Demokrātisko kreiso spēku savienība & \pltxti{Sojusz Lewicy Demokratycznej} & SLD & 1991--1999 (partiju savienība), kopš 1999~--- partija \\
\hline
Eiropas Cilvēktiesību tiesa & \entxti{European Court of Human Rights} & ECT & 1998--mūsdienas \\
\hline
Galīcijas un Cešinas Silēzijas poļu sociāldemokrātiskā partija & \pltxti{Polska Partia Socjalno-Demokratyczna Galicji i Śląska Cieszyńskiego} & --- & 1897--1919 \\
\hline
Komunistiskās Internacionāles Izpildu Komiteja & \entxti{The Executive Committee of the Communist International}; \rutxti{Исполнительный комитет Коммунистического Интернационала} & KIIK & 1919--1943 \\
\hline
Krievijas Federācija & \rutxti{Российская Федерация} & KF & 1991--mūsdienas \\
\hline
Krievijas Komunistiskā (boļševiku) partija & \rutxti{Российская Коммунистическая партия (большевиков)} & KK(b)P & 1918--1925 \\
\hline
Krievijas Sociāldemokrātiskā strādnieku partija & \rutxti{Российская социал-демократическая рабочая партия} & KSDSP & 1898--1918 \\
\hline
Krievijas Sociālistiskā Federatīvā Padomju Republika & \rutxti{Российская Социалистическая Федеративная Советская Республика} & KSFPR & 1918--1922 \\
\hline
Kristīgi Nacionālā apvienība & \pltxti{Zjednoczenie Chrześcijańsko-Narodowe} & ZChN & 1989--2010 \\
\hline
Ķirzakas savienība & \pltxti{Związek Jaszczurczy} & --- & 1939--1942 \\
\hline
Liberāli demokrātiskais kongress & \pltxti{Kongres Liberalno-Demokratyczny} & KLD & 1991--1994 \\
\hline
Likums un taisnīgums & \pltxti{Prawo i Sprawiedliwość} & PiS & 2001--mūsdienas \\
\hline
Nacionālā Demokrātija & \pltxti{Narodowa Demokracja~--- Endecja} & ND & 1887--1947 \\
(Tās nosaukumi dažādos laika posmos: & ~ & ~ & ~ \\
Poļu līga & \pltxti{Liga Polska} & --- & 1887--1893 \\
Nacionālā Līga & \pltxti{Liga Narodowa} & --- & 1893--1928 \\
Nacionāli demokrātiskā partija & \pltxti{Stronnictwo Narodowo-Demokratyczne} & --- & 1897--1919 \\
Nacionāli demokrātiskā savienība & \pltxti{Związek Ludowo-Narodowy} & --- & 1919--1928 \\
Nacionālā partija & \pltxti{Stronnictwo Narodowe} & SN & 1928--1947 \\
\hline
Nacionālās atmiņas institūts & \pltxti{Instytut Pamięci Narodowej} & IPN & 2000--mūsdienas \\
\hline
Nacionālā strādnieku partija & \pltxti{Narodowa Partia Robotnicza} & NPR & 1920--1937 \\
\hline
Nacionālās vienotības padome & \pltxti{Rada Jednosci Narodowej} & --- & 1944--1945 \\
\hline
Nacionālās vienotības pagaidu valdība & \pltxti{Tymczasowy Rząd Jedności Narodowej} & NVPV & 1945--1947 \\
\hline
Nacionālie bruņotie spēki & \pltxti{Narodowe Siły Zbrojne} & NBS & 1942--1947 \\
\hline
Nacionālie bruņotie spēki~--- Ķirzakas savienība & \pltxti{Narodowe Siły Zbrojne~--- Związek Jaszczurczy} & --- & 1944--1945 \\
\hline
Padomju Krievijas / PSRS Tautas komisāru padome & \rutxti{Совет народных комиссаров} & TKP & 1917--1946 \\
\hline
Padomju Savienības Komunistiskā partija & \rutxti{Коммунистическая партия Советского Союза} & PSKP & 1952--1991 \\
\hline
Padomju Savienības Telegrāfa aģentūra & \rutxti{Телеграфное агентство Советского Союза} & TASS & 1925--1991 \\
\hline
Pilsoniskā kustība~--- demokrātiskā darbība & \pltxti{Ruch Obywatelski Akcja Demokratyczna} & --- & 1990--1991 \\
\hline
Polijas Apvienotā strādnieku partija & \pltxti{Polska Zjednoczona Partia Robotnicza} & PASP & 1948--1990 \\
\hline
Polijas armija & \pltxti{Wojsko Poliskie} & --- & 1918--1989 \\
\hline
Polijas Jaunatnes savienība & \pltxti{Związek Młodzieży Polskiej} & PJS & 1948--1957 \\
\hline
Polijas karalistes un Lietuvas Sociāldemokrātiskā partija & \pltxti{Socjaldemokracja Królestwa Polskiego i Litwy} & PKunLSD & 1900--1918 \\
\hline
Polijas karalistes Sociāldemokrātija & \pltxti{Socjaldemokracja Królestwa Polskiego} & PKSD & 1893--1900 \\
\hline
Polijas Komunistiskā strādnieku partija & \pltxti{Komunistyczna Partia Robotnicza Polski} & PKSP & 1918--1925 \\
\hline
Polijas Komunistiskā partija & \pltxti{Komunistyczna Partia Polski} & PKP & 1925--1938 \\
\hline
Polijas Kristīgo demokrātu partija & \pltxti{Polskie Stronnictwo Chrześcijańskiej Demokracji}; \pltxti{Chrześcijańska Demokracja} & --- & 1919--1937 \\
\hline
Polijas Nacionālās atbrīvošanas komiteja & \pltxti{Polski Komitet Wyzwolenia Narodowego} & PNAK & 1944 \\
\hline
Polijas Republikas Nacionālā Padome & \pltxti{Rada Narodowa Rzeczypospolitej Polskiej} & --- & 1939--1941, 1942--1945 \\
\hline
Polijas Republikas pagaidu valdība & \pltxti{Rząd Tymczasowy Rzeczypospolitej Polskiej} & PRPV & 1945 \\
\hline
Polijas Republikas pilsoniskā platforma & \pltxti{Platforma Obywatelska Rzeczypospolitej Polskiej} & PO & 2001--mūsdienas \\
\hline
Polijas Republikas Sociāldemokrātija & \pltxti{Socjaldemokracja Rzeczypospolitej Polskiej} & PRSD & 1991--1999 \\
\hline
Polijas Sociālistiskā partija & \pltxti{Polska Partia Socjalistyczna} & PPS & 1892--1948 \\
\hline
Polijas Sociālistiskā partija~--- ļevica & \pltxti{Polska Partia Socjalistyczna~--- Ļevica} & PPS-ļevica & 1906--1918, 1926--1931 \\
\hline
Polijas Sociālistiskā partija~--- revolucionārā frakcija & \pltxti{Polska Partia Socjalistyczna~--- Frakcja Rewolucyjna} & PPS-frakcija & 1906--1919 \\
\hline
Polijas Strādnieku Partija & \pltxti{Polska Partia Robotnicza} & PSP & 1942--1948 \\
\hline
Polijas Tautas armija & \pltxti{Polska Armia Ludowa} & PAL & 1943--1944 \\
\hline
Polijas Tautas republika & \pltxti{Polska Rzeczpospolita Ludowa} & PTR & 1944--1989 \\
\hline
Polijas uzvaras dienests & \pltxti{Służba Zwycięstwu Polski} & --- & 1939 \\
\hline
Polijas valdība emigrācijā & \pltxti{Rząd Rzeczypospolitej Polskiej na uchodźstwie} & PVE & 1939--1945 (formāli~1990) \\
\hline
Polijas Zemnieku partija & \pltxti{Polskie Stronnictwo Ludowe} & PSL & 1945--1949, 1990--mūsdienas \\
\hline
Polijas Zemnieku partija~--- Jaunā atbrīvošanās & \pltxti{Polskie Stronnictwo Ludowe~--- Nowe Wyzwolenie} & PSL-NW & 1946--1947 \\
\hline
Polijas Zemnieku partija~--- kreisā frakcija & \pltxti{Polskie Stronnictwo Ludowe~--- ļewica} & PSL-levica & 1914--1924 \\
\hline
Politiskā konsultatīvā komiteja & \pltxti{Polityczny Komitet Porozumiewawczy} & --- & 1940--1943 \\
\hline
Poļu Kara organizācija & \pltxti{Polska Organizacja Wojskowa} & POW & 1914--1918 \\
\hline
Poļu Komunistu Centrālais Birojs & \pltxti{Centralne Biuro Komunistów Polskich} & PKCB & 1944 \\
\hline
Poļu patriotu savienība & \pltxti{Związek Patriotów Polskich} & PPtS & 1943--1946 \\
\hline
Poļu Sociālistiskā partija~--- Brīvība, Vienlīdzība, Neatkarība & \pltxti{Polska Partia Socjalistyczna~--- Wolność, Równość, Niepodległość} & PPS-WRN & 1939--1944 \\
\hline
Poļu sociālistu strādnieku partija & \pltxti{Rabotnicza Partia Polskich Socjalistów} & PSSP & 1943--1944 \\
\hline
Poļu sociālistiskā partija Prūsijā & \pltxti{Polska Partia Socjalistyczna Zaboru Pruskiego} & --- & 1893--1919 \\
\hline
Poļu Zemnieku partija „Atbrīvošanās” & \pltxti{Polskie Stronnitctwo Ludowe~--- „Wyzwolenie”} & PSL „\pltxti{Wyzwolenie}” & 1915--1931 \\
\hline
Poļu Zemnieku partija~--- Pjasts & \pltxti{Polskie Stronnictwo Ludowe~--- Piast} & PSL-Piast & 1913--1931 \\
\hline
PSRS Iekšlietu tautas komisariāts & \rutxti{Народный комиссариат внутренних дел СССР} & IeTK & 1934--1946 \\
\hline
PSRS Valsts Aizsardzības Komiteja & \rutxti{Государственный комитет обороны СССР} & PSRS VAK & 1941--1945 \\
\hline
Rietumbaltkrievijas Komunistiskā partija & baltkr.: \betxti{Камуністычная партыя Заходняй Беларусі} & RBKP & 1923--1938 \\
\hline
Rietumukrainas Komunistiskā partija & ukr.: \uktxti{Комуністична партія Західної України} & RUKP & 1919--1938 \\
\hline
Sabiedriskās drošības ministrija & \pltxti{Ministerstwo Bezpieczeństwa Publicznego} & SDM & 1945-1954 \\
\hline
Savstarpējās Ekonomiskās Palīdzības Padome & \rutxti{Совет экономической взаимопомощи} & SEPP & 1949--1991 \\
\hline
Starptautiskā Sarkanā Krusta komiteja & \entxti{International Committee of the Red Cross} & SSKK & 1863--mūsdienas \\
\hline
Strādnieku aizsardzības komiteja & \pltxti{Komitet Obrony Robotników} & KOR & 1976--1977 \\
\hline
Strādnieku aizsardzības komitejas Sociālās pašaizsardzības komiteja & \pltxti{Komitet Samoobrony Społecznej KOR} & KOS-KOR & 1977--1981 \\
\hline
Strādnieku Zemnieku Sarkanā armija & \rutxti{Рабоче-крестьянская Красная армия} & SZSA & 1918--1946 \\
\hline
Tautas armija & \pltxti{Armia Ludowa} & AL & 1944 \\
\hline
Tautas gvarde & \pltxti{Gwardia Ludowa} & GL & 1942--1943 \\
\hline
Ukraiņu nacionālistu organizācija & ukr.: \uktxti{Організація Українських Націоналістів} & OUN & 1929--mūsdienas \\
\hline
Ukrainas Sacelšanās armija & ukr.: \uktxti{Українська повстанська армія} & UPA & 1942--1953 \\
\hline
Ukrainas Tautas Republika & ukr.: \uktxti{Українська Народна Республіка} & UTR & 1917--1920 \\
\hline
Vācijas Demokrātiskā Republika & vācu: \detxti{Deutsche Demokratische Republik} & VDR & 1949--1990 \\
\hline
Vācijas Federatīvā Republika & vācu: \detxti{Bundesrepublik Deutschland} & VFR & 1949--mūsdienas \\
\hline
Valsts Nacionālā padome & \pltxti{Krajowa Rada Narodowa} & KRN & 1944--1947 \\
\hline
Valsts Politiskā pārstāvniecība & \pltxti{Krajową Reprezentację Polityczną} & --- & 1943--1944 \\
\hline
Vēlēšanu akcija Solidaritāte & \pltxti{Akcja Wyborcza Solidarność} & AWS & 1996--2003 \\
\hline
Viskrievijas Centrālā izpildu komiteja & \rutxti{Всероссийский Центральный Исполнительный Комитет} & VCIK & 1917--1938 \\
\hline
Vispolijas arodbiedrību savienība & \pltxti{Ogólnopolskie Porozumienie Związków Zawodowych} & VAS & 1984--mūsdienas \\
\hline
Vissavienības Komunistiskā (boļševiku) partija & \rutxti{Всесоюзная коммунистическая партия большевиков} & VK(b)P & 1925--1952 \\
\hline
Zemes armija & \pltxti{Armia Krajowa} & AK & 1942--1945 \\
\hline
Zemes politiskā pārstāvniecība & \pltxti{Krajowa Reprezentacja Polityczna} & --- & 1943--1944 \\
\hline
Zemnieku bataljoni & \pltxti{Bataliony Chłopskie} & BCh & 1940--1945 \\
\hline
Zemnieku partija & \pltxti{Stronnictwo Chłopskie} & SCh & 1926--1931 \\
\hline
Zemnieku partija „Roch” & \pltxti{Stronnictwo Ludowe „Roch”} & SL Roch & 1940--1945 \\
\hline
Zemnieku partija & \pltxti{Stronnictwo Ludowe} & SL & 1931--1939, 1944--1949 \\
\hline
Zemnieku partija „Tautas griba” & \pltxti{Stronnictwo Ludowe „Wola Ludu”} & SL-WL & 1943--1944 \\
\hline
Zemnieku pašpalīdzības savienība & \pltxti{Związek Samopomocy Chłopskiej} & ZPS & 1944--1957 \\
\hline
\end{tabularx}
\end{footnotesize}

\chapter*{Ievads}
\addcontentsline{toc}{chapter}{Ievads}

\epigraph
{Īsts patriots ir tas, kurš saka patiesību pat savai zemei.}
{Žans Žoress (\frtxti{Jean Léon Jaurès})}

\epigraph
{Tēvzemes mīlestība ir ļoti laba lieta, bet ir kas vēl augstāks pār to~--- patiesības mīlestība.}
{Pēteris Čaadajevs (\rutxti{Пётр Яковлевич Чаадаев})}

\epigraph
{Ir labi būt patiesam vienmēr, arī tad, ja tas skar dzimteni. Katra pilsoņa pienākums ir, ja vajadzīgs, mirt par savu dzimteni, bet nevienu nedrīkst piespiest melot dzimtenes vārdā.}
{Šarls Luijs de Monteskjē (\frtxti{Charles-Louis de Secondat, Baron de La Brède et de Montesquieu})}

\epigraph
{Uzbrukumi tautas trūkumiem un netikumiem nav noziegums, bet nopelns, īstens patriotisms.}
{Visarions Beļinskis\\(\rutxti{Виссарион Григорьевич Белинский})}



\epigraph
{Īsts patriotisms~--- tas pirmkārt ir vēlēšanās zināt un runāt patiesību par savu Tēvzemi.}
{Vjačeslavs Kostikovs (\rutxti{Вячеслав Васильевич Костиков})}

\epigraph
{Visas nelaimes pasaulē ceļas tādēļ, ka cilvēki lieto divējādas mērauklas: vienu sev un savai dzimtai, citu~--- pārējiem.}
{Zenta Mauriņa}

\epigraph
{Es dodu priekšroku savas dzimtenes šaustīšanai, tās sarūgtināšanai, tikai lai nekrāptu to.}
{Pēteris Čaadajevs (\rutxti{Пётр Я́ковлевич Чаада́ев})}

\epigraph
{Jebkurš nacionālisms ir akls, jo meklē savu nelaimju cēloņus citās tautās, tā sākas neiecietība, netaisnība. Nacionālisms nodara ļaunumu vispirms tā paudējiem, jo nenovēršami izraisa pretreakciju no citu tautu puses.}
{Ilga Apine}

\epigraph
{Visa mūsu vēsture~--- tas ir izdomājums, kuram visi piekrīt.}
{Voltērs (\frtxti{Voltaire, Fransuā Marī Aruē})}



\epigraph
{Kas gan ir vēsture, ja ne meli, ar kuriem visi ir vienisprātis?}
{Napoleons Bonaparts (\frtxti{Napoléon Bonaparte})}

\epigraph
{Gala rezultātā mēs pētām vēsturi lai apmierinātu savas intereses un, pēc iespējas, saprastu pie tam savas paša problēmas. Taču nevienu no šiem diviem mērķiem mēs nesasniegsim, ja, atrodoties neauglīgās zinātniskās objektivitātes idejas iespaidā, neuzdrošināsimies zinātniskās problēmas izskatīt no sava skatu punkta.}
{Kārlis Poppers (\entxti{Karl Raimund Popper})}

\epigraph
{Nevar būt vēstures „pagātnei, kāda tā patiesi pastāvēja”, ir iespējamas tikai vēsturiskas interpretācijas, un ne viena no tām nav galīga.}
{Kārlis Poppers (\entxti{Karl Raimund Popper})}

\epigraph
{Vieni no lielākajiem maldiem ir domāt, ka visi jūt, redz un domā tāpat, kā mēs.}
{Pjērs Buasts (\frtxti{Pierre Boiste})}

\epigraph
{Lai kā pieklājas rakstītu vēsturi, ir jāaizmirst par savu ticību, savu tēvzemi, savu partiju.}
{Pjērs Buasts (\frtxti{Pierre Boiste})}



\epigraph
{Mūsu ienaidnieku spriedumi par mums ir tuvāk patiesībai, nekā mūsu pašu.}
{Fransuā de Larošfuko (\frtxti{François de La Rochefoucauld})}

\epigraph
{Ticība tam, ka ir tikai viena, tikai tev zināma patiesība, ir visa ļaunuma sakne.}
{Makss Borns (\entxti{Max Born})}

\epigraph
{Patiesība pasaulē nedara tik daudz laba, cik šķietamas patiesības~--- ļauna. Tieksme izskaistināt vēsturi ir instinktīva tieksme paaugstināt grupas pašnovērtējumu.}
{Mihails Vellers (\rutxti{Михаил Иосифович Веллер})}

\epigraph
{Nobriedis prāts ienīst savas valsts netikumus~--- nepilnīgs noliedz tos. Godprātīgs vēsturnieks redz pienākumu izskaust savas zemes netikumus~--- negodprātīgs grib tikai iznīcināt tās ienaidniekus.}
{Mihails Vellers (\rutxti{Михаил Иосифович Веллер})}

\epigraph
{Poļu tautas rakstura varonīgās īpašības nedrīkst mūs piespiest aizvērt aci uz tās neprātīgumu un nepateicību, kas gadsimtu garumā tai radīja neizmērojamas ciešanas.}
{Vinstons Čērčils (\entxti{Winston Leonard Spencer-Churchill})}



\epigraph
{Katru nacionālistu vajā ideja, ka pagātni ne vien var izmainīt, bet tā ir jāizmaina.}
{Džordžs Orvels (\entxti{George Orwell})}

\epigraph
{Tas, ko mēs saucam par cinismu, bieži vien ir atklāti pateikta patiesība.}
{Jezups Laganovskis}

\epigraph
{Ciniķis ir cilvēks, kurš nebaidās pateikt to, ko citi domā.}
{Aleksandrs Gordons (\rutxti{Александр Гордон})}

\epigraph
{Polijas nelaime ir tā, ka tās vēsture ir tik melīga, kā nekas cits pasaulē.}
{Ježi Gedroics (\pltxti{Jerzy Władysław Giedroyc})}

\epigraph
{Atcerēsimies, ka Polija citu tautu vidū drīzāk ir liela pele, nekā mazs zilonis.}
{Tadeušs Kotarbiņskis (\pltxti{Tadeusz Marian Kotarbiński})}

\epigraph
{Kāda Centrāleiropa? Rietumāzija!}
{Josifs Brodskis (\rutxti{Иосиф Александрович Бродский})}

Polijai vienmēr ir bijusi nozīmīga vieta Eiropas vēsturē, ko noteica gan tās atrašanās pašā Eiropas centrā, gan iespaidīgais iedzīvotāju skaits, gan arī aktīvā politika. Daudzi svarīgi notikumi ir risinājušies Polijas teritorijā, poļu tauta ir bijusi un joprojām ir iesaistīta ne tikai Eiropas, bet arī visas pasaules mēroga procesos. Poļu sabiedrisko kustību, Polijas valsts politikas iedarbība bija un ir jūtama tālu aiz tās robežām, bet īpaši tajās valstīs, kuras atrodas tai kaimiņos, tai skaitā arī Latvijā.

\strong{Poļi} (pašnosaukums \pltxti{Polacy}) ir skaitliski lielākā rietumslāvu tauta. Pēc dažādiem vērtējumiem pasaulē dzīvo no 44 līdz 60 miljonu poļu, no tiem Polijas Republikā~--- 38 miljoni. Nosaukums radies no rietumslāvu cilts (maztautas) poļānu (\pltxti{polanie}) vārda, kas jau no VIII gadsimta dzīvoja Vartas (poļu \pltxti{Warta}) upes baseinā rajonā ap mūsdienu Poznaņu (poļu \pltxti{Poznań}, vācu \detxti{Posen}). Vārda \pltxti{polanie} tulkojums ir vienkārši „lauku (tīrumu) iedzīvotāji” (\pltxti{pole}~--- lauks, tīrums). Tieši poļānu vidū konsolidējās poļu tautas kodols, šeit IX gadsimtā radās pirmā Polijas valsts Pjastu (\pltxti{Piastowie}) dinastijas (10.--14.gs.) vadībā, un šeit atradās pirmās Polijas galvaspilsētas~--- Gņezno (\pltxti{Gniezno}), Poznaņa (\pltxti{Poznań}). Konsolidējot ap sevi citas rietumslāvu ciltis Vislas baseinā, pamazām veidojās kopīga poļu identitāte.

16.~gadsimta vidū poļi kopā ar lietuviešiem un slāvu ciltīm~--- mūsdienu ukraiņu un baltkrievu priekštečiem~--- radīja Polijas-Lietuvas apvienotu valsti jeb Abu Tautu Republiku (1569--1795). To parasti sauca par Žečpospolitu (poļu \pltxti{Rzeczpospolita}, no latīņu valodas: \latxti{Res publica}~--- republika. Pilns nosaukums: poļu: \pltxti{Rzeczpospolita Obojga Narodów}, lietuv.~--- \lttxti{Žečpospolita/Abiejų Tautų Respublika}, baltkr.~--- \betxti{Рэч Паспалітая Абодвух Народаў}). Tā bija muižnieku republika ar vēlētu monarhu priekšgalā, kur poļi bija valdošā tautība.

Mūsdienu poļu historiogrāfijā poļu valstiskuma dažādiem modeļiem pieņemts dot kārtas numurus: 1569.--1795.~gadu periods tiek saukts par I Žečpospolitu, no 1918. līdz 1939.~gadam~--- par II Žečpospolitu, no 1989.~gada~--- par III Žečpospolitu. Polijas Tautas Republikas pastāvēšanas periodu (1944--1989) šajā numerācijā neievēro vai arī lieto terminu \pltxti{Rzeczpospolita Ludowa} (Tautas republika), lai uzsvērtu tās atšķirību no citām poļu republikām.

Sāka veidoties poļu nācija. Taču tās izveide noslēdzās XVIII~gadsimta otrajā pusē un XIX~gadsimtā jau smagākos apstākļos, kad poļi cīnījās par atbrīvošanos no nacionālās atkarības.

Pakļautais Polijas stāvoklis attiecībās ar citām valstīm iezīmējās jau XVIII~gadsimta sākumā. Tā otrajā pusē Žečpospolita ekonomiskā un politiskā iekšējā vājuma, šļahtas (muižniecības) ķīviņu, etnisko nesaskaņu rezultātā tika trīs reizes kaimiņvalstu sadalīta un beidza pastāvēt. XIX gadsimtā tikai atsevišķos periodos Polijā pastāvēja valstiski veidojumi ar dažādu autonomijas pakāpi (Varšavas hercogiste 1807--1814, Polijas karaliste 1815--1830, Krakovas Republika 1815--1846, autonomā Galīcija pēc 1867.~gada). Šo veidojumu esamība ļāva kaut nelielā mērā saglabāt poļu valstiskuma pārmantojamu, tomēr neapmierināja radikālāk noskaņotos poļu patriotus. Viņi nepārtraukti cīnījās par savu neatkarību, vairākkārt sacēlās pret lielvalstu kundzību, taču tas tikai pasliktināja zemes stāvokli. Atbrīvošanās kustība izplatījās visās poļu zemēs, piepildīja augošo nacionālo pašapziņu ar cīņas apofeozi, bruņotas pretošanās, kritušo un nomocīto nacionālo varoņu kultu. Valsts ar 800~gadu vēsturi sadale ienesa arī pastāvīgu nemiera elementu pārējo Eiropas valstu attiecībās.

Īpaši bagāta sarežģījumiem bija Polijas vēsture XX gadsimtā. Polijas valstiskuma atjaunošana (1918) un Versaļas miers (1919) nekļuva par galīgu Polijas jautājuma risinājumu. Jaltas vienošanās (1945) atkal pakļāva Poliju lielvalstu diktātam. Polija ilgstoši atradās t.s. „sociālistiskā” bloka sastāvā, faktiskā atkarībā no PSRS. Tikai 1989.~gadā poļu tauta atkal atguva neatkarību.

Kā raksta Polijas vēsturi daudz pētījušais britu profesors N.~Deiviss: „Polijas nacionālajai kustībai bija vissenākā vēsture, visstiprākais mandāts, vislielākā apņēmība, vissliktākā reputācija un vismazākie panākumi”. Ir saprotams, ka atzīmētās pretrunas ir skaidrojamas gan ar objektīviem, gan subjektīviem, gan starptautiskiem, gan pašu poļu darbības radītiem faktoriem. Bagātās, bet sarežģītās vēstures dažāda uztvere pašu poļu vidū arī pati par sevi ir radījusi problēmas.

Ungāru vēsturnieks I.~Bibo ir atzīmējis īpašu Austrumeiropas tautu kolektīvu psiholoģijas iezīmi~--- sevišķu jūtīgumu, pat bailes no pastāvošām vai tikai šķietamām nacionālās kopības bojā ejas briesmām. Šīm tautām šīs bailes bija saistītas sākotnēji ar turku, tad vācu, dažkārt arī poļu ekspansiju, pēc tam krievu iespiešanos reģionā. Pēdējā gadsimtā šīs bailes galvenokārt saistījās ar Vāciju, Krieviju un PSRS. Lielajām tautām šīs bailes ir mazsaprotamas. Piemēram, krievu tauta ilgstoši ir bijusi apspiesta, bet parasti apspieda viņu savas valsts vara, kuru viņi neuztvēra kā etniski svešu, ja arī imperatoru dzīslās tecēja faktiski vācu asinis. Tāpēc daudzi krievi pagātnē un bieži arī mūsdienās nevar pilnībā saprast tās tautas, kuras baidās galvenokārt no etniski svešiem apspiedējiem, pat ja šo apspiedēju varā dzīvot bija labāk nekā savu valdītāju kundzībā. Tā, nevar strikti apgalvot, ka vairākumam poļu Prūsijas/Vācijas un Austrijas/Austroungārijas kundzībā dzīvot bija sliktāk nekā savu poļu „panu”~--- savas šļahtas varā līdz Žečpospolitas dalīšanai. Taču liela poļu daļa tās pašas šļahtas vadībā XIX~gadsimtā cīnījās pret nacionālo apspiestību, sava valstiskuma atjaunošanu ne tikai pret „aziātisko” Krievijas impēriju, bet arī „eiropeiskajām” Prūsiju/Vāciju un Austriju/Austroungāriju.

\strong{Polijas attiecības ar kaimiņvalstīm} un to tautām vienmēr ir bijušas neviennozīmīgas. Vissarežģītākās attiecības Polijai gandrīz vienmēr veidojās ar kaimiņos esošajām Vāciju un Krieviju.

Vispirms par \strong{Polijas attiecībām ar Vāciju} (\pltxti{Niemcy}). Vācieši ilgstoši ir bijuši vieni no tuvākajiem un arī bīstamākajiem poļu kaimiņiem. Poļi „vācieti” (\pltxti{niemiec}) gadu simteņiem uztvēra kā ienaidnieku.

1217.~gadā pāvests Honorijs III pasludināja krusta karu pret prūšu pagāniem, ar kuriem karoja Mazovijas (poļu \pltxti{Mazowsze, Mazowiecka ziemia}, vācu ~\detxti{Masowien}) hercogs Konrāds. 1225.~gadā hercogs lūdza palīdzību Teitoņu (Vācu) ordenim (latīņu \latxti{Ordo fratrum domus Sanctae Mariae Theutonicorum Ierosolimitanorum}, \latxti{Ordo Teutonicus}, vācu \detxti{Orden der Brüder vom Deutschen Haus St. Mariens in Jerusalem}, saīsināti: \detxti{Deutscher Orden}, \detxti{Deutscher Ritterorden}~--- katoļu reliģiskais bruņinieku ordenis, dibināts XII gadsimtā Palestīnā). Teitoņi ieradās Polijā 1232.~gadā, apmetās Vislas labajā krastā, sāka sagrābt prūšu zemes, pašus tos piespiežot pieņemt kristietību. Iekarotajās zemēs ieradās vācu kolonisti. Ordenis uzurpēja tiesības kristīt pagānus austrumos, noliedzot tādas pat tiesības poļiem. Galvenais ordeņa pretinieks bija Lietuvas lielkņaziste (lietuviešu \lttxti{Lietuvos Didžioji Kunigaikštystė}, poļu \pltxti{Wielkie Księstwo Litewskie}, krievu \rutxti{Великое княжество Литовское, Русское, Жемойтске и иных}; XIII~gs.--1796.). Lai atvairītu teitoņu triecienus, lielkņaziste 1385.~gadā Krēvas pilī (baltkrievu \betxti{Крэўскі замак}, poļu \pltxti{Zamek w Krewie}, lietuviešu \lttxti{Krėvos pilis}, mūsdienu Baltkrievijas teritorijā) noslēdza Polijas un Lietuvas personālo ūniju, kad abām valstīm bija viens valdnieks no Jagelloņu (lietuv.: \lttxti{Jogailaičiai}, poļu: \pltxti{Jagiellonowie}, baltkr.: \betxti{Ягелоны)} dinastijas. Tas mainīja spēku samēru par sliktu teitoņiem. Taču arī pēc Vācu ordeņa novājināšanās, Reformācijas un ordeņa zemju sekularizācijas vācu un poļu valstu attiecības neuzlabojās.

Gan vācieši, gan poļi kopš viduslaikiem netaupīja negatīvās emocijas pret kaimiņiem.

Jau Prūsijas karalis Fridrihs II, apkopojot poļiem naidīgos uzskatus, izteicās, ka sabiedrība, „kuras uzvārdi beidzas ar --ki”, ir „visās nozīmēs nicināma nācija'' (\detxti{Die ganze „Gesellschaft mit dem Namen auf~--- ki” ist „eine in jeder Hinsicht verächtliche Nation”}.) Prūsijas vēsturnieki poļus un citus slāvus apzīmēja par „barbarisma apustuļiem” („\detxti{Apostel der Barbarei}”). Oficiālais Prūsijas valsts historiogrāfs L.~Ranke uzskatīja, ka romāņu-ģermāņu tautām vienmēr ir bijusi noteicošā loma Eiropas vēsturē, bet kas attiecās uz slāviem, to loma bija aizstāvēt „Eiropas civilizāciju” no klejotāju uzbrukumiem. Šis zinātnieks uz slāviem, tai skaitā poļiem, raudzījās no augšas, uzsverot viņu kulturālo atpalicību un Vācijas civilizējošo lomu to vidū. Arī viņa pēctecis amatā H.~Treičke ģermāņu virzību slāvu zemēs nosauca par „kultūras tautu cīņu pret barbariem”. Vācu vēstures grāmatās Polijas vēsture tika reducēta uz anarhijas, iekšējo krīžu un cīņas pret kaimiņiem attēlojumu. Pozitīvie procesi, sasniegumi saimnieciskajā un kultūras jomā nebija pienācīgi novērtēti. (Arī mūsdienās vācu vēsturiskajā literatūrā var atrast šo stereotipu atliekas.) Polijas vēstures materiālam Vācijas valdošo slāņu acīs bija jākalpo vācu imperiālisma ekspansionistisko centienu pamatojumam.

Savukārt poļu vēsturnieki jau no XIX gadsimta oponēja prūsiskajai historiogrāfijai. Šajā pretstāvē piedalījās arī poļu literatūra un māksla, negatīvās krāsās attēlojot vācu pagātni. Viedokļu pretrunīgumu labi ilustrē divi mīti par kauju pie Grīnvaldes~--- Tannenbergas (poļu \pltxti{Bitwa pod Grunwaldem}, vācu \detxti{Schlacht bei Tannenberg}, 1410).

Pēc poļu mīta Teitoņu ordenis falsificēja Romas pāvesta bullu, pēc tam veica karagājienus pret prūšiem, Lietuvu un Poliju, 1308.~gadā ar viltu sagrāba Gdaņsku un Piejūras apgabalu, mēģināja pakļaut Lietuvu, kuru izglāba 1385.~gada personālā ūnija ar Poliju. Kauja pie Grīnvaldes bija mūžīgo Polijas cīņu pret Vāciju kvintesence. Pēc uzvaras Grīnvaldes kaujā Polijā valdījušās Jagelloņu dinastijas kļūda bija atļauja no krustnešu vasaļvalsts radīt mantojamu hercogisti, kura izveidojās par Prūsiju un vēlāk piedalījās Žečpospolitas dalīšanā. Arī Krievijas impērijā ietilpinātajā Polijā 1910.~gadā tika plaši atzīmēta Grīnvaldes kaujas jubileja.

Berlīnes ieņemšana 1945.~gadā bija otra Grīnvalde, Austrumprūsijas pievienošana Polijai uz visiem laikiem pielika punktu teitoņu agresijai. Pēc 1945.~gada Polijas jauniegūtajās rietumu zemēs laikam nebija pilsētas, kurā nebūtu savas Grīnvaldes ielas. Kaut oficiālā Polijas Tautas Republikas propaganda veidoja divus vāciešu tēlus: Poliju apdraudošus, atklāti agresīvus VFR pilsoņus, un sabiedrotos no VDR, jāsaka gan, ka katoliskie poļu iedzīvotāji juta maz simpātiju pret pēdējiem, kurus parasti identificēja kā prūšus.

Arī jau mūsdienās populāro poļu politiķu brāļu Ļeha un Jaroslava Kačinsku pirmsvēlēšanu kampaņa 2005.~gadā tika uzsākta uz J.~Matejko gleznas „Grīnvaldes kauja” fona.

Pilnīgi pretējs ir vācu mīts. Teitoņu ordenis neauglīgajās Eiropas ziemeļaustrumu zemēs, pārvarot pagānu pretestību, ar krustu un zobenu iedibināja kristietību, piesaistīja sava laikmeta modernās tehnoloģijas, nodibināja 96 pilsētas, uzcēla 90 cietokšņus, radīja sava laikmeta paraugvalsti, kuras mantinieces bija Prūsija un Vācijas impērija. (Vēl nacistiskajā t.s. Trešajā reihā darbojās Teitoņu ordeņa kults.) Nedienas sākās tad, kad Polija, jūtot skaudību pret teitoņu valsts panākumiem, noniecināja kristīgo solidaritāti, vienojās ar pagāniem. 1410.~gadā tā sāka karu pret ordeni un diemžēl kaujā pie Tannenbergas, galvenokārt pateicoties lietuviešu spēkiem, sakāva to. Taču vēsturiskais taisnīgums uzvarēja~--- XVII gadsimtā Prūsijas karaļiem izdevās nomest poļu kundzību, 18.~gadsimtā~--- atgūt agrāk zaudētās teritorijas, atbrīvojot tās no poliskās nekārtības.

Vācijā no XVIII gadsimta eksistēja termins „\detxti{polnische Wirtschaft}”~--- poļu saimniecība, kas, vācu acīm raugoties, nozīmēja pilnīgu nesaimnieciskumu. (Tiesa, pēc 1934.~gadā parakstītās Vācijas un Polijas deklarācijas par spēka nepielietošanu 1936.~gadā iznākušajā V.~Noltinga grāmatā bija norādīts, ka tajā laikā minētais termins esot jānodod „grabažu noliktavā”, taču vēl ilgi daudzi vācieši poļu ekonomiskos sasniegumus augstu nevērtēja.) XIX gadsimtā Vācija Prūsijas vadībā kļuva par valstiskuma, ekonomikas un kara mākslas augstāko sasniegumu. 19.gadsimtā otrajā pusē, kad poļi visiem spēkiem cīnījās par nacionālo pastāvēšanu, vācieši kā pirmo un galveno „senseno” ienaidnieku uzlūkoja frančus; tikai valsts austrumu teritorijās, kur dzīvoja daudz poļu, vācieši ar lielākām bažām vērās austrumu virzienā.

Pēc Pirmā pasaules kara bezdibenis starp vācu un poļu vēstures ainām vēl padziļinājās. Kā atzīmējis poļu vēsturnieks M.~Mročko, poļu priekšstatiem par „vācu virzību uz austrumiem” („\detxti{Deutschen Drang nach Osten}”) blakus pastāvēja vācu viedoklis par „slāvu pieplūdumu” („\detxti{Andrang des Slawentums}”) rietumos. Vācu vēstures zinātne Polijas vēsturi un politiku tēloja kā „spiedienu uz Rietumiem” („\detxti{Drang nach Westen”}) un kā „poļu briesmas” („\detxti{polnische Gefahr}”). Ievērojamais vācu vēsturnieks J.~Hallers 1923.~gadā iznākušajā kapitāldarbā „Vācu vēstures laikmeti'' („\detxti{Epochen der deutschen Geschichte}'') rakstīja: „Poļus mūs ar pilnām tiesībām uzskatām par vācu mūžseniem ienaidniekiem Austrumos''. („\detxti{Polen denken wir uns mit Recht als den Erbfeind der Deutschen im Osten}”.)

Arī mūsdienās, kad bundesvēra augstākie virsnieki noliek ziedus Vesterplatē (\detxti{Westerplatte}~--- pussala Baltijas jūrā, kur Otrā pasaules kara sākumā poļi varonīgi pretojās nacistiskajiem iebrucējiem), vāciešu vidū vēl nav izveidojies pamatots un daudz-maz vienots priekšstats par kaimiņtautas vēsturi. Vācu publicistikā arī XX gadsimta 80.~gados varēja lasīt dusmu pilnus vārdus par „nepateicīgajiem” poļiem, kuri gan saņēma no VFR materiālu palīdzību streikotājiem~--- protestētājiem pret „sociālistisko” pārvaldi Polijā, bet tik un tā neatzina pēckara vācu iedzīvotāju deportāciju netaisnīgumu, vietā un „nevietā” atgādināja vācu iebrucēju nodarījumus Polijas teritorijā utt. Kāda pirms Otrā pasaules kara Polijā dzīvojusi un no turienes izraidīta vāciete E.~Losere 1981.~gadā rakstīja: „Kopš kristīšanas [poļu] tauta tika pakļauta stingrai kleriķu uzraudzībai, kura traucēja personības attīstībai. Viņi nespēj izrauties no šiem spaidiem. Viņi tika tā apspiesti, ka uzkrāto agresiju arvien atkal un atkal izlādē bezprecedenta naidā pret brīvākajiem un bagātākajiem vāciešiem. \citespace{} Baznīca triumfē Polijā. Un katoļu baznīca no pašiem sākumiem bija Vācijas valsts niknākais ienaidnieks. Poļi tika un tiks izmantoti kā svira lai nolaistu vācu un Vācijas tautsaimniecības asinis”. Arī XXI~gadsimta sākumā publicistikā var atrast līdzīgus „zinātniskus'' spriedumus par citām tautām.

Kā Hamburgā izdotajā vācu-poļu žurnālā „\detxti{Dialog}” („Dialogs”) rakstījis poļu žurnālists Ā.~Kržeminskis, vāciešiem attieksme pret Poliju un poļiem arī XXI gadsimtā ir atmiņas politikas (\detxti{Gedächtnispolitik}) pārbaudes līdzeklis. Polija taču bija pirmā valsts, kura ar ieročiem cīnījās pret nacistiskās Vācijas agresiju, no pirmās dienas bija antihitleriskās koalīcijas locekle, cieta kara laikā milzīgus cilvēku un materiālos zaudējumus, un, kaut arī ne no uzvarētājām lielvalstīm, ne arī Vācijas puses netika atzīta par ar tām vienlīdzīgu, tomēr kļuva par Vācijai tik sāpīgi zaudēto teritoriju ieguvēju. Taču vāciešiem atmiņās par Otro pasaules karu Polijā ir periferiāla vieta. Kā norādījis ievērojamais vācu vēsturnieks H.U.~Vēlers, Vācijas abiturientiem un studentiem ir zināšanas par Otrā pasaules kara laikā nogalinātajiem sešiem miljoniem ebreju, taču kad viņiem saka, ka tai pat karā dzīvību zaudēja gandrīz katrs piektais polis [precīzāk gan būtu teikt~--- Polijas pavalstnieks~--- V.Š.], un jau kara sākumā no vācu okupētajiem apgabaliem tika padzīti 800~000 iedzīvotāju, nākas sastapties ar nezināšanu un izbrīnu. Sociologs un vēsturnieks Rietumu institūta direktors Poznaņā A.~Saksons raksta, ka vēl daudz ūdens ir jāaizplūst robežupē Oderā līdz vācieši mainīs savus uzskatus par poļiem. XX gadsimta beigās veiktā socioloģiskā aptauja rādīja, ka 50\% vāciešu nav priekšstata par Poliju vai arī viņiem tā neinteresē. 2000.~gadā Polija viņu acīs bija pēdējā vietā no 26 pievilcīgākajām Eiropas valstīm.

[Pamatīgāks materiāls par Polijas vēstures izpēti vācu zinātniskajā literatūrā atrodams G.~Rodes rakstā „\detxti{Der Geschichte Polens in der deutschen Geschitsschreibung}” („Polijas vēsture vācu vēstures literatūrā'') krājumā „\detxti{Nationalgeschichte als Problem der deutschen und polnischen Geschichtsschreibung”}. Braunschweig, 1983.) Par Polijai veltītās vācu historiogrāfijas vērtējumu no pašu poļu viedokļa var izlasīt H.~Olševska rakstā „\detxti{Die deutsche Historiographie über Polen aus polnischen Sicht}” („Vācu historiogrāfija par Poliju no pašu poļu viedokļa”) rakstu krājumā \detxti{Dittmar Dahlmann (Hg)} „\detxti{Hundert Jahre Osteuropäische Geschichte. Vergangenheit, Gegenwart und Zukunft}''. Stuttgart, 2005].

Negatīvi domāja arī poļi par vāciešiem. Ne velti 1939.~gada 1.~septembrī Polijas prezidenta I.~Moscicka runā izskanēja vārdi: „mūsu senais ienaidnieks ir uzsācis karadarbību pret Polijas valsti”. Poļi labi atceras, ka Varšava, faktiskā Polijas galvaspilsēta kopš 1596.~gada, pēdējos divarpus gadsimtos trīsreiz (1795., 1915., 1939.~gg.) krita vācu militāristu rokās. Aušvicas (\detxti{Auschwitz}) jeb Osvencimas, kā arī daudzas citas nacistu ierīkotās koncentrācijas nometnes atradās Polijā un arī šodien atgādina saviem apmeklētājiem par vācu iebrucēju noziegumiem kopš krustnešu laikiem līdz pat Otrajam pasaules karam, kad vārds „vāciešu” kļuva par sinonīmu vārdam „noziedzīgs”.

Protams, vēsturē atrodami daudzi fakti arī par poļu un vāciešu sadarbību. Patiesība ir tā, ka naidīguma izvirdumiem starp vāciešiem un poļiem un, vēl lielākā mērā, starp viņu politiķiem, var nostādīt pretī periodus, kad abas tautas prata sadzīvot draudzīgi. Piemēram, Boļeslavs Drosmīgais sniedza militāru palīdzību Svētās Romas impērijas ķeizaram Ottonam III, pats tika kronēts pateicoties tā atbalstam. Abi personīgi tikās t.s. Gņezno (poļu \pltxti{Gniezno}, vācu \detxti{Gnesen}) kongresā (1~000), kur tika radīta patstāvīga Gņezno arhibīskapija, kas Polijai garantēja baznīcas neatkarību no vācu baznīcas. Mūsdienās pārsvaru gūst uzskats, ka pretēji abām augstāk izklāstītajām leģendām par Grīnvaldes kauju XV~gadsimtā sadursme starp Poliju un krustnešiem nebija konflikts starp nācijām vai pat civilizācijām, bet gan starp valstīm. Pie tam abi konkurējošie bloki: gan Polija un Lietuva, gan Teitoņu ordenis, kuru atbalstīja arī Svētā Romas impērija un Čehijas karaliste, pārstāvēja daudzus etnosus.

Vācu un poļu kultūras ir savstarpēji bagātinājušas viena otru. Lielākoties vācu (un ebreju) amatnieku apdzīvotās Polijas pilsētas pārņēma vācu pilsētu Lībekas, Magdeburgas, Nirnbergas vai Halles tiesības. Neviena valoda, izņemot radniecīgo krievu valodu, nav tik stipri ietekmējusi poļu valodu kā vācu valoda. Vācu „\detxti{Rathaus}'' (rātsnams) poliski ir „\pltxti{ratusz}”, „\detxti{Bürgermeister}” (birģermeistars)~--- „\pltxti{burmistrz}” un „\detxti{Pflug}” (arkls)~--- „\pltxti{plug}”. Poļu valodnieks G.~Korbuts ir atzinis, ka no 100 poļu sarunu valodas vārdiem 16--17 ir vācu izcelsmes. Savukārt vācu valodā tādi vārdi kā „\detxti{Dolmetscher}” (tulks), „\detxti{Droschke}” (važoņa rati), „\detxti{Gurke}” (gurķis), „\detxti{Peitsche}” (pātaga), „\detxti{Quark}” (biezpiens), „\detxti{Zobel}” (sabulis) u.c. ir pārņemti no senās poļu valodas. Polijas nacionālās himnas mūzikas autora ģenerāļa J.~Dombrovska māte bija vāciete. Himnas teksta autors ģenerālis J.~Vibickis cienīja vācu kultūru un izaudzināja savus dēlus Drēzdenē vācu klasikas garā. Tādu piemēru uzskaitījumu varētu turpināt.

Taču, kaut arī šodien „mūžsenā ienaidnieka” tēls abās pusēs ir krietni pabālējis, daži tam raksturīgi priekšstati ir vēl dzīvi. Poļiem ir paruna: „\pltxti{Póki swiat swiatem, Polak Niemcowi nie bedzie bratem}”. („Tik ilgi, kamēr pasaule pastāvēs, polis nekad nebūs vācietim brālis.'') Īpaši negatīvas emocijas poļos izraisa jēdziens ''\pltxti{Prusy}''(prūši). Līdz mūsdienām Varšavā nav Berlīnes vai Prūšu ielas. Toties ir Sakšu pils, Sakšu parks, Sakšu ass (ceļš), Leipcigas iela un Drēzdenes iela. Poļu--sakšu tradicionālās attiecības ir dzīvas vēl šodien. Vārdam „saksis”, atšķirībā no jēdziena „prūsis” poļu ausīs ir pozitīva nozīme. Tomēr, domājot par Vāciju, poļi vispirms atceras Prūsiju un tās iekarojumus Polijā.

Kā atzīmējis populārais poļu publicists A.~Kržeminskis, ciktāl paši poļi skata savas vēstures tumšās lappuses un kritizē daudzus nacionālos mītus, tiktāl viss ir kārtībā, īpašu saasinājumu parasti nav. Taču, ja kritikai pievienojas kāds vācietis, pieminot, piemēram, poļu represijas pret vācu gūstekņiem un nosoda vācu deportācijas no pēckara Polijas, poļu puse nonāk līdz pat publiskiem skandāliem.

Ilgstoša saskarsme poļiem ir bijusi arī ar krieviem (\pltxti{rosjanie}) un Krieviju (\pltxti{Rosja}). Taču divas kaimiņos dzīvojošas lielas radniecīgas tautas vēstures gaita saistīja ar dažādām kristīgās ticības konfesijām, dažādiem kultūras virzieniem, kas saasināja to attīstības nevienmērību. Un tāpat kā daudzos citos gadījumos, kad starp radniecīgām tautām (Izraēlas un arābu valstu iedzīvotājiem, serbiem un horvātiem u.c.), arī starp poļiem un krieviem ilgstoši pastāv it kā nepārvaramas pretrunas. Neatkarīgās Polijas valsts vēsturē grūti atrast periodus, kad tā būtu ilgstoši sadarbojusies ar Krieviju.

Polijas un Krievijas kopīgā vēsture neļauj ne vienu no tām nosaukt tikai par upuri, ne otru tikai par agresoru. Nebūt ne vienmēr poļi ir bijuši austrumu kaimiņa agresijas objekts, kā tas šķiet daudziem Polijas iedzīvotājiem, bieži viņi paši ir tam uzbrukuši. Ne reizi vien vēstures gaitā Polija ir savā labā izmantojusi Krievijai grūtas situācijas.

No XI līdz XVII gadsimtam poļi daudzkārt iebruka Krievzemes robežās, sagrāba milzīgas teritorijas un gadsimtiem ilgi ekspluatēja tās, tai laikā kad krievu karaspēks šai laikā ne reizi nenonāca pašas Polijas teritorijā.

Tā Kijevas Krievzemes laikos poļu karalis Boļeslavs Drosmīgais pēc sava znota bijušā Kijevas kņaza Svjatopolka, kurš bija zaudējis cīņu savam brālim Jaroslavam Gudrajam, aicinājuma devās uz Volīniju, sakāva Jaroslava Gudrā karadraudzi, 1018.~gadā ieņēma Kijevu, bet tā vietā lai atdotu to valdīšanā savas meitas vīram Svjatopolkam, pats sāka tajā valdīt. Taču pilsētnieki, sašutuši par poļu karadraudzes patvaļībām, sacēlās un Boļeslavs bija spiests atstāt pilsētu. Iejaukties Kijevas Krievzemes lietās mēģināja arī Boļeslavs II Drošais, taču pēc sadursmēm ar vietējiem iedzīvotājiem bija spiests savu karadraudzi izvest.

Jau dziļāku poļu un krievu pretrunu rašanās saistāma ar Maskavas Krievzemes jeb Maskavijas ģenēzi. Vēl XIII--XIV gadsimtā Lietuvas valsts atkaroja Zelta Ordai (mongoļu \mntxti{Altan Ord}, tatāru \tttxti{Altın Urda}, mongoļu-tatāru valsts XIII--XV~gs.) lielu daļu no senkrievu zemēm. Tieši Lietuvas valsts toreiz pretendēja uz šo zemju vienotājas lomu, taču pēc jau minētās 1385.~gadā noslēgtās Krēvas ūnijas arī Lietuvā izplatījās katoļu ticība, sākās senkrievu zemju polonizācija. Tāpēc Lietuva senkrievu acīs pakāpeniski zaudēja „savas” valsts tēlu, atbalsta lomu cīņā pret svešzemju jūgu; par tādu kļuva Maskavija, kura palika uzticīga pareizticībai.

1480.~gadā Lietuvas lielkņazs un Polijas karalis Kazimirs IV mēģināja ieņemt Novgorodu un Pleskavu. Protams, arī krievu kņazi veica karagājienus pret poļiem, taču tiem faktiski bija savu zemju aizsardzības un arī atriebības raksturs.

Turpmāk no 1558.~gada (Livonijas kara sākums) līdz 1939.~gadam (oficiālais Otrā pasaules kara sākums) Polija ar Krieviju karoja 13 reizes, pie tam poļi 2 karus uzvarēja un 11 zaudēja.

Pēc 1569.~gadā noslēgtās Ļubļinas ūnijas (poļu \pltxti{Unia Lubelska}, lietuviešu \lttxti{Liublino unija}, baltkrievu \betxti{Люблінская унія}~--- līgums par Polijas karalistes un Lietuvas lielkņazistes apvienošanos konfederatīvā valstī~--- Žečpospolitā ar vēlētu karali) sākās gadsimtiem ilgusī tās un Maskavas valsts sāncensība par austrumslāvu zemju apvienošanu un pakļaušanu.

Var teikt, ka ap 200 gadus XVI --XVII gadsimtā Maskavija izjuta Žečpospolitas spiedienu. Žečpospolita uzskatīja sevi par galveno katoļu ticības izplatītāju austrumos, turpretī Maskavas valsts cīnījās ne tikai par „krievu zemju” vienību, bet arī „visu pareizticīgo” kristiešu interesēm. Maskavija un Polija saskatīja viena otrā antagonistu slāvu zemēs. Polijai kaimiņos Krievijā pastāvēja Maskavas kā „trešās Romas” koncepcija, vadoties no XV~gadsimtā izskanējušās tēzes, ka „Divas Romas [Roma un Bizantija] ir kritušas, trešā [Maskava] stāv, bet ceturtās nekad nebūs” („\rutxti{«Два Рима пали, третий стоит, а четвёртому не бывать}”). XVI gadsimtā Polijas-Lietuvas valsts robeža atradās netālu no Možaiskas. No Livonijas kara (1558--1583) laikiem, kad Maskavija un Žečpospolita cīnījās par ietekmi Baltijā un Stefans Batorijs neveiksmīgi mēģināja ieņemt Pleskavu, abu valstu savstarpējās cīņas gan apdzisa, gan uzliesmoja ar jaunu sparu.

Tās saasinājās it īpaši pēc Polijas iejaukšanās Krievijas lietās t.~s. „Juku laikos” XVII gadsimta sākumā, kad poļu karaspēks okupēja Maskavu: 1604.~gadā kopā ar Viltusdmitriju (\rutxti{Лжедимитрий}, ?--1606) un 1610.--1612.~gadā „cara” Vladislava laikā. XVII gadsimtā poļi un lietuvieši tik bieži un ilgi necīnījās ne ar vienu citu ienaidnieku kā ar Maskavas valsti. XVII gadsimta sākumā Polijā bija populāra atsaukšanās uz spāņu konkiskadoru veiktajiem iekarojumiem Amerikā. Tika spriests: spāņu taču bija ļoti nedaudz salīdzinājumā ar indiāņiem, bet viņi tos uzvarēja. Poļu ir daudz, vai nu viņi netiks galā ar „moskaļiem” [nicīgs krievu nosaukums, lietots poļu, ukraiņu u.c. vidū.~--- V.Š.]? Nabadzīgā poļu šļahta cerībā uz bagātīgu laupījumu devās palīgā Viltusdmitrijam uz Maskavu. Vairākkārt Polija ir bijusi tuvu savam mērķim nostiprināt robežu ar Krieviju pa Dņepras upi.

Nobeļa prēmijas laureāts krievu rakstnieks A.~Solžeņicins, izklāstot abu valstu savstarpējos pāridarījumus, sāka ar Krievijas puses nodarījumiem Polijai, taču pēc tam rakstīja arī par to, ka „iepriekšējos gadsimtos plaukstošā, stiprā, pašpārliecinātā Polija ne īsāku laiku un ne vājāk iekaroja un apspieda mūs [krievus~--- V.Š.]”. Īpaši viņš izcēla „juku laikus” Krievijā XVII gadsimta sākumā, kad „poļi teju neatņēma mums nacionālo neatkarību, šo briesmu dziļums bija ne mazāks kā tatāru iebrukumam, jo poļi apdraudēja arī pareizticību. Pie sevis viņi to sistemātiski apspieda, dzina ūnijā [domāta 1596.~gadā noslēgtā Brestas ūnija~--- V.Š.] \citespace{} Mūsu juku laikos Polijas ekspansiju poļu sabiedrība uztvēra kā normālu un pat pareizu politiku. Paši sevi poļi iedomājās kā Dieva izredzētu tautu, kristietības bastionu, ar uzdevumu izplatīt īstenu kristietību puspāgāniskajos pareizticīgajos, mežonīgajā Maskavijā, būt par universitāšu renesanses kultūras izplatītājiem”.

Ievērojamais krievu vēsturnieks V.~Kļučevskis apgalvoja, ka polis un tatārs bija īstena krievu cilvēka pastāvīgi ienaidnieki arī XVIII gadsimtā. Pēc vēsturnieka un publicista V.~Fiļeviča vārdiem viduslaikos „krievi sātanu iedomājās poļa izskatā” („\rutxti{русские беса представляли в виде ляха}”). Taču katru reizi poļu ekspansijai sekoja prettrieciens.

Pēc Romanovu (\rutxti{Романовы}) dinastijas nodibināšanas (vēlāk Holšteinas-Gotorpas-Romanovu dinastija~--- \rutxti{Гольштейн-Готторп-Романовская династия}), kura valdīja Krievijas caristē, pēc tam Krievijas impērijā laikā no 1613.~gada līdz 1917.~gadam, tā pakāpeniski atkaroja austrumslāvu zemes.

Dažkārt Polija un Krievija gan bija arī sabiedrotās, piemēram, XVII gadsimta beigās pret kopējo ienaidnieku~--- Turciju, Ziemeļu karā (1700--1721) pret Zviedriju. Taču tad atkal priekšplānā izvirzījās nesaskaņas. Pateicoties cara, vēlāk imperatora Pētera I īstenotajai modernizācijai, Krievija izvirzījās priekšā kaimiņvalstij. XVIII~--- XX gadsimtā jau tā diktēja poļiem politiskās spēles noteikumus. Krievijas robežas rietumos atvirzījās aiz Vislas. XVIII~gadsimtā Žečpospolita zaudēja savu valstiskumu. Taču poļu šļahta nevarēja samierināties ne ar savas neatkarības, ne ar citu tautu apdzīvoto, bet līdz tam Žečpospolitas varā esošo teritoriju zaudējumu, meklēja pret Krieviju sabiedrotos Centrālajā un Rietumu Eiropā. 1812.~gadā poļi, karojot nu jau Napoleona I (\frtxti{Napoléon I Bonaparte}, 1769--1821) karaspēka sastāvā, atkal nonāca Maskavā.

Divreiz XIX gadsimtā uzliesmoja varenas poļu sacelšanās pret Krievijas varu. Kaut poļi cīnījās par savu neatkarību pret visām to sadalījušajām valstīm, taču galvenais ienaidnieks bija Krievija, kura bija ieguvusi centrālos Polijas apgabalus. Iekšējās pretrunas gan Krievijā, gan Polijā neļāva tām miermīlīgi izšķirties pēc Pirmā pasaules kara un Oktobra apvērsuma Krievijā. Daudz sarkanarmiešu zaudēja dzīvību 1920.~gadā pie Varšavas. Poļi kā vienu no svarīgākajiem notikumiem Otrā pasaules kara gados atceras 1939.~gada 17.~septembrī notikušo Sarkanās armijas invāziju Polijas teritorijā, pēc kara ar izteiktu neapmierinātību pārcieta PSRS valdošā staļiniskā režīma ietekmes izplatību uz Poliju. Tā tūkstošgadīgajā līdzāspastāvēšanas laikā divu slāvu tautu attiecībās uzkrājās ne mazums notikumu, kurus katra puse traktēja citādi. Uz vēsturisko notikumu subjektīvas uztveres pamata formējās vispārnacionāli mīti. Krievijā~--- mīts par naidīgajiem poļiem-katoļiem, kuri jebkurā situācijā ir gatavi darboties tā, lai ieriebtu „moskaļiem”, Polijā~--- mīts par vienmēr to apdraudošo, barbarisko un impērisko Krieviju.

No dotā darba, kas veltīts Polijas XIX un XX gadsimta vēsturei, autora viedokļa ļoti būtisks ir jautājums par poļu un krievu tautu attiecībām minētajā laikā. Tieši to samežģījumi ir nesuši poļu tautai daudz ciešanu.

Par šīm attiecībām, par daudzo Polijas un Krievijas konfliktu būtību rakstījuši dažādu tautu vēsturnieki un publicisti. Dažu no viņiem skaidrojumi nevar tikt uzskatīti par zinātniskiem.

Piemēram, baltkrievu publicists A.~Tarass, atsaucoties uz nenosauktu zinātnieku ģenētiskajiem pētījumiem, raksta, ka „krievi (no vienas puses), baltkrievi un poļi (no otras puses) ir etniski atšķirīgas tautas, antropoloģiski dažādas rases” un arī „tieši iedzimtas (ģenētiskas) atšķirības izšķiroši veicināja maskaviešu (vēlāk krievu) tradicionālo noraidīšo attieksmi pret Eiropas dzīvesveidu, Eiropas kultūru, katolicismu, uniātismu un protestantismu, viņu naidu pret Lietuvas Lielkņazisti, Kungu Lielo Novgorodu, Livoniju, Polijas Karalisti.” Minētais autors arī apgalvo, ka „tieši poļu un baltkrievu etniskās kopības fakts izskaidro Žečpospolitas izveidi”. Identificējot baltkrievus ar poļiem, A.~Tarass savos darbos vēsturiskos notikumus skata no polonopfīlām un rusofobām pozīcijām. Tā kā autors nenosauc viņa uzskatus apstiprinošo „ģenētiķu” vārdus, hipotēze šai grāmatā netiek komentēta, vienīgi jāsaka, ka šāds vienkāršots sarežģītu procesu izskaidrojums ir tāls no patiesības.

Daļa vēsturnieku uzskata, ka galvenā bija tīri politiska konkurence par dominanci Austrumeiropā.

Citi aizstāv viedokli, ka visu nesaskaņu pamatā bija nesamierināma cīņa starp kristietības austrumu un rietumu atzariem. Reizē līdzās reliģiskajai konfrontācijai XVII gadsimta notikumos tiek izšķirtas arī etniskam konfliktam raksturīgas pazīmes. Tā, pazīstamais poļu izcelsmes amerikāņu vēsturnieks R.~Paips atzīmējis: „Vēsturiskajā perspektīvā poļu--krievu attiecības nekad nav bijušas labas. Starp \citespace{} [šīm] valstīm vienmēr radās sasprindzinājums. \citespace{} Krievija uzskata Poliju par slāvu tradīciju nodevēju, jo \citespace{} [tā] pieņēma katoļu ticību. Savukārt poļi nemīl krievus, jo pārāk daudz ir no viņiem cietuši”.

Nebūt neatbilst patiesībai priekšstats, ka lielas daļas poļu naidīgā izturēšanās pret visu krievisko radās kā sekas trijām Polijas dalīšanām XVIII gadsimtā, poļu sacelšanos apspiešanām XIX gadsimtā, padomiskā „sociālisma” modeļa uzspiešanai, veselai virknei notikumu, kuros cietusī puse bija poļu tauta. Kā parāda poļu vēsturnieces A.~Neviaras 2006.~gadā publicēts pētījums (Niewiara A. Moskwicin—Moskal—Rosjanin w dokumentach prywatnych. Łódź, 2006) „moskaļa-aziāta”, „moskaļa-iebrucēja,” „moskaļa-Kristus nodevēja” tēls parādās poļu rakstītajos avotos vēl sen pirms XVIII~gadsimta, kad vēl nekāda poļu apspiešana no krievu puses nebija iespējama. Līdz Ļubļinas ūnijai (1569) Polijai nebija robežas ar Krieviju, tā bija Lietuvas lielkņazistei. Pēc ūnijas noslēgšanas ar Krieviju konfliktus risināja jau apvienotā Polijas-Lietuvas valsts (Žečpospolita). Ja līdz tam Polijas galvenie ienaidnieki bija rietumos (Vācija) un dienvidos (Turcija), tad tagad tā iesaistījās karos arī austrumos (pret Krieviju). Tad arī itin dabīgi radās poļu šļahtas naids pret „moskaļiem”. Tas bija ne Polijas apspiešanas no Krievijas puses rezultāts (kaut tā to sekmēja), bet jau senāks poļu šļahtas kolektīvo interešu izpausmes veids.

Līdz „sociālisma” padomju modeļa bankrotam, PSRS sabrukumam, Polijas Tautas republikas pārveidei, historiogrāfijā poļu-padomju attiecības tika skatītas no divām pretējām, savstarpēji izslēdzošām pieejām.

Viena bija pārstāvēta PSRS un PTR vēsturnieku darbos, otra Rietumu pētnieku, poļu emigrantu vidū dzīvojošo vēsturnieku, kā arī to poļu pētnieku monogrāfijās un rakstos, kuri, bieži ar pseidonīmiem, publicējās PTR valdības cenzūrai nepakļautajā vēsturiskajā literatūrā.

Pirmā koncepcija sāka veidoties jau Otrā pasaules kara laikā un bāzējās uz padomju vadības oficiālo pozīciju, kura bija pārstāvēta plaši pieejamos politiskajos un diplomātiskajos dokumentos, padomju līderu runās. Autori, kuri pieturējās šai koncepcijai, PSRS realizēto ārpolitiku, arī pret Poliju, vērtēja kā miermīlīgu, progresīvu, internacionālistisku. Turpretī Polijas Republikas starpkaru un Otrā pasaules kara laikā realizētā politika tika skatīta kā Padomju Savienībai atklāti naidīga, nacistiskās Vācijas agresīvos centienus veicinoša. Visspilgtāk šī tendence atklājās padomju vēsturnieku darbos, veltītos Polijas vēstures apkopojumam, Otrajam pasaules karam, ārpolitikas un diplomātijas vēsturei. Analoģisku pozīciju ar nelielām niansēm ieņēma oficiālā Polijas historiogrāfija 1940.~gadu beigās~--- 80.~gados.

Otrās koncepcijas piekritēji par pamatu ņēma uzskatus, gluži pretējus augstāk izklāstītajai pieejai. Viņi viennozīmīgi saskatīja padomju ārpolitikā agresīvo, pretpolisko ievirzi, kā starpkaru, tā arī Otrā pasaules kara gados, īpaši uzsvēra represijas pret poļiem, to deportācijas, poļu virsnieku nošaušanu Katiņā u.~c. Pirmie tādi pētījumi parādījās jau drīz pēc Otrā pasaules kara. Visaptverošāk šī koncepceja izklāstīta V.~Pobog-Maļinovska darbā „Polijas jaunākā politiskā vēsture”. („\pltxti{Najnowsza historia polityczna Polski} 1864--1945”, 1--3, Paryż, Londyn, 1953.--1962.) Jāsaka, ka PSRS vadības nostāja tikai veicināja Rietumos izplatīto uzskatu izplatību arī pašā Polijā. Tā vietā, lai atslepenotu dokumentus, publicētu avotus par Polijas un Krievijas/PSRS attiecību vēsturi, atspēkotu mītus par to, tika vienkārši klusēts. Tas veda pie secinājuma~--- ja jau PSRS klusē, tad to apsūdzošie notikumu izklāsti ir patiesi, bet varbūt patiesība bija vēl briesmīgāka.

Abas koncepcijas pārstāvošajiem darbiem bija raksturīga klaja tendenciozitāte, tieksme padarīt baltu savu pusi un nomelnot pretējo, plaša sabiedriska fona iztrūkums, savstarpēja pretējās puses nacionālo interešu ignorēšana.

Vēstures mantojums arī mūsdienās bremzē labu kaimiņattiecību veidošanos starp Poliju un Krieviju. Pēc Otrā pasaules kara Polijas Tautas Republikas (1952--1989) historiogrāfijā nereti tika uzsvērts, ka neraugoties uz to, ka Krievija un Polija vēsturiski ilgus gadsimtus bija ienaidnieki, abu tautu attiecībās pārsvarā esot bijušas simpātijas un sadarbība. Taču šis apgalvojums nespēja aizsegt to, ka vēsturiskās kolīzijas, valstiskās attiecības nevarēja neietekmēt arī tautu, ierindas cilvēku savstarpējās jūtas. Tā, krievu revolucionārs P.~Lavrovs 1889.~gadā atstāstīja kā gadu pirms Parīzes komūnas (1871) „man nācās piedzīvot, ka poļu mākslas profesors atstāja telpu, dzirdot, ka Dombrovskis griežas pie manis krievu valodā.” Līdzīgi arī XX gadsimta 90.~gados šī darba autoram personīgi nācās piedzīvot, kā pret citām zemēm un tautām, arī Latviju un latviešiem, draudzīgi noskaņots inteliģents pavecāks polis, kurš prata gan krievu, gan vācu valodu, tomēr nevēlējās tajās sarunāties, ar to demonstrējot savu attieksmi pret bijušajiem „okupantiem”.

Tautu savstarpējās attiecības sevišķi traucējoši ietekmēja gan tā saucamā „sociālisma” apstākļos, gan pēc tā sabrukuma skanējušie oficiālie meli.

Kaut XX gadsimta 90.~gados sākās vēstures deideoloģizācija un depolitizācija, tā nebūt nav sekmīgi pabeigta. Kā pamatoti norādījis Daugavpils Universitātes profesors A.~Ivanovs: „\dots{}Dažādas politiskas institūcijas un politiskās elites vēsturi izmantoja un joprojām izmanto kā vienu no politikas veidošanas un īstenošanas instrumentiem, kas palīdz radīt elitei izdevīgus vēsturiskus mītus un nodrošina vienprātību sabiedrībā. \citespace{} vēstures politizēšanas pakāpi nosaka politiskais režīms: antidemokrātiskā režīma apstākļos vēsture kļūst ārkārtīgi politizēta un ideoloģizēta, tieši un atklāti kalpojot valstij un oficiālajai ideoloģijai, turpretim demokrātiskā vara ļauj vēsturniekiem relatīvi brīvi interpretēt vēstures procesu un vērtēt vēstures faktus. Kaut gan arī šī radošā brīvība nav absolūta.” Turpinot šo domu gaitu, var piebilst, ka centienu „nodrošināt vienprātību sabiedrībā” parādīšanās jebkurā valstī ir līdzvērtīga demokrātijas ierobežošanas mēģinājumiem.

Poļu vēsturnieks J.~Duračinskis ir atzīmējis, ka daudzos pēc 1989.~gada publicētajos darbos pamatoti novērsto iepriekšējo melu vietā parādās jauni meli un „gan tie vecie, gan mūsdienu meli izaug uz noteiktu (kardināli pretēju) simpātiju vai arī atsevišķu pētnieku politisko ieskatu pamata. Tas attiecas, galvenokārt uz izšķirošo notikumu un poļu politiskās skatuves galveno „aktieru” vērtējumu”. Vēlme pārskatīt uzkrāto pieredzi diemžēl bieži izpaužas vienkāršotā novecojušo uzskatu caurskatīšanā, agrāko plusu vietā liekot mīnusus un otrādi. Pēc Otrā pasaules kara norisušo sarežģīto iekšējo procesu padziļinātas interpretācijas vietā bieži tiek runāts tikai par tīri vardarbīgu padomju modeļa uzspiešanu Polijai.

[Minētā daudzu pētījumu autora J.~Duračinska historiogrāfiskie darbi var dod lasītājam diezgan pilnīgu priekšstatu par pašu poļu vēsturnieku veikumu savas dzimtenes XIX un XX~gadsimta vēstures izpētē. Daži no tiem ir pieejami arī krievu valodā. Piemēram: \rutxti{Э.~Дурачинский}. „\rutxti{О польской историографии новейшей истории}” („Par jaunāko laiku vēstures poļu historiogrāfiju”) // %http://library.by/portalus/modules/rushistory/readme.php?subaction=showfull&amp;id=1192094160&amp;archive=&amp;start_from=&amp;ucat=19&amp;category=19
% // 2.05.2010.]

Poļu vēsturnieki nereti nevēlas savas vēstures periodizācijā iekļaut arī pašu poļu izraisītās nacionālās sakāves, parasti cenšas uzsvērt, ka Polija vienmēr ir bijusi saistīta ar Rietumu civilizāciju, reizē mazinot sakaru nozīmi ar austrumu kaimiņiem. Acīmredzot to veicinājis arī tas, ka, runājot Polijas valsts darbinieka un rakstnieka S.~Kata-Mackeviča vārdiem, austrumos „mēs vienmēr bijām kungu nācija”, bet rietumos „mēs bijām tikai strādnieku un zemnieku nācija”.

Diemžēl mūsdienu poļu historiogrāfijā, bet īpaši politiskajā publicistikā krievu un poļu pretrunu un savstarpējās nepatikas jūtu cēloņi parasti tiek meklēti galvenokārt jaunāko laiku periodā~--- XVIII--XX gadsimta ietvaros, deviņos gadījumos no desmit uzmanību pievēršot tikai Polijas dalīšanai, uzsverot tās paverdzināšanu, ko veica Krievija, Prūsija/Vācija un Austrija/Austroungārija, arī nežēlīgo 1794., 1830.--1831. un 1863.--1864.~gada poļu sacelšanos apspiešanu, Otrajā pasaules karā Polijai nodarītās patiesās un iedomātās „netaisnības” un padomju modeļa uzspiešanu tai. Mazāk tiek akcentēta Polijas, tikpat cik Krievijas, agresīvā darbība Livonijas karā, poļu īstenotā Pleskavas aplenkšana (1581--1582, 1615), Krievijas un Žečpospolitas cīņa par Smoļensku, Polocku un kreisā krasta Ukrainu, poļu dalība t.s. „juku laikos” Krievijā XVII~gs. sākumā, kad norisa poļu šļahtas atbalstītā Viltusdmitrija I darbība un tam sekojošās divas poļu intervences Krievijā, kā arī krievu-poļu karš 1654.--1667.~gadā, kad Krievija atsaucās Zaporožjes (\uktxti{Запорiзька Січ}) kazaku hetmaņa B.~Hmeļnicka aicinājumam palīdzēt cīņā pret poļu apspiedējiem un pieņemt Zaporožjes kazakus Krievijas pavalstniecībā.

Ja daļa poļu vēsturnieku Polijas rīcību pret Krieviju vērstajos karagājienos piemin arī paškritiski, tad tik un tā vairākumā viņu rakstītajā jūtams zināms lepnums par poļu „varoņdarbiem” šajos iekarošanas karos. Par daudziem citiem, mazāk „spožiem” notikumiem, poļu sakāvēm, arī par poļu iekarotāju nodarītajām ciešanām citu tautu iedzīvotājiem tiek runāts klusināti. Tiesa, ir arī citi piemēri. Jau minētais poļu zinātnieks J.~Tazbirs uz žurnālista jautājumu: „Kurš pirmais sāka nemīlēt otros~--- mēs krievus vai krievi mūs?” atbildēja: „Vienlaikus. Strīdus un sadursmju objektu veidoja Lietuva. Krievzeme bija vāja un Lietuva sagrāba tās zemes, kuras agrāk bija tās (Krievzemes) īpašumā. Sāncensība turpinājās, un uz tās fona noteikti bija jānonāk līdz konfliktam.”

Diemžēl, arī šeit ir vajadzīgs precizējums. Poļu vēsturnieks citētajā frāzē it kā nevilšus visu atbildību par konfliktiem uzkrauj Lietuvai, Poliju pat nepieminot, kaut Lietuvas lielkņazistē ilgu laiku pārsvarā bija slāvu pareizticīgie iedzīvotāji, visai Rietumu Krievzemei Lietuvas lielkņaziste bija dabīgs pretestības centrs kā pret tatāriem, tā Teitoņu ordeni. Par ietekmi Austrumeiropā cīnījās gan poļu karaļi, gan krievu kņazi, gan leišu kunigaiši, taču Lietuvai pakāpeniski nonākot Polijas jaunākā partnera lomā, galveno cīņas smagumu pret Krievzemi uzņēmās tieši Polija.

Žečpospolitas un Krievijas strīds par to, kura no tām valdīs pār Zaporožjes kazakiem, beidzās ar Andrusovas pamiera noslēgšanu 1667.~gadā, ar kuru pirmā zaudēja visus apgabalus Dņepras kreisajā krastā par labu Krievijai. Tas bija ievērojams pagrieziena punkts poļu-krievu attiecībās. No šī laika pusotru gadsimtu līdz 1815.~gadam Krievija izplatīja savu ietekmi uz rietumiem, līdz liela daļa Žečpospolitas nonāca tās sastāvā. Jau 1686.~gadā karalis J.~Sobeskis noslēdza ar Krieviju t.s. Mūžīgo mieru (poļu historiogrāfijā \pltxti{pokój Grzymułtowskiego}), kurā starp citu Žečpospolita apsolīja saviem pavalstniekiem~--- pareizticīgajiem piešķirt ticības brīvību, bet Krievija apņēmās tos aizstāvēt. Tas deva Krievijai ieganstu iejaukties Žečpospolitas lietās.

Mūsdienās savstarpējos poļu~--- krievu pārmetumos parādā poļu nacionālistiski noskaņotajiem vēsturniekiem un publicistiem nepaliek arī liela daļa Krievijas vēsturnieku un publicistu. Uzstājoties vai nu joprojām no padomiskām vai vienkārši impēriskām pozīcijām, viņi nevēlas atzīt Krievijas un PSRS noziegumus pret poļu tautu, visur saskata tikai poļu darbības negatīvās sekas.

[Īsu pārskatu par krievu un padomju vēsturnieku veikumu Polijas vēstures izpētē sniedz S.~Falkovičas raksts: \rutxti{Светлана Фалькович. Польская проблематика в российской историографии} (Poļu problemātika krievu historiogrāfijā) // %http://jazon.hist.uj.edu.pl/zjazd/materialy/falkowicz.pdf
// 24.07.2011.]

Savstarpējos strīdos diemžēl bieži dzimst nevis patiesība, bet jauni aizspriedumi pret kaimiņtautām. Noteiktu liecību par poļu un krievu attiecībām sniedz arī abu valstu oficiālie svētki.

Tagadējās Krievijas Federācijas likumdevēji izvēlējās par Krievijas Tautas vienības dienu noteikt 4.~novembri, t.i.~--- dienu, kad pēc dažiem pieņēmumiem 1612.~gadā krievu zemessardze K.~Miņina un D.~Požarska vadībā atbrīvoja Maskavu no poļu iebrucējiem.

Tiesa, svētku datums tika izraudzīts bez pienācīgām konsultācijām ar speciālistiem. Mūsdienu krievu ekonomists, vēsturnieks un politiskais darbinieks V.~Šeiniss uzsvēris, ka šai dienā poļi tika padzīti tikai no vienas Maskavas pilsētas daļas (\rutxti{Китай-Город}), bet Kremlī viņi vēl palika, un viņu karogi no tā sienām tika nomesti tikai 1612.~gada 7.~novembrī. Plašāku izvērtējumu Krievijas tautas vienības dienai izvēlētajam datumam 1612.~gadā ir devis ievērojamais krievu vēsturnieks V.~Nazarovs. Šai dienā Maskavā nekas ievērojams nenotika, minētā Maskavas daļa (\rutxti{Китай-Город}) tika atbrīvota pēc jaunā stila nevis 4., bet gan 1.novembrī. Nepareizais datējums radās hronoloģiskas kļūdas rezultātā, saskaņojot Pareizticīgās baznīcas un laicīgo kalendāru. 5.novembrī Kremlī ielenktie interventi, kuru rindas bija raibas pēc nacionālā sastāva un poļi, visticamāk, bija mazākumā, parakstīja kapitulāciju, bet nākamajā dienā Kremļa garnizona padevās. Tieši Kremļa atbrīvošana kļuva par ievērojamu notikumu. Tādejādi tagad Krievijā spēkā esošā svētku diena balstās uz kļūdainu datējumu, neprecīzu vēsturisko avotu traktējumu un notikumu interpretāciju, kura neatbilst avotiem.

Lai arī kam būtu taisnība, svētku datuma izvēles politizēto raksturu nevar noliegt.

Polijas Republikā kā Polijas armijas diena (\pltxti{Świeto Wojska Polskiego}) tiek atzīmēts 15.augusts, kad Polijas armija 1920.~gadā uzvarēja Sarkano armiju kaujā pie Varšavas.

Šāda masu vēsturiskās atmiņas politizēta virzība traucē atklātības apstarotu, mūsdienīgu attiecību izveidi gan atsevišķu cilvēku, gan veselu tautu starpā. Savukārt, mūsdienās pastāvošās politiskās problēmas traucē objektīvi izvērtēt Polijas vēsturi.

Nākas ar nožēlu konstatēt, ka vēsturē diemžēl var atrast argumentus par labu gandrīz jebkuram viedoklim. Īpaši tas attiecās uz divu tautu attiecībām, kuras dzīvojušas blakus vairāk nekā tūkstoš gadu un kuras gan vieno, gan šķir tūkstošgadu savstarpējās cīņas vēsture. Šai gadījumā galīgi aplami ir jautājumi „kurš pirmais sāka?” un „kurš ir vairāk vainīgs?”. Jau attiecībā uz XX gadsimta sākuma situāciju krievu filozofs N.~Berdjajevs runāja par to, ka poļiem un krieviem ir jāatbrīvojas no atmiņām par gadsimtiem ilgo pretstāvi, grēkus, atzīstot savu vainu, ir jānožēlo visiem. Šai sakarā dziļu patiesību nes poļu vēsturnieka S.~Keņeviča doma: „Nācijai īsta patiesība, lai kāda arī tā būtu, nevar kaitēt \citespace{} Aprakstot pagājušo gadsimtu konfliktus, ir jāņem vērā ne tikai mūsu labā esošie argumenti, bet arī argumenti, ka nāk par labu mūsu bijušajiem pretiniekiem, lai neaprobežotos tikai ar sašutumu par mums nodarītajām pārestībām, bet parunātu arī par pāridarījumiem, kurus mēs esam nodarījuši citiem, lai, galu galā, nestiprinātu mūsu pārliecību par pārākumu pār citām nācijām, atceroties to, ka analoģiska citu nāciju pārliecība attiecībā par poļiem mūs aizvaino”.

Bijušais Polijas komunistu līderis un Valsts prezidents V.~Jaruzeļskis, ir stāstījis, kā viņš savas vizītes laikā Londonā tikās ar Lielbritānijas premjerministri M.~Tečeri un ieminējās tai, ka viņš ir liels Napoleona I cienītājs. M.~Tečere viņu uzveda savas mājas bēniņos, parādīja plānu ādas mapīti un teica: „Jūs zināt, kam tā piederēja? Napoleonam. Taču francūzi es nekad šurp neatvestu!” V.~Jaruzeļskis to komentēja: „Skatieties, 200~gadu ir pagājis, bet angļi nevar aizmirst [Napoleona karus]. Viņi ar frančiem dzīvo savienībā, divus pasaules karus karoja kopā. Un neraugoties uz to, katrai valstij palicis diametrāli pretējs Napoleona redzējums, sava vēstures versija. Taču tas nedrīkst traucēt sadarbību.”

Kā atzīmējis Polijas Zinātņu akadēmijas zinātniskais līdzstrādnieks, Berlīnes Brīvās universitātes goda profesors R.~Traba, poļiem ir gan vajadzīga Eiropas telpa dialogiem, taču pirmkārt viņiem ir jāatrisina „iekšējie strīdi”, kritiski jāizvērtē sava pagātne, lai „skeleti skapī” netraucētu tikt vaļā no iemantotajiem kompleksiem un politiskā balasta. Var piebilst, ka poļiem, daudz cietušiem no staļiniskajiem PSRS drošības orgāniem, nevajadzētu aizmirst, ka pirmie divi to vadītāji bija poļu šļahtiču kārtai piederīgie F.~Džeržinskis un V.~Menžinskis, tāpat no šīs kārtas nāca arī viena no staļinisma laikmeta drūmākajām figūrām~--- staļiniskais prokurors t.~s. „atklātajos procesos”~--- A.~Višinskis.

Ilgstoši apspiestas tautas nacionālais jūtīgums traucē tai un arī tās vēsturniekiem objektīvi izvērtēt vēsturi.

Izcilais mūsdienu krievu filozofs un sociologs A.~Zinovjevs ir norādījis uz interesantu likumsakarību: pēc lieliem vēsturiskiem notikumiem, tajos sakāvi cietusī puse, jau <em>post faktum</em> analizējot situāciju, apstākļus, kas noveda pie šīs sakāves, nespēj būt objektīva. Raudzīties pagātnē no zaudētāju skatu punkta ir mazproduktīvi, jo viņi taču cieta sakāvi, tātad izrādījās netālredzīgi, nesaprata vai vismaz slikti saprata, kurp ved notikumu gaita, tai skaitā to notikumu, kurus izsauca viņu pašu aktivitātes. To vispirms var teikt par revolūcijās varu zaudējušajiem, taču tāpat arī attiecināt uz ar Otro pasaules karu saistītajos notikumos varu, ietekmi zaudējušajiem Polijas, un ne tikai Polijas, bijušajiem valdošajiem slāņiem.

Poļu zinātnieki par XX gadsimta otrās puses lielāko lūzumu uzskata 1989.~gada notikumus, kas ne tikai mainīja Polijas valsts raksturu, atjaunoja Polijas Republiku (\pltxti{Rzeczpospolita Polska}), bet arī atveseļoja situāciju Eiropā kopumā. Jāuzsver, ka šis lūzums, tāpat kā PSRS sabrukums 1991.~gadā, ir pavēris arī daudz labvēlīgākas iespējas padziļināti izpētīt Polijas vēsturi, tuvināties vēsturiskajai patiesībai. Tomēr tas nenozīmē, ka šī patiesība atklāsies pati no sevis. Pēdējās desmitgadēs pēc Polijas Tautas Republikas likvidācijas, kad oficiāla cenzūra vairs neeksistē, pie varas esošie spēki ir parādījuši, ka arī ar attiecīgu finanšu politiku var panākt, lai daudzi vēsturnieki, publicisti rūpētos ne tik daudz par objektīvās patiesības atklāšanu, cik politiskās elites pasūtītas vēstures interpretācijas izplatību. Daudzos mūsdienu poļu publicistu, arī zinātnieku, tāpat kā citu Austrumeiropas tautu vēsturnieku darbos, dominē politiskās nostādnes, nevis vēsturiskā patiesība.

2004.~gadā grupa poļu vēsturnieku izvirzīja iniciatīvu izstrādāt un realizēt aktīvu vēsturisko politiku (\pltxti{polityka historyczna}). Termins (\detxti{Geschichtspolitik}) bija aizgūts no Vācijas, kur tas radās jau XX gs. 80.~gados, tiesa, ar nedaudz citu saturu. Polijā vēsturiskās politikas realizācijas piekritēji vēsturi un vēsturisko atmiņu lielākoties uzskata par politiskās cīņas arēnu pret ārējo un iekšējo ienaidnieku. Ar to viņi faktiski attaisno atkāpes no profesionālās ētikas, arī domu brīvības ierobežošanu. Apriori tiek uzskatīts, ka „ārējais pretinieks” cenšas nostiprināt savu vēstures notikumu interpretāciju, tāpēc „savu” vēsturnieku pienākums ir solidāri pretoties šādām briesmām, aizstāvēt argumentus, pretējus pretinieku lietotajiem. Pie varas esošie politiskie spēki, izmantojot valsts administratīvos un finanšu resursus, cenšas nostiprināt tiem tīkamas vēsturisko notikumu interpretācijas. Valdošie slāņi cenšas uzurpēt un monopolizēt tiesības interpretēt vēsturi, tagadni uzskatot par mērlīdzekli pagātnes izvērtēšanai. Parasti speciālas vēsturiskas politikas realizēšanas nepieciešamībai kā attaisnojums tiek minēta nepietiekamā patriotisma audzināšana vēstures stundās skolā. Nacionāli noskaņotie radikāļi, lai stiprinātu „nacionālo pašapziņu”, prasa radīt „harmonisku” vēstures ainu, panākt vēsturiskās atmiņas vienveidību.

Tā 2005.--2007.~gadā politizēti publicisti mēģināja iestāstīt Polijas sabiedrībai, ka humanitārās zinātnes pēc 1989.~gada, tai laikā radītās mācību grāmatās esot nepietiekami patriotiskas, bezierunu patriotisma vietā izplatot kritisko patriotismu. Lai īstenotu vēsturisko politiku, tiek radītas speciālas valstiskas un sabiedriskas iestādes un muzeji, iesniegti likumprojekti lai iemūžinātu „vienīgi pareizo” vēsturisko notikumu traktējumu, ieviešot pat kriminālatbildību par tā neievērošanu, kontroli pār izglītības sistēmu. Izmantoti ievērojami finansiālie resursi dažādu politiski tendētu projektu īstenošanai, komisiju dibināšanai, augsti atalgotu amatu radīšanai, ieviešot vēsturisko „varoņu” panteonu, atceres dienas utt. Godājot un nereti arī pārspīlējot savus upurus, kas nesti cīņā pret „svešajiem”, bieži tiek tīšuprāt aizmirsts, ka savi noziegumi un to rezultātā nestie citu tautu upuri nevar tikt attaisnoti pat ar savas tautas interešu aizstāvību. Bieži politisko uzskatu maiņa ved pie tā, ka tiek izcelti vieni un noniecināti citi vēstures personāži. Piemēram, Poznaņā viena no garākajām ielām tagad nes nevis Jaroslava Dombrovska, bet Jana Henrika Dombrovska vārdu, jo pirmais mūsdienu „pilsētas tēviem” bijis pārāk kreiss.

Šāda vēsturiskās politikas īstenošana ved pie diskusiju ierobežošanas savā valstī un konfliktsituāciju radīšanas ar ārpasauli. Kritiskās pieejas trūkums pret savas zemes vēsturi un savu patriotismu traucē demokrātiskas politiskās kultūras izveidei, ved pie atteikšanās no plurālisma vēsturnieku darbos. Vēsture un vēsturiskā atmiņa ir svarīga kolektīvās identitātes sastāvdaļa, taču skatot to tikai caur vēsturiskās politikas prizmu, tiek tikai sarežģīta strīdu un problēmu atrisināšana. Poļu vēsturniece A.~Volfa-Poveska norāda, ka arī mūsdienu Polijas vēsturnieku un žurnālistu vidū netrūkst cilvēku, kuri demonstrē, ka var „taisīt” politiku ar vēstures palīdzību, ka poļu vēsturiskajai politikai ir ģeopolitiskas koordinātes, tā atrodas starp divu savu lielāko kaimiņvalstu vēsturiskajām politikām.

Vērtējot vēsturiskos notikumus, jāņem vērā arī tās pārvērtības, kas notiek ar apspiestām tautām, kad tās atgūst valstiskumu un kļūst par noteicējām zemē, kur dzīvo citas mazākumtautības. Bijušajiem cietējiem ir vajadzīgs patiešām augsts kultūras un demokrātiskās attīstības līmenis, lai no apspiestajiem nepārvērstos par apspiedējiem. Ne velti neatkarīgās Polijas valsts vēsturi XX gadsimta 20--30.~gados tik atšķirīgi vērtē, no vienas puses, poļi un, no otras puses, tajā dzīvojušie ukraiņi, baltkrievi, lietuvieši, vācieši, ebreji un krievi.

Vēstures jautājumiem ir jāpaliek par speciālistu un vēstures entuziastu meklējumu un diskusiju objektu, bet politiķiem jāmeklē ceļi valstu un tautu attiecību uzlabošanai. Kā atzinis Krievijas ZA loceklis A.~Čubarjans: „Attiecības starp cilvēkiem un starp tautām kļūst stiprākas, dziļākas un godīgākas, ja tās pamatojas uz vēsturiskās patiesības atzīšanu un gatavību kritiskai analīzei, arī savas vēstures pārvērtēšanai.” Acīmredzams, ka saprašanās starp kaimiņtautām~--- poļiem un krieviem~--- būs iespējama tikai tad, ja abas puses pilnībā apzināsies tās grūtības, kuras šai ceļā nosaka vēsturiskais mantojums, kā arī ievēros un cienīs pretējās puses tradīcijas un uzskatus. Mūsdienās poļu un krievu zinātnieki no Polijas un Krievijas Zinātņu akadēmijām cenšas radīt objektīvu Polijas un Krievijas sarežģīto attiecību versiju, taču joprojām abu grupu pieejā ir daudz subjektīvisma. Krievu akadēmiķis I.~Kovaļčenko norādījis: „Nacionālā vēsture~--- tā ir autobiogrāfija, bet katra autobiogrāfija ir subjektīva, tāpēc, lai piedotu nacionālajām vēsturēm objektīvu raksturu, to uzrakstīšanai ir jāpiesaista citu valstu vēsturnieki.”

Blakus pašiem poļu pētniekiem, kuri ne vienmēr ir spējīgi būt pietiekami kritiski pret savu valsti, daudz par Polijas vēstures problēmām rakstījuši arī tās kaimiņvalstu vēsturnieki. Tas gan atvieglo, gan sarežģī tās līdzsvarotāku izvērtēšanu. Atvieglo tāpēc, ka ļauj izzināt vienas puses noklusēto, sarežģī~--- jo liek meklēt patiesību bieži diametrāli pretējās faktu interpretācijās. Bieži vien vairāk nekā pašiem poļiem un arī Polijas kaimiņvalstu vēsturniekiem var uzticēties neieinteresēto ārzemju pētnieku, piemēram, šveicieša A.~Kapellera, kanādieša M.~Dž.~Karleija u.c. vērtējumiem.

XX un XXI gs. mijā starp Poliju un Krieviju attīstījās vēsturnieku sadarbība, notika savstarpēja apmaiņa ar dokumentiem un materiāliem. Apritē nāca un joprojām nāk arvien jauni arhīvu materiāli, taču faktu uzkrāšana nevar automātiski atrisināt to interpretācijas problēmu.

Starp citu, vēsturnieku aprindās bieži skan gaušanās, ka pārāk lēnu tiek atvērti vai arī nemaz netiek atvērti Krievijas arhīvi, kas vismaz daļēji atbilst patiesībai. Taču, kā norādījis krievu diplomāts, politiķis un vēsturnieks V.~Faļins, tai pat laikā tiek aizmirsts, ka ASV uz nenoteiktu laiku ir liegusi jebkādu pieeju dokumentiem, kurus amerikāņu karavīri Otrā pasaules kara beigās ieguva nacistiskās Vācijas vadības mītnē Tīringijā. Acīmredzot tie satur pārāk daudz nepatīkama materiāla par amerikāņu politiku Eiropā, varbūt arī Polijā.

Ir vajadzīgs laiks, lai atbrīvotos no aizspriedumiem un liekām emocijām, neizbēgamām īpaši XX gadsimta vētraino notikumu vērtēšanā. Diemžēl, starptautiskā situācija ne vienmēr ir labvēlīga patiesības noskaidrošanai. Tā, Rietumvalstu un Krievijas interešu pretstāve Ukrainas teritorijā, faktiskais NATO valstu un Krievijas hibrīdkarš, īpaši pēc tiešas militāras sadursmes sākuma 2022.~gada februārī, kurš objektīvi veicināja Polijas kā Ukrainas kaimiņvalsts lomas pieaugumu, tikai sekmē arvien jaunu vairāk vai mazāk būtisku pretrunu, kā arī jau minēto aizspriedumu un emociju saasināšanos.

Tikai patiesības noskaidrošana, falsifikāciju atmaskošana spēj radīt pamatu normālām, cieņas pilnām, draudzīgām attiecībām starp tautām. Tas attiecas arī uz poļiem, vāciešiem, krieviem, ukraiņiem, baltkrieviem, lietuviešiem, ebrejiem un arī latviešiem, kuri gan ilgstoši dzīvoja kaimiņos, bet kuru attiecības veidojās dažādu iekšpolitisku un ārpolitisku apstākļu ietekmē. Lai atceramies t.s. ''poļu laikus” Latvijas teritorijā, arī to, ka XX gadsimta divdesmitajos gados par Poliju Latvijā inteliģences aprindās skanēja novērtējums „panu Polija”. Profesors J.~Tazbirs vēl XXI gadsimta sākumā norādīja, ka no vairākuma Polijā dažādos izglītības līmeņos lietojamo vēstures grāmatu grūti uzzināt, ka daudzu gadsimtu gaitā Žečpospolita nacionālā ziņā atgādināja mozaīku. Kā atzinis E.~Duračinskis, arī mūsdienās daudzi poļi bijušās Polijas iedzīvotājus: ukraiņus, baltkrievus un lietuviešus joprojām uzlūko kā „mazākumtautības”, kuras nav pelnījušas neko vairāk kā Otrās Žečpospolitas realizētās politikas pret „nacionālajiem mazākumiem” uzlabotu variantu. Tāpēc arī šajā darbā nacionālajām attiecībām starp dažādajām Polijā un tās kaimiņos dzīvojošajām tautām ierādīta nozīmīga vieta.

Bijušā Polijas prezidenta B.~Komarovska padomnieks vēstures jautājumos T.~Nalenčs intervijā presei uzsvēra, ka valstij un sabiedrībai jākoncentrē uzmanība to vēsturisko motīvu aprites veicināšanai, ar kuriem Polija var lepoties, piemēram, jāpopularizē pasaulē lielākā Malborkas (poļu \pltxti{Malbork}, vācu \detxti{Marienburg}) pils vai Ļ.~Valensas, kurš stāvēja pie „jaunās Polijas” sākumiem, personība. Pēc T.~Nalenča vārdiem, vēsturi nedrīkst izmantot kā avotu negatīvu emociju formēšanai gan valsts iekšienē, gan ārpus tās. Piekrītot tam, ka vēsture jāizmanto pozitīvu ideālu audzināšanai, autoram gan jāpiezīmē, ka ar tās palīdzību nedrīkst arī redzēt un popularizēt tikai savas un tā noniecināt citu tautu sniegumus, kā arī „neredzēt” negatīvo savas valsts vēsturē.

Lasītāja priekšā esošā darba autors centies cik iespējams objektīvi un daudzpusīgi atspoguļot Polijas vēstures XIX un XX gadsimtā sarežģītās problēmas, taču uzreiz jāsaka, ka viņš tomēr piedāvā tikai vienu~--- savu redzējumu. Jāuzsver, ka mūsdienās ir ļoti grūti pētīt, analizēt, vienkārši rakstīt par XX gadsimta notikumiem, pilnībā atsakoties no sava subjektīvā vērtējuma. Cerams, ka pēc gadiem, kad kaislības būs norimušas, objektivitāti sasniegt būs vieglāk. Taču apstākļos, kad uz politiskās skatuves darbojas spēki, kuri sevi ar lielākām vai mazākām tiesībām pozicionē kā to vai citu XX gadsimtā darbojošos partiju, strāvojumu mantiniekus, prasību raudzīties uz XX gadsimta notikumiem „pilnīgi neitrāli”, droši var nosaukt par utopisku.

Godīgi jāpasaka, ka grāmatā visi vēsturiskās izšķiršanās brīži, notikumi, to veicēji vērtēti no iespējami demokrātiskāka, vardarbību izslēdzoša attīstības ceļa sasniegšanas viedokļa. Visdažādākajos nacionālajos strīdos autora personīgās simpātijas pieder apspiestajiem nacionālajiem mazākumiem, bet ne lielvalstiskas, bieži šovinistiskas nacionālās politikas realizētājiem. Tai pašā laikā autoram nav pieņemama to pagātnē (un dažkārt arī mūsdienās) apspiesto tautu nacionālistisko spēku darbība, kuri savu mērķu sasniegšanai nereti gatavi pārkāpt citu tautu, citas tautības cilvēku tādas pat tiesības uz brīvību un vienlīdzību, neapstājoties pat noziegumu priekšā.

Stādīts mērķis veicināt no aizspriedumiem un stereotipiem brīvu un līdzsvarotu izpratni par Polijas un arī visas Eiropas vēsturi Otrā pasaules kara gados, palīdzēt lasītājiem kritiski uztvert un izvērtēt dažādus apzināti un neapzināti tendenciozus sacerējumus par Poliju Otrajā pasaules karā, kuri pēdējās desmitgadēs izplatās Austrumeiropā.

Otrā pasaules kara notikumos autora simpātijas ir antihitleriskās koalīcijas sabiedroto pusē, kuri cīnījās par vispārcilvēciskiem mērķiem. Tāpēc viņš kritiski raugās uz tādiem Austrumeiropas politiķiem, kuri, piesedzoties ar „patriotiskiem” lozungiem, kara laikā faktiski atbalstīja nacistus, bet pēc tam mēģināja uzspiest šī vispasaules nozīmības izšķiršanās punkta vērtējumam savu šauri nacionālistisko skatījumu. Arī mūsdienās darbojas viņu ideju mantinieki.

Runājot par pēckara situāciju pasaulē, autors neuzskata par iespējamu staļinisko (arī pēc J.~Staļina nāves uzlaboto) „sociālisma” modeli identificēt ar to mērķi, kuru stādīja tā daļa cilvēces attīstības ceļu meklētāju, kuri par galveno uzskatīja sociālā taisnīguma ideālu sasniegšanu. Sociālisms kā sabiedriska iekārta, kurā cilvēks neekspluatē citu cilvēku, kur pastāv sabiedrisks, bet ne valdošās elites piesavināts īpašums uz ražošanas līdzekļiem, netika uzcelts ne PSRS, ne citās t.~s. „sociālisma” valstīs. Ražošanas līdzekļu koncentrācija valsts (faktiski~--- partijiski-birokrātiskā aparāta) rokās veda pie virsmonopolizācijas, kas bremzēja ražošanas attīstību. „Sociālisma” valstīs nepastāvēja sociālais taisnīgums. Pastāvēja pat superekspluatācija, kad daļa sabiedrības (staļiniskajās nometnēs ieslodzītā) praktiski nesaņēma nekādu atlīdzību, daļa strādāja par niecīgu atalgojumu, bet sabiedrības virsslānis~--- partijiski-birokrātiskā elite saņēma savam veikumam neadekvāti augstu atalgojumu. Sociālisma princips „no katra pēc spējām, katram pēc viņa darba”, „sociālisma” valstīs netika īstenots, jo nebija jau objektīvas mērauklas, kā šo darbu novērtēt. Tirgus vietā vērtējumu deva kā valsts birokrātijas slānis kopumā, tā atsevišķi tā pārstāvji. Tāpēc, runājot par PSRS, tās satelītvalstīm, arī Polijas Tautas Republiku, par tur it kā uzcelto sabiedrisko iekārtu autors parasti raksta kā par „sociālismu”, t.i.~--- lietojot pēdiņas. Tikai runājot par sociālismu kā mērķi, kuru pieņēma un centās sasniegt daļa uz labāku dzīvi cerošo, bet lielākoties nekritiski domājošo tautas masu, tas lietots bez pēdiņām.

Pēckara notikumos autora visdziļākā cieņa pieder cīnītājiem par demokrātiju un tautas labklājību, kuri sava mērķa sasniegšanai bija gatavi lietot dažādus cīņas līdzekļus, taču atteicās izmantot vardarbību, „šķirisko”, „sociālistisko” vai „nacionālo” interešu vārdā piekopt teroru pret savas un citu tautu piederīgajiem. Ar to viņi krasi norobežojās no saviem pretiniekiem~--- totalitāro, autoritāro, etnokrātisko varas sistēmu aizstāvjiem, kuriem savukārt visi līdzekļi bija derīgi it kā „valstisko”, „nacionālo”, „šķirisko”, „sociālistisko”, arī „demokrātisko” u.tml., bet faktiski~--- savu savtīgo mērķu sasniegšanai. No šādām pozīcijām arī skatīti gan poļu, gan citzemju darbinieki, viņu pārstāvētie sabiedriskie spēki, viņu ietekmētie notikumi.

Patiesam Polijas vēstures redzējumam lasītājs tuvināsies, salīdzinot dažādus pētījumus, kritiski vērtējot kā šo, tā arī citus Polijas vēsturei veltītus darbus.

Ilustrācijas ievadam

\chapter{Polija kaimiņvalstu varā. 1795~--- 1918}

\epigraph
{Karot poļi neprot. Taču dumpoties!}
{Hugo~Kollontajs (\pltxti{Hugo Kołłątaj})}

\epigraph
{Diemžēl mēs [poļi] neprotam strādāt! Kauties, lieliski cīnīties, nomirt, uz to vienmēr esam gatavi; taču uzcītīgi strādāt, ilgstoši, bez trokšņa un uzslavām, strādāt varbūt ne priekš sevis, tas mums ir par daudz.}
{Ludvika Plātere (\lttxti{Ludwika Plater})}

\epigraph
{Polija ir unikāla valsts ar tieksmi pēc impērijas, kuras tai nekad nav bijis.}
{Dmitrijs Kuļikovs (\rutxti{Дмитрий Куликов})}



\epigraph
{Nelaimīga ir valsts, kurai nav varoņu.~--- Nē! Nelaimīga ir tā valsts, kurai ir vajadzīgi varoņi.}
{Bertolts Brehts (\detxti{Eugen Bertolt Friedrich Brecht})}

\epigraph
{Lepnums mums izmaksā dārgāk nekā bads, slāpes un aukstums.}
{Tomass Džefersons (\entxti{Thomas Jefferson})}

\epigraph
{Starptautiskajā politikā morāles nav bijis, nav un nebūs.}
{Jakovs Kedmi (\hetxti{יעקב קדמ})}

\epigraph
{Nabags nav tas, kam maz pieder, bet tas, kurš daudz grib.}
{Angļu sakāmvārds}

\epigraph
{Ja visi vainīgi, neviens nav vainīgs.}
{Pēteris Krupņikovs}

\epigraph
{Vistālāk iet tas, kurš nezina, kurp iet.}
{Olivers Kromvels (\entxti{Oliver Cromwell})}



\epigraph
{Gandrīz visi dižie līderi savai dzimtenei ir maksājuši asins jūras.}
{Mihails Vellers (\rutxti{Михаил Иосифович Веллер})}

\epigraph
{Nācijai zaudēt savu valsti, savu patstāvību un neatkarību ir liela nelaime. To var salīdzināt ar smagu slimību, kas kropļo nācijas dvēseli.}
{Nikolajs Berdjajevs (\rutxti{Николай Бердяев})}

\epigraph
{Dažas tautas traģēdijas tiek uzvestas bez starpbrīžiem.}
{Staņislavs Ježijs Lecs (\pltxti{Stanisław Jerzy Lec})}

\epigraph
{Kādreiz dedzinātāji ir pārliecināti, ka tautai priekšā nesuši lāpu.}
{Staņislavs Ježijs Lecs (\pltxti{Stanisław Jerzy Lec})}

\epigraph
{Pēc neveiksmīgām revolūcijām vienmēr seko ienīstas un atriebīgas valdības.}
{Pjērs Buasts (\frtxti{Pierre Boiste})}

\epigraph
{Visas revolūcijas beidzas ar reakciju. Tas nav novēršams. Tas ir likums. Un jo negantākas un niknākas bijušas revolūcijas, jo stiprāka bija reakcija. Revolūciju un reakcijas nomaiņā ir kāds maģisks riņķis.}
{Nikolajs Berdjajevs (\rutxti{Николай Бердяев})}



\epigraph
{Neprasme pārciest nelaimi ir liela nelaime.}
{Bions no Borisfēnas (\eltxti{Βίων Βορυσθενίτης})}

\section{Poļu zemes XVII gadsimta beigās un XIX gadsimtā}

\epigraph
{Laimīga tā tauta, kurai ir garlaicīga vēsture.}
{Šarls Luijs de Monteskjē (\frtxti{Charles-Louis de Secondat, Baron de La Brède et de Montesquieu})}

\epigraph
{Vēsture ir vislabākais skolotājs, kuram ir paši sliktākie skolnieki.}
{Indira Prijadaršinī Gandija (\entxti{Indira Priyadarshini Gandhi})}

\epigraph
{Nabadzība noved pie revolūcijas, revolūcija pie nabadzības.}
{V.~Igo (\frtxti{Victor Marie Hugo})}

\epigraph
{Tur, kur ir divi poļi, pastāv trīs viedokļi.}
{Poļu paruna}

\epigraph
{Pateicoties vienprātībai aug mazas valstis, ķildu dēļ iet bojā lielvalstis.}
{Henriks Senkevičs (\pltxti{Henryk Adam Aleksander Pius Sienkiewicz})}



\epigraph
{Piesien kādu skrandu pie spieķa, turi to augstu, un tu redzēsi, cik daudzi sekos tai kā karogam.}
{Staņislavs Ježijs Lecs (\pltxti{Stanisław Jerzy Lec})}

\epigraph
{Lielā Kartāga veda trīs karus. Pēc pirmā tā joprojām bija spēcīga. Pēc otrā tā vēl bija apdzīvota. Pēc trešās tā vairs nebija atrodama.}
{Bertolts Brehts (\detxti{Bertolt Brecht})}

\epigraph
{Pat visa zeme nav vienas veltīgi izlietas asins lāses vērta.}
{Aleksandrs Suvorovs (\rutxti{Александр Васильевич Суворов})}

\epigraph
{Kad tiek meklēti kontrrevolūcijas panākumu cēloņi, no visām pusēm var saņemt parocīgu atbildi, ka X kungs vai pilsonis Y nodeva tautu. Šī atbilde var būt pareiza un arī nē \dots{} katrā gadījumā tā nekādi nepaskaidro, kā tas notika, ka tauta ļāva sevi nodot.}
{Fridrihs Engelss (\detxti{Friedrich Engels})}

\epigraph
{Naids ir baiļu sekas, mēs vispirms baidāmies un tikai pēc tam ienīstam.}
{Sirils Konolijs (\entxti{Cyril Vernon Connolly})}



\epigraph
{Patriotisms~--- tas ir „mīlu savu”, nacionālisms~--- „nīstu svešu”.}
{Šarls de Golls (\frtxti{Charles André Joseph Marie de Gaulle})}

\epigraph
{Patriotisms ir mīlestība pret savējiem, nacionālisms~--- naids pret citiem.}
{Rihards fon Vaiczekers (\detxti{Richard Karl von Weizsäcker})}

\epigraph
{Nacionālisms daudz vairāk asociējas ar naidu pret svešu tautu nekā ar mīlestību pret savējo.}
{Nikolajs Berdjajevs (\rutxti{Николай Александрович Бердяев})}

\epigraph
{Cilvēks, kas ienīst citu tautu, nemīl arī savējo.}
{Nikolajs Dobroļubovs (\rutxti{Николай Александрович Добролюбов})}

\epigraph
{Novests līdz galējam sasprindzinājumam, nacionālisms dzen postā tautu, kas tam ļāvusies, padarot šo tautu par cilvēces ienaidnieci, jo cilvēce vienmēr izrādīsies stiprāka par atsevišķu tautu.}
{Vladimirs Solovjovs (\rutxti{Владимир Сергеевич Соловьёв})}

\subsection{Polijas dalīšanas, Varšavas hercogiste un Vīnes kongress}


Poļu etnosa attīstības un poļu nācijas veidošanās apstākļi līdz XVIII gadsimta beigām bija samērā labvēlīgi. Poļu apdzīvotā teritorija bija kompakta, nesadalīta ar dabīgiem kalnu vai ūdens šķēršļiem. Senseno poļu zemju kodols gadsimtu gaitā ietilpa vienotā valstī. Tiesa, arī aiz šīs valsts robežām atradās vairākas poļu teritorijas, bet pati valsts bija daudzetniska, taču lielākā iedzīvotāju masa tajā runāja vienā~--- poļu valodā, piederēja vienai~--- katoļu ticībai. Mūsdienu poļu vēsturnieks J.~Tazbirs gan raksta, ka pēc aptuveniem aprēķiniem tikai ap 40\% Žečpospolitas (\pltxti{Rzeczpospolita Obojga Narodów}~--- Polijas-Lietuvas apvienotā valsts jeb Abu Tautu Republika) iedzīvotāju bija poļi, pie tam tikai daļai no tiem piemita nacionālā apziņa. (Pēc šī darba autora domām attiecībā uz šo laiku pareizāk būtu runāt par etnisko, nevis nacionālo apziņu.) Sākot ar Ļubļinas ūniju 1569.~gadā, kad tika izveidota Žečpospolita un Lietuvas lielkņaziste zaudēja savu krievisko politisko komponenti, Žečpospolitā sākās Pareizticīgās baznīcas apspiešana, rietumkrievu rakstiskās valodas izraidīšana no darbvedības, visa krieviskā vajāšana. Tā īpaši pastiprinājās pēc Brestas baznīcas ūnijas (1596), kad pastiprinājās katoļu ekspansija (Brestas ūnijas rezultātā daļa Žečpospolitas pareizticīgo garīdznieku pakļāva Romas pāvestam, saglabājot daļu pareizticīgo ritu un savu hierarhiju) tajās pareizticīgo austrumslāvu zemēs, kuras bija Žečpospolita sastāvdaļa.

XVII--XXVIII gadsimtā Polijas sabiedrība dalījās trijās kārtās: muižniecībā jeb šļahtā, sīkpilsoņos un zemniekos.

Šļahta bija karojošu kungu kārta, kura sevi krasi norobežoja no pārējām~--- zemākajām kārtām, saucot to piederīgos par liellopiem (\pltxti{bydło}). Šļahtai piederēja monopols uz varu, zemes īpašumu un sabiedrisko prestižu. XV--XVI gadsimtā izveidojās, bet nākamajos gadsimtos par visas šļahtas ideoloģijas sastāvdaļu kļuva t.s. sarmatisms. Tas balstījās pieņēmumā, ka šļahtas izcelsme ir saistāma ar sarmatiem (sena nomadu tauta, kura runāja indoirāņu valodā), kas antīkajā laikmetā pakļāvuši slāvu ciltis un izveidojuši virsslāni. Sarmatisms it kā pamatoja Žečpospolitas muižniecības tiesības norobežoties no etniski ``svešajiem''~--- slāvu un lietuviešu zemniekiem. Sarmatisma ideoloģijas galvenie elementi bija~--- šļahtas neierobežota brīvība, nacionālā augstprātība, ticība Polijas vēsturiskajai izredzētībai, ksenofobija (neiecietība pret svešo). Sarmatisms kultivēja īpašas parašas. Tā, ``īsts šļahtičs'' labāk mira badā, bet neaptraipīja rokas ar tam nepiedienīgo fizisko darbu.

Starp citu, poļu publicists un politiķis A.~Vasiļevskis atzīmējis, ka poļu kultūras vēsturē gadsimtu gaitā izveidojušās divas nacionālo problēmu aprakstīšanas skolas, kur katra no tām vadījās no savas patriotisma izpratnes. Sarmatiskais virziens identificēja patriotismu ar pašslavināšanu, ar nekritisku, visa, kas bija savs, attaisnošanu. Otrs~--- demokrātiski reformistiskais virziens nebaidījās teikt tautai rūgtu patiesību, bet patriotismu identificēja ar atklātu norādīšanu uz nacionālajiem trūkumiem, kaut šādas pārdomas par saviem grēkiem bija arī sāpīgas. Šī darba autors var piebilst, ka arī mūsdienu Polijas vēstures literatūrā šie virzieni turpina pastāvēt.

Žečpospolitas spilgtākā īpatnība bija nekur citur Eiropā neredzēts šļahtas~--- šī priviliģētā slāņa daudzskaitlīgums. XVIII gadsimta vidū gandrīz ik desmitais valsts iedzīvotājs bija šļahtičs. Kaut kārtas uzvedības noteikumi paredzēja visu augstdzimušo šļahtiču pilnīgu vienlīdzību, patiesībā tā bija visai iluzora. Dižciltīgo ģerboņu īpašnieku vidū lielai daļai jau nepiederēja nedz zeme, nedz dzimtcilvēki. Šie nemantīgie šļahtiči parasti pelnīja iztiku un pajumti pie šļahtas bagātās daļas~--- t.s. magnātiem (no latīņu \latxti{magnatus}~--- liels cilvēks, \latxti{magnatis}~--- dižciltīgs cilvēks, parasti zemes lielīpašnieks. Žečpospolitā pastāvēja ap 60 magnātu dzimtu), kuru politiskais svars bija atkarīgs no t.s. ``klientu''~--- viņu atbalstītāju šļahtas rindās skaita.

Viens no Polijas neveiksmju cēloņiem bija tas, ka jebkura spēku koncentēšana centrālās varas vadībā kādu tālu mērķu sasniegšanas vārdā rada kategoriskus iebildumus Polijas nacionālajā elitē. No vienas puses, XVII gadsimtā poļu aristokrātijai bija pilnīgi nesaprotami, kāpēc saspringt, kad viss ir pieejams: gan galmi, lai tur ``spīdētu'', gan teritorijas ar līdz pusvergu stāvoklim nospiestiem iedzīvotājiem austrumos, kurus varēja pērt un no kuriem varēja vākt nodevas. No otras puses, centrālās varas nostiprināšanās, pakļaujot sev magnātus un šļahtu, nebija pēdējo interesēs.

Piemēram, atbalsta trūkuma centrālai varai dēļ Polija XVII gadsimtā palika bez kara flotes, kas bija viena no galvenajām militāro panākumu atslēgām (Šī perioda Polijas karaļi mēģināja izveidot reālu floti, taču viņu mēģinājumi atkārtoti izgāzās līdzekļu trūkuma dēļ karaļa kasē, jo Polijas šļahta neredzēja nepieciešamību pēc flotes un atteicās paaugstināt nodokļus tās celtniecībai. Vēl esošie kuģi tika pārdoti 1641.--1643.~gadā, kas nozīmēja Žečpospolitas flotes beigas.)

Nākamajā~--- XVIII gadsimtā šļahta nesaprata, kāpēc tai būtu jāatsakās no \latxti{liberum veto} (no latīņu, brīvais veto jeb brīvais aizliegums~--- Žečpospolitas parlamentārās uzbūves princips, kurš ļāva jebkuram Seima loceklim pārtraukt jautājuma apspriešanu. Deputātam vajadzēja tikai skaļi izsaukties latīniski ``\latxti{Sisto activitatem!}'' (Es pārtraucu darbības) vai poliski ``\pltxti{Nie pozwalam!}'' (Es neļauju!) un jādod vienlīdzīgas tiesības visiem valsts iedzīvotājiem. Rezultātā no 1573.~līdz 1763.~gadam, kad sanāca apmēram 150 Seimu, aptuveni trešdaļā no tiem netika pieņemts lēmums. Protams, ar vienu faktora, lai arī tik svarīga kā šļahtas nostāja, nevar visu izskaidrot.

Abas pārejās kārtas bija galvenās nodevu maksātājas, bet bez politiskām tiesībām. Pie sīkpilsoņiem piederēja pilsētu iedzīvotāji, uz kuriem attiecās pilsētu tiesības un kuri guva pastāvīgus ienākumus no tirdzniecības un amatniecības. 1790.~gadā pilsētās dzīvoja ap 16\% iedzīvotāju (bez ebrejiem gan tikai 6\%). Zemniecība sastādīja ap 75\% visu iedzīvotāju, lielākā tās daļa (ap 65\%) dzīvoja uz šļahtai piederošas zemes, pārējie~--- baznīcai un karalim piederošas. Pilsonībai (buržuāzijai)~--- lieltirgotājiem, manufaktūru īpašniekiem, baņķieriem XVIII gadsimta beigās varēja pieskaitīt tikai dažus desmitus cilvēku. Zemnieki (ap 85--90\%) pildīja klaušas, pārējie maksāja renti~--- t.s. činšu (poļu \pltxti{czynsz}, vācu \detxti{Zins}~, no latīņu \latxti{census}~--- procents~--- Polijā, Lietuvā un arī Latgalē tā sauca zemnieku nodevas jeb renti naudas vai produktu veidā). Pēc feodāļu-muižnieku gribas pāreja no činša atpakaļ uz klaušu saimniecību nesastapa grūtības un bija bieža parādība. Uz 100 zemnieku saimniecībām bija 20, kurās tika apsaimniekoti 7 līdz 8 ha zemes, 62 saimniecībās tās bija mazāk, 16 bija bezzemes saimniecības.

Ārpus kārtām stāvēja garīdzniecība, pilsētu nabagi un beztiesiskie un diskrimenētie ebreji. Augstākā un vidējā garīdzniecība izcelsmes ziņā bija tuva šļahtai, bet zemākā (īpaši no XIX gadsimta otrā ceturkšņa)~--- zemniecībai. Tā kā reliģija bija cieši saistītā ar nacionālo pašapziņu, pēc Polijas neatkarības zaudēšanas liela garīdzniecības daļa kļuva par aktīvu nacionālās atbrīvošanās kustības spēku.

Tiesa, vienprātības par garīdzniecības vērtējumu nav. Piemēram, vācu autors R.~Vingendorfs pirms Otrā pasaules kara rakstīja, ka katoļu garīdzniecībai gan esot bijusi mazāk svarīga nacionālā ideja, cik iespēja to izmantot kā cīņas līdzekli pret pareizticīgo Krieviju un protestantisko Prūsiju. Krievu vēsturnieks V.~Djakovs savukārt norādīja, ka sociālie motīvi garīdzniecībai bija mazāk nozīmīgi nekā citiem iedzīvotāju slāņiem, un kritiskos momentos tā parasti pieslējās šļahtai, tikai atsevišķi indivīdi nostājās zemniecības un citu darba cilvēku pusē.

Citi minētie slāņi, kaut piedalījās vēsturiskajos notikumos (sacelšanās u.c.), kādu patstāvīgu sabiedriski-politisku pozīciju neieņēma.

Pārdzīvojusi ``ziedu laikus'' XVII gadsimtā, Žečpospolita XVIII gadsimtā nonāca ekonomiskā un politiskā panīkumā, cīnījās par savu izdzīvošanu. Žečpospolita soli pa solim tuvojās katastrofai. Konfrontācija ar kaimiņvalstīm un pārmērīgi lielas kara izmaksas izsūca zeltu no valsts kases. Daudzie kari: ar Krieviju par ukraiņu un baltkrievu zemēm, ar Zviedriju~--- par Livoniju, ar Prūsiju~--- par tās austrumu apgabaliem, Žečpospolitu novājināja. Pirmā Žečpospolita izrādījās vāja valsts, kas nespēja aizstāvēt savu neatkarību un teritoriālo integritāti, bet tajā pašā laikā tik ķildīga, lai visos apkārtējos kaimiņos pamodinātu dedzīgu vēlmi atbrīvoties no šāda kaimiņa. Krievijai gan Žečpospolita droši piesedza tās robežas no Eiropas problēmām, un tā nekad nebūtu piekritusi šīs bufervalsts iznīcināšanu, ja tā parādītu zināmu saprātīgumu un necenstos apspiest pareizticīgos un pretoties Krievijas ietekmei.

Nebeidzamie iekšējie kari pret pareizticīgajiem iedzīvotājiem Žečpospolitas austrumu zemēs veicināja panīkumu. Karos lija poļu asinis, graujot valsts cilvēcisko potenciālu.

Valsti nomocīja dažādu iekšējo politisko spēku savstarpējās cīņas, pretrunas starp priviliģētajiem katoļiem un nevienlīdzīgajiem protestantiem, pareizticīgajiem un jūdu ticības piekritējiem, laikmetam neatbilstošā, novecojusī valsts iekārta, kad katrs pārstāvniecības iestādes~--- Seima loceklis varēja izmantot t.s. \latxti{liberum veto} tiesības, nobloķējot vairākuma pieņemta lēmuma izpildi. Poļu paruna pat ar zināmu lepnumu konstatēja, ka valsts pastāv pateicoties nekārtībām: ``\pltxti{Polska nierzadem stoi}''.

Nebija vienotības poļu un lietuviešu izcelsmes muižniecības vidū. Poļu vēsturnieks H.~Visners ir uzsvēris, ka poļu valdošie slāņi kopš Jagelloņu dinastijas (poļu \pltxti{Jagiellonowie}, lietuv. \lttxti{Jogailaičiai}~--- valdīja Lietuvā un Polijā no XIV līdz XVI gadsimtam) valdīšanas sākuma plānoja vienota Polijas politiskā organisma izveidi ar Lietuvas kunigaitiju. XVIII gadsimtā poļu programma, balstoties uz gadsimtiem ilgo savienības tradīciju, vairs neredzēja vietu suverēnai Lietuvas valstij. Arī vēlāk~--- jau XIX gadsimtā poļi nesaskatīja, ka lietuvieši rada savu etnosociālu un etnopolitisku kopību. Lietuviešu vēsturnieks un politologs A.~Kulakauskas secinājis, ka poļi uz XVIII gadsimta pārmaiņām Žečpospolitas teritrorijā, kuru viņi uzskatīja par savu īpašumu, reaģēja vēsturiski nepareizi [t.i~--- neatbilstoši savām tālākajām interesēm.~--- V.Š.]. Viņi neatzina lietuviešu tautu par pilnvērtīgu, uzskatīja Lietuvu par Polijas provinci un lietuviešus kā nacionālo mazākumu, kuri nedrīkst pat izvirzīt prasību pēc kulturālas autonomijas. Lai aizstāvētu savas tiesības pastāvēt kā patstāvīgai tautai savā valstī lietuvieši \lttxti{volens nolens} (gribot negribot) bija spiesti izvēlēties savu nacionālās attīstības ceļu, norobežojoties no poļiem.

Vēl viena problēma bija augstais ebreju iedzīvotāju īpatsvars Žečpospolitā. Ir aprēķini, ka 1800.~gadā 70\% visu pasaules ebreju dzīvoja Polijā un aptuveni 25\% bijušās Žečpospolitas iedzīvotāju bija ebreji, šeit izveidojās bagāts ebreju kultūras mantojums. Taču autohtonās etniskās grupas, kuras izsenis dzīvoja šai teritorijā, nekādi nevēlējās atzīt ebrejus par līdztiesīgiem.

Etniskās pretrunas savijās ar kārtu pretišķībām. Augstdzimusī ``poļu tauta'' (\pltxti{narod Polski}), kā viduslaikos sauca poļu šļahtu, centās turēt paklausībā vienkāršo ``kalpu tautu'' (\pltxti{narod chlopskie}).

Dažādās pretrunas Žečpospolitā labi saskatīja tālaika gaišākie Eiropas prāti. Jau Ž.~Ž.~Russo rakstīja: ``Lasot poļu valdīšanas vēsturi, ar grūtībām var saprast, kā tik dīvaini uzbūvēta valsts varēja pastāvēt tik ilgi.'' Arī lielais Apgaismības klasiķis Voltērs izteicās: ``\frtxti{Un Polonais~--- c´est un charmeur; deuz polonais~--- une begarre; trois polonais, eh bien, c´est la question polonaise.}'' (Viens polis ir apburošs cilvēks, divi poļi~--- tracis, trīs poļi, nu jā, tas jau ir Polijas jautājums.). Pagāja nedaudzi gadu desmiti, un virknes karu izpostītā Žečpospolita pat nespēja pati sevi aizstāvēt.

Tālākais Polijas liktenis, valsts sadale saistījās ar Eiropas valstu pūliņiem saglabāt spēku līdzsvaru starptautiskajā arēnā. Kompromisus varēja atrast uz tai laikā vājās Polijas rēķina. Kaimiņvalstis~--- Austrija, Prūsija un Krievija, izmantojot konfesionālās nesaskaņas, t.s. disidentu (no latīņu \latxti{dissidens (disidentis)}~--- tāds, kas nepiekrīt. Polijā, kur valdīja katolicisms, tā sauca kristiešus, kuri neatbalstīja valdošo konfesiju) cīņas pret katoļu privilēģijām, iejaucās valsts iekšējās lietās, traucēja nepieciešamo politisko reformu pieņemšanu, kas varētu stiprināt valsti.

Analizējot Žečpospolitas sadalīšanas iekšējos cēloņus, ievērojamais krievu vēsturnieks S.~Solovjovs pētījumā ''Polijas krišanas cēloņi'', (\_rutxti{Соловьёв С.М. История падения Польши, Москва, 1863}) uzsvēra, ka pirmajā vietā starp galvenajiem tās katastrofas cēloņiem liekami ne kaimiņvalstu agresīvie centieni, bet gan spēcīgā krievu (tai skaitā baltkrievu un ukraiņu) nacionālās atbrīvošanās kustība ''zem reliģiskā karoga'' pret poļu jūgu, par savām tiesībām, par vienlīdzību. Domājams, jāpiekrīt S.~Solovjova uzskatam, ka pašu poļu vaina bija tā, ka Polijas pareizticīgo vidū radās disidentu kustība, kuru galvenokārt atbalstīja Krievija, bet ne tikai. Polijas katoļu vairākums pat negribēja dzirdēt par atteikšanos no savām privilēģijām un tiesību vienlīdzību ar nekatoļiem un neuniātiem. Krievu vēsturnieks norādīja, ka 1653.~gadā, kad Maskavijas cara Alekseja Mihailoviča sūtnis kņazs B.~Repņins no Polijas valdības pieprasīja, lai pareizticīgie krievu cilvēki Žečpospolitā neciestu reliģiskos spaidus, pēdējās valdība nepiekrita šai prasībai, un sekas bija Mazkrievijas atkrišana no Žečpospolitas. Pēc simts gadiem Krievijas ķeizarienes Katrīnas~II vēstnieks, arī kņazs [N.]~Repņins, izteica tādu pašu prasību, bet saņēma atteikumu, un rezultāts bija pirmā Polijas dalīšana. S.~Solovjovs ar to gan vienpusīgi vienkāršoja vēsturisko procesu, taču būtisku tās aspektu atspoguļoja. Aizbildniecība pār t.s. disidentiem, vienlīdzīgu viņu tiesību ar katoļiem kā krievu tautā vispopulārākās lietas aizstāvība, bija īpaši svarīga Katrīnai~II,~--- rakstīja cits krievu vēsturnieks V.~Kļučevskis. Sākotnēji runa pat vairāk bija par, kā mūsdienās teiktu, cilvēktiesību aizstāvības politiku, nevis par Krievijas valsts teritoriālo paplašināšanu un liel-, maz- un baltkrievu apvienošanu Krievijas impērijas robežās.

Krievijai labvēlīgu politiku gan ieturēja 1764.~gadā ar tās atbalstu par Žečpospolitas karali ievēlētais viens no Krievijas imperatores Katrīnas~II favorītiem S.~Poņatovskis. Karalis Staņislavs~II Augusts sāka savu politisko darbību, patiesi ticot, ka tikai ar Krievijas palīdzību Polijā var īstenot viņa iecerētās reformas. Viņš atbalstīja poļu rūpniecības attīstību, tirdzniecības kompāniju dibināšanu, daudz vērības veltīja literatūrai un zinātnei. Reizē viņš mēģināja ierobežot magnātu patvaļu un nostiprināt centrālo varu. Taču drīz pierādījās, ka tāda Polijas attīstība ir nevēlama ne tikai lielai daļai poļu feodāļu, bet arī kaimiņivalstīm. Krievija pieprasīja atrisināt disidentu nevienlīdzības jautājumu. Tikai 1768.~gadā milzīga Krievijas spiediena ietekmē Seims bija spiests atzīt pareizticīgo vienlīdzību ar katoļiem Žečpospolitā. 24.~februārī tas pieņēma jaunu ``mūžīgu'' miera līgumu ar Krieviju, kurā garantēja disidentiem toleranci un vienlīdzību, bet poļu valdošajam slānim tādas ``tiesības'' kā brīvas karaļa velēšanas un \latxti{liberum veto}. Tomēr poļu virsslānis šādu vienlīdzību nepieņēma. Tam vienlīdzība tiesībās ar citticībniekiem bija līdzvērtīga visu poļu brīvību zaudēšanai. Ar līgumu neapmierinātā daļa garīdzniecīnas, magnātu un šļahtas Baras cietoksnī izveidoja konfederāciju (\pltxti{Konfederacja barska}, 1768--1772) un uzsāka sacelšanos. V.~Kļučevskis to raksturoja kā ''poļu šļahtiču ''Pugačova dumpi'' \citespace{} apspiedēju laupīšanas gājienu par tiesībām uz apspiešanu''. Izraisījās pilsoņu karš, kas savukārt izsauca kaimiņvalstu intervenci.

1772.~gadā Austrija un Prūsija, kuras baidījās, ka Krievija varētu patstāvīgi sagrābt poļu un lietuviešu zemes, iniciēja \strong{pirmo Žečpospolitas sadali}, kuras rezultātā tā šķirās no vairākām svarīgām teritorijām. Cariskā Krievija sākotnēji iebilda pret sadali, cenšoties panākt visas Žečpospolitas pakļaušanu savai ietekmei, jo pie savām rietumu robežām tai bija izdevīgāk saglabāt vāju Žečpospolitu nekā spēcīgu Prūsiju, kura jau 1648.--1721.~gadā bija pievienojusi sev daļēji poļu apdzīvotās Rietumu Pomorjes jeb Pomerānijas \pltxti{(poļu Pòmòrzé}, vācu \detxti{Pommern}) un 1740.~gadā Silēzijas (poļu \pltxti{Śląsk}, vācu \detxti{Schlesien}) teritorijas.\footnote{Par jau pirms Polijas trijām dalīšanām Prūsijas rokās nonākušajām poļu apdzīvotajām teritorijām šajā darbā ies runa tikai kopsakarā ar citām, XVIII~gs. otrajā pusē Žečpospolitā ietilpstošajām zemēm.} Taču kad starptautiskā situācija (1768.~gadā Turcija pieteica karu Krievijai, kurš ilga līdz 1774.~gadam, draudēja arī Austrijas iesaistīšanās tajā pret Krieviju) virzīja Krieviju uz savienību ar Prūsiju, tā piekrita sadalei. Katrīna II raksturoja Polijas dalīšanas cēloņus no sava skatu punkta: ``Šīs [poļu] tautas nepastāvības, tās ļaunprātības un naida pret mūsu [tautu], nemitīgas tieksmes uz izvirtību un franču negantībām rezultātā, mēs nekad tajā neatradīsim nedz mierīgu, nedz drošu kaimiņu, vienīgi kā novedot to būtiskā nespēkā un nevarenībā''.

Austrija ieguva Austrumgalīcijas (bieži literatūrā apgabalu sauc arī vienkārši par Galīciju) ar Ļvovu (ukraiņu \uktxti{Львів}, poļu \pltxti{Lwów}, vācu \detxti{Lemberg}) un t.s. Mazpoliju (\pltxti{Małopolska}~--- apgabals Polijas dienvidaustrumos Vislas vidus un augštecē, tiek saukts arī par Rietumgalīciju. Vēsturiski galvenā pilsēta~--- Krakova (Krakow), tikai bez Krakovas, kura palika Žečpospolitā. Mūsdienās Austrumgalīcija atrodas Ukrainas, bet Mazpolija jeb Rietumgalīcija Polijas teritorijā. Lai aizmaskotu Polijas sadali, Austrija atcerējas, ka kādreiz (XIV gadsimtā) Austrijas pakļautā Ungārija valdīja pār Galīciju un tāpēc esot notikusi ``atkalapvienošanās''. 1774.~gadā Austrijā pat izkala medaļu ar uzrakstu ``\latxti{Antigua jura Vindicata Galicia et Lodomeria in Fidem recepetis MDCCLXXIII}'' (Senās Galīcijas un Lodomērijas tiesības atgūtas 1873.~gadā.

Prūsija, kura sevišķi bija ieinteresēta apvienot divas līdz tam atsevišķi pastāvējušās savas valsts daļas: Brandenburgu ar Pomerāniju un Austrumprūsju, saņēma Rietumprūsiju (vācu~--- \detxti{Westpreußen}~--- apgabals Vislas upes lejteces abos krastos), gan bez Dancigas (vācu \detxti{Danzig}, poļu \pltxti{Gdansk}) un Toruņas (poļu \pltxti{Toruń}, vācu \detxti{Thorn}), Kujāvijas (poļu \pltxti{Kujawy}, vācu \detxti{Kujawien}) reģiona ziemeļdaļu (Vislas kreisajā krastā) un daļu Lielpolijas (poļu \pltxti{Wielkopolska}~--- apgabals Polijas rietumos Vartas (poļu~--- \pltxti{Warta}, vācu~--- \detxti{Warthe}) upes baseinā, kuru agrāk apdzīvoja visļanu un poļanu ciltis). 1773.~gadā Prūsijas karalis Fridrihs~II vēstulē Voltēram, tēlojot šo provinču šķietamo atpalicību, rakstīja: ``Poļu provinces nevar salīdzināt ne ar vienu no Eiropas valstīm. Augstākais, tās ir salīdzināmas ar Kanādu. Būs jāpieliek daudz pūļu un laika, lai panāktu to, kas garos gadsimtos sliktas pārvaldes dēļ ir nokavēts.'' Prūsija tagad ar Žečpospolitai atņemto piejūras apgabalu savienoja Austrumprūsiju ar pārējo valsts teritoriju. Ar to Polija zaudēja arī savus svarīgākos ārējās tirdzniecības ceļus, bet Prūsija ieguva iespēju kontrolēt vairāk nekā piektdaļu Polijas ārējās tirdzniecības, gūt no muitas nodevām, ar kurām tika apliktas pa Vislu vestās poļu preces (pie Dancigas muitas nodoklis sastādīja 12\% no preču vērtības), ienākumus, kuri pēc dažiem vērtējumiem bija lielāki par pārpalikušās Polijas ienākumiem.

Krievija savukārt ieguva Latgali ar \lttxti{Dinaburgu} (tagadējo Daugavpili) un Austrumbaltkrieviju ar Polocku, Vitebsku un Mogiļevu un t.s. Melno Krieviju (lietuviešu \lttxti{Juodoji Rusia}~--- agrākās Lietuvas lielkņazistes daļu Daugavas labajā un Berezinas upes kreisajā krastā). Prūsijas karalis Fridrihs~II atzina: ``\dots{}Krievijai ir daudz tiesību tā rīkoties ar Poliju, ko gan nevar teikt par mums ar Austriju''. Karaļa teiktais prasa komentāru. Zemes, ko sagrāba Prūsija, bija pārsvarā poļu apdzīvotas. Austrijai piešķirtajās zemēs dzīvoja galvenokārt rietumukraiņi un poļi. Vāciski runājošo tajās bija nedaudz. No šī viedokļa Krievijai, pievienojot sev slāvu apdzīvotas zemes, bija uz tām vairāk tiesību nekā tās sabiedrotajām. 1772.~gadā, ko poļi atzīmē kā Polijas pirmo dalīšanu, ievērojama Baltkrievijas daļa tika atbrīvota no poļu jūga, un Krievijai šis notikums bija ne mazāk nozīmīgs kā daļas Mazkrievijas (Ukrainas) atbrīvošanās 1654.~gadā no tāda pat jūga un apvienošanās ar Krieviju. Krievu vēsturnieki to vērtē kā faktisku visu trīs bijušo senkrievu atzaru~--- baltkrievu, lielkrievu un mazkrievu apvienošanu vienas Krievijas valsts ietvaros pēc vairāku gadsimtu šķelšanās. Par iespēju atbrīvot daļu ticības brāļu~--- baltkrievu no katoļu diskriminācijas, par viņu pievienošanu Krievijai gan nācās piešķirt brīvas rokas Prūsijai un Austrijai attiecībā uz citām poļu zemēm.

Taču jau padomju vēsturnieki uzsvēra, ka nedz Prūsija, nedz Austrija neuzdrošinājās sagrābt Žečpospolitas teritorijas, kamēr to nebija sankcionējusi cariskā Krievija. Tāpēc carismam tāpat kā tā sabiedrotajiem jānes pilna atbildība par 1772.~gada Polijas dalīšanu, kas nesa prūšu un austriešu varas nodibināšanu pār to valdījumos nonākušajām poļu un ukraiņu tautas daļām.

Jau tūlīt pēc pirmās Polijas dalīšanas neapmierinātie poļu šļahtiči uzsāka cīņu pret carisko Krieviju. Tā, daži pret Krievijas ietekmi Polijā karojošās un sakautās šļahtiču Baras konfederācijas dalībnieki, kuri tika izsūtīti uz Krievijas iekšieni, J.~Pugačova vadītā dumpja jeb Zemnieku kara (1773--1775) laikā pieslējās tam.

Austrija kopumā saņēma ap 83 tūkstošus km$^{2}$ ar 2,6 miljoniem iedzīvotāju. Prūsija kopā ieguva ap 36 tūkstošus km$^{2}$ teritorijas ar 580~000 iedzīvotāju. Krievijas guvums bija ap 92 tūkstoši km$^{2}$ ar 1,3 miljoniem iedzīvotāju. (Literatūrā sastopami arī nedaudz citādi dati. Tautas skaitīšana, kuru veica Poliju dalījušās valstis līdz 1776.~gadam, rādīja, ka Prūsijas iedzīvotāju skaits pieaudzis par 0,6~miljoniem, Austrijas~--- 2,1~miljonu, Krievijas~--- 1,3~miljonu cilvēku.)

Teritoriālās pārmaiņas apstiprināja Žečpospolitas Seims 1773.~gadā. Tikai Novogrudokas (baltkrievu \betxti{Навагрудак}, poļu \pltxti{Nowogródek}) deputāts T.~Rejtans pret to protestēja. XIX~gadsimtā poļu sabiedriskā doma viņu pacēla nacionāla varoņa augstumos. Izcilais poļu mākslinieks J.~Matejko šim notikumam veltīja vienu no savām gleznām.

Tādejādi pirmajā dalīšanā Žečpospolita šķīrās no vairāk nekā 28\% savas teritorijas un vairāk nekā trešdaļas tās iedzīvotāju. Tomēr apmēram 527~tūkstošu km$^{2}$ plašais Polijas valsts pārpalikums vēl joprojām bija tikpat liels, kā Francija vai Lielbritānija un lielāks nekā 1918.~gadā radītā Polijas valsts. Pēc pirmās dalīšanas lielākā daļa poļu, visas lietuviešu, daļa ukraiņu un baltkrievu zemju joprojām palika Žečpospolitas sastāvā. Pēc poļu vēsturnieka T.~Korzona vērtējuma pārpalikušajā Žečpospolitā dzīvoja 7,4~miljoni iedzīvotāju. (Tādejādi, pirms pirmās dalīšanas Žečpospolitā bija ap 11,4~miljoni iedzīvotāju).

Valsts sadalīšana izsauca poļu sabiedrībā šoku, patriotisma uzplūdus un vēlmi modernizēt valsti. Polijas vēstures pētnieks I.~Balabans gan rakstīja, ka pēc pirmās dalīšanas ``visi saprata, ka ir nepieciešamas reformas'', taču tad jāmin arī poļu vēsturnieka M.~Bobržinska konstatējums, ka Polijas tautu valsts dalīšana ne pārāk satrauca. Lielāku ietekmi atstāja Apgaismības kustība Eiropā, vēlāk~--- Franču revolūcijas sākums (1789). Norisa reformas izglītības un administratīvās pārvaldes jomā. Tā, Polijā pēc karaļa Staņislava Augusta priekšlikuma tika radīta laicīga Izglītības komisija (\pltxti{Komisja nad Edukacją Młodzi Szlacheckiej Dozór Mająca}, 1773--1794) pirmā Eiropā iestāde, kura pēc savām funkcijām līdzinājās visu valsti aptverošai izglītības ministrijai. Pēc Prūsijas parauga tika reorganizēta armija. Tomēr pietiekami lielas un modernas armijas radīšanu, kura varētu nodrošināt Polijas politisko neatkarību, nepieļāva valsts finansiālais vājums. (Pēc M.~Bobržinska datiem vairāk kā puse Žečpospolitas ienākumu tika tērēta armijas uzturēšanai.)

Savukārt finansiālo vājumu lielā mērā noteica valsts politiskā iekārta. Absolūtie monarhi kaimiņvalstīs daudz brīvāk varēja aplikt savus pavalstniekus ar nodokļiem nekā Polijas Seima deputāti savus vēlētājus. Kā rakstīja poļu vēsturnieks J.~Rutkovskis, galvenais traucēklis enerģiskas finansiālās politikas realizācijai bija Seimu ekonomiskā pozīcija. Seimu deputāti pārstāvēja šļahtas intereses, kura vēlējās savās rokās koncentrēt maksimāli lielu nacionālā ienākuma daļu. Agrārpolitikā šļahta aizstāvēja dzimtbūšanas saglabāšanu. Rūpniecības un tirdzniecības jomā Seimi pirmkārt rūpējās par iespēju pārdot lauksaimniecības ražojumus par iespējami augstākām un iepirkt rūpniecības preces par iespējami zemākām cenām. Tas deva šļahtai iespēju celt savu dzīves līmeni. Zemnieki nekādu labumu no šīs politikas neguva. Amatnieki par savu darbu tika atalgoti arvien mazāk. Tas traucēja amatnieciskās ražošanas pāraugšanu par kapitālistisko ražošanu.

Ap 1780.~gadu sabiedriskā aina Polijā gan bija mainījusies uz labo pusi. Neraugoties uz teritoriālajiem zaudējumiem un tirdzniecības ierobežojumiem, saimniecība Žečpospolitā attīstījās. Ja 1776.~gadā valstī ieveda preces par 48 miljoniem zlotu, bet izveda tikai par 22 miljoniem, tad 1785.~gadā eksports (pamatos lauksaimnieciskā produkcija) pirmoreiz ilgu gadu laikā pārsniedza importu. Pieauga valsts ienākumi. Lauksaimniecības ražotā galvenā prece iekšējā tirgū bija rudzi, ārējā~--- kvieši. Tomēr pakāpeniski graudkopība zaudēja savas pozīcijas. No XVIII gadsimta beigām sākās kartupeļu audzēšana. To izplatība iezīmēja progresu lauksaimniecības preču ražošanā, jo kartupeļus izmantoja pārtikā, lopbarībā un kā izejvielu pārtikas rūpniecībā. No tehniskajām kultūrām plaši tika audzēti lini un kaņepāji. Pastāvēja manufaktūras un rūpali. Paplašinājās saražoto preču sortiments. Sākās pāreja uz jaunu kurināmā veidu~--- akmeņoglēm. XVIII gadsimta 90.~gados darbojas ap 280~uzņēmumu, kuri izmantoja galvenokārt algotu darbaspēku. Dzimtcilvēki raka kanālus, kuri savienoja Baltijas un Melnās jūras baseina upju augšteces, kas ļāva daļu tirdzniecības novirzīt uz Melno jūru. Auga pilsētas kā ārējās tirdzniecības centri~--- Gdaņska (poļu \pltxti{Gdańsk}, vācu \detxti{Dancig}) un Poznaņa (poļu \pltxti{Poznań}, vācu \detxti{Posen}). Pēc jau minētā T.~Korzona aprēķiniem, kopš 1775.~gada iedzīvotāju skaits pieauga par 1,4 miljoniem un sasniedza 8,8 miljonus. Varšavas iedzīvotāju skaits 1791.~gadā sasniedza 120~tūkstošus. Tiesa, iekšējā tirdzniecība bija attīstīta vēl vāji. Izplatītākā tās organizācijas forma bija vietējie un gadatirgi. Galvenais šaurā tirgus cēlonis bija zemnieku saimniecību pusnaturālā rakstura saglabāšanās, brīvā tirgus attiecību vājā iespiešanās laukos. Zemnieki sastādīja 72,7\%, pilsētnieki-kristieši~--- 6,8\%, ebreji, kuri mita galvenokārt pilsētās un miestos,~--- 10,2\%, šļahta~--- 8\%, garīdzniecība~--- 0,6\%, pārējie~--- 1,7\%.

Uzplauka kultūra. Karalis Staņislavs II Augusts (Poņatovskis) bija Apgaismības un rokoko mākslas atbalstītājs. Viņa ciešie kontakti ar Drēzdeni veicināja poļu mūzikas dzīvi. Karalis uzstājās arī kā mākslas mecenāts un veicināja celtniecības darbus Varšavā. Viņa laikā uzbūvētā ievērojamākā celtne ir Karaļa pils jeb ``Pils uz ūdens'' (\pltxti{Pałac Na Wodzie}) Lazenkos (\pltxti{Łazienki}).

Taču valsts tautsaimniecības un kultūras attīstību traucēja dzimtbūšana, brīva darbaspēka trūkums, arī novecojusī politiskā iekārta.

Demokrātiski-patriotiskais virziens iestājās par mantojamas monarhijas ieviešanu, \latxti{liberum veto} tiesību likvidēšanu. Kad 1787.~gadā sākās jauns Krievijas~--- Turcijas karš, kas saistīja Krievijas spēkus, poļos radās cerība izmantot situāciju, lai likvidētu Polijas atkarību no tās. Krievu vēsturnieks S.~Solovjovs šai sakarā rakstīja: ``Redzēja [poļi] Krieviju apgrūtinātā stāvoklī~--- un gribēja to izmantot; nespēja to izmantot lai iedvestu jaunus spēkus paralizētajā [Polijas] valsts ķermenī, toties pilnībā izbaudīja prieku iespert lauvam, netikuši skaidrībā, ka lauva ne tikai nav tuvs nāvei, bet nav pat saslimis.'' Kopā ar Krieviju pret Turciju karoja arī Austrija. Savukārt Prūsija baidījās no Austrijas un Krievijas nostiprināšanās.

Krievijas valdniece Katrīna II cerēja iesaistīt arī Žečpospolitu savienībā pret Turciju, 1788.~gadā sanāca konfederatīvs (šeit~--- tāds, kurā lēmumus pieņēma ar vairākumu balsu) Seims (t.s. Četrgadu Seims~--- \pltxti{Sejm Czteroletni}, 1788--1792), lai apspriestu šo jautājumu. Pret Žečpospolitas savienību ar Krieviju iebilda Prūsijas karalis Fridrihs Vilhelms II, kurš baidījās no Krievijas nostiprināšanās kara rezultātā. Seims atteicās noslēgt līgumu ar Krieviju. Apstākļos, kad Eiropā brieda lieli notikumi (1789.~gada 14.~jūlijā Parīzē tika ieņemta Bastīlija), patriotiskie poļu spēki Seimā uzsāka cīņu par valsts politisku un ekonomisku reformēšanu. Viens no patriotu pārstāvjiem filozofs un rakstnieks S.~Stašics 1790.~gadā griezās pie sabiedrības ar uzsaukumu ``\pltxti{Przestroi dla Polsi}'' (``Brīdinājums Polijai''), kurā bija vārdi: ``Cik tālu Polija ir atpalikusi! Polijā ir sācies tikai XV gadsimts, kad pārējā Eiropā beidzas XVIII gadsimts!''

To, ka šī atpalicība tomēr bija visai nosacīta, rādīja \strong{1791.~gada 3.~maijā} Četrgadu Seima pieņemtā jaunā valsts \strong{Konstitūcija}, kura likvidēja Žečpopolitas konfedaratīvo raksturu, Lietuvas lielkņazistes nosacīto patstāvību, ieviesa franču filozofa un rakstnieka Š.~Monteskjē proponēto varas dalīšanas principu, nodalot likumdošanas, izpildvaru un tiesu varu, pasludināja t.s. pilsoniskās brīvības. Polijas Konstitūcija iedibināja arī citus jaunievedumus: kontrasignācijas (no latīņu \latxti{contra} pret + \latxti{signare} parakstīt) principu, kad valdības vadītājs vai atsevišķs ministrs līdzās valsts galvam parakstīja tā izdotus aktus, tā uzņemdamies politisku un juridisku atbildību; parlamentārās (politiskās) atbildības principu un neuzticības votumu (no latīņu \latxti{votum}~--- vēlēšanas, lēmums, domas, kas izteiktas balsojot) ar sekojošu ministru atsaukšanu no amata. Tika noteikta Seima kā divpalātu parlamenta uzbūve (deputātu un senatoru palāta, ar deputātu palātas izšķirošu nozīmi), Seima sēžu kārtība, likumdošanas procedūra un likumprojektu balsošanas kārtība, kvoruma (latīņu \latxti{quorum}~--- nepieciešamais sapulces dalībnieku skaits, lai tā būtu pilntiesīga) princips. Tāda parlamenta struktūra darbojas arī mūsdienās.

Šļahtas un garīdzniecības privilēģijas gan lielā mērā saglabājās, bet, likvidējot \latxti{liberum veto} un nostiprinot Seima tiesības pieņemt lēmumus ar balsu vairākumu, tika mazināta feodālā anarhija, magnātu varenība, paplašinātas augošās pilsonības (buržuāzijas) tiesības. Kā norādīja poļu vēsturnieks un valsts darbinieks H.~Jablonskis, patriotiskajām aprindām, kuras centās glābt Žečposplitu, bija skaidrs, ka, lai saglabātu jau satricināto Polijas neatkarību, pirmkārt bija jācenšas iznīcināt magnātu varu. Tāpēc 3.~Maija Konstitūcijas ieviestajai nemantīgās šļahtas, kura sastādīja magnātu galveno balstu, politisko tiesību ierobežošanai bija progresīva nozīme, kaut vecās iekārtas aizstāvji skaļi vaimanāja par ``šļahtiskās demokrātijas'' iznīcināšanu.

Polijas 1791.~gada 3.~Maija Konstitūcija bija neapšaubāms solis uz priekšu, tā bija otrā rakstītā satversme pasaulē pēc ASV 1787.~gada Konstitūcijas un pirmā Eiropā (Francijā 1791.~gada Konstitūcija tika pieņemta 3.~septembrī.) Franči Polijas satversmi izmantoja, izstrādājot savas valsts 1791., 1793. un 1795.~gada Konstitūcijas. Neraugoties uz ierobežotību, Polijas Konstitūcijai bija progresīvs raksturs un tās pieņemšanas gadadiena Polijā tiek svinīgi atzīmēta vēl joprojām. Četrgadu Seima veikums un īpaši jaunā Konstitūcija izsauca līdzjūtīgu vērtējumu ``visā Eiropā''. Tomēr, kaut arī Polijā tika ieviesta konstitucionāla monarhija, Konstitūcija bija novēlota, tā būtiski nemainīja sabiedrisko iekārtu, valstī joprojām dominēja feodālā kārtība, pilsonība palika ļoti vāja. Turpināja pastāvēt dzimtbūšana un ebreju īpašais nelīdztiesīgais stāvoklis.

Pie tam jaunās satversmes pieņemšanas ar karaļa piekrišanu diena bija izraudzīta tad, kad tās pretinieki Seimā vēl atradās brīvdienās un nevarēja piedalīties balsošanā. Pēc vācu vēsturnieka E.~Meijera datiem bija ieradušies tikai ap 100 no 500 Seima deputātiem. Ar to Konstitūcijas likumību varēja apšaubīt un tās pieņemšanu pielīdzināt valsts apvērsumam. Pie tam Konstitūcijas nostiprināšanai tika draudēts lietot arī nekonstitucionālus līdzekļus. Tā, 13.~maijā visās Varšavas ielās bija piestiprināti drukāti nezināmas izcelsmes uzsaukumi, kuri aicināja nokaut katru, kurš runāja, rakstīja, pretojās vai arī gatavojās to darīt pret 3.~maija konstitūciju. Katram tādam slepkavam tika solīta atlīdzība. Policija gan uzsaukumus noplēsa. Tomēr visi vietējie seimeļi (\pltxti{sejmik}) atzina Konstitūciju.

Taču izveidojās Žečpospolitai nelabvēlīga starptautiskā situācija. Ja poļu reformatori cerēja izmantot Austrijas un Prūsijas nesaskaņas, viņi vīlās. Pēc Austrijas ķeizara Jozefa II nāves 1790.~gada 20.~februārī jau 27.~jūlijā Austrija un Prūsija savstarpēji izlīga, bet 1791.~gada augustā noslēdza vienošanos pret Franciju. Mainījās Krievijas nostāja. Vēl 1790.~gada 12.~oktobrī Krievijas imperatore Katrīna II paziņoja ``Tā kā mēs raugāmies uz Poliju kā uz valsti, kura atrodas starp četrām spēcīgām valstīm [Prūsiju, Krieviju, Austriju, Turciju] un kalpo kā šķērslis to strīdiem, tad šo šķēršļi ir jāsaglabā un jāsargā, lai arī ko tas mums maksātu. Un mēs par to parūpēsimies. Taču tikai līdz brīdim, līdz mūsu ienaidnieku un pašas Polijas ienaidnieku ļaunie nodomi nepiespiedīs mūs mainīt mūsu politiku.'' Taču pēc krievu--turku kara (1787--1791) beigām 1792.~gada maijā Katrīna~II, kura baidījās no Polijas nostiprināšanās un tās mēģinājumiem atjaunot 1772.~gada robežas, kā arī revolucionārā ``franču mēra'' izplatīšanās, bija gatava iejaukties Polijas iekšējās lietās, aicinot poļus uz pilsonisku nepakļaušanos.

Saprotams, ka, baidoties par savām pozīcijām, magnāti un no tiem atkarīgie ``klienti'' meklēja Katrīnas~II atbalstu. Ievērojamais vācu vēsturnieks H.f.~Treičke par toreizējo situāciju Polijā rakstīja: ``Pār malām plūstošs cīņas prieks un gatavība upurēties, dedzīgas runas un brālīgi apskāvieni, vaimanājoši priesteri un augstprātīgas, skaistas sieviete, pie tā klāt punšs un mazurka, cik tik sirds vēlas, bet arī savstarpējais partiju ienaids, nepaklausība, dusmīgas apsūdzības un drošsirdīgu, sajūsminātu vīru viļņošanās vidū nevienas valstiskas galvas, neviena liela rakstura.'' Krievijas piekritēji 1792.~gada maijā organizēja t.s. Targovices konfederāciju (\pltxti{Konfederacja targowicka}, pēc miesta nosaukuma), kura pasludināja jauno Konstitūciju par spēkā neesošu un ar krievu karaspēka palīdzību vērsās pret karali un Seimu. 1792.~gada 18.~maijā Krievijas karaspēks iemaršēja Polijas teritorijā. Atsaucoties uz 1791.~gada 3.~maija poļu ``revolūciju'', to pašu darīja arī Prūsijas armija. (Interesanti, ka vācu vēsturnieks H.f.~Zitzevics šo rīcību salīdzinājis ar PSRS un tās sabiedroto, arī Polijas, karaspēka ieiešanu Prāgā 1968.~gadā.)

Poļu karaspēku komandēja karaļa brāļa dēls J.~Poņatovskis un Amerikas atbrīvošanās kara dalībnieks, ASV pilsonis ģenerālis \strong{T.~Kostjuško} u.c.

T.~Kostjuško bija dzimis mūsdienu Baltkrievijas senā, bet ne pārāk turīgā augstmaņu dzimtā. Iespējams, ka viņa pirmā valoda bija baltkrievu, un viņš tika kristīts gan pareizticībā, gan katolicismā. 1769.~gadā Kostjuško tika piešķirta karaļa stipendija mācībām Parīzē. Šeit viņš kā eksterns mācījās Parīzes kara akadēmijā. Parīzē pavadītie pieci gadi būtiski ietekmēja T.~Kostjuško vēlākos uzskatus. Piedalījās Amerikas neatkarības karā. 1776.~gada augustā ieradās ASV, dažus mēnešus vēlāk kļuva par Kontinentālās armijas galveno inženieri un vadīja Filadelfijas fortifikācijas darbus, sadraudzējās ar T.~Džefersonu. Pēc septiņiem gadiem armijas dienestā Kongress 1783.~gada 13.~oktobrī T.~Kostjuško piešķīra brigādes ģenerāļa dienesta pakāpi. Viņam tika piešķirta ASV pilsonība, īpašumi. Tomēr 1784.~gadā viņš devās atpakaļ uz Poliju, bet savu īpašumu novēlēja izmantot melno vergu izpirkšanai un izglītošanai. 1784.~gada augustā K.~Kostjuško ieradās Polijā un apmetās savā dzimtas īpašumā. Viņš būtiski uzlaboja savu dzimtcilvēku stāvokli, atvieglodams klaušas, tādejādi izpelnīdamies liberāļa slavu.

Poļi guva panākumus Zeleņces kaujā (\pltxti{Bitwa pod Zieleńcami}). Pēc kaujas karalis Staņislavs II Augusts nodibināja \latxti{Virtuti Militari} (kara nopelnu) ordeni, ar kuru apbalvoja J.~Poņatovski un T.~Kostjuško. Arī mūsdienās šis ordenis ir Polijas augstākais militārais apbalvojums. Tomēr tā kā atbilstoši stratēģiskajai situācijai poļi pārspēka priekšā turpināja atkāpties, krievu avotos kauja skaitījās kā viņu uzvarēta. Turpmāk Krievijas armija vairākās kaujās sakāva poļu un lietuviešu spēkus. 1792.~gada jūlijā arī Polijas karalis Staņislavs~II Augusts pievienojās konfederātiem, izdeva rīkojumu par savas armijas atlaišanu. Krievijas karaspēks ieņēma Varšavu.

1793.~gada janvārī Krievija un Prūsija, kuras vienoja bailes no revolucionārajiem notikumiem Francijā, to ietekmes izplatības Polijā, (Prūsija vēl arī nevēlējās pieļaut Krievijas robežu tālāku izplatību uz rietumiem) parakstīja slepenu vienošanos par jaunu, jau \strong{otro Polijas dalīšanu}. Austrija, kura tajā laikā bija cietusi vairākas sakāves karā ar Franciju, kā arī centās sagrābt Bavāriju, Polijas otrajā dalīšanā nepiedalījās. Tās noteikumi poļiem tika paziņoti 1793.~gada 27.~martā.

Krievija saņēma Rietumbaltkrieviju ar Minsku (baltkr. \betxti{Мінск)}, Labā krasta Ukrainu ar Žitomiras (ukraiņu \uktxti{Житомир}) pilsētu, Volīnijas (ukraiņu \uktxti{Волинь}, poļu \pltxti{Wołyń}) austrumu daļu un daļu Podolijas (poļu \pltxti{Podole}, ukr. \uktxti{Поділля}, \uktxti{Podilla}, krievu \rutxti{Подолье}), ar Braslavas (poļu \pltxti{Wrocław}, vācu \detxti{Breslau}, krievu \rutxti{Бреславль}) pilsētu, Prūsija~--- t.s. Lielpolijas apgabalu ar Gņezno (poļu \pltxti{Gniezno}, vācu. \detxti{Gnesen}), Poznaņas, Gdaņskas un Toruņas pilsētām. Katrīna~II, kura jau 1792.~gada decembrī Krievijas sūtnim Polijā J.~Siversam rakstīja, ka Krievija nolēmusi ``atbrīvot kādreiz Krievijai piederējušās, tās tautiešiem (\rutxti{единоплеменники}) un vienai ticībai piederīgajiem apdzīvotās zemes un pilsētas no viņus apdraudošās kārdināšanas un apspiešanas'', tagad par godu ukraiņu un baltkrievu apdzīvoto zemju ``atgriešanai'' Krievijas sastāvā lika izkalt speciālu medaļu. Tās viena pusē bija viņas pašas profils, otrajā~--- Krievijas ērglis, kurš savieno agrāko Krievijas un pievienoto zemju kartes ar uzrakstu ``\rutxti{Отторженная возвратихъ}'' (Atņemtā atgriešana).

Mūsdienās baltkrievu vēsturnieki lielākoties uzskata, ka tikai pateicoties baltkrievu zemju iekļaušanai Krievijas impērijās sastāvā tika pārtraukta to katolizācija un polonizācija, kas ļāva baltkrieviem saglabāties kā etnosam. Tieši Krievijas impērijas sastāvā noformējās patstāvīga baltkrievu nācija.

Žečpospolitas teritorija samazinājās vēl divas reizes. Tagad tā aptvēra vairs tikai 230~000 km$^{2}$ ar 4,4~miljoniem iedzīvotāju. Īpaši sāpīgs Žečpospolitai bija Gdaņskas pievienošana Prūsijai, jo ar to tika zaudēta arī pieeja jūrai. Valstiski-tiesiskās attiecības starp Krieviju un Žečpospolitu noteica traktāts, ko tās parakstīja 1793.~gada oktobrī. Pēc traktāta noteikumiem abas valstis apņēmās, ka, uzbrukuma gadījumā vienai no tām, otra palīdzēs tai ar visiem spēkiem. Apvienoto karaspēku komandēt tiesības saņemtu valsts, kura piešķirtu vairāk karaspēka. Praktiski tas nozīmēja, ka poļu karaspēks tika pakļauts krievu pavēlniecībai, jo Polijai bija tiesības turēt tikai 15~000 lielu armiju. Krievija saņēma tiesības vajadzības gadījumā ievest savu karaspēku Polijā, tiesības apstiprināt tās ārpolitiskos līgumus. Tika atcelta arī 3.~maija Konstitūcija. Faktiski pēc otrās dalīšanas Žečpospolita zaudēja savu neatkarību. Tā no iespējamā Krievijas pretinieka kļuva par tās vasaļvalsti.

Žečpospolitas sadalīšanu u.c. noteikumus apstiprināja t.s. Grodņas Seims 1793.~gada rudenī. Panākt šī lēmumu pieņemt Seima deputātus Krievijas sūtnis J.~Siverss centās ar kukuļiem, draudiem, tiešu spiedienu (Grodņas pils bija pilna krievu kareivju). Kad Seima maršals prasīja nobalsot par līgumu, deputāti vairākas stundas klusēja. Galu galā atskanēja Krakovas deputāta J.~Ankviča balss: ``Klusēšana ir piekrišana''. Tad nu Seima maršals paziņoja, ka līgums pieņemts vienbalsīgi. Krievu vēsturnieks S.~Solovjovs uzsvēra, ka šādu notikumu gaitu noteica gadsimtiem ilgā poļu tautas klusēšana, kad trokšņoja vienīgi šļahta seimos.

Taču poļu patriotu cīņas griba vēl nebija sagrauta, cerot uz revolucionārās Francijas palīdzību, viņi slepus gatavoja \strong{sacelšanos}. Par savu vadītāju viņi izvirzīja T.~Kostjuško, kurš sevi jau bija pierādījis par drosmīgu karavadoni Amerikas brīvības cīņās un arī Polijā. 1794.~gada 16.~martā Krakovas iedzīvotāji pasludināja T.~Kostjuško par republikas diktatoru un nacionālo bruņoto spēku virspavēlnieku. T.~Kostjuško komandētie poļu spēki sasniedza līdz 70~000 cilvēku (kopumā, ieskaitot mobilizētos pilsētniekus un zemniekus, tautas miliciju, poļu vienības sasniedza vairākus simtus tūkstošus cilvēku), taču bija ļoti vāji apbruņoti. 1794.~gada 4.~aprīlī T.~Kostjuško vadītajām vienībām izdevās gūt panākumus kaujā pie Raclaviciem (\pltxti{Bitwa pod Racławicami}), kur krievu karaspēks uzbruka ciešā ierindā, bet T.~Kostjuško, izmantojot pieredzi, gūtu ASV Neatkarības karā, poļu strēlniekus izvietoja izklaidus, izmantojot dabīgos aizsegus. Pats T.~Kostjuško izveda ar izkaptīm bruņoto zemnieku vienības krievu daļu aizmugurē, kur tie negaidītā straujā uzbrukumā sagrāba artilēriju un piespieda krievu kājniekus atkāpties. Taču iznīcināt tos neizdevās un kara darbība turpinājās.

Poļu panākumam vēlāk tika veltīta J.~Matejko glezna. 1894.~gadā Lembergā tika atklāta 114 metru gara, 15 metru augsta kaujas panorāma ar 38 metru diametru. 1944.~gadā kara gaitā panorāma tika daļēji bojāta un to sāka atjaunot tikai 1980.~gadā, bet 1985.~gadā atklāja Vroclavā. Nosauktie mākslas darbi joprojām kalpo romantizētam poļu vēstures izklāstam.

Pēc uzvaras pie Raclaviciem sacelšanās sākās arī citviet. Sacēlušos rokās nonāca Varšava un Viļņa. Varšavas sacelšanās gaitā liela daļa krievu garnizona gāja bojā, (viens bataljons~--- ap 500 karavīru atradās baznīcā, protams, neapbruņots. Sacēlušies ielauzās baznīcā un lielāko daļu karavīru nogalināja), otrai izdevās atstāt pilsētu. Kā rakstīja I.~Balabans, pēc krievu garnizona iznīcināšanas ``asiņainajiem skatiem'' ``franču revolūcijas piemēra iejūsminātā tauta'' izvilka no mājām arī ``Tēvzemes nodevējus'' un pakāra uz laternu stabiem. Vēsts par briesmu darbiem Varšavā sasniedza krievu armijas daļas, radot tajās spēcīgas atriebības jūtas.

Savā laikā padomju vēsturnieki V.~Djakovs un I.~Millers PSRS un Polijas sakariem veltītā rakstu krājumā rakstīja, ka ``humānā un labvēlīgā sacēlušos attieksme pret viņu gūstā nonākušajiem krievu virsniekiem un karavīriem lika tiem iestāties revolucionārās armijas rindās''. Šiem krievu karavīriem poļi nelika karot pret saviem tautiešiem, bet nosūtīja uz divīziju, kura atvairīja prūšu uzbrukumu. Nenoliedzot šādu faktu esamību, domājams, ka visiem revolucionāriem simpatizējošie padomju autori sacēlušos humānismu pārspīlēja un apžēloti kā reiz tika tikai tie, kuri izteica gatavību karot poļu pusē.

T.~Kostjuško stāvokli sarežģīja nepieciešamība rēķināties kā ar kustības rojālistisko, tā revolucionāro spārnu, kurš Varšavā jau organizēja revolucionāros tribunālus un politisko pretinieku pakāršanu. Piemēram, tika sodīts ar nāvi Viļņas bīskaps I.~Masaļskis, Lietuvas hetmanis J.~Zabello u.c. Vācu vēsturnieks H.f.~Zitzevics šo linča tiesu (angļu~--- \entxti{the Lynch law}, aizdomās par noziegumiem turēto cilvēku nogalināšanu bez tiesas un izmeklēšanas) pret īstiem vai iedomātiem sacelšanās pretiniekiem pastāvēšanu, kuras nobiedēja mēreno elementus, pat minējis kā tās sakāves galveno cēloni.

T.~Kostjuško mēģinājumi iesaistīt cīņā arī plašākus zemnieku spēkus izsauca šļahtiču neapmierinātību. 1794.~gada 7.~maijā Poļaņicas (\pltxti{Połaniec}) pilsētas apkārtnē tika pasludināts manifests (\pltxti{Uniwersał połaniecki}, poļu \pltxti{uniwersal}~--- vēstījums visiem), kurš prasīja, lai zemniekiem, kuri bija samaksājuši parādus un izpildījuši virkni citu nosacījumu, tiktu piešķirta personīgā brīvība, lai klaušu dienu skaits tiktu samazināts, lai zemes īpašnieki vai tās pārvaldnieki atbildētu tiesas priekšā par zemnieku apspiešanu kā vainīgi vēlmē pazudināt nacionālās sacelšanās lietu utt. T.~Kostjuško biedēja šļahtu, ka Maskava grib pret to sacelt poļu zemniekus, norādot uz to smago dzīvi un solot Katrīnas II labvēlību. Šļahta bija sašutusi par šādu mēģinājumu graut viņu īpašuma tiesības uz zemniekiem un universālam nebija praktisku seku, bet tikai simboliska nozīme.

Kad T.~Kostjuško nosūtīja savu pārstāvi uz Parīzi lūgt palīdzību Sabiedriskās glābšanas komitejai (\frtxti{Comité de salut publicē}, 1793--1795), tā apšaubīja poļu revolucionaritāti un atteica palīdzību. Komiteja poļu pārstāvim uzdeva jautājumus: ``Kā izskaidrot, ka jūsu Kostjuško, tautas diktators, cieš sev blakus karali, kuru pie tam, kā būtu jāzin Kostjuško, tronī iesēdināja Krievija? Kā izskaidrot, ka jūsu diktators aiz bailēm aristokrātu priekšā, kuri nevēlas atteikties no ``darba rokām'', neuzdrošinājās veikt zemnieku masu mobilizāciju? Kā izskaidrot, ka viņa proklamācijas zaudē savu revolucionāro nokrāsu, attālinoties no Krakovas? Kā izskaidrot, ka viņš nekavējoties apspieda karātavām tautas sacelšanos Varšavā? Atbildiet!'' Poļu pārstāvim nācās klusēt.

Toties pret T.~Kostjuško armiju karoja arī Austrijas un Prūsijas bruņotie spēki (visai neaktīvi, kaut daži vācu vēsturnieki apgalvo, ka tieši Prūsijas karaspēka pievienošanās Krievijas armijai noteikusi kara iznākumu).

Sacelšanās apspiešanā piedalījās arī slavenais krievu karavadonis A.~Suvorovs. 28.septembrī poļu galvenie spēki tika smagi sakauti, T.~Kostjuško ievainots un saņemts gūstā.

Pēc nostāstiem, tad viņš izteicis vārdus ``\latxti{Finis Poloniae}'' (no latīņu: ``Beigas Polijai'', literatūrā ir arī cits šo vārdu variants: ``\latxti{Finis regni Poloniae''}~--- ``Beigas Polijas karaļvalstij.''). Pats T.~Kostjuško vēlāk, 1803.~gadā gan rakstīja, ka viņš bija smagi ievainots (kājā un galvā), gūstā krita jau bez samaņas, ko atguva tikai pēc divām dienām, un nekādus pravietiskus vārdus nespēja pateikt. Viņš arī neuzskatījis sevi par pēdējo poli, ar kura nāvi Poljai pienāktu beigas. Pēc citas versijas, kad Varšavā uzzināja par notikušo, tad gan atskanējuši saucieni; ``Nav Kostjuško! Beigas tēvzemei!''.

Interesanti, ka T.~Kostjuško vadībā cīnījās arī jauns virsnieks M.~Oginskis, kurš pēc sakāves bija spiests atstāt dzimteni. Pēc vienas mūzikas zinātnieku versijas tieši 1794.~gadā arī radās viņa pasaulslavenā polonēze ``\pltxti{Pożegnanie Ojczyzny}`` (Atvadas no dzimtenes), saukta arī par ``\pltxti{Polonez Oginsky}'' (Oginska polonēzi). (Pēc citas versijas polonēze tapa pēc 1820.~gada, kad M.~Oginskis jau bija Krievijas imperatora Aleksandra I amnestēts, saņēma atpakaļ konfiscētās muižas, ieņēma senatora amatu, bet, neapmierināts ar imperatora politiku, devās uz Itāliju.)

Pēc tam A.~Suvorova vadībā krievu karaspēks devās uzbrukumā Varšavai 24.~oktobrī, ieņemot tās priekšpilsētu Prāgu (\pltxti{Praga}).

Uzbrukumā piedalījās arī tie pulki, kuri bija izcietuši poļu pēkšņo sacelšanos Varšavā. Cīņas Prāgā bija asiņainas. A.~Suvorovs ziņojumā par kauju rakstīja: ``Pārvarot visas grūtības un uzveicot pretinieka sīvo aizstāvēšanos trijos nocietinājumos, mūsu karaspēks ielauzās Prāgā. Briesmīga bija asinsizliešana, katrs solis uz ielām bija nosegts ar kritušajiem. Visi laukumu noklāti ķermeņiem, pēdējā un pati briesmīgākā iznīcināšana notika Vislas krastā, ko redzēja Varšavas tauta. Šis briesmīgais skats iedvesa viņiem šausmas…''

Poļu vēstures literatūrā uzsvērts, ka krievu karavīri zvērīgi izrīkojās ar civiliedzīvotājiem. Polijā stingri nostiprinājies viedoklis par mierīgo poļu iedzīvotāju ``masu slaktiņu''. Poļu vēsturnieki Varšavas piepilsētas civiliedzīvotāju nogalināšanu piemin bieži, neminot gan daudzus līdzīgus gadījumus tā laika karu un sacelšanos vēsturē. Šis sižets ietverts daudzās periodam veltītajās monogrāfijās un arī skolai domātās mācību grāmatās, kaut autoru vidū nav vienprātības par slaktiņa mērogiem. Tiek runāts par ``pilnīgu'' izkaušanu, ``ievērojamas daļas'' apslaktēšanu un vienkāršu ``slaktiņu''. Piemēram, ievērojamais poļu publicists, vēsturnieks, vēlāk arī diplomāts L.~Vasiļevskis rakstīja, ka Krievijas karaspēks ``apkāva un noslīcināja Vislā vairāk nekā 10 tūkstošu sieviešu un bērnu''. Viedoklis ir pārvērties par aksiomu, kuru Polijā vairs neviens neapstrīd. Jāatzīmē gan, ka mūsdienu krievu vēsturnieks A.~Širokorads atzīmējis, ka neviens no poļu vēsturniekiem nav devis atbildi uz jautājumu, kāpēc, ilgus mēnešus gatavojoties pilsētas aizsardzībai, poļu vadītāji neevakuēja Prāgas iedzīvotājus vismaz pāri Vislai un neizmitināja pārējo varšaviešu namos.

PSRS šī epizode netika pieminēta nedz vēsturnieku darbos, nedz uzziņu literatūrā. No pirmspadomju Krievijas vēsturniekiem par šo kara epizodi visplašāk rakstījis N.~Kostomarovs, kurš secināja, ka poļu stāsti par šo epizodi ``neiztur kritiku''. Atsaucoties uz krievu ``tā laika avotiem'', vēsturnieks raksta, ka pavisam gāja bojā ap 12 tūkstošu poļu (karavīru un civiliedzīvotāju), pie tam daudzi, ``glābjoties no krievu durkļiem'' noslīka Vislā, gūstā krita ap 1~tūkstotis cilvēku. Pēc krievu vēsturnieka D.~Bantiša-Kamenska datiem, Prāgas ieņemšanā tika nogalināti 13,5 tūkstoši poļu karavīru, ap 11,5~tūkstošu tika saņemti gūstā, līdz diviem tūkstošiem noslīka Vislā, cenšoties tai pārkļūt, ap tūkstotim tas izdevās. No 22~tūkstošiem krievu karavīru, kas piedalījās uzbrukumā Prāgai, 580 tika nogalināti un 960 ievainoti. (1894.~gadā Sanktpēterburgā Krievijas ģenerālštāba pulkveža N.~Orlova izdotā darbā par Prāgas ieņemšanu poļu kritušo skaits minēts no 9 līdz 10 tūkstošiem, ievainoto no 11 līdz 13 tūkstošiem, krievu zaudējumi: 300--450 nogalināto un ap 2~000 ievainoto.) Jau poļu-krievu spēku samērs vien liecina, ka, ja poļu karavīri kaut cik drosmīgi cīnījās, krievu karavīriem nebija laika nodarboties ar civiliedzīvotāju ``slaktēšanu''.

Memuāros stāstīts par briesmīgiem nežēlības gadījumiem, kādi sastopami gandrīz katra cietokšņa ieņemšanas gaitā, gan arī uzvarētāju augstsirdības gadījumiem. Ir atrodamas publicista F.~Bulgarina pierakstītas krievu ģenerāļa I.~fon Klugena atmiņas, kurš stāstīja: ``Uz mums šāva no ēku logiem un jumtiem, un mūsu karavīri, ielaužoties mājās, nogalināja visus, kas viņiem gadījās ceļā \citespace{} virsnieki vairs nespēja apturēt asins izliešanu \citespace{} Pie tilta atkal sākās slaktiņš. Mūsu karavīri šāva pūļos, neņemot vērā p~--- un spalgie sieviešu kliedzieni, bērnu brēcieni stindzināja dvēseli \citespace{} Saniknotie mūsu karavīri katrā dzīvā radībā redzēja mūsējo slepkavnieku Varšavas sacelšanās laikā. ``Nav nevienam piedošanas''~--- kliedza mūsu karavīri un nogalināja visus, neievērojot ne gadus, ne dzimumu. \citespace{} Četrās stundās notika briesmīga atriebība par mūsējo apslaktēšanu Varšavā.'' Šeit gan jāņem vērā, ka Prāgas aizstāvju vidū bija vairāki tūkstoši brīvprātīgo vietējo iedzīvotāju, kuri upuru noteikšanas gaitā varēja tikt pieskaitīti civiliedzīvotājiem, kaut karoja ar ieročiem rokās. Prāgas ieņemšanas laikā bija arī vēsturnieku minētā epizode, kad daļa tās aizstāvju izlauzās līdz Vislai un krievu karavīri tos, kuriem neizdevās upi pārpeldēt, iznīcināja kreisā krasta iedzīvotāju acu priekšā. Šie fakti varēja kalpot par pamatu leģendai par civiliedzīvotāju ``slaktiņu''. Jāatzīst, ka krievu karavīru vidū, kuriem bija izdevies sacelšanās sākumā atstāt Varšavu, dzīvas bija atmiņas par tur zvērīgi nogalinātājiem ieroču biedriem, valdīja atriebības jūtas, kuras acīmredzot atsevišķos gadījumos varēja izpausties arī pret civilistiem, taču tad nu jārunā par savstarpējiem ``slaktiņiem''.

Prāgas ieņemšana nobiedēja Varšavas iedzīvotājus, daļa metās bēgt projām no pilsētas, citi nosūtīja deputātus pie karaļa, pieprasot kapitulāciju. Viņš atbalstīja lūgumu un pie A.~Suvorova tika nosūtīta Varšavas maģistrāta delegācija. Kad daži poļu virsnieki pirms tam mēģināja ar spēku izvest no pilsētas karali un iepriekš saņemtos krievu gūstekņus, lai cīņu turpinātu, pilsētnieki to nepieļāva. Tā paši Varšavas iedzīvotāji nepieļāva sacelšanās dalībnieku mēģinājumus turpināt cīņu. Kapitulācija tika pieņemta.

Varšavas maģistrāts kopā ar sālsmaizi pasniedza pilsētas atslēgas A.~Suvorovam. 26.~oktobrī viņa armija iegāja pilsētā. Kad A.~Suvorovs novērsa iespējamo pilsētas izlaupīšanu, atbrīvoja no gūsta sākotnēji 6~000 poļu zemessargu, pēc tam vēl arī 500 sagūstīto poļu virsnieku, Varšavas maģistrāts iedzīvotāju vārdā viņam uzdāvināja zelta tabakas dozi ar briljantiem un uzrakstu ``Varšava~--- savam glābējam''. Imperatorei Katrīnai~II par Varšavas ieņemšanu A.~Suvorovs ziņoja vēstulē, sastāvošā no trim vārdiem: ``Urā! Varšava mūsu!'' Katrīnas~II atbilde bija tikpat īsa: ``Urā! Feldmaršals Suvorovs!'' Tātad par savu veikumu Polijā A.~Suvorovs saņēma feldmaršala pakāpi, kas apliecina ne tikai viņa militāro spēju novērtējumu, bet arī to nozīmi, kādu Katrīna~II piešķīra Polijas pakļaušanai.

Daļa poļu patriotu, kam izdevās izglābties no Krievijas karaspēka, gan mēģināja vēl turpināt cīņu, bet drīz tika galīgi sakauti. Var atzīmēt, ka poļu šļahtiči 1795.~gadā neveiksmīgi mēģināja izraisīt sacelšanos ukraiņu un baltkrievu zemnieku vidū ar lozungu ``Par mūsu un jūsu brīvību'' (\pltxti{Za naszą i waszą wolność).} Daļa poļu bēga uz ārzemēm un no šī laika tur pastāvīgi dzīvoja poļu emigranti. 1796.~gadā Krievijas cars Pāvils I apžēloja T.~Kostjuško un līdz ar viņu arī citus 20~000 poļu politiskos ieslodzītos, kuri bija nometināti Sibīrijā. T.~Kostjuško emigrēja uz ASV.

Kaut cietusi sakāvi, 1794.--1795.~gada sacelšanās ievadīja virkni poļu nacionālās atbrīvošanās kustības dalībnieku sacelšanās vairāku gadu desmitu garumā, nostādot Polijas jautājumu Eiropas valstu politikas dienas kārtībā. Jau minētie padomju vēsturnieki V.~Djakovs un I.~Millers uzskatīja, ka šī sacelšanās iezīmē poļu atbrīvošanās kustības sākumrobežu.

Kā vēsturisku mītu var pieminēt baumas, kas klīda par Katrīnas~II nāvi, (pēc oficiālās versijas viņa mira no asinsizplūduma smadzenēs), kas it kā bija saistīta ar pēc T.~Kostjuško sakāves un trešās Polijas dalīšanas uz Pēterburgu atvesto Pjastu (\pltxti{Piasty}~--- poļu karaļu dinastija, kura valdīja no X līdz XIV gadsimtam) dinastijas troni. Katrīna~II esot personīgi likusi savos apartamentos Ziemas pilī to apvienot ar tādu toreizējo jaunievedumu kā savu personīgo ūdens klozetu. (Poļu garīdznieks un vēsturnieks, 1830--1831~gada sacelšanās dalībnieks V.~Kaļinka raksta par S.~Poņatovska troni, uz kura sēdējuši arī S.~Batorijs, Sigizmungs Vaza un viņa dēls Vladislavs. 1796.~gada 6.(17.) novembra rītā ķeizariene pēc pamošanās devās to apmeklēt, bet pēc kāda brīža galminieki izdzirdēja krītoša ķermeņa troksni. Kādu brīdi sulaiņi šaubījās līdz tomēr iedrošinājās atvērt durvis. Ķeizariene gulēja uz grīdas bez samaņas un noasiņoja. Tūlīt izsauktie ārsti neko vairs nespēja līdzēt, pēc dažām stundām Katrīna~II nomira no vaginālās asiņošanas. Pēterburgas aristokrātiskajos salonos vēl ilgi mēļoja, ka ķeizarienes ūdens klozetā zem Pjastu troņa esot bijis noslēpies kāds poļu fanātiķis, iespējams, punduris, un no apakšas viņu ievainojis ar šķēpu vai zobenu, bet pēc tam, izmantojot apjukumu, aizbēdzis no Ziemas pils. Loti jāšaubās, vai šai mītā ir kaut grans patiesības. Iespējams, baumu izcelsmi veicināja poļu svētuma Pjastu troņa izvešana no Polijas un it kā notikusī zaimošana, kas poļu acīs brēca pēc soda.

T.~Kostjuško vadītās sacelšanās, kura ilga 238~dienas, sakāve kalpoja par pamatojumu \strong{Polijas-Lietuvas valstiskuma likvidācijai}. 1795.~gada 24.~oktobrī valstis, kuras piedalījās Polijas sadalē, noteica savas jaunās robežas. \strong{Tā bija Polijas trešā dalīšana}. Karalis Staņislavs~II Augusts Poņatovskis 1795.~gada 25.novembrī nolika savas pilnvaras. Bijušās Žečpospolitas zemju valstiski-tiesiskā situācija būtiski mainījās: nodibinājās absolūtismam raksturīga pārvalde, poļu šļahta zaudēja politisko varu. Ja līdz Žežpospolitas sadalei jau sāka veidoties kopējs Vispolijas tirgus, tās sadale šo procesu pārtrauca, sarāva daudzus tradicionāli izveidojušos ekonomiskos sakarus starp atsevišķiem Polijas apgabaliem. Nepieciešamība veidot jaunus sakarus un pielāgoties Poliju sadalījušo valstu ekonomiskajām struktūrām nostādīja poļu apgabalus nevienlīdzīgā situācijā ar šo valstu teritorijām. Poļu apgabalu ekonomika nonāca atkarībā no minēto valstu politikas, to valdību centieniem ievērot vai neievērot (mazievērot) jauno pakļauto apgabalu intereses. Taču magnātu pozīcijas tika maz skartas. Jaunajos apstākļos, lielvaru aizsardzībā viņiem vairs nebija vajadzīgs sīkās šļahtas atbalsts. Pēdējā zaudēja iespēju saņemt no saviem labvēļiem līdzekļus par tiem sniegtajiem pakalpojumiem.

Austrijas varā pārgāja Galīcijas ziemeļu daļa līdz Bugas upei, ko sāka saukt par Rietumgalīciju, (atšķirībā no tās jau agrāk iegūtās Austrumgalīcijas ar Krakovu), daļa Mazovijas (\pltxti{Mazowsze}) un daži citi novadi ar 147 tūkstošu km$^{2}$ kopējo teritoriju un 1,2~miljoniem iedzīvotāju.

Prūsija ieguva zemes uz rietumiem no Piļicas (\pltxti{Pilica}), Vislas (\pltxti{Wisła}), Bugas (\pltxti{Boh}) un Nemunas (poļu \pltxti{Niemen}, lietuviešu \lttxti{Nemunas}, vācu \detxti{Memel}) upēm ar Varšavu, no kuras tās dumpīguma dēļ atteicās citas Polijas dalītājvalstis, kā arī Rietumlietuvas (Žemaitijas, poļu \pltxti{Żmudź}) zemes, kas kopumā sastādīja 55 tūkstošus km$^{2}$ ar 1~miljonu iedzīvotāju. Krievu vēsturnieks A.~Pogodins uzsvēra, ka ``pašas vecākās, īsteni poļu zemes'' 3.~dalīšanā ieguva Prūsija.

Tātad, kad runā par XVIII gadsimtā notikušajām trijām Polijās dalīšanām, faktiski tiek runāts par Žečpospolitas~--- apvienotās Polijas un Lietuvas~--- dalīšanām.

Etnisko poļu apdzīvotās teritorijas faktiski savā starpā sadalīja Austrija ar Prūsiju.

Krievija saņēma lietuviešu (daļa lietuviešu apdzīvoto teritoriju ar Suvalkiem gan ieguva Prūsija), baltkrievu un ukraiņu zemes uz austrumiem no Bugas upes ar kopējo platību 120 tūkstošu km$^{2}$ un 1,2 miljoniem iedzīvotāju.

Tā Žečpospolitas triju dalīšanu rezultātā Krievijas rokās nonāca latviešu (Kurzeme), lietuviešu, baltkrievu (izņemot teritorijas daļu ar Belostoku (\pltxti{Białystok}), kas pārgāja Prūsijas īpašumā) un ukraiņu zemes (izņemot Austrumgalīcijas apgabalu ar Lembergu (Ļvovu), kurš palika Austrijas rokās).

Tiesa, Krievijai bija jāsamaksā smaga cena~--- jāļauj nostiprināties Prūsijai un Austrijai. Jautājums nebija par Polijas likteni. Kāpēc Katrīnai~II vajadzēja ņemt vērā Polijas intereses, ja pēdējā nevēlējās ņemt vērā Krievijas un Polijā dzīvojošo krievu, baltkrievu un ukraiņu intereses? Krievijai svarīgāks bija kas cits~--- pazuda buferis starp Krieviju un vācu valstīm. Austrija un Prūsija tagad atradās tieši pie Krievijas robežām. Bet alternatīva būtu tikai atteikšanās pievienot Krievijai radniecīgos baltkrievus un ukraiņus. Karot pret Prūsiju un Austriju, ko, iespējams, atbalstītu Anglija, lai saglabātu Poliju tās etnogrāfiskajās robežās, nu nekādi neatbilda Krievijas interesēm. Var teikt, ka XVIII gadsimtā Prūsijas un Austrijas agresīvā politika pavēra iespēju Krievijai atrisināt Rietumkrievijas jautājumu bez asiņaina kara ar Eiropas lielvalstīm.

Krievu sabiedriskā doma gan toreiz, gan vēlāk apsveica ieguvumus. Kā savās 1899.~gadā izdotajās lekcijās par Krievijas vēsturi rakstīja krievu vēsturnieks S.~Platonovs, ``attiecībā pret Poliju Krievijas uzdevums bija atbrīvot pareizticīgos krievu iedzīvotājus no katolisko poļu valdīšanas, t.i.~--- atņemt Polijai vecās krievu zemes un no šīs puses sasniegt krievu tautības etnogrāfiskās robežas.'' Kā redzam, ``krievu tautībai'' šeit tika pieskaitītas arī citas (baltkrievu un mazkrievu jeb ukraiņu) tautības.

Līdzīgs viedoklis bija arī ievērojamajam angļu vēsturniekam, kulturologam un sociologam A.~Toinbi. Viņš 1947.~gadā rakstīja: ``Rietumos valda uzskats, ka Krievija ir agresors, \citespace{} XVIII gadsimtā Polijas dalīšanas laikā Krievija sagrāba teritorijas lielāko daļu: XIX gadsimtā tā ir Polijas apspiedējs \citespace{} Novērotājs no malas, ja tāds eksistētu, teiktu, ka krievu uzvaras pār zviedriem un poļiem ir tikai pretuzbrukums, \citespace{} XIV gadsimtā labākā daļa īsteno Krievijas teritoriju~--- gandrīz visa Baltkrievija un Ukraina~--- tika atrauta un pievienota rietumu kristietībai \citespace{} Poļu iekarotās īstenās krievu teritorijas \citespace{} tika atgrieztas Krievijai tikai 1930.--1945.~gada pasaules kara pēdējā fāzē.'' Citējot šo vērtējumu kā centienu panākt objektivitāti apliecinājumu, krievu literatūrzinātnieks un publicists V.~Kožinovs norādījis, ka Rietumu uzbrukums Krievijai sākās vēl agrāk, jau ar Polijas karaļa Boļeslava Drosmīgā iebrukumu Kijevā 1018.~gadā (par to runāts šī darba ievadā). V.~Kožinovs uzsvēris, ka nevar runāt par Krievijas dalību Polijas [kā poļu apdzīvotas zemes~--- V.Š.] sadalē. Patiesībā poļu zemes dalīja Austrija un Prūsija, Krievija sev pievienoja tikai senās austrumslāvu [neredzot lietuviešus un latviešus~--- V.Š.] apdzīvotās zemes, kuras arī mūsdienās ietilpst Baltkrievijas un Ukrainas sastāvā. Tāpēc V.~Kožinovs Krievijas dalību Polijas sadalīšanā nosaucis par ``liberālu mītu.''

Arī mūsdienu krievu vēsturnieks O.~Ņemenskis 2012.~gada 21.~decembrī konferencē ''Baltkrievijas un Krievijas apvienošanās'', veltītā Žečpospolitas pirmajai sadalīšanai, uzsvēra: ''Krievija visos trijos Žečpospolitas dalīšanas posmos nesaņēma ne pēdas poļu zemes, nepārkāpa Polijas etnogrāfisko robežu. Krievijas līdzdalības ideoloģija [Žečpospolitas] dalīšanās sastāvēja tieši no iepriekš vienotās Krievzemes atkalapvienošanās''. Tādejādi pēc vēsturnieka domām apvainojumi Krievijai par dalību poļu zemju dalīšanā XVIII gadsimtā pēc būtības neatbilst patiesībai.

Taču ievērojamais krievu vēsturnieks S.~Kļučevskis savukārt atzīmējis, ka Krievija, tā vietā lai pievienotu sev Rietumkrieviju, piedalījās Polijas sadalīšanā, tādejādi ``atrisinājums neatbilda uzdevumam''. Krievija sev pievienoja ne tikai Rietumkrieviju, bet arī Lietuvu un Kurzemi, taču daļu Rietumkrievijas~--- Galīciju atdeva Austrijai. Pēc S.~Kļučevska domām Polija nebija lieks loceklis Ziemeļaustrumu Eiropas valstu ģimenē, kalpojot par vāju starpnieku starp trijiem stipriem kaimiņiem. ``Atbrīvojusies no to vājinošās Rietumkrievijas un pārveidojusi savu valsts iekārtu, kā centās tās labākie sadales laikmeta ļaudis, tā varētu sniegt lielisku pakalpojumu slāvu lietai un starptautiskajam līdzsvaram, kļūstot par balstu pret ar visiem spēkiem uz austrumiem tiecošos Prūsiju. Pēc Polijas krišanas sadursmes starp nosauktajām trijām valstīm nevājināja vairs nekāds starptautisks buferis un tām bija vissāpīgāk jāatsaucas uz Krieviju, kuras robeža pa Nemunu nekļuva drošāka tāpēc, ka tai kaimiņos atradās prūšu priekšposteņi \citespace{} Bez krievu apgabaliem, savās nacionālajās robežās, pat ar izlabotu valsts iekārtu patstāvīga Polija būtu mums nesalīdzināmi mazāk bīstama, nekā tā pati Polija austriešu un vācu provinču veidā. Visbeidzot, poļu valsts iznīcināšana neatsvabināja mūs no cīņas pret poļu tautu: nepagāja ne 70~gadu kopš trešās Polijas dalīšanas, bet Krievija jau trīs reizes karoja ar poļiem (1812., 1831. un 1863.~gg.). Žečpospolitas rēgs, ceļoties no vēsturiskā kapa, radīja dzīva tautas spēka iespaidu. Varbūt, lai izvairītos no kara pret tautu, vajadzēja saglabāt tās valsti.'' S.~Kļučevska secinājumos faktiski visdziļāk izvērtēti Krievijas politikas rezultāti attiecībā pret Poliju.

Situācija, kādā nonāca poļi, tai laikā nebūt nebija ārkārtēja. Daudzas pasaules tautas gan pirms, gan arī pēc tam nonāca daudznacionālu impēriju sastāvā. Šoreiz šis liktenis bija piemeklējis poļus. Jāpiemetina, ka ar Žečpospolitas sadalīšanu sašķelta tika ne tikai poļu tauta. Arī citu tajā dzīvojošo tautu pārstāvji (tai skaitā ap 800~000 ebreju, kam nebija savas kompaktas teritorijas), nonāca dalītājvalstu varā, kur viņu statuss un dzīves apstākļi arī bija dažādi.

Pretsparu veselas Eiropas tautas nomākšanai, tās teritorijas sadalīšanai trijās valstīs pārējā Eiropa nevēlējās dot. Laikabiedru vairākums attaisnoja notikušo ``Eiropas kartes racionalizēšanu''.

Daudzi citu tautu vēsturnieki ir pauduši viedokli, ka poļi paši bija vainīgi savā nelaimē.

Viens no marksisma pamatlicējiem F.~Engelss tieši vainoja poļu feodālo aristokrātiju~--- magnātus savienībā ar trim lielvalstīm, kuras dalīja Žečpospolitu, lai tikai izbēgtu no revolūcijas, kā rezultātā tika sadalīta valsts, kas bija tikpat liela cik Francija. Domājams, ka bailes no revolūcijas, ko it kā izjuta magnāti, F.~Engelss pārspīlēja, jo viņi taču joprojām turpināja savstarpējās ķildas. Visi poļu magnāti vēlējās panākt stipras unitāras valsts eksistenci, taču tikai tai gadījumā, ja viņi paši tajā būtu noteicēji. Tādā situācijā visas reformas bija nolemtas neveiksmei.

Arī krievu vēsturnieki parasti uzsvēra Polijas sabrukuma iekšējos cēloņus, bet apskatot ārējo faktoru, galveno atbildību par Žečpospolitas sadali piedēvēja Prūsijai, turpretī Krievijas dalību tajās dažkārt pat raksturoja kā neatbilstošu tās nacionālajām interesēm. Tā, jau minētais krievu vēsturnieks A.~Pogodins rakstīja, ka Polijas valsts gāja bojā, jo tā nepaguva laikus nostāties uz tai laikā vienīgi pareizā monarhiskās varas un reizē ar to militārisma, rūpniecības un tirdzniecības attīstības ceļa.

Kopumā sākotnēji arī poļu vēsturnieku vidū lielākā uzmanība tika pievērsta valsts neatkarības zaudēšanas iekšējiem cēloņiem. Piemēram, autoritatīvais poļu vēsturnieks un valstsvīrs (viņš 1908.--1913.~gadā bija Galīcijas un Lodomērijas vietvaldis (\detxti{Statthalter})) M.~Bobržinskis, atzīmējot, ka nav bijis valsts Eiropā, pret kuru jaunajos laikos tās kaimiņi nebūtu turējuši ļaunus nodomus, uzsvēra, ka pat salīdzinoši mazas nācijas spēja no šīs cīņas iziet kā uzvarētājas, turpretī ``poļi, kad viņus dalīja kaimiņi, nebija vājāki par katru no tiem ne pēc zemes platības, ne iedzīvotāju skaita, ne materiālās labklājības, ne garīgās attīstības, pat bija pārāki pār tiem tai vai citā jomā, taču tikai vieni poļi krita, pie tam bez cīņas, bez īstas cīņas, uz kādu tie bija spējīgi''. Atsevišķu personību gatavība nest upurus valsts labā ``nevarēja aizstāt gatavības trūkumu sabiedrībā nest upurus''. Vēsturnieks secināja ``Ne [neizdevīgās] robežas un ne kaimiņi, bet vienīgi iekšējās nesaskaņas noveda poļus pie politiskās pastāvēšanas zaudēšanas''.

Krievu vēsturnieks un filozofs N.~Karejevs 19.~gadsimta 80.~gadu beigās secināja, ka visi pētnieki ir vienoti~--- Polija gāja bojā iekšējo cēloņu rezultātā. ``Diagnoze~--- likumdevējas varas bezspēcība un pilnīgs izpildvaras sajukums.''

Tomēr poļu vēsturnieku viedokļi XIX gadsimta beigās~--- XX gadsimta sākumā sadalījās divās skolās. T.s. Krakovas skolas pārstāvji uzskatīja, ka XVIII gadsimta beigās notikušās Žečpospolitas sagrūšanas cēloņi meklējami valsts iekšējā vājumā, otras~--- Varšavas skolas uzskatu aizstāvji tos saskatīja tai apstāklī, ka pēc 1772.~gadā notikušās dalīšanas valsts nonāca intensīvas modernizācijas fāzē, bet krita par upuri savu kaimiņvalstu~--- Krievijas, Prūsijas un Austrijas negausībai. Arī ievērojamais vēsturnieks no Lembergas O.~Balcers, kurš kā pirmais no poļu zinātniekiem 1921.~gadā tika apbalvots ar Baltā ērgļa (\pltxti{Orła Białego}) ordeni, apgalvoja, ka`` \dots{} īstais, izšķirošais mūsu valstiskuma pagrimuma cēlonis, īsts šī notikuma \latxti{causa efficiens} (izraisītājs) bija apvienoto, tātad vareno, kaimiņu alkatība, kuri noslēdza savienību, lai pazudinātu Poliju.''

Neviens no virzieniem tā arī neieguva izšķirošu pārsvaru. Vairākums nonāca pie secinājuma, ka Polijas dalīšanas izsauca kā iekšējo, tā ārējo faktoru mijiedarbība. Ir izteikts diezgan pārliecinošs viedoklis, ka atsevišķi par sevi ne vieni, ne otri nenovestu pie valstiskuma zaudēšanas.

Jau pieminētais vēsturnieks no Poznaņas J.~Rutkovskis rakstīja: ``Polijas iekšējā iekārta bija pietiekami stipra, lai tā varētu realizēt savu pašas iekšējo politiku, taču bija par vāju, lai varētu aizstāvēties pret ārējo spiedienu.'' Viņš arī uzsvēra Žečpospolitas militāro vājumu, kuru lielā mērā noteica tās politiskais vājums, kad Seims nevēlējās stiprināt armiju, baidoties, ka karalis to izmantos absolūtās monarhijas radīšanai. Poļu vēsturnieks J.~Tazbirs, kurš uzsver, ka Poliju XVIII gadsimtā līdz bojā ejai noveda trūkumi valsts uzbūvē, nevis nacionāli netikumi, ir norādījis, ka poļu tauta pēc pirmās Žečpospolitas sadales ar reformām, 1791.~gada 3.~maija Konstitūcijas pieņemšanu, T.~Kostjuško vadīto sacelšanos pierādīja, ka tā ir pati spējīga labot savu valsti, cīnīties par to, līdz ar to apstrīdot tēzi, ka ``mēs kritām mūsu trūkumu rezultātā.'' Jāpiekrīt, ka norisa cīņa par pretrunu, valsts trūkumu pārvarēšanu, bet reizē jāatzīmē, ka notikušās Žečpospolitas dalīšanas taču tieši rādīja, ka panākumi šai cīņā bija nepietiekami lai saglabātu valsti. Domājams, ka pārmetumus nav pelnījusi tauta, bet gan galvenokārt tās augšslāņi. Nosacījumus Polijas sadalīšanai radīja tās bezspēcība, tās valdošo slāņu nespēja kopot tautas spēkus vispārnacionālam uzdevumam~--- suverenitātes un teritoriālās integritātes aizstāvēšanai, tā vietā nodarbojoties ar cīņu par savu privilēģiju sargāšanu un paplašināšanu. Vienkāršie zemnieki neko neieguva nedz no karaļu nomaiņām, nedz varas pārdalēm, nedz arī Žečpospolitas sadalīšanas.

Mūsdienu vēsturnieku darbos autoram nav gadījies atrast Polijas valstiskuma iznīcināšanu attaisnojošus uzskatus, bet domstarpības pastāv par tās novērtējumu XVIII gadsimta notikumu kontekstā, par citu valstu tiesībām uz tām vai citām teritorijām.

Prūsija no bijušajām poļu zemēm izveidoja trīs provinces: Rietumprūsiju, Dienvidprūsiju un Jauno Austrumprūsiju. Prūsijas poļu valdījumos sākās poļu asimilācija. Daudzas valsts un baznīcas muižas nonāca prūšu muižnieku rokās. Prūsijai piederošajā Polijas daļā jau no 1776.~gada darbojās likums, kurš šļahtiču zemes atļāva iepirkt ne tikai vācu muižniekiem, bet arī pilsētniekiem. Ar dažādām privilēģijām poļu zemēm tika piesaistīti vācu zemnieki, amatnieki un tirgotāji. Poļu ciemu vidū veidojās vācu zemnieku ciemi. No 1797.~gada visa tiesvedība un administratīvā lietvedība notika vācu valodā, poļi tika izraidīti no valsts dienesta, nodibināja vācu skolas. Poļiem tika uzspiesta visas sabiedriskās dzīves reglamentācija. Poļu vēsturnieks J.~Feldmans uzsvēra, ka Prūsija uz Polijas rēķina palielinājās vairāk nekā divas reizes un kļuva par ``modru Polijas jautājuma kapraci'' Eiropas politikā, par sīkstāko poliskuma apkarotāju agrākās Žečpospolitas teritorijā.

Zemes, kuras nonāca Austrijā, ieguva nosaukumu Galīcija un Lodomērija (vācu \detxti{Königreich Galizien und Lodomerien}, poļu \pltxti{Królestwo Galicji i Lodomerii}, kur \latxti{Lodomeria} bija latīņu valodā lietots Galīcijas un Vladimiras kņazistes nosaukums XIII--XIV gs.) un tika sadalītas 12 apvidos. 1806.~gadā Austrija no Galīcijas guva 19\% visu valsts ienākumu. Arī Galīcijā sākās poļu asimilācijas mēģinājumi, gan ne tik enerģiski kā Prūsijas zemēs. 1790.~gadā Vīnes augstmaņu slepenā memorandā sakarā ar visai biklu Konstitūcijas projektu, kuru ierosināja provinces seims un kuru noraidīja imperators Leopolds~II, bez aplinkiem bija norādīts, ka Austrijas mērķim jābūt ``galīciešu pakāpeniskai pārvēršanai par vāciešiem''.

Krievijas rokās nonākušās teritorijas tika sadalītas Kurzemes, Viļņas un Grodņas guberņās. Saglabājās kārtu pašpārvalde, poļu muižnieku kundzība pār zemniekiem nebija satricināta. Uz laiku saglabājās vietējie likumi, arī Lietuvas statuta (\lttxti{Lietuvos Statutas}~--- likumu kodekss bija radīts XVI~gs. un civillietās palika spēkā daļā no bijušās Lietuvas lielkņazistes teritorijas līdz 1840.~gadam) nozīme. Krievijas dalība Polijas sadalē uz diviem turpmākajiem gadsimtiem lielā mērā noteica Krievijas ārējo politiku, jo starptautiskā stabilitāte bija lielā mērā atkarīga no valstu~--- triju dalīšanas dalībnieču attiecībām.

Dažādi Polijas apgabali vēl līdz tās dalīšanai attīstījās nevienmērīgi, tas bija redzams arī tikai etnisko poļu apdzīvotajās zemēs. Taču šis nevienmērīgums, kā uzskatīja poļu ekonomikas vēsturnieks V.~Kula, arī stiprināja valsts vienotību, radot priekšnoteikumus darba dalīšanai starp atsevišķām zemēm. Pēc Polijas sadales ekonomiskās attīstības nevienmērīgums pieauga. Tā, t.s. Lielpolija~--- ekonomiski attīstītākais Žečpospolitas apgabals XVIII gadsimta beigās, kurš acīmredzot saglabātu šo lomu neatkarīgā valstī, pēc tās sadales kļuva par Vācijas impērijas ``aizmuguri'' (vācu \detxti{Hinterland}~--- zemi, kuru raksturoja mazāks iedzīvotāju blīvums un sliktāka ekonomiskā un infrastruktūras attīstība kā blakus esošajos attīstītākajos apgabalos), bet Austrumgalīcija kļuva par ekonomiski atpalikušāko no visām poļu provincēm. Ar sadali tika sarauti vai vismaz apgrūtināti veidojamie ekonomiskie sakari starp dažādām poļu teritorijām. Tiesa, vēlreiz jāatgādina, ka ne tikai pirms 1795., bet arī pirms 1772.~gada ne visas poļu apdzīvotās teritorijas atradās vienas valsts sastāvā. Ārpus Žečpospolitas palika tādas zemes, uz kurām tā pretendēja, kā jau pieminētās Silēzijas (poļu \pltxti{Śląsk}, vācu \detxti{Schlesien}) un Rietumu Pomorjes jeb Pomerānijas (poļu \pltxti{Pòmòrzé}, vācu \detxti{Pommern}) teritorijas. Reizē jāuzsver, ka dalīšanas, sagraujot vienus sakarus, veicināja citus. Piemēram, sakarus starp Pozenes un Silēzijas apgabaliem. Tie pat stiprināja poļu elementus Silēzijā, kas guva izpausmi jau XIX gadsimta pirmajā pusē.

Par vienu no poļu zemju attīstības nevienmērības rādītājiem var kalpot iedzīvotāju pieauguma dinamika. Tā, 1816.--1856.~gadā iedzīvotāju skaits gandrīz dubultojās Prūsijas rokās esošajā Pomerānijā un Augšsilēzijā (vācu \detxti{Oberschlesien}, poļu \pltxti{Górny Śląsk}), par 73\% pieauga Pozenes hercigistē, par 36\% Polijas karalistē un tikai nedaudz vairāk par 20\% Austrijai piederošajā Galīcijā. V.~Kula uzsvēris, ka Polijas sadale paildzināja feodālisma pastāvēšanu tās zemēs. Vājā poļu feodālā valsts, kāda bija Žečpospolita savas pastāvēšanas pēdējos gados, nespējusi feodāļiem garantēt viņu privilēģijas revolucionāras situācijas priekšā, tāpēc tie bija ieinteresēti iegūt absolūtisko kaimiņvalstu aizbildniecību, kuras tad arī nodrošināja to privilēģiju pastāvēšanu vēl vairāk nekā pusgadsimta garumā. Taču šis spriedums nevar būt viennozīmīgs. Pats V.~Kula bija spiests atzīt, ka Žečpospolitas sadale neapturēja kapitālisma elementu nobriešanu feodālisma iekšienē un ja arī to aizturēja, tad tikai uz īsu laiku, ka kapitālisma attīstība daudzās nozarēs un teritorijās pēc Žečpospolitas sadales pat paātrinājās. Viņš arī norādījis, ka reģionālās ekonomikas iezīmes lielā mērā veidojās daudzu sociāli-ekonomisku faktoru ietekmē, starp kuriem politiskajām robežām nebija galvenā loma. Tā, vēlāk Krievijas atkarībā pastāvošajā Polijas karalistē (1815--1915) esošais Kališas (poļu \pltxti{Kalisz}, vācu \detxti{Kalisch}) rajons ekonomiskā ziņā maz atšķīrās no Pozenes provinces. Dombrovas baseins karalistē faktiski veidoja vienu ekonomisku rajonu ar Augšsilēziju Prūsijā, bet Austrumgalīcijai ar Lembergu bijis mazāk kopīgā ar Rietumgalīciju ap Krakovu, kas abas atradās Austrijas sastāvā, nekā starp Rietumgalīciju un tai Polijas karalistes pusē pieguļošo Mehovas (poļu \pltxti{Miechów}) rajonu. Ar kapitālisma attīstību pieauga reģionālās atšķirības, kas nesakrita ar politiskajām robežām. V.~Kula arī norādījis, ka nevis Žečpospolitas sadale, bet kapitālisma attīstība tās sadalītajās daļās noveda pie Polijas ``A'' (ekonomiski attīstītāko rajonu) un Polijas ``B'' (ekonomiski vājāko, atpaliekošo rajonu'') izveides. Pat vienā pašā Polijas karalistē bija novērojams šāds dalījums. Karalistē abos Vislas krastos pastāvēja dažādas Polijas, kur atšķirības bija lielākas nekā starp Pozenes provinci Prūsijā un Kališas rajonu Polijas karalistē.

\asterism


Taču \strong{Žečpospolitas valstiskuma likvidācija} notika apstākļos, kad poļu nācijas konsolidācijas process jau bija izvērsies, tāpēc tā sastapa stipru tautas pretestību, \strong{izsauca poļu nacionālo kustību}. Poļu emigrācija atgādināja pasaulei par Polijas pastāvēšanu. No šī laika starptautiskajās attiecībās sāka pastāvēt ``poļu jautājums'', kuru ar Polijas dalītājām konkurējošās valstis centās izmantot savās interesēs.

Kaut Polijas valsts pārstāja pastāvēt, daudzi poļi, īpaši šļahtiči, neatmeta cerības atjaunot neatkarību, kaut dažādie valsts atjaunošanas plāni visbiežāk izrādījās nereāli. Tieši \strong{šļahtiči nostājās nacionālās atbrīvošanās kustības priekšgalā}. Polijas īpatnība bija šļahtiču lielais īpatsvars~--- ap 6 līdz pat 10\% no visiem iedzīvotājiem. Tiesa, par šļahtičiem Žežpospolitā par īpašiem nopelniem varēja kļūt arī zemāko kārtu pārstāvji, taču tas notika reti, jo jautājumu izlēma Seims. Tā, ir dati, ka 1788.--1792.~gadā, kad Polijā darbojās jau minētais Četrgadu Seims (\pltxti{Sejm Czteroletni}), par šļahtičiem kļuva ap 400 cilvēku, tikpat cik visos iepriekšējos XVIII gadsimta gados.

Tomēr bija arī ceļi, kā apiet likumu. Viens no līdzekļiem, kā pierādīt savu piederību šļahtiči kārtai, bija 12 liecinieku uzrādīšana. Bagātam plebejam vispirms bija jāaiziet pie bārddziņa un jālūdz iegriezt pāris brūču, kuras nebūtu bīstamas, bet radītu skaidri saskatāmas rētas~--- jo, kas gan par šļahtiču bez kaujās gūtām rētām! Pēc tam bija jānoalgo kāds, kas apvainotu šļahtiča statusa pretendentu, ka viņš nav šļahtičs, un tam pretī jānostāda 12 liecinieki, kuri šo apgalvojumu noliegtu. Tā ar tiesas spriedumu varēja iekļūt priviliģēto kārtā. Bez tam Lietuvas lielkņazistē līdz 1764.~gadam darbojās likums, pēc kura katrs ebrejs, kurš pārgāja kristīgajā ticībā, uzreiz kļuva par šļahtiču. Tas bija apbalvojums par ticības maiņu. Bija arī gadījumi, kad plebeji pārgāja judaismā, lai pēc dažiem gadiem atgrieztos katolicismā un tā iegūtu šļahtiča ģerboni. Tāpēc arī minētais likums galu galā tika atcelts.

Kopējā šļahtiču masā zemes īpašnieki sastādīja mazākumu. Pēc Žečpospolitas pirmās sadales tādu muižnieku bija vairāk nekā 300~000, bet t.s. sīko šļahtiču, kuru īpašumā bija tikai daļa ciema vai vispār nebija sava īpašuma~--- vairāk nekā 400~000. Pēc poļu vēsturnieka T.~Ļepkovska aprēķiniem XVIII~gadsimta beigās bezzemes šļahta sastādīja 55\%, bet 40\% šļahtiču īpašumā nebija dzimtcilvēku. Lielie zemes īpašnieki~--- magnāti, kuru īpašumā bija vairāki simti ciemu, dzīvoja galvenokārt valsts austrumos, tai skaitā zemēs, kur poļi bija etnisks mazākums. Viduspolijā par magnātu jau skaitījās šļahtičs, kura īpašumā bija vairāki desmiti ciemu. Sīko šļahtiču īpašumā parasti bija muiža ar dažiem klaušu zemniekiem. Viņi parasti algoja strādniekus, jo savu klaušinieku nepietika. Šādu šļahtiču dēli bieži nodarbojās ar tirdzniecību (1775.~gadā tika atcelts aizliegums šļahtičiem nodarboties ar amatniecību un tirdzniecību), kļuva par garīdzniekiem, dienēja pie magnātiem, pārejot bezīpašuma šļahtiču statusā. Daļa šļahtiču magnātu dienestā līdz vecumam nopelnīja pensiju, bet vairākums cieta trūkumu, atšķiroties no zemniekiem tikai ar savām iluzorajām pilsoniskajām un politiskajām tiesībām. Nabadzīgākie bezīpašuma šļahtiči rentēja zemes gabalus, tai skaitā arī no zemniekiem, kurus arī paši apstrādāja. Īpaši Mazovijas (poļu \pltxti{Mazowsze}, vēsturisks apgabals Polijas centrā) un Podlases (poļu \pltxti{Роdlаsiе}~--- no poļu ``\pltxti{Pod lasem}'' (zem meža)~--- vēsturisks apgabals tagadējās Polijas austrumos) novados varēja redzēt cilvēku ar zobenu pie sāniem, kurš ara zemi. Taču šos arājus-šļahtičus nekad neatstāja pārākuma apziņa pār saviem kaimiņiem dzimtcilvēkiem.

Jau XVIII gadsimta otrajā pusē ar magnātu karadraudžu likvidāciju magnātu aizbildniecībā esošā t.s. šļahtiču klientūra strauji mazinājās, tā vairs magnātiem nebija vajadzīga. Ar neatkarības zaudēšanu un ar kapitālistisko attiecību attīstību sīkā šļahta izrādījās vairs nevajadzīga savā agrākajā~--- karotāju veidolā. Pāreja uz činšu (renti) padarīja nevajadzīgus šļahtiču amatus magnātu saimniecībās. Arvien vairāk šļahtiču pārcēlās uz pilsētām. Ja turīgākie pilsētu šļahtiči papildināja augstāko ierēdņu, advokātu un brīvo profesiju pārstāvju rindas, tad nabadzīgākie līga darbā par apsargiem, mājkalpotājiem, reizēm kļūstot arī par klaidoņiem.

Angļu vēsturnieks N.~Deiviss pat uzskata, ka juridiskā ziņā Polijas aristokrātijai pienāca gals, kad pēc Žečpospolitas sadalīšanas tika anulēti likumi, kas noteica tās statusu. Taču N.~Deiviss nav precīzs. Protams, magnātu un šļahtas ietekme citām valstīm pievienotajās poļu teritorijās mazinājās, taču pilnībā neizzuda.

Daudzas magnātu muižas tika pārdotas, lai nomaksātu parādus. To valstu valdības, kuras sadalīja Poliju, pārņēma savā īpašumā vairākumu valsts un baznīcas zemju (tā atņemot iztiku daudziem šļahtičiem~--- karaļa zemju nomniekiem), bet par dalību 1794.~gada sacelšanās konfiscētās zemes sadalīja savas valsts augstmaņiem. Administratīvā pārvalde un tiesa nonāca nepoļu ierēdņu rokās. Prūsijā šļahta zaudēja personas neaizskaramības tiesības. Austrija arī atcēla šļahtiču personīgās brīvības un to īpašuma neaizskaramības tiesības, privilēģiju būt atbrīvotiem no nodokļiem. Kā raksta jau minētais poļu vēsturnieks J.~Tazbirs, XVIII gadsimta pirmajā pusē valdošais valsts aparāta vājums, nespēja iekasēt nodokļus nāca par labu šļahtai, kura iedzīvojās uz valsts rēķina. Kad Polija tika sadalīta, šļahtiči uzreiz izjuta, ka okupanti nodokļus prot ievākt daudz labāk nekā Žečpospolita. Lielā mērā šis apstāklis bijis pamatā vienam no šļahtas kultivētajiem mītiem~--- ka Žečpospolitas valsts iekārta bijusi gandrīz vai ideāla. Īpašu sašutumu šļahtā viesa tas, ka poļu zemes sadalījušo valstu aparāts iejaucās muižnieku un zemnieku attiecībās.

Taču šo triju lielvalstu, kuras pašas bija feodālas kārtu monarhijas, politika nevarēja būt radikāla. Tās poļu zemēs saglabāja feodālās attiecības un kārtu struktūru. Daļai poļu magnātu savu dižciltību izdevās apstiprināt Austrijā un Prūsijā. Īpaši Austrijas imperators labprāt apmaiņā pret likvidējamajiem senatoru, vojevodu, kastelānu (\pltxti{Kasztelan}, no latīņu \latxti{castellum}~--- pils, XVIII gadsimtā tie dienesta hierarhijā ieņēma otro vietu pēc vojevodām) u.c. amatiem piešķīra poļu lielmuižniekiem savulaik Žečpospolitā mazizplatītos grāfu un baronu titulus. Vairākums poļu grāfu un baronu ģimeņu savus titulus ieguva Vīnē un Berlīnē pirmajā desmitgadē pēc Žečpospolitas krišanas. Poļu šļahta zaudēja daudzas savas tiesības, taču saglabājās kā kārta.

Pēc spāņu izcelsmes angļu vēsturnieces I.~De~Madariagas datiem Krievijas muižnieku (krievu \rutxti{дворяне}) kopskaits 1795.~gadā sastādīja ap 111~600. Pēc Polijas sadales tiem pievienojās 250~974 šļahtiču, kas Krievijas impērijā sastādīja 66,22\% no visiem muižniekiem. Tātad Krievijai piederošajā Polijas daļā šļahtiču bija vairāk nekā muižnieku visā pārējā Krievijas impērijā. Līdz 1830.~gadam Polijas karalistē saglabājās bagāto zemes īpašnieku pārsvars sabiedriskajā un ierēdņu hierarhijā. Tika apstiprināti prūšu un austriešu piešķirtie tituli, tika dāvāti arī jauni. Tā kā šļahtiču īpatsvars bija salīdzinoši liels, nav brīnums, ka viņu uzskati, tikumi lielā mērā ietekmēja veidojošos poļu nāciju. Šļahtas psiholoģiskais veidols un tās tradīcijas arī turpmāk atstāja lielu iespaidu poļu kultūras attīstībā, kad no to rindām nākusī inteliģences daļa saistīja sevi ar buržuāzijas ekonomiskajām un sabiedriskajām interesēm.

Nedaudz apsteidzot notikumus, jāsaka, ka 1817.~gadā Krievijai piederošajā Polijas daļā tika izdots likums, kurš, tāpat kā to paredzēja Krievijas impērijā spēkā esošā vēl 1722.~gadā Pētera I ieviestā Rangu tabula, deva iespējas par nopelniem valsts labā iegūt muižnieka (šļahtiča) tiesības. Katrs 10 gadus nokalpojis skolotājs, katrs karavīrs, kurš uzdienēja līdz kapteiņa pakāpei, katrs, kam bija ievērojami nopelni valsts priekšā, ieguva šļahtiča tiesības. Avīze ``\pltxti{Dziennik Praw}'' (``Likumu Dienasgrāmata'') publicēja šo cilvēku sarakstus. Tas veda pie šļahtas stāvokļa pakāpeniskas nonivelēšanas, ar ko ``īstie'' šļahtiči bija neapmierināti. Vēl krasāk stāvoklis mainījās pēc 1830.~gada sacelšanās. 1836.~gadā Polijas karalistē tika izdots dekrēts par muižniecību, kura mērķis bija radīt jaunu, saistītu ar carismu muižnieku kārtu. Mantojamās muižnieku (šļahtas) tiesības bija likumīgi jāapstiprina: dokumentāli jāpierāda, ka kāda no vīriešu kārtas senčiem īpašumā līdz 1775.~gadam bija vismaz viens ciems, vai līdz 1795.~gadam viņš bija Seima deputāts, senators, valsts darbinieks, poļu ordeņa kavalieris. Poļu šļahta saglabāja zināmas privilēģijas (kara dienestā, vidējās un augstākās izglītības ieguvē), taču zaudēja monopolu uz zemes īpašumu, nodokļu atlaidēm, atbrīvošanu no kara klausības. Līdz 1861.~gadam, kad tika beigts izskatīt jautājumu par šļahtiču tiesību apstiprināšanu, kārtas privilēģijas bija zaudējuši apmēram ¼ dzimušo šļahtiču. Polijas karalistē XIX gadsimta vidū tikai ap 5~000 šļahtiču ģimeņu piederēja muižas. Saprotams, ka jau tas vien radīja pamatu cietušo neapmierinātībai.

Tiesa, arī turpmāk Krievijas muižniecība bija daudznacionāla. Pēc 1897.~gada skaitīšanas datiem tikai 53\% dzimtmuižnieku par savu dzimto nosauca krievu valodu. 28,3\% no viņiem sevi uzskatīja par poļiem. Tātad vēl XIX gadsimta beigās vairāk nekā ceturtdaļa Krievijas impērijas muižnieku bija poļi. Tomēr poļu šļahtiču īpatsvars Krievijas muižnieku vidū bija ievērojami mazinājies. Daļa bagātās šļahtas~--- zemes īpašnieki, uzkrājot kapitālu, paplašinot preču ražošanu, izmantojot algotu darbaspēku, kļuva par kapitālistiskiem uzņēmējiem, saglabājot arī sociālo prestižu un sabiedrisko ietekmi. Bezīpašuma šļahta piegādāja zemākā un vidējā līmeņa ierēdņus, tās pārstāvji ieņēma daudz virsnieku posteņu armijā, taču tikai nelielai tās daļai bija vēlēšanu tiesības vietējās pašvaldības iestādēs. Sīko un vidējo muižu īpašnieki, bet vēl jo vairāk zemes īpašumus zaudējušie šļahtiči asi izjuta briesmas, kuras viņu privilēģijām nesa kapitālisma attīstība. Liela daļa no viņiem zaudēja savu mantu, to stāvoklis tuvinājās valsts zemnieku statusam. Daļa bijušo šļahtiču veidoja deklasētu ļaužu grupu. (Piemēram, bija gadījumi, ka nabadzībā nonācis poļu šļahtičs~--- dzimtmuižnieks 19.~gadsimta beigās Rīgā strādāja par amatnieku vai strādnieku.)

Šļahta ``bija galvenā pretkrieviskā noskaņojuma barotne visa XIX gadsimta laikā'', kā to raksturojis N.~Deiviss. Šļahtas centieni panākt tādu valsts iekārtu, kuras attīstību viņi varētu ietekmēt sev labvēlīgā virzienā, galvenokārt saistījās ar Polijas valsts neatkarības atjaunošanu, pie tam atjaunošanu agrākajās robežas, kad poļu feodāļi ekspluatēja arī citu tautu zemniekus. Šļahtiči un viņu sekotāji vēl XX gadsimtā par katru cenu centās atjaunot Poliju 1772.~gada robežās ``no jūras līdz jūrai'' (\pltxti{Od morza do morza}), tātad~--- vienā valstī apvienojot gan poļu, gan lietuviešu, gan ukraiņu un baltkrievu zemes. Sociālās priekšrocības, kuras šī kārta izmantoja gadsimtu gaitā (izglītība, prestižs sabiedrībā), ļāva tai ieņemt vadošo vietu poļu nacionālajā kustībā.

Tomēr jāuzsver, ka šļahtiču politiskie uzskati bija daudzveidīgi: dažas grupas ieņēma klerikālas, feodālas pozīcijas, citas, ``maksājot nodevas'' romantismam un lielvalsts atjaunošanas ideālam, pēc uzskatiem tuvinājās buržuāziskajiem revolucionāriem, izvirzīja arī progresīvas prasības par zemnieku stāvokļa uzlabošanu, to feodālās atkarības likvidēšanu, kārtu privilēģiju likvidēšanu, vēlēšanu tiesību paplašināšanu un citu buržuāzisko politisko brīvību ieviešanu.

Par poļu šļahtičiem dažādos laikmetos ir izteikti visdažādākie viedokļi, no cildinošiem līdz paļājošiem. Piemēram, kritiski pret tiem bija noskaņots krievu publicists un beletrists F.~Bulgarins, no mātes puses polis, viņa tēvs bija karojis poļu pusē pret Krieviju un tāpēc izsūtīts uz Sibīriju. Pats viņš bija dienējis gan Krievijas armijā, gan poļu leģionos Napoleona armijā, bet pēc tam bija ierēdnis un nodarbojās ar literāru darbību Pēterburgā. F.~Bulgarins rakstīja: ``Polijā no laika gala mēļoja par brīvību un vienlīdzību, kuru īstenībā nebija nevienam, tikai bagātie pani bija pilnīgi neatkarīgi no visām varām, taču tā nebija brīvība, bet patvaļa \citespace{} Sīkā šļahta, nevaldāma un neizglītota, vienmēr atradās pilnīgā atkarībā no katra, kurš to baroja un dzirdināja, un pat stājās viszemākajos amatos pie paniem un bagātās šļahtas, un pacietīgi pacieta kāvienus,~--- ar noteikumu, lai tas notiktu ne uz kailas zemes, bet uz paklāja, taču muļķīgas lepnības dēļ nicināja nodarbošanos ar tirdzniecību un amatiem kā neatbilstošu šļahtiča nosaukumam.''

Nepakļāvīgo šļahtas attieksmi pret iekarotājiem, to monarhiem lielā mērā balstīja tās senās tradīcijas. Pat zaudējuši tiesības uz šļahtiča nosaukumu, daudzi no tiem turpināja cienīt savu ģerboni. Ilgstoši saglabājās šļahtas prestižs sabiedrībā. Pat sīkie šļahtiči, kuriem vidējās izglītības un ierēdņa vietas ieguve pilsētā nozīmēja jau lielu personīgu panākumu, centās norobežoties no nedižciltīgajiem pilsētniekiem. ``Īsts'' šļahtičs karali uzskatīja par sev līdzvērtīgu, nevis augstākstāvošu personu, aizstāvēja savas \latxti{veto} tiesības, bija gatavs uz nepakļaušanos līdz pat atklātam dumpim (\pltxti{rokosz}) savu tiesību un brīvību aizsardzības vārdā. Politiska opozīcija esošajai varai tika uzskatīta par cienījamu tikumu, pilsonisko pienākumu. Faktiski princips, ka vara valdniekam nav Dieva dota, bet viņš to saņēmis no vēlētājiem, par kuru it kā iestājās šļahta, XIX gadsimta cīņās pret nacionālajiem apspiedējiem, neliecināja par šļahtas progresivitāti, bet par tradicionālismu, jo kā pilntiesīgi valdnieka vēlētāji tika atzīti tikai paši šļahtiči. Tas arī bija viens no galvenajiem šļahtas bezspēcības cēloņiem XIX gadsimta brīvības cīņās. Kā rakstīja ievērojamais krievu vēsturnieks S.~Solovjovs: ``Valstisko un sabiedrisko aizturu iztrūkums, sava spēka apzināšanās, savas pilntiesības un neatkarības vienreizīguma apzināšanās bija par pamatu galējai personības attīstībai poļu šļahtā, tieksmei pēc neierobežotas brīvības, nemācēšanai ar savu es piekāpties vispārēja labuma priekšā.''

Pēc Polijas sadales daļa muižnieku, labi saprotot, ka ārējā vara nodrošina viņiem pastāvošās sabiedriskās iekārtas un privilēģiju saglabāšanu, piekopa samierniecisku politiku. Taču daudzi magnāti un ievērojama daļa šļahtas piedalījās atbrīvošanās kustībā. Parasti viņus vadīja ne tikai patriotisma jūtas, bet arī personīgās intereses. Tomēr, runājot par atbrīvošanās kustību Polijā, V.~Ļeņins līdz XIX gadsimta 60.~gadiem to sauca par ``šļahtas atbrīvošanās kustību'', norādot, ka tā ``ieguva milzīgu, pirmšķirīgu nozīmi ne vien no visas Krievijas, ne vien no visu slāvu, bet arī no visas Eiropas demokrātijas viedokļa.''

Katra jauna šļahtiču paaudze (vismaz daļa tās) mēģināja atbrīvoties no apspiedējiem. Vieni cerēja kooperēties ar Krieviju vai Prūsiju, lai ar to palīdzību atjaunotu Polijas vienotību. Piemēram, poļu ģenerālis J.H.~Dombrovskis, kurš jau bija karojis T.~Kostjuško vadībā, 1796.~gadā piedāvāja Prūsijas karalim Fridriham Vilhelmam II poļu karavīrus nostādīt Prūsijas pavēlniecībā. Taču karalis par to neizrādīja nekādu interesi. Citi poļu šļahtiči vai nu pievienojās to valstu, kuras sadalīja Poliju, pretiniekiem, vai paši rīkoja sacelšanās. Pēc 1794.~gada daudzi poļi devās uz Itāliju un Franciju, viņu apmešanās centri bija Venēcija un Parīze. Īpašas cerības tika saistītas ar Franciju, kura veda nacionālus revolucionārus karus pret kontrrevolucionāro monarhiju koalīciju. Izveidojoties Napoleona I vadītajai Francijas impērijai, tā pakļāva virkni Eiropas nacionālo valstu, veda imperiālistiskus iekarošanas karus, kas savukārt izraisīja pret to nacionālus atbrīvošanās karus. Taču to daudzi poļu patrioti, norūpējušies tikai par savas dzimtenes brīvību, vairs nevēlējās saskatīt.

Vīlies Prūsijā, 1796.~gada oktobrī ģenerālis J.~H.~Dombrovskis piedāvāja Francijas Direktorijai (valdībai, 1795--1799) organizēt \strong{poļu leģionu}. Nākamā gada sākumā tika izveidoti divi tādi Francijas vasaļvalsts Cisalpīnas Republikas (itāļu \latxti{Repubblica Cisalpina}) Napoleona komandētās armijas sastāvā. Tajos galvenokārt dienēja poļi (1797.~gadā~--- ap 6~000), kuri bija dienējuši Austrijas armijā un saņemti gūstā vai arī bija dezertējuši no tās. Poļu leģionāri savu cīņu jēgu saskatīja Polijas valstiskuma atjaunošanā ar Francijas palīdzību. J.~H.~Dombrovskis izteicās: ``Kā uzvarētāji mēs uzcelsim no jauna mūsu Tēvzemi''. Leģioni kļuva par poļu nacionālisma audzināšanas skolu. Poļu leģionu karavīri nēsāja poļu mundierus ar franču kokardēm. J.~Dombrovskis ierosināja Napoleonam nosūtīt leģionus caur Turcijai pakļautajām teritorijām Balkānos uz Austrijai piederošo Galīciju, kur tika cerēts izraisīt poļu šļahtas sacelšanos. Tiesa, drīz (1797.~gada aprīlī) tika noslēgts Francijas un Austrijas miera līgums, kurš leģionāriem nāca pilnīgi negaidīts. Tas izsauca dziļu vilšanos poļu emigrantu vidū Francijā. (Daži pat devās uz Krieviju un iestājās tās armijā.) Tika kaldināti arī citi plāni kā atjaunot Polijas neatkarību. Tomēr lielākā daļa leģionāru naivi ticēja, ka Francija agri vai vēlu palīdzēs īstenot sapni par Polijas brīvību. Ne bez pamata ir krievu vēsturnieku izteiktā doma, ka poļu emigrācijas vadībai leģioni bija vajadzīgi nevis kā bāze nākamo Polijas bruņoto spēku veidošanai, bet kā ``avanss'' Francijai, lai pierādītu, cik svarīgi tai būtu atjaunot Poliju kā savu sabiedroto Eiropā.

Starp citu, poļu publicists un sabiedriskais darbinieks J.~Vibickis, lai iedvesmotu leģionārus, 1797.~gadā sacerēja vārdus plaši pazīstamajai ``Dombrovska mazurkai'' (``\pltxti{Mazurek Dąbrowskiego}''), sauktai arī par Dombrovska maršu, (Pirmais nosaukums gan bija ``Poļu leģionāru Itālijā dziesma''~--- ``\pltxti{Pieśń Legionów Polskich we Włoszech''}) Dziesmas pirmais pants un piedziedājums skanēja:

\vspace{1.5em}

\noindent
\begin{minipage}{0.45\textwidth}
\pltxti{
Jescze Polska nie zginela,\\
kiedu my źyjemu!\\
Co nam obca przemosc wzęeła,\\
szabla odbierzemy.\\
Marsz, marsz, Dąbrowski,\\
z ziemi wloskiej do poliski,\\
za twoim przewodem\\
złązcym się z narodem!}
\end{minipage}
\hspace{2em}
\begin{minipage}{0.45\textwidth}
Polija nav zudusi,\\
kamēr mēs vēl dzīvi!\\
Ko mums sveša vara ņēma,\\
Zobeniem atgūsim.\\
Marš, marš, Dombrovski,\\
no Itālijas uz Poliju,\\
mēs tavā vadībā\\
ar tautu vienosimies!
\end{minipage}

\vspace{1.5em}


Dziesmu izpildīja ar tautas mūziku mazurkas ritmā. Šī dziesma visur pavadīja poļu leģionārus, to viņi dziedāja, kad ar Napoleona armiju atgriezās dzimtenē, vēlāk slepenās sazvērnieku sapulcēs, 1831.~gadā tā kļuva par poļu sacelšanās himnu. Neraugoties uz to, ka poļu nebrīves gados tā tika aizliegta, kā aicinoša uz dumpi, viņi to turpināja dziedāt. 1926.~gadā ``Dombrovska mazurka'' kļuva par oficiālo Polijas valsts himnu.

Otrā pasaules kara laikā PSRS izveidotās T.~Kostjuško vārdā nosauktās poļu divīzijas karavīri himnu dziedāja bez pantiem, kuros bija rindas ``vācietis un krievs neizturēs, kad zobenu rokā ņemsim'' un ``visi vienā balsī runā, pietiks mums šīs nebrīves''. Taču mazurkā bija arī pants par poļu karavadoni S.~Čerņecki, kurš kļuva pazīstams ne tikai ar savu varonību, bet arī ar savu nežēlību, apspiežot pret poļu varu vērstās sacelšanās Ukrainā. Viņš arī apgānīja ukraiņu nacionālā varoņa B.~Hmeļņicka kapu un mirstīgās atliekas.

Pēc Otrā pasaules kara 1950.~gadā Polijas tautas republikā (PTR) kā valsts himna tika publicēti tikai pirmie divi panti, bet 1952. un 1953.~gadā tikai 1.~pants ar piedziedājumu. Izskanēja priekšlikums sarīkot konkursu jaunai himnai, taču tas tā arī netika realizēts. Mūsdienās himna tiek dziedāta bez ``vācieša un krieva'' pieminēšanas. Toties pants par S.~Čerņecki ir palicis.

Poļu leģionāri demonstrēja izcilu varonību kaujās, taču viņu liktenis arī parādīja, cik nesavienojama ir cīņa par brīvību ar kalpošanu iekarošanas politikai. 1798.~gadā pie diviem esošajiem leģioniem tika pievienots arī trešais, kurš nesa ``Donavas'' leģiona nosaukumu, kaut tika izvietots pie Reinas. Leģionāri, ciešot ievērojamus zaudējumus, piedalījās kara darbībā pret Napoleona pretiniekiem. Poļu leģionāriem bija visai nozīmīga vieta karos Itālijā, 1798.~gadā tie iegāja Romā, piedaloties Pāvesta valsts (itāļu \latxti{Stati Pontifici}) pagaidu likvidācijā. Kad 1799.~gadā Napoleons kļuva par pirmo konsulu~--- Francijas valsts galvu, viņš leģionus pārformēja, tie arī nomināli kļuva par Francijas armijas sastāvdaļu. Daļa leģionāru palika Itālijā un 1805.--1807.~gadā piedalījās Francijas vestajos karos, daļa (ap 5~000 ``durkļu'') 1801.~gadā, neraugoties uz pretestību, tika nosūtīta uz Haiti un tika izmantota cīņā pret nēģeru brīvības cīnītājiem F.~Tusēna-Luvertīra vadībā, kaujās un no slimībām zaudējot 2/3 sastāva. Pēc dažām ziņām tikai ap trīs simtu poļu atgriezās Francijā. (Pēc citām ziņām 1814.~gadā jau kā angļu gūstekņi Eiropā atgriezās ap 500 cilvēku.) Kopumā 1797.--1807.~gadā caur poļu leģioniem izgāja ap 35~000 cilvēku. Daļa no viņiem vēlāk veidoja Varšavas hercogistes armijas kodolu, daļa (ap 8~tūkstošiem) palika Francijas dienestā. No šī laika poļu leģiona ideja līdz pat valsts atjaunošanai starptautisku sarežģījumu brīžos atkal un atkal kļuva populāra poļu patriotu aprindās. Piemēram, Krimas kara (angļu \entxti{Crimean War}, franču \frtxti{guerre de Crimée}, krievu \rutxti{Крымская война}, 1853--1856) laikā izcilais poļu dzejnieks Ā.~Mickevičs Turcijas armijā mēģināja izveidot poļu daļas karam pret Krieviju.

Tāpat kā Francija, arī tās pretinieki: Krievija, Prūsija un Austrija, diplomātisku apsvērumu vadītas, dažkārt runāja par nākotnes plāniem atjaunot Polijas valstiskumu. Tādas baumas labprāt uzklausīja zināmas poļu aristokrātijas un arī zemākās šļahtas aprindas. Krievijā šādu plānu iniciators bija kņazs Ā.~Čartorijskis. Būdams Krievijas imperatoram Aleksandram~I tuva persona, viņš 1804--1806.~gadā bija pat Krievijas ārlietu ministrs. Ā.~Čartorijskis izstrādāja Žečpospolitas atjaunošanas plānu tās agrākajās robežās Aleksandra I protektorātā. Ar dažādu kompensāciju palīdzību Austrija būtu jāpierunā atdot Krievijai Galīciju, bet Prūsijai Lielpolija (\pltxti{Wielkopolska}~--- apgabals Polijas rietumos Vartas upes baseinā) būtu jāatņem ar spēku. Aleksandrs~I šo plānu tieši nenoraidīja, bet skatījās uz to kā uz diplomātisku Prūsijas ietekmēšanas līdzekli.

Jāuzsver, ka jau Krievijas imperators Pāvils I, tāpat kā vēlāk viņa dēls Aleksandrs I, mēģināja realizēt politiku, lai Krievija kļūtu it kā par Polijas labo aizbildni uz Austrijas un Prūsijas stingrākās varas fona. Īpaši Aleksandra I valdīšanās laikā Krievijas valsts veica pasākumus, lai radītu sev piekritējus poļu valdošajā slānī. Aleksandra~I liberālie žesti (konfiscēto muižu atdošana to agrākajiem īpašniekiem vai arī zaudējumu atlīdzināšana, nodokļu atvieglinājumi), pat privilēģiju piešķiršana (neraugoties uz vispārējo labības izvešanas aizliegumu, Aleksandrs~I piešķīra poļu muižniekiem atsevišķas atļaujas eksportēt labību, izmantojot Melnās jūras ostas) rada atbalsi poļu sabiedrībā. Nemazsvarīga loma bija ekonomisko sakaru nodibināšanai, agrāko Žečpospolitas apgabalu iekļaušanai Viskrievijas tirgū. Taču tai pat laikā Krievijas valdnieki nevarēja atteikties no bijušās Žečpospolitas baltkrievu un ukraiņu apgabaliem. Polija bija tradicionāls Krievijas ienaidnieks un Krievijas imperatori nevarēja atļauties atjaunot stipru varenu Poliju, kas varētu apdraudēt Krieviju.

\asterism

1806.~gadā, kad Napoleons uzsāka karu pret Prūsiju, Polija viņa plānos ieguva ievērojamu lomu. Napoleonu tā interesēja kā militārs placdarms, izdevīgs līdzeklis diplomātiskajā spēlē. Karā iesaistījās arī Krievija. Jauno kara fāzi Napoleons nosauca par ``Pirmo Polijas karu'' un uzdeva ģenerālim J.~H.~Dombrovskim organizēt poļu spēkus. J.~H.~Dombrovskis izveidoja vairākas divīzijas (ap 30~000 vīru) un tās sekmīgi karoja pret prūšiem un krieviem. 1806.~gada novembrī J.H.~Dombrovska ietekmē Pozenē notika poļu sacelšanās, poļi paši vairākās pilsētas atbruņoja Prūsijas garnizonus. Tas, protams, atviegloja Napoleona karaspēka darbību. 28.~novembrī franči iegāja Varšavā un 19.~decembrī tās iedzīvotāji uzgavilēja Napoleonam. Taču īstenot poļu cerības un atjaunot Polijas valsti viņš nesteidzās. 1807.~gada janvārī agrākajos Polijas apgabalos, kurus tagad okupēja Francija, Napoleons nodibināja ``Valdošo komisiju'' (``\pltxti{Komisja Rządząca''}), kura darbojās franču virsvadībā.

1807.~gada jūlijā, sakāvis Prūsiju un Krieviju, Napoleons par lielu vilšanos poļiem noslēdza Tilzītes mieru, pēc kura no poļu zemēm, kuras līdz tam ietilpa Prūsijas sastāvā (izņemot Rietumprūsiju un Pomerāniju (\pltxti{Pòmòrzé}), kas palika Prūsijā, un Dancigu (Gdaņsku), kura pirmo reizi [vēlāk tas notika pēc Pirmā pasaules kara] tika atzīta par ``brīvpilsētu'', taču ar franču garnizonu), izveidoja miniatūru \strong{Varšavas lielhercogisti} (\pltxti{Księstwo Warszawskie}, 1807--1815, faktiski~--- 1813), kurā ap 104~000 km$^{2}$ lielā apgabalā dzīvoja 2,6 miljoni iedzīvotāju. Hercogistes nosaukumam nebija nekāda vēsturiska pamata. Tas radās tikai tāpēc, ka abi imperatori: Napoleons I un Aleksandrs I, parakstot Tilzītes miera līgumu, nevēlējās lai jaunās Francijas vasaļvalsts oficiālajā nosaukumā figurētu vārds ``poļi'', ``Polija'' vai to atvasinājumi. 1809.~gadā pēc Napoleona uzvaroša kara pret Austriju hercogistei pievienoja teritorijas, kuras pēc trešās Žečpospolitas dalīšanas iegāja Austrijas sastāvā. Radās ap 160~000 km$^{2}$ liela ``valsts'' ar 4~350~000 iedzīvotāju. Nacionālā ziņā iedzīvotāju sastāvs bija daudz kompaktāks nekā Žečpospolitā. Pēc 1810.~gada statistikas hercogistē dzīvoja: 79\% poļu, ap 7\% ebreju, ap 6\% vāciešu, 8\%~--- lietuviešu un baltkrievu. Varšavas hercogistē administrācijā, tiesās, skolās tika lietota poļu valoda.

Pēc zināmām Napoleona svārstībām par hercogistes nominālo valdnieku kļuva Žečpospolitas priekšpēdējā karaļa Fridriha Augusta mazdēls, Saksijas karalis Fridrihs Augusts, kuru par likumīgu Polijas troņa mantinieku atzina vēl 1791.~gada Konstitūcija. Polijas valsts gan līdz tam jau nepastāvēja, taču poļu apdzīvotās teritorijas ar to nepilnu četrdesmit gadu laikā tika kārtējo reizi sadalītas un, kaut gan vēstures literatūrā to nav pieņemts darīt (Piemēram, Vācijas~--- Polijas attiecību speciālists vācu vēsturnieks V.~Jakobmeijers 1939.~gadā notikušo Polijas dalīšanu uzskata par piekto, pirms tam pieminot 1772., 1793., 1795. un 1815.~gada dalīšanas, bet aizmirstot par 1807.~gadā notikušo Napoleona īstenoto Varšavas hercogistes izveidi, kārtējo (ceturto) reizi pārdalot agrākās Žečpospolitas teritoriju), tomēr pēc autora viedokļa \strong{Napoleona veikto pārdali var kvalificēt par ceturto Polijas dalīšanu}. Pie tam Varšavas hercogiste pastāvēja vairākus gadus, šīs ceturtās dalīšanas rezultātā izveidotās robežas saglabājās ilgāk nekā pēc otrās dalīšanas radītās. Var jau uzskatīt Varšavas hercogistes radīšanu par Polijas atjaunošanas mēģinājumu, taču aiz šī valstiskā veidojuma palika daudzas etniskās poļu zemes.

Faktiski Varšavas hercogiste bija Francijas protektorāts, kurš piegādāja pēdējai pārtiku un ``lielgabalu gaļu''. Napoleons 1806.~gadā paziņoja: ``Polija ir pārāk sarežģīts jautājums: [poļi] pieļāva sadalīšanu, pārstāja būt par [vienotu] tautu, zaudēja sabiedrības garu, šļahtai tur ir pārāk liela loma, tautai pārāk maza. Tas ir līķis, kurā vispirms ir jāiedveš dzīvība, pirms es sākšu domāt, ko ar to darīt \citespace{} Es iegūšu no viņiem karavīrus, virsniekus, bet pēc tam paskatīšos [ko darīt tālāk].'' Tomēr pilnībā no Francijas atkarīgās hercogistes radīšanu daudzi poļi uzskatīja par Polijas atdzimšanas sākumu. 1807.~gada 22.~jūlijā Drēzdenē Napoleons izdeva hercogistes ``Konstitūciju'' (``\frtxti{Statut konstitutionennel du Duché de Varsovie''}). Napoleons Polijas pārvaldē balstījās uz tās muižniecību, taču visos oficiālos aktos izvairījās no vārdu ``polis'' vai ``poļu'' lietošanas, lai neradītu Austrijā un Krievijā bailes par to piesavināto poļu zemju drošību un pārliecinātu tās, ka Varšavas hercogistes radīšana nebūt vēl nenozīmē Polijas valsts atjaunošanu tās vēsturiskajās robežās. Dāvājot Konstitūciju ``Varšavas un Lielpolijas'' iedzīvotājiem, Napoleons uzņēmās saistības ``saskaņot to ar kaimiņvalstu mieru''. Konstitūcijā nebija norāžu uz vārda, preses, sapulču, biedrošanās brīvībām, personas un mantas neaizskaramību. Tiesa, atšķirībā no 1791.~gada 3.~maija Konstitūcijas visi iedzīvotāji bez kārtu atšķirībām tika pasludināti vienlīdzīgi likuma priekšā, taču šī vienlīdzība netika attiecināta uz ebrejiem. Seims tika vēlēts, otrā palāta~--- Senāts~--- monarha iecelts. Parlamenta kompetence bija ierobežota, tas varēja apspriest tikai budžeta, civilās un kriminālās likumdošanas jautājumus. 1807.~gadā tika atcelta dzimtbūšana, ieviesta zemnieku personīgā brīvība, taču zeme palika šļahtas rokās kā privātīpašums. Atstājot savu muižnieku, zemniekam bija pienākums ``atdot muižniekam pēdējā zemes īpašumu, kas sastāv no ēkām, inventāra un sējumiem''. Šļahtiči varēja zemniekus padzīt no viņu saimniecībām. Reformas īstenošana veicināja zemes koncentrāciju šļahtas rokās. 1810.~gadā hercogistē 54\% zemes atradās folvarkos, 2\%~--- baznīcas un 44\%~--- zemnieku rokās. Ap 20\% valsts muižu Napoleons izdāvāja saviem maršaliem un ģenerāļiem.

Interesantu Napoleona un poļu šļahtiču attiecību aspektu atklājis jau pieminētais poļu ekonomikas vēsturnieks V.~Kula. Tā Polijas daļa, kura pēc trim dalīšanām bija Prūsijas varā, bija nonākusi arī tās kredītiestāžu darbībā sfērā. Šīs iestādes bija radītās, lai atvieglotu prūšu muižniekiem (junkuriem) pāreju uz intensīvāku saimniekošanu. Iespēju saņemt valsts kredītus ieguva arī poļu muižnieki. Viņi to arī pilnā mērā izmantoja. Taču pēc Napoleona uzvaras pār Prūsiju viņš ar uzvarētāja tiesībām pasludināja sevi par šo kredītu īpašnieku un pieprasīja tos atmaksāt viņam. Izrādījās, ka atšķirībā no prūšu muižniecības lielākās daļas poļu muižnieki pārsvarā bija izmantojuši kredītus nevis savu muižu attīstībai, bet vienkārši izšķērdējuši. Poļu muižnieciskā historiogrāfija pat radīja tēzi, ka kredīti poļu šļahtičiem tikuši piešķirti ar nolūku tos izputināt, novest līdz bankrotam un tādā veidā iznīdēt arī ``poļu gara cietoksni''~--- šļahtiču muižas. Kad noskaidrojās, ka poļu šļahta nespēj samaksāt savus parādus Napoleonam, sākās sarunas. Varšavas hercogistes valdības delegācija sekoja Napoleonam pa visu Eiropu, lai vestu ar viņu cinisku kaulēšanos. Napoleons parādus pamazām atlaida~--- kā maksu par katru jaunu poļu pulku, kurš devās karot uz Spāniju. Tikai krietni vēlāk, XIX gadsimta vidū poļu muižnieki iemācījās izmantot naudu lai, atbilstoši tirgus principam, tā nestu jaunu naudu.

Arī hercogistei kopumā nācās ciest no Napoleona pastāvīgajām naudas prasībām. Tās saimniecisko attīstību traucēja viņa ievestā kontinentālā blokāde (1806--1814).

1808.~gadā tika ieviests Napoleona kodekss (\frtxti{Code Napoléon, arī Code Civil des Français}) kā hercogistes civillikums, kurš privātīpašuma principu sabiedriskajā dzīvē padarīja par izšķirošo, radot privātīpašniekiem sabiedriskās augšupejas iespējas neatkarīgi no kārtu piederības. Tas deva pirmo pamatīgo triecienu kārtu iekārtai, ievērojami sekmēja kapitālistisko attiecību attīstību feodālo vietā. Tomēr nedrīkst noklusēt, ka Napoleona kodeksa ieviešanu Varšavas hercogistes Seims pat īsti neapsprieda, uzstājās trīs deputāti, kuri ierosināja to pieņemt, uzreiz notika balsošana un ar 105 pret 2 balsīm kodekss tika pieņemts.

Tradicionālās pašvaldības tika atceltas un hercogisti pārvaldīja pēc Francijas departamentu sistēmas parauga. Hercogiste sākotnēji dalījās sešos departamentos un tie katrs desmit apriņķos. Pēc Napoleona uzvaras 1809.~gadā pār Austriju klāt nāca vēl četri departamenti no bijušajām Austrijas teritorijām. Bez tam Napoleons arī radīja atsevišķu 30~000 cilvēku lielu poļu armiju, no kuras daļa devās Napoleonam palīgā karot Spānijā. 1810.~gadā tās karavīru skaitu palielināja līdz 60~000 un 1812.~gadā līdz 75~000. Kopumā Varšavas hercogistes armijā ilgāku vai īsāku laiku dienēja ap 200~000 poļu. Komandēja armiju hercogistes kara ministrs J.~Poņatovskis, pēdējā Žečpospolitas karaļa Staņislava II Augusta Poņatovska brāļa dēls.

Kritiskāk noskaņotie poļi ironiski runāja par jauno valsti: ``Hercogiste Varšavas, monētas prūšu, armija poļu, karalis sakšu, bet kodekss franču'', taču apzinoties savu atkarību no Napoleona labvēlības, neuzdrošinājās atklāti kurnēt.

Par Varšavas hercogistes pastāvēšanas laiku eksistē anekdote.

\begin{quote}
Kāds ārzemnieks, nonācis Varšavas hercogistē, vietējiem iedzīvotājiem apjautājās:

``Kas tā ir par zemi?''

``Varšavas hercogiste,''~--- skanēja atbilde.

``Un kas ir tās valdnieks?''

``Saksijas karalis.''

``Un kāda viņam ir armija?''

``Poļu.''

``Bet kādas tiesības šeit pastāv?''

``Franču.''

``Kāda nauda?''

``Prūšu.''

``Tad gan jūs dzīvojat kā pie Bābeles torņa,''~--- nošūpoja galvu ārzemnieks.
\end{quote}


Poļu dižciltīgie baidījās par savu stāvokli, taču Napoleons, paredzot tos izmantot pret Krieviju, centās īstenot viņu interesēm atbilstošu politiku.

Lai arī Varšavas hercogiste nebija poļu sapņu piepildījums, tās radīšanai bija pozitīva nozīme. Hercogistē attīstījās nacionālā kultūra, zinātne, izglītība. Jau pašam faktam, ka pastāv valstisks veidojums, kurš turpina poļu valstiskuma tradīcijas, bija liela nozīme laikabiedru acīs. Polijas jautājums atkal tika izvirzīts aktuālās politikas ierindā, ar to nācās rēķināties Vīnes kongresam 1815.~gadā. Daudzi franču ieviestie likumi un institūcijas saglabāja savu iespaidu arī pēc hercogistes likvidācijas līdz pat XX gadsimtam.

Daļa poļu leģionāru arī pēc Varšavas hercogistes izveides palika Francijas armijā un palīdzēja Napoleonam okupēt Spāniju. Spāņi šo karu sauca~--- ``karš par Spānijas neatkarību'' (\estxti{Guerra de la Independencia Española}). Poļi cerēja, ka tādā ceļā viņi tuvina savas valsts atjaunošanu, kaut mūsdienās ir grūti izprast, kā, apspiežot citu tautu, var cīnīties par savas tautas neatkarību.

1810.--1811.~gadā Napoleons, cenšoties piesaistīt savā pusē poļu šļahtu, solīja tai Polijas atjaunošanu 1772.~gada robežās. Tiesa, ja solījums arī tiktu īstenots, šai valstij bija nolemts palikt Francijas atkarībā. 1810.~gada Francijas un Krievijas konvenciju, kurā Francija solīja nepaplašināt Varšavas hercogistes teritoriju, Napoleons atteicās ratificēt, taču šis solis vairāk vērtējams kā Aleksandra I šantažēšanas līdzeklis, nekā liecība par īstenajiem Napoleona nodomiem. Atbildot uz to, Aleksandrs I inspirēja baumas par autonomas Lietuvas lielkņazistes radīšanu no Krievijas rietumu guberņām, kura varētu kļūt par nākamās poļu valsts kodolu. Taču sarežģīto diplomātisko spēli pārtrauca 1812.~gadā sācies karš.

Neilgi pirms tā sākuma~--- 1812.~gada 10.~februārī poļu ģenerālis M.~Sokolņickis iesniedza Napoleonam detalizētu kara vešanas plānu pret Krieviju ar nosaukumu ``Par līdzekļiem kā Eiropai atbrīvoties no Krievijas ietekmes, bet pateicoties tam, arī no Anglijas ietekmes''. Tajā bija izstrādāts ne tikai Napoleona armijas virzīšanās maršruts, bet arī Krievijas sadalīšanas un kā neatkarīgas valsts likvidēšanas scenārijs. Francijas imperators varēja būt apmierināts, ka tā iekarošanas plānu realizēšanai atradās tāds iegansts kā Polija. 1812.~gada 22.jūnijā Napoleons parakstīja savu uzsaukumu Francijas ``Lielajai armijai'' (\frtxti{Grande Armée}): ``Karavīri, otrais poļu karš ir sācies. Pirmais beidzās Fridlandē [pilsēta~--- tagad Pravdinska, pie kuras notika uzvaroša Napoleona kauja ar Krievijas armiju 1807.~gada 14.~jūnijā] un Tilzītē. \citespace{} Tātad dosimies uz priekšu, pāriesim Nemunu, ienesīsim karu tās [Krievijas] teritorijā. Otrais poļu karš nesīs tādu pat slavu franču ieročiem kā pirmais. Taču miers, kuru mēs noslēgsim, tiks nodrošināts un nesīs beigas tai postošajai ietekmei, kurus Krievija nu jau 50 gadu kā atstāj uz Eiropu''. Sarunās ar poļu darbiniekiem Napoleons tieši norādīja, ka gaida poļu šļahtas uzstāšanās pret Krieviju lietuviešu, baltkrievu un ukraiņu zemēs. Taču 1812.~gada martā noslēgtajā līgumā ar Austriju Napoleons tai garantēja Galīcijas paturēšanu. Kāda varētu kļūt poļu valsts Napoleona uzvaras gadījumā, tā arī nebija skaidrs.

Naktī uz 24.~jūniju sākās pārcelšanās pār Nemunu un 300 poļu huzāru kā pirmie to šķērsoja. Poļu sajūsma bija liela. Ā.~Mickevicš to vēlāk (1834) tēloja poļu nacionālajā eposā ``\pltxti{Pan Tadeusz}'' (Tadeuša kungs). 26.~jūnijā Varšavas hercogistes Seims pasludināja, ka priekšā stāv Polijas ``atkalapvienošanās''. Poļu karavīri no Spānijas atgriezās Varšavas hercogistē. Napoleona pusē karojošais poļu korpuss bija visuzticamākais no ``Lielās armijas'' cittautiešu daļām. Polijas atbrīvošana kļuva par vienu no lozungiem, kas ļāva Napoleona armijā ievilināt daudz poļu. 1812.~gadā pirms karagājienu pret Krieviju Francijas imperatora rīcībā bija ap 85~000 (pēc citiem datiem~--- ne mazāk par 120~000) poļu karavīru. Kara gājienā uz Maskavu Napoleona \frtxti{Grande Armee} (Lielā armija) sastāvā bija ap 100~000 poļu karavīru (Citi dati: pret Krieviju Napoleona armijas sastāvā karoja 70 tūkstošu poļu ar 105 lielgabaliem.) Tādejādi katrs piektais Napoleona karavīrs bija polis. Viņi sastādīja gandrīz vai pusi no Napoleona kavalērijas. Taču visus poļus apvienot vienkopus Napoleons nevēlējās. Daļa no tiem tika izkaisīti starp citām franču armijas vienībām. Tikai ap 30 tūkstošu (citi dati~--- 37 tūkstošu) vīru lielu poļu kontingentu komandēja J.~Poņatovskis. Kopā ar viņu cīnījās arī poļu ģenerāļi J.H.~Dombrovskis, J.~Zaijončeks u.c.

T.~Kostjuško, kurš, atgriezies Eiropā, 1798.~gadā apmetās Parīzes apkaimē un piedalījās poļu politiskajā dzīvē, arī atbalstīja Poļu leģionu izveidi, 1799.~gada nogalē T.~Kostjuško tikās ar Napoleonu, taču, nesaņēmis no viņa solījumu par konstitucionālas Žečpospolitas atjaunošanu bijušajās robežās, uzskatīja par sev pazemojošu akli kalpot Francijas imperatoram. Viņš nepārcēlās arī uz Varšavas hercogisti, kaut 1807.~gadā sarunā ar Francijas policijas ministru J.~Fušē, piesolīja Napoleonam savu palīdzību, ja tas dotu rakstisku solījumu, publicētu avīzēs, ka Polijā tiks izveidota valsts iekārta līdzīga Anglijas iekārtai, ka zemnieki tiks atbrīvoti ar zemi un Polijas robežas sniegsies no Dancigas līdz Ungārijai, iekļaujot arī Galīciju. Atbildei uz to Napoleons rakstīja J.~Fušē: ``Es nepiešķiru nekādu nozīmi Kostjuško. Viņam savā zemē nav tās ietekmes, kurai viņš pats tic. Vispār, visa viņa uzvedība vieš pārliecību, ka viņš ir vienkārši muļķis. Vajag ļaut viņam darīt, ko vēlas, nepiegriežot viņam nekādu uzmanību''. Par Napoleonu T.~Kostjuško izteicās: ``Viņš domā tikai par sevi, nicina ikvienu lielu tautību un vēl vairāk neatkarības garu. Viņš ir tirāns.''

Poļu daļas pirmās iegāja krievu karaspēka pamestajā Viļņā. Taču poļi drīz bija vīlušies. Atkarojis Krievijai Lietuvu, Napoleons to nepievienoja Varšavas hercogistei, bet radīja patstāvīgu hercogisti, kura gan ilgi nepastāvēja.

Poļi piedalījās arī Borodinas u.c. kaujās pret krievu karaspēku, ciešot lielus zaudējumus. Pēc tam, kad jau bija pierādījusies krievu tautas pretestība iebrucējiem, Napoleona franču ģenerāļi ieteica viņam pasludināt dzimtbūšanas atcelšanu Krievijā, ko imperators tomēr nedarīja. Vācu vēsturnieks E.~Veiss uzskata, ka tas, iespējams, notika tāpēc, ka viņš negribēja kaitēt savai sabiedrotajai~--- poļu šļahtai. Tomēr jāsaka, ka pēc tam, kad krievu zemnieki bija izjutuši Napoleona karavīru pārestības, kuri atņēma tiem pēdējos mājlopus un pēdējo gabalu maizes, cerēt uz vietējo iedzīvotāju atbalstu iebrucēji diezin vai varēja, pat ja paziņotu par dzimtbūšanas jūga atcelšanu.

1812.~gada beigās, kad Varšavā nonāca ziņas par Napoleona armijas grūto stāvokli Krievijā, tika mēģināts savākt jaunus poļu brīvprātīgos, bet cerēto 30~tūkstošu vietā pieteicās vien ap 400~cilvēku. Atgriezušās Polijā, izretinātās poļu daļas ar J.~Poņatovski priekšgalā gan vēlējās aizstāvēt Varšavu pret Krievijas armiju, taču spēku nepietika un poļu karavīri devās līdzi Napoleona armijai, pametot Poliju.

Napoleons, atkāpjoties no Krievijas, izstrādāja divus variantus nākotnei. Pēc pirmā viņš bija gatavs kompromisa mieram ar Krieviju, solot tai nepieļaut Polijas valsts atjaunošanu. Viņš teica: ``Ja poļi neizpildīs savu pienākumu [domāta brīvprātīga masu mobilizācija karam pret Krieviju], tad Francijai un visai pasaulei miera jautājums vienkāršosies, jo tad ar Krieviju noslēgt mieru būs viegli.'' Otrais variants paredzēja piekāpšanos Austrijai, piešķirot tai teritorijas Balkānos, bet tai piederošo Galīciju atdodot Varšavas hercogistei un pasludinot Polijas atjaunošanu. Par Polijas karali tad vajadzētu kļūt kādam no Napoleona tuvākajiem līdzgaitniekiem. Izmantojot poļu sajūsmu par valstiskuma atgūšanu, Napoleons vēl cerēja vest uzvarošu karu pret Krieviju. Taču neviens no variantiem neīstenojās, jo Eiropas monarhi nevēlējās slēgt nekādas vienošanās ar Napoleonu.

1813.~gadā Napoleona pusē vēl karoja ap 40~000 poļu. J.~Poņatovskis kļuva par vienīgo ārzemnieku, kurš Leipcigas t.s. ``Tautu kaujas'' laikā (1813.~gada 16.--19.~oktobrī) no Napoleona rokām saņēma Francijas maršala zizli, taču pēc tam~--- arī pavēli piesegt franču daļu atkāpšanos. Franči priekšlaikus uzspridzināja tiltu un viņš, nevēloties padoties, ievainots mēģināja pārpeldēt Elsteres upi un noslīka.

Karojot Napoleona vadībā, 1812.~gadā poļi atstāja smagas atmiņas Krievijas civiliedzīvotājiem. Galvenie vardarbību un laupīšanu veicēji Krievijā bija nevis Napoleona disciplinētie francūži, bet vācieši un poļi. Par to rakstījuši daudzi krievu autori, sākot no A.~Puškina līdz J.~Tarlem. Kā rakstīja baltkrievu izcelsmes vēsturnieks profesors M.~Kojalovičs: ``Maskavā tauta neieredzēja un baidījās no tiem francūžiem, kuri saprata krieviski un runāja krieviski. Tie bija poļi.'' Krievijas apmija, sekojot Napoleona armijai un nonākot Polijā, bija gatava to pārvērst par tuksnesi, taču poļus no tā paglāba imperators Aleksandrs I, kurš Rietumeiropā vēlējās ieiet kā atbrīvotājs, nevis atriebējs. Krievu muižnieku aprindās gan klīda baumas, ka imperatora labvēlīgā attieksme pret poļiem ir saistīta ar viņa faktisko otro, neoficiālo sievu, dzimušo polieti A.~Nariškinu, ar kuru viņam bija vairāki bērni. Poļu šļahta kā savu līderi pie Aleksandra I nosūtīja Ā.~Čartorijski, taču viņš tā arī nesaņēma no Krievijas imperatora skaidru atbildi par Polijas nākotni. Līdz jautājuma galīgai starptautiskai izlemšanai Aleksandrs I kā augstāko pārvaldes iestādi lika radīt pagaidu Augstāko padomi (\pltxti{Rada Tumsczasowa Najwyźsza}), kuras sastāvā bija divi krievi, divi poļi un viens prūsis.

1814.~gadā Napoleona pusē vēl karoja ap 4~000 poļu. Pat pēc Napoleona atteikšanās no troņa viņam uzticīgie poļu brīvprātīgie izveidoja vieglās kavalērijas eskadronu, kurš pavadīja bijušo imperatoru uz Elbas salu. Šis eskadrons piedalījās arī Napoleona ``100 dienu'' epopejā un pilnībā gāja bojā Vaterlo (\entxti{Waterloo}, 1815.~gada 18.~jūnijā) kaujā. Vēl visu XIX gadsimtu poļu sabiedrībā pastāvēja kas līdzīgs Napoleona kultam.

Vācu vēsturniece A.~Šmidte-Roslere uzskata, ka to nosacīja Napoleona mīlas sakars ar poļu muižnieci M.~Vaļevsku un viņu kopējais dēls A.~Vaļevskis, kurš kļuva par Francijas pavalstnieku, bet tomēr piedalījās 1830--1831.~gada poļu sacelšanās norisē, tika apbalvots, atgriezās Francijā, kur Napoleona III laikā kļuva par tās ārlietu ministru. Vēsturniece atsaucas uz B.~Prusa romānu ``Lelle'' (``\pltxti{Lalka}''). Domājams, ka tomēr lielāka nozīme bija poļu patriotu utopiskajām cerībām, ka Napoleons uzvaras gadījumā pār Krieviju atjaunos Žežpospolitu, tāpēc simpātijas pret viņu neizdzisa vēl ilgi.

Šai sakarā gan krievu dzejnieks kņazs P.~Vjazemskis, kurš savu karjeru sāka Varšavā, jaunībā bija liberāli noskaņots un 1830.~gadā uzstājās pret krievu karaspēka ievešanu Polijā, savā dienasgrāmatā rakstīja: ``Lai cik arī poļi būtu truli, taču nevar iedomāties, ka vesela tauta brīvprātīgi ietu nāvē, pretī neglābjamai bojā ejai \dots{} Napoleons viņus sagūstīja ar divām, trijām frāzēm \citespace{} Ko viņš ir izdarījis Polijas labā? Griezies pie tās ar vairākiem madrigāliem savās proklamācijās, izdalījis tai dažus Goda Leģiona krustus, nopirktus ar poļu asins straumēm. Tas arī viss.'' Taču poļu patrioti kā slīcējs pie salmiņa pieķērās jebkurai cerībai, kas tiem solīja valstiskuma atjaunošanu.

Jau 1812.~gada decembrī, ieradies Viļņā, Krievijas imperators Aleksandrs I pasludināja vispārēju amnestiju Krievijas pavalstniekiem~--- poļiem, kuri bija karojuši Napoleona pusē, vēlāk atļāva izdzīvojušajiem poļu karavīriem atgriezties dzimtenē. Tur viņi tika iekļauti jaunveidojamās Polijas karalistes (\pltxti{Królestwo Polskie}) armijā. Ģenerālis J.~H.~Dombrovskis saņēma no Aleksandra~I kavalērijas ģenerāļa pakāpi, piedalījās Krievijas pakļautībā esošas Polijas karalistes jaunveidojamās armijas radīšanā un kļuva par Polijas senatoru. Ģenerālis J.~Zaijončeks kļuva par Aleksandra I vietvaldi (\rutxti{наместник}) Polijas karalistē, saņēma kņaza titulu.

Arī T.~Kostjuško 1814.~gadā griezās pie Aleksandra I ar vēstuli, kura saturēja padomus, kā labiekārtot Poliju. T.~Kostjuško lūdza Aleksandru I pasludināt sevi par Polijas karali, dot Polijai Konstitūciju, līdzīgu tai, kāda pastāvēja Anglijā utml. Sākotnēji Krievijas imperators pret poļu ģenerāli izturējās labvēlīgi, pat piedāvāja viņam vadīt Polijas karalistes administrāciju, taču pēc būtības atbildēja izvairīgi, ka cerot ``paveikt varonīgās tautas atdzimšanu''. T.~Kostjuško, uzzinot, ka Polija netiks atjaunota 1772.~gada robežās, atteicās sadarboties ar uzvarētājiem karā un devās uz Šveici, kur arī mira. T.~Kostjuško ķermenis tika iebalzamēts un 1818.~gadā pārvests uz Krakovu, apglabāts Vāvelas (\pltxti{Wawel}) pils katedrālē, kur atrodas Polijas karaļu kapenes.

\asterism

% page 69


1815.~gadā pēc Napoleona galīgas sakāves un uzvarētājvalstu sasauktā \strong{Vīnes kongresa} Napoleona radītā Varšavas hercogiste pazuda no kartēm. Padzenot Napoleona armiju, tās teritoriju (kas līdz hercogistes radīšanai piederēja Prūsijai un daļēji Austrijai) bija ieņēmis Krievijas karaspēks. Līdz Vīnes kongresam Krievijas rokās faktiski atradās 9/10 bijušās Žečpospolitas. Vīnes kongresā Aleksandrs I vēlējās panākt Varšavas hercogistes pievienošanu Krievijai, tikai vissliktākajā gadījumā piekāpjoties Prūsijai un atdodot tai daļu Pozenes novada. Taču Rietumvalstis, īpaši Austrija, Francija un Anglija, baidoties no Krievijas nostiprināšanās, nevēlējās visas Varšavas hercogistes zemes atstāt tai. Tā netika ievērotas arī poļu tautas objektīvās intereses radīt pēc iespējas vienotu teritoriju, kuras iedzīvotājiem būtu vieglāk cīnīties arī par savu valstiskumu.

Pēc Vīnes kongresa lēmuma Varšavas hercogistes rietumu daļa tika nodota Prūsijai un sāka saukties par Pozenes lielhercogisti (vācu \detxti{Großherzogtum Posen}; poļu \pltxti{Wielkie Księstwo Poznańskie}, 1818--1848). Tās teritorija gandrīz sasniedza 29~000 km$^{2}$, tur dzīvoja ap 0,9 miljoniem cilvēku. Varšavas hercogistes lielāko, austrumu daļu ar pašu Varšavas pilsētu pasludināja par Polijas karalisti (krievu~--- \rutxti{Царство Польское}, poļu~--- \pltxti{Królestwo Polskie}) un nodeva Krievijai (No 161~600 km$^{2}$ lielās Varšavas hercogistes 128~500 km$^{2}$ ar 2,7~miljoniem iedzīvotāju nonāca Krievijas rokās). Tā Krievija starptautiska lēmuma rezultātā kļuva arī par poļu etnisko zemju īpašnieci.

Krievu militārās tehnikas speciālists, kurš publicē arī darbus par vēsturi, A.~Širokorads raksta, ka ne Krievijas cars Aleksejs Mihailovičs, ne Katrīna II, ne J.~Staļins vēstures gaitā nepaņēma ne pēdu īstenas poļu zemes, bet atgrieza tikai Krievijai pirms tam piederošās, bet zaudētās teritorijas. Minētais publicists arī norādījis, ka zaudētu teritoriju atgūšanas piemēru Eiropas vēsturē ir daudz. Un nevienam neienāk prātā saukt Spānijas un Portugāles monarhus par agresoriem par to, ka rekonkististas (\estxti{Reconquista}~--- atkarošana) vairāku gadsimtu (722--1492) laikā viņi atkaroja Ibērijas pussalu mauriem, vai Francijas karaļus, kuri Simtgadu (1337--1453) kara laikā atguva britu sagrābtās zemes, kaut gan spāņi, gan francūži veica arī daudz zvērību, tostarp pret gūstekņiem un civiliedzīvotājiem.

Tad nu nākas atgādināt, ka Krievijā bija arī valdnieki, sākot ar Aleksandru I, kuri tomēr faktiski pievienoja poļu zemes Krievijas impērijai. Tiesa, šai gadījumā Eiropas valdnieki faktiski ar Polijas karalisti tikai atlīdzināja Krievijai par tās ieguldījumā karā pret Napoleonu, kura uzticami sabiedrotie bija poļi. Pēc šīs loģikas poļu zemju nonākšanas krievu rokās vaininieki bija paši poļi. Viņu zemju ar Varšavu nodošanu Krievijai varēja uzskatīt kā ``sodu'' poļiem par aktīvo Napoleona atbalstīšanu. Protams, viņiem bija cita loģika~--- poļu tautai bija tādas pat tiesības uz valstiskumu kā citām to ieguvušām tautām. Taču sabiedrotie tika meklēti pēc principa: ``Mana ienaidnieka ienaidnieks~--- mans draugs''. Tika aizmirsts, ka šim ``ienaidnieka ienaidniekam'' pirmajā vietā vienmēr bija (un būs) savas, nevis poļu intereses.

Polijas zemju pievienošana Krievijas impērijai no tās puses bija starptautiskās sabiedrības atbalstīta uzurpācija, neapstrīdama tik tiešām etnisko poļu zemju sagrābšana. Protams, imperators Aleksandrs I domāja pirmkārt par savām un Krievijas impērijas, nevis poļu interesēm. Turklāt Aleksandrs I 1815.~gadā formāli atjaunoja Polijas valstiskumu, tā saukto Polijas karalisti, un tā tika apvienota ar Krievijas impēriju uz personiskās ūnijas noteikumiem. Pašu poļu uztverē tas gan bija tikai jauns Krievijas solis Polijas atsevišķu daļu aneksijā.

PTR pastāvēšanas laikā oficiālie poļu autori rakstīja, ka poļu varonība Napoleona kara laikā esot nesusi savus augļus un tāpēc Vīnes kongresa esot nolemts saglabāt Polijas valstiskuma iedīgli, radot Polijas karalisti. Apgalvojums gan netika argumentēts. Faktiski Polijas karaliste radās dažādu valstu interešu sadursmju rezultātā, dalot poļu apdzīvotās teritorijas atbilstoši Eiropā pastāvošajam spēku samēram.

Mūsdienu baltkrievu autors, pedagoģijas un informātikas speciālists A.~Tarass, acīmredzot balstoties uz Aleksandra I Polijas karalistei piešķirto valstisko atribūtiku, uzskata, ka poļi ne velti karojuši zem Napoleona karogiem, jo 25 gadus no 1807. līdz 1831.~gadam viņu zeme bijusi suverēna valsts, lai arī atkarīga sākotnēji no Francijas, pēc tam no Krievijas. Var jau teikt, ka nav nevienas pilnībā suverēnas valsts, visas ir zināmā mērā atkarīgas no citām, taču minētais autors izliekas neredzot, ka suverēnas valstis vismaz piedalās savu robežu noteikšanā un tieši suverenitātes nepilnība noveda poļus pie 1830.~--- 1831.~gada sacelšanās.

Austrijai tika jau agrāk tai piederējusī Galīcija un vēl Veļičkas (\pltxti{Wieliczka}) apriņķis ar sāls raktuvēm. Austrija zaudēja teritorijas, ko tā bija guvusi trešajā un daļu~--- pirmajā Žečpospolitas dalīšanā. Tās nonāca Polijas karalistē. Austrijas Galīcijas apgabals bija 77~000 km$^{2}$ liels, tajā dzīvoja līdz 3,5~miljoniem iedzīvotāju. Krakova tika pasludināta par brīvpilsētu visu triju valstu aizbildniecībā. Tās apgabala (ap 1~000 km$^{2}$ ar 96~000 iedzīvotāju, no kuriem vairāk kā 20 tūkstošu dzīvoja pašā pilsētā, bet pārējie~--- tās apkārtnes ciemos) ``brīvība'' gan bija iluzora, jo faktiski valdīja triju uzvarētāju valstu rezidenti. Pēc Lielbritānijas pieprasījuma dalītājvalstis gan apņēmās ievērot poļu tautības un poļu valodas tiesības, taču stingri noteiktas tās nebija.

Vīnes kongress 1815.~gadā radīja tādu kārtību Eiropā, kad patstāvīga Polija nebija pastāvošās sistēmas nepieciešama sastāvdaļa. Polija palika sadalīta, Polijas jautājums~--- neatrisināts. Tā atbilstoši Vīnes kongresa lēmumiem 1815.~gadā faktiski atkal notika nu jau kārtējā \strong{Polijas teritorijas dalīšana}. (Kaut vācu vēsturnieki: A.~Šmidte-Roslere, L.~Dralle, poļu vēsturnieks G.~Labuda u.c. to sauc par ceturto Polijas dalīšanu, šī darba autors tomēr uzskata, ka pareizāk būtu runāt par trijām Polijas-Lietuvas valsts (Žečpospolitas) dalīšanām, kas notika XVIII gadsimtā, taču, ja šo uzskaiti turpina, par kārtējām bijušās valsts teritorijas dalīšanām jāuzskata gan Varšavas hercogistes radīšanu no agrāk Prūsijai un Austrijai piederošajiem pārsvarā poļu apdzīvotajiem apgabaliem, gan tās likvidēšanu. Tā XIX gadsimtā notika divas~--- ceturtā un piektā~--- šīs teritorijas dalīšanas, bet XX gadsimtā jau nākamās~--- sestā (1939.~gadā) un septītā (1945). Tiesa, jāuzsver vēl viena aspekts. XVIII gadsimtā trīs reizes tika dalīta Polijas-Lietuvas valsts (Žečpospolitas) daudzu tautu apdzīvota teritorija, XIX gadsimtā tika dalīta nu jau vienīgi poļu teritorijas, bet XX gadsimtā (1939. un 1945.~gadā) jau Polija zaudēja 1918.~--- 1920.~gadu laikā ar spēku sagrābtās teritorijas, kuras apdzīvoja kā poļi, tā arī vācieši, lietuvieši, ukraiņi, baltkrievi.

G.~Labuda uzsvēris, ka Vīnes kongresa veiktā dalīšana daudz būtiskāk kā iepriekšējās salauza saimniecisko, politisko un etnisko poļu zemju vienotību.

Marksisma teorētiķi Polijas dalīšanu kvalificēja kā ``vislielāko noziegumu''. Līdzdalība Polijas dalīšanā vairākus gadu desmitus saistīja valstis~--- tās dalībnieces ar kopējām interesēm ne tikai poļu tautas nacionālās atbrīvošanās kustības, bet zināmā mērā arī citu tautu nacionālās kustības apspiešanā. Tieši ar Vīnes kongresu radās triju monarhiju: Austrijas, Krievijas un Prūsijas t.s. Svētā savienība (franču \frtxti{La Sainte-Alliance}, vācu \detxti{Heilige Allianz}, krievu \rutxti{Священный союз}), izteikti konservatīvs spēks Eiropā. Dalība Polijas dalīšanā līdz pat 1914.~gadam cementēja triju monarhiju sadarbību un sargāja Austrumeiropu pret kariem un revolucionāriem satricinājumiem.

Kopumā Prūsija saņēma 8\%, Austrija 10\% un Krievija 82\% agrākās Žečpospolitas teritorijas. Padomju vēsturnieks A.~Manusevičs ne bez pamata apgalvoja, ka Polijas karalistes radīšana nesa zināmu atvieglojumu poļu tautas liktenī salīdzinājumā ar iepriekšējo stāvokli un radīja labvēlīgākus apstākļus attīstībai. Taču tas nevar mīkstināt secinājumu, ka faktiski kārtējā Polijas teritorijas pārdale, ko veica Vīnes kongresā, bija tāds pats noziegums pret poļu tautu kā Žečpospolitas dalīšanas XVIII gadsimtā. Pie tam ar centrālā Polijas apgabala nonākšanu Krievijas impērijas rokās poļu tautai radās vēl viens, jauns apspiedējs~--- Krievijas carisms, pret kuru poļiem nācās cīnīties par savu brīvību. Padomju vēsturnieki gan uzsvēra, ka līdz ar to poļu tauta ieguva arī sabiedroto cīņā pret carismu~--- krievu tautu (pareizāk gan būtu teikt~--- Krieviju apdzīvojošās tautas), taču autoram jāatzīmē, ka šis sabiedrotais tolaik bija vājāks par jauniegūto apspiedēju, tāpēc no poļu tautas viedokļa notikumu attīstību nekādi nevar vērtēt kā pozitīvu.

Žečpospolitas sadales, Napoleona kari postoši ietekmēja gandrīz visus banķierus un lieluzņēmējus. Rezultātā pilsonībai (buržuāzijai), kas veidojās 19.~gadsimtā, nebija pamatu iepriekšējā laikmetā, nebija savu tradīciju. Otra jaunveidojamās pilsonības īpatnība bija tā, ka lielā mērā tā veidojās no nepoļu izcelsmes elementiem. Tāda pilsonība politiski ilgi vēl nevarēja konkurēt ar šļahtu. Līdz XIX gadsimta 60.~gadiem buržuāzijā valdošie bija valdībai labvēlīgi uzskati, ko lielā mērā noteica tās atbalsts rūpniecības un tirdzniecības attīstībai un dažviet (piemēram, Lodzā) nepoļu pilsonības pārsvars.

Pēc Polijas sadales svešzemnieku vara veda pie tā, ka poļu etniskās teritorijas nomalēs norisa pretrunīgi procesi. No vienas puses, lauku iedzīvotāji sāka zaudēt nacionālo (etnisko) raksturu. Denacionalizējās daļa sīkās šļahtas un inteliģences, kura saistīja savu pastāvēšanu ar kalpošanu svešajām varām. Aristokrātijai, lielajiem magnātiem vispār bija raksturīgas kosmopolītiskas tendences. Taču no otras puses, tai pat laikā nacionālā apspiešanas politika, kuru piekopa Žečpospolitu sadalījušās valstis, izsauca pretestību un stimulēja poļu nacionālās pašapziņas izaugsmi. Tādejādi, ja nacionāla valstiskuma, nacionālas valodas un kultūras pastāvēšanai poļu nācijas, tās pašapziņas veidošanā bija nenoliedzami pozitīva loma, tad Polijas sadalei un nacionālu spaidu ieviešanai bija divejāda: gan bremzējoša, gan stimulējoša ietekme.

\asterism

\strong{Poļi XIX gadsimtā dzīvoja arī ārpus nosauktajām teritorijām}. Viņu kopējais skaits sniedzās līdz 9 miljoniem. Salīdzinoši daudz viņu bija \strong{Silēzijā} (vācu \detxti{Schlesien}, poļu \pltxti{Śląsk}), kura jau iepriekš ietilpa Prūsijas sastāvā un veidoja īpašu Silēzijas provinci (\detxti{Provinz Schlesien}).

Lejassilēzijā (vācu \detxti{Niederschlesien}, poļu Dolny Śląsk) sakarā ar tekstilrūpniecības, bet kopš XIX gadsimta sākuma arī metalurģijas strauju attīstību un valdības pasākumiem vācu kolonizācijas veicināšanai ieplūda daudz vācu iedzīvotāju un plaši izplatījās vācu valoda. Šeit pastāvēja Prūsijā lielākie muižnieku zemes īpašumi. XIX gadsimta beigās vairāk nekā 90\% Lejassilēzijas iedzīvotāju par dzimto uzskatīja vācu valodu.

Augšsilēzijā (Vācu \detxti{Oberschlesien}, poļu \pltxti{Górny Śląsk}) kur jau XVIII gadsimta otrajā pusē tika atklātas pirmās akmeņogļu ieguves šahtas (pati pirmā~--- 1751.~gadā), agrāk kā citur (jau XIX gadsimta 40.~gados) notika rūpniecības apvērsums, šahtās sāka pielietot tvaika mašīnas ūdens atpumpēšanai, attīstījās melnā un krāsainā metalurģija. XIX gadsimta pirmajā pusē akmeņogļu ieguves šahtu skaits pieauga trīs reizes, dzelzs ražošana~--- sešas reizes, cinka ražošana~--- trīs reizes un sastādīja ap 75\% no Prūsijas un 40\% no visas pasaules ieguves. 1862.~gadā Augšsilēzijā parādījās Prūsijā pirmās martena krāsnis. Raktuves, metalurģiskie uzņēmumi u.c. atradās Prūsijas valdības un prūšu aristokrātijas rokās, kas ekspluatēja poļu darbaļaudis. Augšsilēzijā jau sākotnēji šķiriskais dalījums sakrita ar nacionālo. Poļu valoda gan ilgstoši kalpoja kā mājas sarunu līdzeklis, bet arī šeit sākās poļu ģermanizācijas process. Tāpēc līdz Polijas valsts atjaunošanai 1918.~gadā sabiedriskie procesi Silēzijā tikai atbalsoja nacionālās cīņas citos poļu apdzīvotajos rajonos.

\strong{Poļi dzīvoja arī} Polijas karalistē neietilpstošajās \strong{Krievijas Ziemeļrietumu novada} Viļņas, Kauņas, Grodņas, Minskas, Mogiļevas un Vitebskas (XIX gadsimtā dalījums nedaudz mainījās) \strong{guberņās}, tai skaitā arī Vitebskas guberņā ietilpstošajā Latgalē, \strong{un Dienvidrietumu novada} (jeb labā krasta Ukrainas) Podolijas, Volīnijas un Kijevas \strong{guberņās}. Gandrīz pusi, ap 40\% no visiem šo apgabalu iedzīvotāju sastādīja katoļi un uniāti, ap 10\% ebreji, 5\% protestanti. Izglītības sistēmā valdošā bija poļu valoda. Viļņas, Grodņas, Minskas, Volīnijas un Podolijas guberņās sākotnēji poļi civilajā un militārajā pārvaldē ne tikai saglabāja, bet arī nostiprināja savu lomu. Šeit poļu muižnieki baudīja privilēģijas uz pārējo iedzīvotāju rēķina, taču sakarā ar poļu nacionālās atbrīvošanās cīņu pakāpeniski viņu stāvoklis sarežģījās. Citāds stāvoklis bija Vitebskas, Mogiļevas un Kijevas guberņās, kur gandrīz visu ierēdniecību veidoja krievi vai pārkrievojušies ārzemnieki.

Atsevišķo poļu apdzīvoto apgabalu būtiskās atšķirības ekonomiskās attīstības līmenī noteica ne tik daudz darba dalīšana starp vienas zemes novadiem, cik dažādie politiski-ekonomiskie apstākļi, kādos tie nonāca. Līdzās vāciešiem un itāļiem poļi bija pēc skaita trešā lielākā Eiropas tauta, kura XIX~gadsimtā cīnījās par nacionālo apvienošanos un vienota valstiskuma izveidi. Atšķirībā no pirmajām divām, poļiem tas neizdevās. Pārējās valstis, panākušas savu neatkarību, to nedeva Polijai, jo tas apdraudētu stabilitāti Eiropā.

Jāatzīmē, ka ārpus Polijas karalistes teritorijas palikušās agrākās Polijas-Lietuvas zemes, kuras apdzīvoja ukraiņi, baltkrievi, lietuvieši un latvieši (latgalieši) un kuras tagad ietilpa tieši Krievijas impērijas, bet nevis Polijas karalistes sastāvā, paši poļi toreiz sauca par sagrābtajām, t.i.~--- viņiem atņemtajām zemēm. Turpretī, piemēram, no baltkrievu nacionālā viedokļa šo novadu pievienošana Krievijai pārsvarā jāvērtē pozitīvi, jo notika radniecīgu, vēsturiski, kultūras un valodas ziņā tuvu slāvu tautu apvienošanās vienā spēcīgā valstī. Šeit stabilizējās politiskā situācija, pārtrūka piespiedu katolizācija un polonizācija (citu, nepoļu tautu asimilācija, pārpoļošana). Tas gan notika ne uzreiz. Šajās bijušajās Polijas-Lietuvas valsts austrumu zemēs poļi bija ievērojamā mazākumā, tomēr ierēdņu vidū viņu bija daudz, arī augstākajos amatos, jo pārējie iedzīvotāji izglītības līmeņa ziņā no poļiem atpalika. Poļu iecelšanu ierēdņu amatos veicināja arī lielkņazs Konstantīns, kuram Aleksandrs I piešķīra militāro un civilo varu arī Krievijas rietumu guberņās. Tikai Nikolaja I valdīšanas laikā (1825--1855) poļu ietekmi šeit sāka ierobežot.

Protams, pievienošana Krievijai arī ukraiņiem un baltkrieviem nesa ēnas puses, carisms savas politikas realizācijai izmantoja poļu valdošos slāņus, ukraiņu un baltkrievu tautai netika piešķirta valstiskums kaut vai autonomijas formā, to apdzīvotie apgabali tika sadalīti dažādās guberņās, saglabājās feodālā apspiestība. Poļu muižnieki, ja viņi neatteicās dot uzticības zvērestu Krievijas imperatoriem, saglabāja savu varu pār vietējiem zemniekiem.

Arī Latgalē stāvoklis bija līdzīgs kā lietuviešu, ukraiņu un baltkrievu apdzīvotajos apgabalos. Kā to parāda latviešu vēsturnieks Ē.~Jēkabsons, poļu muižnieku attieksme pret vietējiem zemniekiem neatšķīrās no vācu muižniecības attieksmes pret viņiem citās Latvijas daļās. Katram, kurš kaut ko gribēja dzīvē sasniegt, bija jārunā poliski un jākļūst par poli. Pieņemot darbiniekus muižās, priekšroka tika dota tiem, kuri runāja poliski. Atbildīgu amatu pildītājus, piemēram, muižnieku bērnu skolotājus, bieži aicināja no Polijas.

\subsection{Galīcija Austrijas / Austroungārijas valdījumā}

Austrija (no 1866.~gada Austroungārija) izrādījās visvājākā no trim lielvalstīm, tas ietekmēja arī poļu stāvokli tās pārvaldītajās teritorijās. Jau imperators Jozefs II 1781.~gadā uzsāka zemnieku brīvlaišanu (tika mazināta zemnieku personīgā atkarība no muižniekiem, zemnieku bērni ieguva iespēju doties uz pilsētām u.tml., taču klaušas vēl saglabājās), veicinot zemnieku zemes izpirkšanu, taču šis process ievilkās, jaunu impulsu saņemot tikai 1848.~gada revolūcijas laikā. Galīcija jeb pēc oficiālā nosaukuma~--- Galīcijas un Lodomērijas karaliste (vācu \detxti{Königreich Galizien und Lodomerien,} poļu \pltxti{Królestwo Galicji i Lodomerii),} kur karalistei bija pievienots arī Bukovinas novads (ukraiņu \uktxti{Буковина}, rumāņu \rotxti{Bucovina}, 1775.~gadā Krievija to atdeva Austrijai, tā to 1786.~gadā pievienoja Galīcijai, bet 1849.~gadā pārvērta par atsevišķu apgabalu; mūsdienās daļa Bukovinas ietilpst Ukrainā, otra daļa Rumānijā) pēc Vīnes kongresa tika sadalīta 18 apgabalos (\detxti{Kreis}), atjaunota centralizēta pārvalde, birokrāti bija galvenokārt etniskie vācieši (austrieši) vai pārvācoti čehi. Valsts valoda bija vācu, tajā darbojās pārvaldes iestādes. Provincei nebija vietējās autonomijas. Tika gan atjaunots Seims (\pltxti{Sejm Stanowy Królewstwa Galicji i Lodomerii}), kurā ievēlēja augstāko slāņu~--- garīdzniecības, muižniecības~--- magnātu un šļahtas~--- un arī Lembergas (Ļvovas) pilsētas~--- Galīcijas administratīvā un politiskā centra~--- pārstāvjus, taču tā pilnvaras bija niecīgas. Tāpēc vācu vēsturnieks E.~Meijers uzskatīja, ka pēc 1815.~gada poļiem Galīcijā bija mazāk brīvību nekā pārējās Polijas daļās.

Ekonomika šai novadā attīstījās vājāk kā Prūsijai un Krievijai pakļautajos poļu apgabalos. Galīcija bija arī viena no Austrijas visvairāk atpalikušajām provincēm. Vācieši to dēvēja par Habsburgu monarhijas nabagmāju (\detxti{Armenhaus}). Poļu valodā to bieži sauca par \pltxti{Golicija i Głodomorija} [vārdu spēle: kails \pltxti{(goły)} un izbadējies (\pltxti{głodny})]. Atpalicība no citiem poļu novadiem sāka mazināties tikai pēc 1830.--1831.~gada sacelšanās Polijas karalistē, kad daudzi poļu emigranti bēga uz Galīciju. Tiesa, liela daļa no viņiem Austrijas valdības spiediena rezultātā bija spiesta aizbraukt uz Franciju, taču daļa tomēr ieguva pilsonību. Šeit visilgāk valstī saglabājās zemnieku dzimtbūtnieciskā atkarība (personīgo brīvību zemnieki Galīcijā saņēma tikai 1848.~gadā), novecojušas saimniekošanas formas. Dzelzceļa celtniecība šeit sākās vēlāk nekā pārējās poļu zemēs. Krakova tikai 1855.~gadā tika savienota ar Vīni, 1861.~gadā~--- ar Lembergu, nākamajos gados dzelzceļa stigu turpināja līdz Krievijas un Rumānijas robežai. Pēc tam pāri Karpatu kalniem tika uzbūvētas četras dzelzceļa līnijas, kuras savienoja Galīciju ar Ungāriju, taču tām bija galvenokārt militāri-stratēģiska nozīme.

Kā uzskata krievu autori, tieši iepriekšējā gadsimtiem ilgā poļu kundzība bija novedusi novadu pie pilnīga panīkuma. Galīcijā zemniekus nospieda ļoti augsts nodevu slogs, kas neļāva modernizēt saimniecību. Ebreju izcelsmes poļu vēsturnieks Š.~Aškenazi rakstīja, ka Galīcijas sīkpilsonis visa veida nodokļos maksāja 16 reizes vairāk nekā sīkpilsonis Polijas karalistē. Pie tam, ja pēdējā valsts kases ieņēmumi tika izlietoti tikai vietējām vajadzībām, tad no Galīcijā ievāktajiem līdzekļiem 25\% ``aizpeldēja'' uz Vīni. 80\% Galīcijas iedzīvotāju bija zemnieki, 27\% no viņiem piederēja saimniecības mazākas par 1 ha, 42\% no tām bija starp 1 un 5 ha. Tā kā nenotika lauksaimnieciskās ražošanas modernizācija un mehanizācija, produkcijas ražošana nepieauga. XX gs sākumā Galīcijā vajadzēja pat ievest labību no Ungārijas. Ja 1869.~gadā Galīcijā bija 5,45~miljoni iedzīvotāju, līdz 1910.~gadam viņu skaits pieauga līdz 8 miljoniem. No 1880. līdz 1910.~gadam 600~000 cilvēku devās uz ārzemēm, lielākoties pāri okeānam. Nabadzība un higiēnas trūkums veda pie augstākās bērnu mirstības Habsburgu monarhijā. Arī izglītības ziņā Galīcija, gan kopā ar Bukovinu un Dalmāciju, atradās pēdējā vietā Austroungārijas impērijā. 1890.~gadā 64,8\% vīriešu un 71,6\% sieviešu Galīcijā neprata lasīt un rakstīt. Muižnieki pat iedomāties nevarēja, ka viņi brīvprātīgi varētu likvidēt klaušu sistēmu, viņi arī neko nedarīja, lai uzlabotu savu zemnieku stāvokli.

Rūpniecība praktiski neattīstījās, jo bija vājš iekšējais tirgus. Turklāt vairāk attīstītajās austriešu un čehu apdzīvotajās zemēs pastāvēja uzņēmumi, kuru konkurenci Galīcijas uzņēmumi nevarēja izturēt. 1822.~gadā Galīcijā gan pastāvēja ap 40 nelielas metālizstrādājumu fabrikas, kas pārstrādāja zemas kvalitātes dzelzsrūdu, taču to skaits pastāvīgi samazinājās. Pēc 1841.~gada datiem Galīcijā un Bukovinā, kur dzīvoja 28,7\% visu impērijas iedzīvotāju, atradās tikai 3,4\% visu Austrijas (bez Ungārijas) uzņēmumu. 22\% Galīcijas rūpnieciskās produkcijas veidoja šņabis. XIX gadsimta otrajā pusē, dzelzceļa satiksmes ar Austriju izveides un ar to saistītās metālizstrādājumu ievešanas rezultātā Galīcijā iezīmējās krasa metalurģijas krīze.

Tomēr atzīmējams, ka kopš XIX gadsimta otrās puses Galīcijā Drohobičā (\pltxti{Drohobycz}) darbojās lielākās naftas ieguves Eiropā, atklātas 1846.~gadā. 1866.~gadā šeit tika uzcelta pirmā naftas pārstrādes rūpnīca Centrāleiropā, 1910.~gadā~--- otrā. Līdz 1900.~gadam iegūstamais naftas daudzums sasniedza 1.~miljonu tonnu, bet 1912.~gadā Austroungārijā, galvenokārt Galīcijā, jau ieguva 2,9~miljonus tonnu. Ar to Austroungārija bija trešā lielākā naftu iegūstošā valsts pasaulē pēc ASV un Krievijas.

Stāvokli sarežģīja \strong{raibais nacionālais sastāvs}. Muižnieki~--- zemes īpašnieki vairumā bija poļi jeb ungāri, bet zemnieki, īpaši provinces austrumos~--- rusīni jeb \strong{rutēņi} (Habsburgu impērijā no 1848.~gada oficiālajos dokumentos vācu valodā tika lietots nosaukums \detxti{Ruthenen}, tā izdalot atsevišķu no poļiem un krieviem nacionālo grupu. Tāpēc, runājot par XIX gadsimtu, autors lieto terminu \lttxti{rutēņi}, kaut arī krievu vēsturnieki dod priekšroku terminam ``rusīni'', kas uzsver to kopību ar krievu tautu). Vietējā muižniecība, arī tirgotāji un augstākā garīdzniecība bija pārpoļota, tikai mājās vēl lietojot tautas valodu. Savai tautībai uzticīgi bija palikuši tikai sīkie amatnieki, zemnieki un zemākie garīdznieki. Pēc dažiem datiem Galīcijā dzīvoja 47\% poļu, 45\% rutēņu un 6\% ebreju (Pēc citām ziņām poļu bija 45,9\%, rutēņu~--- 42\%, ebreju~--- 10\%. Krievu vēsturniece A.~Bahturina raksta, ka XIX gadsimta vidū novada iedzīvotāju vairākumu sastādījuši rusīni~--- 43,7\% un ebreji~--- 11,8\%. Jāsaka, ka sarežģītā Austrumgalīcijas attīstība ļāva krieviem tur saskatīt krievu, poļiem~--- poļu vairākumu.). Rutēņu inteliģences Galīcijā gandrīz nebija, tāpat kā nebija augstāko un vidējo mācību iestāžu. Skolās, kuru sistēma šai laikā Austrijā vēl nebija sevišķi attīstīta, notika pārvācošana (ģermanizācija), taču sākumskolās, īpaši lauku, mācības norisa dzimtajā valodā. Habsburgu monarhija īpaši necentās pārvācot vietējos iedzīvotājus. Tomēr pakāpeniskā politiskās sistēmas liberalizācija veda arī pie Galīcijas rutēņu nacionālās kustības pacēluma. Tās priekšgalā nostājās uniātu baznīca.

No otras puses, poļu nacionālajai kustībai šeit bija senas tradīcijas. 1817.~gadā atjaunotajā Lembergas (Ļvovas) universitātē (dibināta 1784.~gadā) mācības gan norisa latīņu un vācu valodā, bet pēc diviem gadiem tur nodibināja poļu valodas katedru. Poļu muižniecības bērnu vajadzībām bija atvērtas vairākas ģimnāzijas. No Galīcijas aktīvākie poļu patrioti devās uz Polijas karalisti, lai piedalītos 1830.~gada sacelšanās. Tieši no šīs kustības izplatības baidījās Austrijas valdošie slāņi, jo Galīcija iedzīvotāju skaita ziņā sastādīja gandrīz 1/7, bet teritorijas ziņā~--- 1/8 daļu no toreizējās Habsburgu impērijas. Austrieši ievēroja rutēņu masu noskaņojumu un veicināja to pretpoļu un proaustriešu aktivitātes.

Galīcijā bija trīs katoļu eparhijas, kurās darbojās ap 700~draudžu un 900~baznīcu, ap 50~vīriešu un 15~sieviešu klosteru ar ap 500~mūkiem un mūķenēm. Bez tam novadā pastāvēja arī divas uniātu eparhijas ar vairāk nekā 200~draudžu un ap 300~baznīcu. Uniātu baznīcas rokās atradās gandrīz visas rutēņu nacionālās kustības aktivitātes, vērstas uz atbrīvošanos no poļu un krievu ietekmes.

Jaatzīmē, ka Austrijas sastāvā līdz ar Polijai atsavinātajām teritorijām nonāca arī ap 200~000 ebreju. Austrijas imperators Jozefs II jau 1781.~gadā izdeva dekrētu par reliģisko toleranci, ar ko uzlaboja arī ebreju stāvokli. Taču mērķis bija ebreju asimilācija. Tikai pēc tās viņi varētu iegūt līdztiesību.

Blakus esošajā Krakovas brīvpilsētā (\pltxti{Rzeczpospolita Krakowska}), kura 1815.~gadā saņēma savu satversmi, saglabājās demokrātiskā kustība un Polijas valsts atjaunošanas ideja. 1364.~gadā dibinātajā Krakovas universitātē veidojās nacionāli noskaņoto pulciņi. Autonomā Krakova kalpoja par vispolijas tirdzniecības centru. Kad no kaimiņos esošās Polijas karalistes pēc 1830./1831.~gada sacelšanās šurp bēga tās dalībnieki, Krievijas karaspēks uz laiku okupēja brīvpilsētu. Tās autonomijas tiesības tika ierobežotas. Taču drīz atkal radās konspiratīvas poļu patriotu grupas. Tad pilsētu līdz 1841.~gadam okupēja Austrijas karaspēks. Tomēr Krakova turpināja eksistēt kā brīvdomības un poļu patriotisma centrs.

Stāvoklis Galīcijā saasinājās jau 1846.~gadā. Poļu šļahta mēģināja sacelties pret Austriju. Iniciatīvu parādīja poļu revolucionārie demokrāti, kuru vidū liela loma bija filozofam, literātam un sabiedriskam darbiniekam E.~Dembovskim. Šī grupa cīņu par Polijas nacionālo atbrīvošanos centās apvienot ar zemnieku antifeodālo prasību apmierināšanu. Sacelšanos atbalstīja arī Galīcijas liberāļi. Revolucionārās cīņas sākās Krakovā (\pltxti{Rewolucja Krakowska}). 22.~februārī sacēlušies izveidoja t.s. Polijas Republikas Nacionālo valdību (\pltxti{Rząd Narodowy Rzeczypospolitej Polskiej}), kura manifestā ``Visiem poļiem, kuri prot lasīt!'' aicināja tautu celties cīņai pret trijām Polijas sadalītājvalstīm par nacionālo neatkarību, pirmo reizi Polijas vēsturē pasludināja radikālas sociālās reformas: feodālo klausību atcelšanu, nekavējošos zemnieku apstrādātās zemes nodošanu viņu īpašumā, zemes piešķiršanu bezzemniekiem, algas paaugstināšanu strādniekiem un amatniekiem, sabiedrisko darbu ieviešanu tiem. Uzsaukumā ``Brāļiem izraēlītiem'' tika pasludināta ebreju emancipācija. Tā Krakovā 1846.~gadā, divus gadus agrāk nekā pārējā Eiropā, izvērsās revolucionāras cīņas ar vispārnacionālu nozīmi. Taču t.s. Nacionālā valdība tā arī neko būtisku neizdarīja lai pārnestu revolucionāro kustību ārpus Krakovas, izplatītu sacelšanos zemniecībā. Neliela E.~Dembovska vadīta vienība, kura devās iesaistīt cīņā apkārtējos zemniekus, 27.~februārī cieta sakāvi sadursmē ar austriešu karaspēku, pats viņš gāja bojā. 2.~martā ap 1~500 poļu~--- Polijas Republikas Nacionālās valdības piekritēju bēga uz Pozeni, 4.~martā austriešu armija ieņēma Krakovu.

Blakus Krakovai Galīcijas laukos austriešu administrācijai izdevās pārliecināt vietējo zemniecību nostāties pret poļu ``paniem''~--- saviem muižniekiem, kuri šos zemniekus nežēlīgi apspieda. Jau pirms sacelšanās Krakovā bija sākušies zemnieku nemieri (\pltxti{rabacja}). Tagad austriešu varasvīru uzkūdītie zemnieki pavasarī izvērsa tā saucamo ``Galīcijas slaktiņu'' (\pltxti{Rzeź galicyjska). Tā bija visasākā zemnieku antifeodālā uzstāšanās XIX gadsimtā,} virzīta galvenokārt pret poļu muižniekiem, daļēji arī inteliģenci: tika izlaupītas muižas, nogalināti muižnieki, nereti arī vienkārši nepazīstami iebraucēji, kurus pūlis uzskatīja par sacēlušos aģitatoriem. Pat poļu tautības zemnieki Galīcijas rietumdaļā atbalstīja nevis savus panus~--- muižniekus, bet austriešu administrāciju. Zemnieki izdemolēja vairāk nekā 400~muižu un nogalināja vairāk nekā tūkstoti (pēc dažiem datiem līdz diviem tūkstošiem) poļu zemes īpašnieku un viņu ģimenes locekļu. Šis zemnieku trieciens poļu neatkarības aizstāvjiem bija ļoti sāpīgs kā politiski, tā morāli. Aprīlī sacelšanos apspieda austriešu karaspēks. Sacēlušos zemnieku vadītājs J.~Scela tika internēts un izsūtīts uz Bukovinu. Poļu XIX gadsimta daiļliteratūrā viņš iegājis kā antivaronis. Piemēram, S.~Vispjaņska populārajā drāmā ``\pltxti{Vesele}'' (``Kāzas'') viņš tēlots kā asinīm aptraipīts gars. Turpretī XX gadsimtā Polijas Tautas republikas laikā viņš tika vērtēts kā zemnieku sacelšanās pret apspiedējiem simbols.

Pēc sacelšanās apspiešanas Krievija, Prūsija un Austrija 1846.~gada 6.~novembrī parakstīja līgumu par Krakovas kā brīvpilsētas likvidāciju. Tā pēdējā formāli neatkarīgā poļu teritorija tika pievienota Austrijai. Austrijas Habsburgu dinastijas pārstāvji turpmāk saviem tituliem pievienoja vēl vienu~--- ``Krakovas hercogs'' (``\detxti{Herzog von Krakau}''). Pēc šiem notikumiem Krakovas poļu šļahtiču un inteliģences attieksme pret austriešu administrāciju vēl vairāk pasliktinājās. Nacionālās un sociālās pretrunas nebija likvidētas.

Krakovas sacelšanās neveiksme ietekmēja Galīcijas poļu izturēšanos turpmākajos notikumos. Bija stipri novājināts nacionālās kustības revolucionārais spārns. Galīcijas iedzīvotāji salīdzinoši vāji piedalījās 1848.~gada revolūcijā, jeb ``tautu pavasarī'', kur pārējās Eiropas tautas blakus sociālajiem izvirzīja arī nacionālos mērķus. Demokrāti, kuri iestājās par zemnieku atbrīvošanu, politiskajām brīvībām, Galīcijā bija ļoti vāji. Tomēr ziņas par revolucionāriem notikumiem Vīnē 1848.~gada martā izsauca demonstrācijas Krakovā un Lembergā. Tā, 1848.~gada 19.~martā Lembergā notika masu manifestācija un 12~000 cilvēku (aptuveni 1/6 no visiem pilsētas iedzīvotājiem) parakstīja petīciju imperatoram Ferdinandam I, ar prasību piešķirt pilsoniskās un nacionālās tiesības, ieviest mācības poļu valodās, kā arī piešķirt tai oficiālu statusu, pilnībā atcelt klaušas zemniekiem. Taču imperators atteicās pat apspriest prasības par pārmaiņām Galīcijā. 5.~aprīlī Krakovā tika radīta Nacionālā komiteja (\pltxti{Komitet Narodowy}), bet Lembergā~--- Nacionālā padome (\pltxti{Rada Narodova}). Uz vietām radās komitejas, kas pakļāvās Nacionālajai padomei.

Galīcijas poļu liberāļi joprojām kā galveno izvirzīja neatkarīgas poļu valsts atjaunošanas lozungu, bet demokrātiski pārveidojumi viņiem palika otrajā vietā. Tomēr daļa poļu liberāļu, tai skaitā viens no poļu autoritatīvākajiem politiskajiem darbiniekiem emigrācijā, bijušais Krievijas ārlietu ministrs un imperatora Aleksandra I personīgais draugs, kurš aktīvi piedalījās 1830.~gada sacelšanās, tās laikā bija Pagaidu valdības (\pltxti{Rząd Tymczasowy), pēc tam Nacionālās valdības (Rząd Narodowy) priekšsēdētājs} kņazs Ā.~Čartorijskis griezās pie muižniecības ar aicinājumu atteikties no savām vecajām tiesībām uz zemi. Taču Galīcijas šļahtiči kopumā baidījās tā pazaudēt arī tiesības uz atlīdzību par to. Tai pat laikā sākās zemnieku nemieri, tie masveidā atteicās pildīt klaušas. To izmantoja austriešu ģenerālgubernators grāfs F.~Stadions, kurš, saņēmis imperatora piekrišanu, 22.~aprīlī paziņoja par klaušu atcelšanu, zemnieku apstrādātās zemes nodošanu to īpašumā ar vēlāku valsts atlīdzību agrākajiem tās īpašniekiem. Varas iestādes veica zemes sadali zemniekiem lai atņemtu poļu patriotiem popularitāti, ko viņi daļēji bija guvuši pateicoties sociāli-ekonomiskajām prasībām. Galīcijā no augšas ievestā agrārā reforma bija liberālāka nekā blakus esošajās Krievijai un Prūsijai piederošajās ukraiņu un poļu zemēs. Tas vēl vairāk nošķīra vietējos zemniekus no poļu nacionālās kustības, novājināja to. Galīcijā no 1848. līdz 1857.~gadam tika īstenota agrārā reforma, kura pārvērta zemniekus par viņu apstrādājamās zemes īpašniekiem, taču to rokās bija tikai nelieli zemes gabali, vairāk nekā 2/3 no tiem piederēja mazāk par 10 morgiem (5,7 ha). Zemnieki asi izjuta aramzemes, bet īpaši pļavu un mežu trūkumu. Tas tikai pieauga, dalot zemi mantojumā daudzbērnu ģimenēs.

1848.~gada 25.~aprīlī Krakovā sākās nekārtības, iedzīvotāju sadursmes ar austriešu karavīriem. Nākamajā dienā austrieši apšaudīja pilsētu no Vavelas (\pltxti{Wawel}~--- kalns Krakovā pie Vislas ar monumentālu celtņu ansambli, tai skaitā karaļa pili, celtu XIII--XIV gadsimtā, un XIV gadsimta gotisku katedrāli, kur apglabāti karaļi un vēlāk arī citas izcilas Polijas personības) un aplenca pilsētu ar karaspēku. Sacēlušies kapitulēja un Nacionālā komiteja pārtrauca savu darbību. Tas bija pirmais kontrrevolūcijas panākums Eiropā.

Taču Lembergā poļu nacionālā kustība turpinājās. Tur tika radīta 20~000 liela poļu Nacionālā gvarde. Apstākļos, kad visā Habsburgu monarhijā notika revolūcija, tas deva cerības uz panākumiem. Taču Austrijas armija vispirms apspieda sacelšanos Prāgā, tad itāļu provincēs, oktobrī pašā Austrijā un varēja pievērsties arī Galīcijai. 2.~novembrī Nacionālā Padome Lembergā tika padzīta, pilsēta apšaudīta ar artilēriju, visā Galīcijā ievests kara stāvoklis, visas nacionālās organizācijas atlaistas. Revolūcija Galīcijā bija apspiesta. Vēl gan palika cerības uz revolūciju Ungārijā, kur ģenerāļa J.~Bema vadībā cīnījās arī tūkstošiem poļu, bet 1849.~gada maijā tajā pēc jaunā Austrijas imperatora Franča Jozefa aicinājuma iegāja Krievijas karaspēks ģenerālfeldmaršala I.~Paskeviča vadībā un triju mēnešu laikā sacelšanos apspieda. Ar to galīgi sabruka poļu cerības uz Eiropas palīdzību. Vairs nebija izredžu tuvākajā laikā izraisīt visas tautas bruņotu brīvības cīņu. Kara stāvoklis Galīcijā gan turpinājās vēl līdz 1854.~gadam.

Nākamo triecienu poļu nacionālajai kustībai Galīcijā deva \strong{rutēņu nacionālās kustības rašanās}. Kā raksta daži poļu vēsturnieki, Austrijā varas iestāžu mākslīgi atbalstītais konflikts starp šļahtu un zemniekiem paralizēja poļu nacionālo aktivitāti. Tiesa, var strīdēties par to, vai šis konflikts pie tā paša rezultāta nevestu arī bez varas iestāžu atbalsta. Pie tam jāuzsver, ka zemnieku kustībai bija ne tikai sociāla, bet arī nacionāla~--- rutēņu nokrāsa.

Austriešu ģenerālgubernators grāfs F.~Stadions, baidoties no tā, ka Galīcija varētu pārvērsties par ``poļu Pjemontu'' (itāļu \ittxti{Piemonte}, Pjemontas apgabals Itālijā XIX gadsimtā ieguva nozīmīgu lomu itāļu tautas nacionālās atbrīvošanās kustībā, kur ap Sardīnijas karalisti (itāļu \ittxti{Regno di Sardegna}), faktiski~--- Pjemontu 1859.--1860.~gadā notika Itālijas apvienošanās),~--- placdarmu, no kura varētu sākties neatkarīgas poļu valsts atjaunošana (Tik tiešām, vēlāk, 1918.~gadā, Galīcija, pateicoties saviem kultūras centriem, pēc Polijas neatkarības atjaunošanas kļuva par sava veida ``Polijas Pjemontu'', kura spēja sagādāt nākamajai Polijas valstij daudz politiķu, pārvaldes darbinieku, virsnieku, zinātnieku), meklēja pretsvaru poļiem un atrada tādu Galīcijas rutēņos.

Rutēņu intereses šai laikā vismaz daļēji sakrita ar Austrijas imperatora un valdības interesēm. 1847.~gadā Galīcijā iznāca 32 rutēņu izdevumi, bet 1848.~gadā~--- jau 156 (tiesa, tas bija rekords un netika pārspēts nākamo 30~gadu laikā). Ar viņa atbalstu 1848.~gadā tika radīta pirmā rutēņu politiskā organizācija~--- Galvenā Tautas padome (ukraiņu \uktxti{Головна Руска Рада}, krievu \rutxti{Галицийская рада}), kuru vadīja bīskaps G.~Jahimovičs. Tā laika rutēņu nacionālās kustības īpatnība bija naidīgums pret poļu liberāļiem, kuri izteica poļu muižnieku intereses, un uzsvērta lojalitāte Austrijas imperatoram. Pēc Galīcijas padomes iniciatīvas veidojās arī vietējās rutēņu komitejas no garīdzniecības un vēl tikai topošās inteliģences pārstāvjiem. Tika izvirzīta prasības paplašināt rutēņu tiesības Galīcijā un sadalīt provinci divās daļās: rietumu~--- poļu, un austrumu~--- rutēņu daļā. 1848.~gada jūnijā šo jautājumu apsprieda Slāvu tautu kongresā Prāgā, tur arī tika atzīta visu Galīcijas tautību līdztiesība.

Savu ietekmi atstāja arī Krievijas armijas dalība sacelšanās apspiešanā Galīcijai blakus esošajā Ungārijā, izmantojot arī Galīcijas teritoriju. Kā atzīmē daži mūsdienu krievu vēsturnieki, krievu armijas, kura runāja rutēņiem saprotamā valodā, parādīšanos Galīcijā vietējie zemnieki uztvēra ar sajūsmu un tā kļuva par impulsu krievu kultūras atdzimšanas sākumam. Tas esot izsaucis arī vēlmi tuvināties Krievijai, kas varēja apdraudēt Austrijas, bet vēlāk (no 1867.~gada) Austroungārijas vienotību. Tiesa, sajūsma par krievu armijas parādīšanos neatturēja pazīstamo rutēņu aktīvistu, juristu, rakstnieku Ā.~Dobrjaņski, kurš 1848.~gadā tika ievēlēts Ungārijas parlamentā, norobežoties no panslāvisma. Viņš paziņoja: ``Ungāru brīvība mums ir tuvāka nekā krievu patvaldība, tāpat kā mīkstais Ungārijas klimats ir tuvāks nekā Sibīrijas ziema.'' Vēlāk gan uz etnisko mazākumu asimilāciju vērstā Ungārijas varas iestāžu politika atstūma no Budapeštas rutēņu kustības vadītājus un pat pārvērtā viņus par rusofīliem. Arī minētais Ā.~Dobrjaņskis emigrēja uz Krieviju.

Uz rutēņu prasībām pēc lielākām tiesībām tika dota Galīcijas ģenerālgubernatora F.~Stadiona atbilde: ``Jūs varat rēķināties ar valdības atbalstu tikai tajā gadījumā, ja gribēsiet būt patstāvīga tauta un atteiksieties no apvienošanā ar tautu ārpus valsts, proti~--- Krievijā, ja gribēsiet būt rutēņi, bet ne krievi. Jums nekaitētu pieņemt jaunu nosaukumu, lai atšķirtos no krieviem, kuri dzīvo ārpus Austrijas''. Pēc dažu krievu vēsturnieku domām, austriešu varas iestādes, dažādi veicinot rutēņu etnisko atšķirību attīstību, atšķirīgu no krievu valodas rakstību, palīdzējušas rasties ukraiņu valodai, ukraiņu vēsturei, lai tikai atšķeltu vietējos iedzīvotājus no krievu tautas. Domājams, ka pretkrievisko nodomu vadītās Austrijas nacionālās politikas efektivitāte un arī rutēņu tendence apvienoties ar krieviem šeit tiek ievērojami pārspīlētas, taču ir nenoliedzams, ka Austrijas varas iestādes starpnacionālo attiecību laukā sekmīgi īstenoja saukli ``skaldi un valdi''. Tāpat nenoliedzams, ka 1848--1849.~gada revolucionārais pacēlums sekmēja Galīcijas rutēņu kultūras un izglītības darbības organizatorisku noformēšanos un pārtapšanu nacionāli--politiskā kustībā.

1861.~gadā tika izveidots vietējais Galīcijas Seims (poļu \pltxti{Sejm Krajowy}, vācu \detxti{Galizischer Landtag}) vietējo jautājumu risināšanai. Baidoties no 1863.~gadā Polijas karalistē sākušās sacelšanās izplatības arī Galīcijā, daži vietējie poļu muižnieki prasīja ieviest šeit aplenkuma stāvokli. 1864.~gadā Galīcijā arī tika ieviests kara stāvoklis, kurš pastāvēja līdz 1865.~gada maijam. Austrija pastiprināja robežas apsardzību ar Polijas karalisti, pār kuru virzījās palīdzība sacelšanās dalībniekiem.

No otras puses, vietējie poļu politiķi baidījās, ka Krievijas un Vācijas antipoliskā politika (Pēc 1863.--1864.~gada poļu sacelšanās sakāves cara valdība uzsāka represijas ne tikai pret poļu nacionālo kustību, bet arī pret ukraiņu valodas un kultūras propagandu) var tikt īstenota arī Galīcijā, ka varas iestādes var pret viņiem izspēlēt ``ukraiņu kārti''~--- sākušos ukraiņu nacionālās neatkarības kustību. Acīmredzot nemazsvarīga loma bija arī tam, ka Austrija, atšķirībā no Krievijas un Vācijas, bija katoliska valsts. Tāpēc viņi pakāpeniski atteicās nodalīt Galīciju no pārējās Austrijas/Austroungārijas.

Pēc 1863--1864.~gada sacelšanās sakāves Polijas karalistē, Austrijas sakāves karā ar Prūsiju (1866) un Austroungārijas izveides (1867) tās varas iestādes arī mainīja savu politiku, atteicās no ģermanizācijas. Austroungārijā praksē tika realizēta \strong{trīs ``kungu tautu''}~--- vāciešu (austriešu), ungāru (maģāru) un Galīcijas poļu~--- uzkundzēšanās pārējām tautām. Administratīvās struktūras bija izveidotas tā, ka vienā tās daļā~--- Cisleitānijā~--- vadošā tauta bija austrieši, bet otrā~--- Transleitānijā~--- ungāri. Galīcija ietilpa Cisleitānijā. Nacionālajā ziņā austrieši Bohēmijā (čehu \cstxti{Čechy}, vācu \detxti{Böhmen}~--- vēsturisks reģions mūsdienu Čehijas rietumdaļā) un Morāvijā (čehu \cstxti{Morava}, vācu \detxti{Mären}, vēsturisks novads Čehijas austrumdaļā) varēja turēt pakļautībā čehus, Istrijā (horvātu un slovēņu \sltxti{Istra}, itāļu \ittxti{Istria}~--- pussala Adrijas jūrā, Horvātijas ziemeļos)~--- slovēņus un horvātus, maģāri Ungārijā~--- slovākus, rumāņus un horvātus, bet poļi Galīcijā~--- rutēņus. Maģāru tīkojumus palielināt savu ietekmi novadā austrieši atvairīja, sadarbojoties ar konservatīvajiem poļu aristokrātiem.

Galīcijā vairāk kā citur Austroungārijā tika meklēti kompromisi ar cittautiešiem, tajā netika realizēta denacionalizācijas politika, kuru varētu salīdzināt ar ģermanizāciju Vācijai un rusifikāciju Krievijai piederošajās poļu teritorijās.

Arī Galīcijas poļu vidū auga vēlme pārskatīt iespējamos attīstības ceļus liberālisma gaismā. Buržuāziskie liberāļi, apzinoties poļu ekonomiskā potenciāla vājumu salīdzinājumā ar austriešu, prūšu un krievu, pirmkārt meklēja ceļus kā paaugstināt poļu kapitāla konkurences spēju. Pasludinot lozungu ``Ekonomika ir svarīgāka par politiku'', viņi ne tikai nosodīja romantisko ideju par poļu nacionālo vienreizīgumu, kura bija šļahtas revolucionaritātes pamatā, bet aktuālo cīņas problēmu pret nacionālo apspiešanu centās piepildīt ar citu saturu.

Jau 1864.~gada beigās Krakovas žurnālists L.~Povidajs rakstā ``\pltxti{Polacy i Indianie}'' (``Poļi un indiāņi''), kurš izsauca diskusijas visās poļu zemēs, brīdināja poļus neatkārtot Ziemeļamerikas indiāņu likteni, kuri esot noniecinājuši materiālo labklājību, lepojušies ar savu ``folkloras pārākumu'' un rezultātā gājuši bojā kolonizatoru spiediena rezultātā. (Protams, indiāņu nolemtības cēloņu attēlojums bija vairāk nekā apšaubāms.~--- V.Š.) Izmantojot prūšu politiķu izteikumus par poļiem kā ``Eiropas indiāņiem'', nolemtiem iznīcībai, L.~Povidajs minēja, pēc viņa viedokļa, briesmīgās prūšu realizētās ģermanizācijas sekas poļu zemēs, īpaši Silēzijā, kur rūpniecība un tirdzniecība bija vācu rokās. ``Svešu''~--- nepoļu kapitālu ieplūšanu rūpniecībā un tirdzniecībā viņš uzskatīja par nacionālā jautājuma asuma saglabāšanās cēloni visās poļu zemēs. Jautājuma atrisināšanas ceļu viņš saskatīja ``savu''~--- poļu uzņēmēju darbības aktivizēšanā. Taču šī aktivizācija nevarēja notikt bez politiskiem priekšnoteikumiem.

Viens no Galīcijas poļu liberālajiem līderiem F.~Smolka 1868.~gadā Galīcijas seimā iesniedza rezolūciju, kura aicināja uz cīņu par Galīcijas autonomiju un Austroungārijas federāciju, kur Galīcija būtu viena no sastāvdaļām. Tuvā franču-vācu kara, Austroungārijas dalības tajā un novājināšanās perspektīvas Galīcijas gan konservatīvajos, gan liberālajos poļu politiķos stiprināja cerības par sava mērķa panākšanu. Taču Francijas sakāve 1870.~gada karā un Austroungārijas neitralitāte tajā apraka visus federālistiskos plānus. Cīņa turpinājās par autonomiju.

Galīcija pakāpeniski ieguva vairāk politisku tiesību nekā citas Austroungārijas provinces. 60.--70.~gados Vīne ar daudziem likumdošanas aktiem faktiski ieviesa t.s. \strong{Galīcijas autonomiju}. 1873.~gadā, kad tika ieviestas tiešas (bez vietējā Seima starpniecības) vēlēšanas reihsrātā, Galīcijas poļu mantīgo slāņu prasību vairākums tika realizēts. Austrijas parlaments sastāvēja no divām palātām. Augšpalātā, kurā ietilpa imperatora ģimenes locekļi, aristokrātisko dzimtu pārstāvji un speciāli imperatora nozīmētas personas, Galīciju pārstāvēja trīs arhibīskapi un vēl 11 aristokrāti. Apakšpalātā, kuru veidoja vietējo seimu delegācijas, Galīcija ieguva 38~vietas.

Līdz pat XX gadsimta sākumam Galīcijas poļus politiski pārstāvēja lielie zemes īpašnieki, kuri bija sociāli tuvi austriešu aristokrātijai. Viņi bija galvenokārt konservatīvi orientēti. Pastāvēja vietējā pašvaldība (\pltxti{samorząd}). Poļi provincē ieguva saimnieciski un politiski vadošo lomu. Lembergā (Ļvovā) darbojās Galīcijas Seims. Tas tika vēlēts pēc kūriju principa, t.~i.~--- pēc profesionālā un mantas statusa, un tikai nelielai iedzīvotāju daļai bija vēlēšanu tiesības, priviliģētākie bija šļahtiči un bagātie pilsētnieki. No 150 Seima deputātiem 141 tika vēlēti. 9 vietas ``pēc amata'' saņēma bīskapi un universitāšu rektori. Seimā bija poļu vairākums. Ukraiņiem, kuri sastādīja pusi no Galīcijas iedzīvotājiem, bija tikai līdz ¼ vietu. Tikai XX gs. sākumā Austroungārijā realizēto reformu rezultātā Galīcijas Seimā paplašinājās arī trūcīgo iedzīvotāju pārstāvniecība. Seima kompetencē bija Galīcijas lauksaimniecības un mežsaimniecības jautājumi, novada budžets, labdarības, skolu un baznīcas lietas, Galīcijā izvietoto armijas daļu apgāde.

1866.~gadā Seims pieņēma lēmumu par poļu valodas ieviešanu Galīcijas skolās. 1867.~gadā Galīcijā tika izveidota novada skolu padome (\pltxti{Rada Szkolna Krajowa}), kura organizēja izglītības iestāžu darbu. 1869.~gadā poļu valoda Galīcijā ieguva oficiālu vietējās jeb zemes valodas (\detxti{Landessprache}) statusu. Poļu valodu atzina arī vietējās administratīvajās un tiesu iestādēs, tā tika iekļauta skolu programmās. Valodu, kādā notika mācības skolās, varēja izvēlēties. 1870.~gadā Krakovas (Jagelloņu) universitātē no vācu pārgāja uz poļu mācību valodu, 1871.~gadā tika paplašināta pasniegšana poļu valodā Lembergas universitātē, tehniskajās mācību iestādēs. Poļu mācību iestādes Austroungārijā kļuva populāras arī Pozenes un Polijas karalistes poļu vidū. Tā, 1909./1910.~mācību gadā starp 3~250 Krakovas universitātes studentiem 639 bija Krievijas pavalstnieki.

Lemberga kā Galīcijas administratīvais un politiskais centrs auga visai strauji. 1772.~gadā, kad pilsēta nonāca Austrijas sastāvā, tajā bija 20~000 iedzīvotāju. 1900.~gadā~--- jau 159~000, bet 1910.~gadā pat 200~000. 1861.~gadā dzelzceļš Lembergu saistīja ar Vīni. XIX gadsimta otrajā pusē Lemberga bija jau salīdzinoši moderna tirdzniecības pilsēta, ko kādreiz sauca arī par ``Austrumu Vīni''. Tajā darbojās Vācijas, Francijas, Lielbritānijas un Dānijas konsulāti. Ielas bija bruģētas, ierīkota kanalizācija, pacēlās modernas ēkas~--- Landtāga nams (1881), teātris (1900), dzelzceļa stacija (1904). 1894.~gadā tajā kā ceturtajā pilsētā Eiropā ierīkoja elektrisko tramvaju.

Uzplauka arī otra lielākā Galīcijas pilsēta~--- Krakova. 1869.~gadā Krakovā Vavelas (\pltxti{Wawel}) pils katedrālē tika restaurētas poļu karaļu kapenes, 1890.~gadā šeit svinīgi no Francijas tika pārapbedīts dzejnieks Ā.~Mickēvičs. 1905.~gadā, kad austriešu militāristi atstāja Vavelas pili, sākās tās atjaunošanas darbi. Krakovā tika sarīkoti nacionālo varoņu T.~Kostjuško un J.~Poņatovska atceres svētki, 1910.~gadā uzvaras Grīnvaldes kaujā 500~gadu jubilejai par godu par pianista un politiķa J.~Paderevska līdzekļiem tika uzstādīts karaļa Vladislava II Jagello piemineklis (1939.~gadā to iznīcināja vācu nacisti, atjaunots tika 1975.~gadā). Krakovā 1871.~gadā tika nodibināta Zinību Akadēmija (\pltxti{Akademia Umiejętności}), 1873.~gadā~--- Mākslas akadēmija. Tas viss pārvērta Krakovu par ``svētceļojumu'' vietu, īpaši poļiem no Pozenes provinces un Polijas karalistes.

Poļu ietekme Galīcijā bija nesalīdzināmi spēcīgāka nekā rutēņu. Poļu muižniecība ar grāfu A.~Goluhovski priekšgalā enerģiski cīnījās par to, lai visās dzīves jomās nodrošinātu poļu prioritāti. Poļu muižniecība un pilsonība atteicās atzīt ukraiņu valodas vienlīdzību ar poļu valodu Seimā. A.~Goluhovskis sākotnēji bija Galīcijas vietvalža padomnieks, pēc tam pats vairākkārt ieņēma šo posteni. 1871.~gadā Austroungārijas valdībā tika nodibināts īpašs Galīcijas lietu ministra amats, kurš līdz Austroungārijas sabrukumam parasti bija polis. Poļi varēja iegūt arī citus ministru posteņus. Ministru un Galīcijas vietvaldi imperators nozīmēja pēc saskaņošanas ar poļu deputātiem no t.s. ``poļu kolo'' (``\pltxti{Koło Polskie}''~--- poļu loks) Austrijas parlamentā. XIX gadsimta pēdējā trešdaļā Galīcijas poļu politiķi ieguva samērā lielu ietekmi Vīnē, kur 12 ministri un 2 ministru prezidenti nāca no Galīcijas. Var teikt, ka poļi Galīcijā baudīja priviliģētu stāvokli pretstatā ģermanizācijai Vācijā un carisma jūgam Krievijā. Lai aizsargātos no šiem ļaunumiem poļu virsslāņi atbalstīja Habsburgu monarhiju. Taču, kā norādījis angļu vēsturnieks A.~Dž.~P.~Teilors, tai pašā laikā viņi uzlūkoja Galīciju kā poļu valsts paraugu kādā attālā nākotnē un vēlējās saglabāt Habsburgu monarhiju tādā formā, lai Galīciju varētu atšķelt no tās jebkurā brīdī.

Galīcijas autonomija radīja poļiem vislabvēlīgākos apstākļus kopš tiem laikiem, kad eksistēja konstitucionālā Polijas-Lietuvas valsts. Austroungārijas valdošie slāņi faktiski tā atlīdzināja poļu muižniecībai par atbalstu Habsburgu monarhijai un uzstāšanos pret citu nacionālo grupu autonomijas prasībām. Poļi līdz pat impērijas sabrukumam palika pats uzticamākais un nelokāmākais tās elements. Šādas politikas būtību saskatīja arī padomju diktators un nacionālo lietu speciālists J.~Staļins. Viņš rakstīja: ``Austrieši pietuvināja sev poļus, deva tiem privilēģijas, lai poļi palīdzētu austriešiem stiprināt savas pozīcijas Polijā, un par to deva poļiem iespēju žņaugt Galīciju. Tā ir īpaša, tīri austriešu sistēma~--- izcelt dažas tautības un dot tām privilēģijas, lai pēc tam tiktu galā ar pārējām''.

Tādejādi Galīcija kļuva par vienīgo bijušās Polijas daļu, kas ieguva daļēju politisku un kulturālu brīvību, kā rezultātā Lemberga un Krakova kļuva par galvenajiem poļu politiskajiem un kultūras centriem.

Toties Galīcijas poļu zemnieku vidū nacionālās pašapziņas veidošanās norisa īpaši grūti. Autonomajā Galīcijā vara atradās poļu muižnieku rokās, kuri gan gribētu iegūt zemnieku atbalstu nacionālajā jautājumā, taču tikai ne ar politisku vai ekonomisku piekāpšanos. Konservatīvo muižnieku un buržuāzijas elementu nostiprināšanās pie varas noveda pie tā, ka Galīcijas lauki palika ``sprādziena bīstami'' kā sociālajā, tā nacionālajā aspektā. Ekonomiskā stagnācija Galīcijā veicināja sociālo attiecību arhaisku formu saglabāšanos, kas savukārt vājināja visu vietējo poļu tautības iedzīvotāju kopību. Tiesa, daļa proletarizēto vai pusproletarizēto zemnieku meklēja darbu Polijas karalistē, Pozenes lielhercogistē, arī Silēzijā~--- visur, kur attīstījās rūpniecības uzņēmumi, tā veidojot vispolijas darbaspēka tirgu, un līdz ar to veicinot dažādo poļu apgabalu iedzīvotāju nacionālo kopību.

Lembergā attīstījās arī ukraiņu kultūra, daudzās jomās pilsēta bija paraugs pārējām ukraiņu apdzīvotajām teritorijām. Rutēņu kustībā konkurēja divi strāvojumi. To piekritējus varētu saukt par ``ukrainofīliem'' un ``maskavafīliem''. Pirmais bija lojāls Habsburgu monarhijai, otrs orientējās uz Krieviju. 70.--80.~gados rutēņu aktīvisti, tā saucamie ``tautībnieki'' (\rutxti{народовцы}) arvien vairāk orientējās uz kopīgas ukraiņu nācijas attīstību. Noraidot kompromisu ar Galīcijas poļiem, viņi arī nevarēja pieņemt Krievijas orientācijas piekritēju uzskatus, kuri rutēņus uzskatīja par krievu tautas sastāvdaļu. Kā raksta angļu vēsturnieks N.~Deiviss, XIX~gs. otrajā pusē rutēņi sāka sevi saukt par ukraiņiem, lai izvairītos no maldinošā un aizskarošā apzīmējuma ``mazkrievs''. ``Ukrainis'' bija vienkārši politiski apzinīgs rutēnis. Ja nomaina vārdu ``politiski'' ar ``nacionāli'', tad konstatējumam var piekrist.

XX gadsimta sākumā vēsturnieks S.~Tomašivskis uzsvēra, ka Galīcijā pēc 1866.~gada poļi, palikdami nacionāls mazākums Austroungārijas valstī, bet iegūdami gandrīz neierobežotu varu pār apgabala vēsturiskajiem pamatiedzīvotājiem~--- ukraiņiem, kļuva par šķērsli ukraiņu nacionālajai kustībai, kas noveda pie pastāvīgām cīņām abu tautu starpā. Krievijas propaganda, vērsta uz vietējo rutēņu piesaistīšanu Krievijas ietekmei, lika poļiem, izvēloties starp rusofīlo un ukrainisko virzienu rutēņu attīstībā, atbalstīt otro. Taču ārpus valodas jautājuma poļi nevēlējās atbalstīt ukrainiskos elementus. Pēc nosauktā vēsturnieka domām, ukraiņu vairākums Galīcijā pirms Pirmā pasaules kara ilgstošas cīņas rezultātā vietējā pārvaldē bija ieguvuši ietekmi, ko salīdzinājumā ar poļu ietekmi izteica proporcija 1:4, taču, piemēram, ukraiņu vidusskolu bija tikai 1/10 no poļu vidusskolu skaita. Ukraiņu valodas tiesības sabiedriskās dzīves jomā joprojām bija ierobežotas, tā tika vienīgi pieciesta. Austroungārijas varas iestādes pretojās ukraiņu inteliģences centieniem vecā vietējo iedzīvotāju nosaukuma ``rutēņi'' nomaiņai ar ``ukraiņi''. Veco terminu mākslīgi uzturēja arī skolas. Savukārt saprotams, ka Krievija Galīcijas polonizāciju uzņēma kā visai netīkamu, vēl jo vairāk tāpēc, ka tā sekoja īsi pēc Polijas karalistes autonomijas likvidēšanas.

Neatkarīgas ukraiņu valsts radīšanas ideja apdraudēja kā Romanovu, tā Habsburgu impēriju vienotību, taču otrajai tā bija mazāk bīstama, jo tās sastāvā bija daudz mazāk ukraiņu apdzīvotu zemju nekā Krievijā. Galīcijā konflikts starp rutēņiem un poļiem te saasinājās, te atkal pieklusa. Atbalstot te vienu, te otru pusi, Austroungārijas valdība radīja provincē zināmu līdzsvaru. Rutēņu bērni varēja sākumskolās mācīties dzimtajā valodā. Šeit tika izdota literatūra ukraiņu valodā (Krievijā tā bija ierobežota). Lembergā strādāja vairāki ievērojami ukraiņu sabiedriskie darbinieki. Piemēram, šurp uz laiku pārvācās vēsturnieks, folklorists, publicists M.~Dragomanovs. Lembergas universitātē ilgstoši strādāja viens no ukraiņu nacionālās kustības līderiem vēsturnieks, profesors M.~Gruševskis. Viņš apgalvoja, ka Galīcija ir ``ukraiņu tautas avangards, kurš jau sen ir apdzinis nabago Krievijas Ukrainu'', ka ``līdz šim Galīcija gāja, bet Ukraina stāvēja vai sekoja Galīcijai''.

Vairāk nekā 10\% iedzīvotāju Galīcijā sastādīja \strong{ebreji}. Viņu tirgotāji un amatnieki veidoja vidusslāni starp lielajiem zemes īpašniekiem (galvenokārt~--- poļiem) un nabagajiem zemniekiem (galvenokārt rutēņiem). Ebreji šeit, tāpat kā citās Polijas daļās, pelnīja iztiku galvenokārt divējādi. Pirmkārt, viņu darbojās kā starpnieki starp muižniekiem~--- zemes īpašniekiem un zemniekiem, pildīja muižu pārvaldnieku, krodzinieku uzdevumus, kas krīžu periodos izraisīja pret viņiem dusmas no abām pusēm. Otrkārt, mazpilsētiņās viņi centās nodrošināt savu eksistenci ar sīktirdzniecību un amatniecību. Ebreju vairākums Galīcijā dzīvoja pilsētās vai arī savos miestiņos (jidišā \pltxti{Štetl}). 1880.~gadā ar vairum-, mazum- un ārējo tirdzniecību nodarbināto tirgotāju vidū ebreji sastādīja 84,5\%.

No poļu nacionālisma viedokļa ebreji bija jāizstumj no savām saimnieciskajām un sabiedriskajām pozīcijām. Saimniecībai bija jākļūst ``poliskai'' un jāpanāk attīstītākās valsts daļas industrializācijas gaitā, ko Galīcija kā Austrijas un Austroungārijas \latxti{de facto} lauksaimnieciska kolonija līdz šim bija ``nogulējusi''. Poļu muižniecība pati dibināja saimnieciskas sabiedrības (\pltxti{kólka rolnicze}) un, lai izstumtu no tautsaimniecības ebreju tirgotājus, manufaktūru īpašniekus un amatniekus, atbalstīja citu kārtu poļus, kuri vēlējās nodibināt savus uzņēmumus. Varas iestādes sistemātiski darbojās pret ebreju interesēm, atbalstīja antisemītisku aģitāciju. Katoļu baznīca pieļāva, ka atdzima vecas leģendas par it kā ebreju veiktām rituālajām slepkavībām. Jau no 1871.~gada skanēja aicinājumi boikotēt ebrejus, bet 90.~gados jau vairojās vardarbīgi uzbrukumi ebrejiem. Tas viss radīja grautiņiem labvēlīgu noskaņojumu sabiedrībā un virzīja ebrejus uz domām par emigrēšanas nepieciešamību. Pēc Polijas sadalīšanas ebrejiem pavērās iespēja migrēt uz līdz tam viņiem neaizsniedzamiem apgabaliem. Galīcijas ebreji, kas kļuva par Austrijas pavalstniekiem, sāka pārcelties uz citām Habsburgu monarhijas provincēm, bet vēlāk arī tās galvaspilsētu Vīni. Tomēr uz vietas palikušie Galīcijas ebreji saglabāja savu nacionālo identitāti.

\strong{Politisko partiju veidošanos Galīcijā} traucēja tās nošķirtība no pārējiem poļu apdzīvotajiem apgabaliem, tomēr līdz XX gs. sākumam arī šeit radās poļu politiskās partijas: sociālistiskā un zemnieku.

Jau 80.~gados šeit pastāvēja sociāldemokrātiskas grupas. 1890.~gadā no dažiem strādnieku pulciņiem izveidojās Galīcijas strādnieku partija, (``\pltxti{Galicyjska Partia Robotnicza}''). 1892.~gadā tā pārveidojās par Galīcijas sociāldemokrātiskā partiju (\pltxti{Galicyjska Partia Socjaldemokratyczna}, arī \pltxti{Socjaldemokratyczną Partią Galicji}). Partija formāli darbojās kā vienotās Austrijas sociāldemokrātiskās partijas sastāvdaļa.

1896.~gadā strādnieku un zemnieku kustības pacēluma apstākļos Austroungārijas valdība nolēma ieviest valstī jaunu vēlēšanu sistēmu. Jau esošajām četrām pēc mantas cenza sadalītajām vēlēšanu kūrijām tika pievienota piektā, uz vispārējo vēlēšanu tiesību pamata dibināta. Tajā ievēlēja 72~deputātus, 15~vietas piešķirot Galīcijai. Taču tas nenozīmēja tālejošu demokratizāciju, jo 1~deputātu pirmajā (muižnieku) kūrijā ievēlēja 63~vēlētāji, bet piektajā~--- ap 70~tūkstošu vēlētāju. Tomēr rezultātā poļu sociāldemokrātu pārstāvji I.~Dašinskis un J.~Kozakevičs pirmo reizi tika ievēlēti Austrijas parlamentā.

1897.~gadā Austrijas sociāldemokrātiskās partijas kongress pārveidoja partiju par sešu nacionālo partiju federāciju. Katra no tām tika veidota nevis pēc teritoriālā, bet nacionālā principa. Daudzi marksisti uzskatīja to par strādnieku šķelšanu. 1897.~gadā Galīcijas strādnieku partija arī pieņēma Galīcijas un Cešinas Silēzijas poļu sociāldemokrātiskās partijas (\pltxti{Polska Partia Socjalno-Demokratyczna Galicji i Śląska Cieszyńskiego}) nosaukumu. Turpmāk partija panāca vairāku savu pārstāvju ievēlēšanu Galīcijas Seimā un Austrijas parlamentā. Ievērojamākais partijas līderis bija minētais I.~Dašinskis.

Attīstījās \strong{zemnieku kustība}. Kapitālistiskās ražošanas attīstības lauksaimniecībā apstākļos sīko un vidējo zemes īpašnieku masas īpaši sāpīgi izjuta tradicionālās dzīves straujās pārvērtības, sava sociālā stāvokļa nestabilitāti, savas nākotnes nedrošību. Tāpat kā feodālisma laikmetā zemniecība palika pats lielākais iedzīvotāju slānis, kurš cieta no zemes trūkuma, smagas ekspluatācijas, bieži pirmskapitālistiskās formās. Krasā sociālā nevienlīdzība ar citiem slāņiem noveda pie tā, ka neraugoties uz citu partiju centieniem piesaistīt zemniecību sev sabiedrības politiskās organizācijas sistēmā radās īpašs virziens. Tas uzstājās pret feodālisma paliekām, par labvēlīgu nosacījumu radīšanu lauksaimniecības attīstībai, sociālo problēmu atrisināšanu par labu lauku sīkražotājiem, par zemnieku pielīdzināšanu tiesībās citiem slāņiem. Pati dzīve virzīja zemniecību uz apvienošanos organizācijās lai aizstāvētu savas ekonomiskās un politiskās intereses. Ar zemniecību saistītā inteliģence savukārt uzsāka kustības ideoloģisko pamatu izstrādi.

1886.~gadā Pēterburgā inženiera izglītību ieguvušais poļu šļahtičs B.~Vislouhs publicēja rakstu sēriju, kurā viņš izteica viedokli, ka politiskai organizācijai, kas vēlas panākt Polijas neatkarību, jābalstās uz darba cilvēkiem, pirmkārt~--- zemniekiem, kuri sastāda nācijas vairākumu. Nākotnes neatkarīgo Poliju viņš saskatīja etniskajās robežās, atzīstot ukraiņu, baltkrievu un lietuviešu tautu tiesības uz valstiskumu, izteicās par poļu un ukraiņu zemniecības sadarbību cīņā pret muižniekiem un veidojošos buržuāziju. B.~Vislouha programma bija virzīta uz feodālo palieku likvidāciju un par mērķi stādīja zemniecības politisku aktivizēšanu.

1888.~gadā Galīcijā izveidojās zemnieku grupa, kura darbojās ``kristīgā sociālisma'' garā. Jau radikālākās pozīcijās stāvēja no 1889.~gada Lembergā B.~Vislouha izdotā avīze ``\pltxti{Przyjaciel ludu}'', (``Tautas draugs''). Ap avīzi grupējās cilvēki, kuri centās piešķirt zemnieku kustībai visas tautas kustības raksturu. Tā ilgstoši varēja izplatīties tikai Galīcijā. (Īpaši Polijas karalistē zemnieku organizācijām praktiski bija liegtas iespējas darboties legāli.) Galīcijas Seima vēlēšanās 1889.~gada jūlijā 18~povjatos (poļu \pltxti{powiat}~--- vidēja lieluma administratīvi teritoriāla vienība, līdzīga apriņķim) radās zemnieku vēlēšanu komitejas, kuras ar inteliģences demokrātisko pārstāvju atbalstu panāca četru zemnieku deputātu ievēlēšanu, kuru darbībai tajā gan nebija reālas nozīmes.

1895.~gadā Rešuvā (poļu \pltxti{Rzeszow}, krievu \rutxti{Жешув}, ukraiņu \uktxti{Ряшiв}, vācu \detxti{Reichshof}) Galīcijā tika izveidota zemnieku partija ``\pltxti{Stronnictwo Ludowe}'' (No poļu \pltxti{lud}~--- tauta, zemniecība,~--- Zemnieku jeb Tautas partija). Tās piekritējus tautā sauca par \strong{``ļudoviešiem''}. (Tiesa, lietojot vārdu ``partija'', jāsaprot zināma īpatnība. Programmas dokumentos, zinātniskajā literatūrā, publicistikā šis vārds tiek lietots. Taču Polijā izplatītais nosaukums ``\pltxti{stronnictwo}'' burtiski nozīmē ``piekritēju savienība''. Tieši šis apzīmējums precīzāk izsaka zemnieku tā laika organizētības pakāpi, ja to, piemēram, salīdzina ar revolucionārajām strādnieku partijām.) Par tās priekšsēdētāju kļuva advokāts K.~Ļevakovskis, kurš gan pēc diviem gadiem atteicās no priekšsēdētāja amata. Partijas vadībā darbojās arī B.~Vislouhs, H.~Revakovičs, J.~Stapiņskis u.c.

Ļudovieši pārstāvēja nacionāli orientētos apzinīgākos lauku iedzīvotāju, galvenokārt sīkzemnieku, slāņus. Partijas galvenās prasības bija neatkarīgas Polijas valsts atjaunošana; Galīcijas autonomijas panākšana; vēlēšanu likumu maiņa parlamentā Vīnē un Galīcijas Seimā, lai radītu demokrātiskāku parlamentāro sistēmu; vienlīdzīgas tiesības zemniecībai, vispārējo vēlēšanu tiesību ieviešana un sabiedriskās dzīves demokratizācija; plaša zemniecības pārstāvniecība likumdevējās iestādēs; agrārās reformas īstenošana un zemnieku saimniecību skaita palielināšana. Ļudovieši centās likvidēt muižnieku privilēģijas ekonomiskajā, politiskajā un sabiedriskajā dzīvē. Seima vēlēšanās apgabalos ar jauktu~--- poļu un rutēņu iedzīvotāju sastāvu partija uzstājās pret šļahtas kandidātiem. 1895.~gada Seima vēlēšanās tajā ievēlēja 9~ļudoviešus, no tiem 7~zemniekus. Tā kā konservatoriem, muižniecībai bija cieši sakari ar baznīcu, ļudovieši drīz nonāca konfliktā ar to. Baznīcai piederēja ievērojamas zemes platības. Tā ļudoviešu parādīšanos zemnieku vidē uzskatīja par apdraudējumu savām monopoltiesībām uz ticīgo poļu zemnieku prātu kontroli, kā ksendza tiesību ierobežošanu. No kanceles pār ļudoviešiem sāka līt apvainojumi, taču viņi, aizstāvot tiesības uz apziņas brīvību, pasvītroja savu dziļo cieņu pret baznīcu, pilnīgu atbalstu kristīgajai morālei, taču noraidīja baznīcas iejaukšanos valsts politiskajā dzīvē. Kustības kreisais spārns prasīja baznīcas atdalīšanu no valsts. Praktiskajā darbībā ļudovieši centās nejaukt politiku ar reliģiju.

Ļudoviešu idejiskās pozīcijas pamatā bija Eiropas pilsoniskās un sīkpilsoniskajā kustībā XIX~gadsimtā visai plaši izplatītā \strong{agrārisma teorija}. Tā radās kā atbildes meklējumi uz pārdzīvojamajām tirgus saimniecības grūtībām. Teorijas pamatā atradās zemniecības garīgo, morālo, kulturālo un sociālo īpatnību un tās darbības atzīšana par unikālu un nepārejošu vērtību visas cilvēces attīstībā. Rūpniecības preču pārprodukcija, rūpnīcu slēgšana, masu bezdarbs daļā ekonomistu radīja uzskatu, ka lielrūpniecības attīstības iespējas ir izsmeltas, toties lauksaimniecība, ar valsts protekcionisma (no latīņu \latxti{protectio}~--- aizbildniecība; iekšējā tirgus aizsardzības sistēma pret ārzemju konkurenci ar augstu muitas nodevu u.c. līdzekļu palīdzību) politikas atbalstu var cīnīties ar ekonomiskajiem satricinājumiem.

Nometinot ``liekos'' strādniekus laukos, pārvēršot viņu par sīkzemniekiem, kuri apstrādā savu lauciņu, pavērās cerība ``nolīdzināt'' krīžu gaitu, novērst sociālo satricinājumu cēloņus. Agrārisma teorētiķi lika priekšā veicināt lauksaimniecības attīstību ar lētu kredītu piešķiršanu, nodokļu samazināšanu, ``cenu šķēru'' uz rūpniecības un lauksaimniecības preču likvidēšanu, ``kooperatīvu asociāciju'' izveidi. Šos agrārisma postulātus piemērojot dažādu valstu apstākļiem, zemnieku kustībā tika izstrādātas dažādas pastāvošās iekārtas uzlabošanas koncepcijas sīkīpašnieku un sīkražotāju labā. Rekonstruējot pastāvošo agrāro iekārtu un dažādiem līdzekļiem attīstot lauksaimniecību, zemnieku partijām parlamentārā ceļa iekarojot varu, tika plānota pakāpeniska, bez satricinājumiem pāreja no pastāvošā kapitālisma uz ``zemnieku kooperatīvo republiku''.

Vēlreiz jāuzsver, ka ka ārpus Austroungārijas citās poļu zemēs zemnieku kustība XIX gadsimtā nespēja iegūt stingrus organizatoriskus pamatus. Polijas karalistes un Pozenes novada teritorijā ļudoviešu kustība plašāk izvērsās tikai pēc Polijas neatkarības atgūšanas 1918.~gadā.

\subsection{Poznaņa Prūsijas / Vācijas sastāvā}

Pēc Vīnes kongresa radītā Pozenes province (vācu \detxti{Provinz Posen}, poļu \pltxti{Prowincja Poznańska),} aptvēra tās Polijas daļas, ko Prūsija bija jau saņēmusi 2. un 3. Žečpospolitas dalīšanas reizē. Šīs agrākās Polijas zemes kādreiz veidoja t.s. ``Lielās Polijas'' (\pltxti{Wielkopolska}) kodolu, tur X gadsimtā Gņezno (poļu~--- \pltxti{Gniezno}, vācu~--- \detxti{Gnesen}) bija poļu galvaspilsēta, līdz XIV gadsimtam tur kronēja karaļus, no XI~gadsimta tur atradās arhibīskapa sēdeklis. (Zemes, ko Prūsija bija sev pievienojusi jau agrāk~--- Rietumprūsija, Pomerānija u.c., bija tieši iekļautas Prūsijas sastāvā.)

Pozenes zeme, kas kā tilts savienoja Austrumprūsiju ar Pomerāniju un Silēziju, faktiski tika iekļauta Prūsijā kā autonoma province un līdz 1866.~gadam neietilpa Vācu savienībā. Tā sākotnēji gan tika dēvēta par Pozenes lielhercogisti (vācu \detxti{Großherzogtum Posen,} poļu \pltxti{Wielkie Księstwo Poznańskie}), taču faktiski province atradās Prūsijas karalistes sastāvā. Prūsijas karalim Pozenes lielhercoga tituls bija vajadzīgs vienīgi tāpēc, lai to pretstatītu Polijas karaļa titulam, kuru pieņēma Krievijas imperators. Ar 1848.~gada revolūciju tika atmests arī lielhercogistes nosaukums. Tāpat kā Rietumprūsijā, arī Pozenes hercogistē, īpaši tās rietumu daļā, vēl no Teitoņu ordeņa laikiem saglabājās liels skaits vācu iedzīvotāju. Parasti Rietumprūsiju un Pozenes zemi, neraugoties uz to atšķirīgo statusu, minēja reizē, tāpēc XIX gadsimta otrajā pusē tās sāka saukt par ``\detxti{Ostmark}'' (Austrumu apgabals).

Pozenes lielhercogistē dzīvoja ap 0,9 miljoniem cilvēku, pēc citiem datiem~--- ap 521~000 katolisku poļu, ap 218~000 lielākoties protestantisku vāciešu (pēc konfesionālās piederības Pozenes vāciešu vidū dominēja luterāņi, kaut pastāvēja arī maza grupiņa t.s. reformātu-kalvinistu), un ap 50~000 ebreju. Tajos novados, kas robežojās ar Krievijai piederošo Polijas karalisti, poļu daļa iedzīvotāju vidū sniedzās līdz 85\%, bet rietumu novados bija vācu vairākums. Kopumā provincē varēja runāt par attiecību 1:2 par labu poļiem.

XIX~gadsimta sākumā nevar runāt par kādu nacionālu naidu starp vietējiem vāciešiem un poļiem. Abas tautības nebija skaidri nošķirtas, eksistēja daudz polonizētu vāciešu un ģermanizētu poļu. (Ebreji kopumā labprātāk sliecās uz vācu nekā poļu ietekmes pieņemšanu.) Par savstarpējiem kontaktiem liecināja jauktās laulības. XIX~gadsimta pirmajā pusē tās sasniedza 20\% no visām. Jāatzīmē gan, ka poļi biežākus kontaktus uzturēja ar vācu katoļiem nekā ar vācu protestantiem un otrādi~--- vācieši labprātāk kontaktējās ar poļu protestantiem nekā katoļiem. Savstarpējā tuvināšanās veda pie asimilācijas, sākotnēji vairāk polonizējās vācieši, pat prūšu ierēdņi.

Sākotnēji Pozenē nacionālo attiecību laukā valdīja ``klusie gadi'', jo Prūsija 1815.~gadā bija ieguvusi arī Vestfāli un Reinzemi, kur arī bija jāorganizē jauna pārvalde, tāpēc nevarēja visu uzmanību veltīt poļu apgabaliem. Bez tam Prūsijai bija jārēķinās arī ar pārējo Poliju sadalījušo valstu realizēto politiku, īpaši Krievijas politiku Pozenes novadam blakus esošajā Polijas karalistē. Tur Aleksandrs I bija atļāvis nosacīti liberālas Konstitūcijas pieņemšanu (par to sīkāk stāstīts nākamajā apakšnodaļā), karalistes pavalstnieki baudīja zināmu autonomiju, tikai ārlietas bija stingri Krievijas imperatora noteiktas. Prūsija neuzdrošinājās Pozenes poļu iedzīvotājiem radīt sliktākus apstākļus nekā Polijas karalistē, jo tad vajadzētu rēķināties ar viņu simpātijām Krievijai.

Prūsijas toreizējais karalis Fridrihs Vilhelms III solīja jaunajiem pavalstniekiem sargāt ne tikai viņu reliģiju, bet arī valodu, nodrošināt vietējiem iedzīvotājiem pieejamību amatiem pārvaldē. Viņa uzsaukums poļiem skanēja: ``Jūs tiksiet iekļauti manā monarhijā tā, lai nevajadzētu noliegt savu tautību''. Šādu attieksmi noteica arī tas, ka trūka prūšu ierēdņu, kuri zinātu arī poļu valodu. Pozenes lielhercogistē darbojās pašvaldība: landtāgs. Tajā 24 (vēlāk~--- 26) vietas piederēja muižniekiem, 16~--- pilsētām un 8~--- lauku kopienām. Landtāgā bija poļu vairākums un viņi to izmantoja, lai aizstāvētu savas tiesības pret vēlākajiem valdnieku un ierēdņu tīkojumiem veikt ģermanizāciju. Landtāga sēdēs tika izmantota gan vācu, gan poļu valoda. Tās abas izmantoja arī pārvaldē, tiesu un izglītības iestādēs. Prūsija pat atļāva izveidot Pozenes korpusu Prūsijas armijas sastāvā ar poļu virsniekiem un poļu komandu valodu.

1815.~gadā par Prūsijas karaļa vietvaldi Pozenē kļuva poļu un prūšu politiķis, zemes lielīpašnieks un komponists firsts (kņazs) A.~Radzivills. Par savu uzdevumu viņš uzskatīja samierināt poļus ar Prūsiju, kaut Pozenes poļi pirmkārt centās iegūt pašpārvaldi. Viņa karjera gan beidzās ar sacelšanos Polijas karalistē (1830), kur viens no vadītājiem bija viņa brālis Mihals. Prūsijas valdībā kā izglītības ministrs darbojās liberāls un tālredzīgs politiķis K.~Altenšteins, kurš 1822.~gada decembrī, atbildot uz Pozenes ierēdņu jautājumu par vācu valodas ieviešanu, sacīja: ``Protams, valdība ir ieinteresēta, lai poļu iedzīvotāji zinātu vācu valodu un varētu izmantot valsts iestāžu pakalpojumus; taču tas nebūt nenozīmē, ka poļus ir jāpārvāco \citespace{} Jo kā cilvēka, tā arī tautas audzināšanu var nostiprināt tikai dzimtā valoda: atņemt to \citespace{} un censties mākslīgi iepotēt svešu valodu, tāds ceļš ir nepareizs pat attiecībā pret atsevišķu cilvēku, nemaz nerunājot par veselu tautu, pat ja tai nebūtu tik bagātas, tik izstrādāts un nobeigtas valodas, kāda ir poļu [valoda] \citespace{} Vara, kura atzīst, ciena un aizstāv tautas reliģiju un tās valodu, divus lielākos tās svētumus, var būt pārliecināta, ka piesaistīs sev pavalstnieku sirdis. Taču ja vara ir vienaldzīga pret tiem, vai vēl vairāk~--- uzliek tiem savu roku, tā apvaino un uzbudina tautu un reizē ar to gatavo sev neuzticamus un saniknotus pavalstniekus''. Diemžēl šo dziļo patiesību dažādu tautību šovinisti nav sapratuši līdz pat mūsdienām.

Neatbilst patiesībai J.~Staļina valdīšanas periodā tapušajā padomju vēstures literatūrā propagandētais apgalvojums, ka visvairāk cieta tie Polijas iedzīvotāji, kuri pēc valsts sadalīšanas nonāca Prūsijas varā. Prūsijai piederošajos apgabalos ekonomika attīstījās sekmīgi, īpaši jau pirms Žečpospolitas sadales Prūsijas rokās nonākušajā Silēzijā, lauksaimniecība arī Pozenē attīstījās raitāk nekā Krievijai piederošajās poļu zemēs. Prūsijā poļu iespējas iegūt izglītību savā valodā XIX gadsimta pirmajā pusē bija krietni labākas kā Krievijas vai Austrijas teritorijās. Ja 1815.~gadā Pozenes provincē bija 164 (citi dati~--- 540) skolas, tad 1845.~gadā to jau bija 1~142. Sākotnēji tikai dažas vācu politiskās grupas iebilda pret samierināšanās kursu un vēlējās realizēt poļu asimilāciju pēc Rietumprūsijas parauga. Tās virsroku sāka gūt tikai pēc poļu sacelšanās 1830.~gadā, kura ietekmēja situāciju arī Pozenē.

Tikai turpmāk arvien uzskatāmāk parādījās vācu un poļu interešu pretstati. Sākotnējo diezgan plašo Pozenes iekšējo autonomiju Prūsija pakāpeniski sašaurināja. Jau no XIX gadsimta 20.~gadu sākuma hercogistes autonomās iestādes sāka darboties kā Prūsijas administrācijas turpinājums, amatos tajās nozīmēja tikai etniskos vāciešus. Dažus gadus pēc Vīnes kongresa tika precizēta provinces pārvaldes sistēma. 1819.~gadā tika radīts Pozenes pārvaldes iecirknis ar 17 apgabaliem (\detxti{Kreis}) un Brombergas (\pltxti{Bydgoszcz}) iecirknis ar 9 apgabaliem. Tāds iedalījums pastāvēja līdz pat 1887.~gadam, kad Prūsijas Ministru padomes priekšsēdētājs un Vācijas impērijas kanclers O.~f.~Bismarks reorganizēja provinces. Līdz Pirmajam pasaules karam Pozenes province bija sadalīta Pozenes iecirknī ar 28 apgabaliem un Brombergas iecirknī ar 14 apgabaliem. Gnezenes arhibīskapija tika pārveidota par Gnezenes-Pozenes arhibīskapiju, kur pats arhibīskaps rezidēja Pozenē. Luterāņu baznīcas priekšgalā atradās Konsistorija ar superintendantu.

Prūsijā jau 1807.~gadā bija oficiāli atcelta zemnieku personīgā dzimtbūtnieciskā atkarība, 1811.~gadā zemniekiem tika atļauts izpirkt savas feodālās saistības, samaksājot 25-kārtīgu to gada apjomu vai atsakoties no 1/3 līdz 1/2 savas apstrādājamās zemes. Taču vēl ilgi zemnieki neieguva līdztiesību, gadu desmitiem saglabājās izpirkuma maksājumi. 1823.~gadā šo kārtību attiecināja arī uz Pozenes lielhercogisti. Arī tur notika agrārā reforma, turīgie zemnieki varēja iegūt privātīpašumā daļu apstrādājamās zemes, ar atteikšanos no otras daļas izpērkot savas feodālās saistības~--- klaušas un nodevas. Izpirkšanas maksājumus noteica valdības ieceltas speciālas komisijas. Zemes cenas Pozenē bija zemākas nekā pašā Prūsijā, kur varas iestādes vairāk rēķinājās ar muižnieku interesēm. Tādejādi valdība starp zemniekiem un muižniekiem šeit it kā izveidoja zemniekus sargājošu sienu. Kaut reformas notika ļoti lēni, zemes izpirkšana ievilkās līdz 1865.~gadam, zemnieki zaudēja apmēram 1/3 viņu agrāk apstrādājamās zemes, tomēr radās zemnieku~--- zemes īpašnieku slānis, kurš jau XIX gadsimta beigās kļuva par ietekmīgu politisku faktoru. Jāpasvītro, ka poļu zemniecība Pozenē turpmāk ļoti auksti izturējās pret šļahtas sacelšanās mēģinājumiem. Labklājības pieaugums šai zemniecībai vēlāk, kad tā nonāca pie nacionālās pašapziņas, ļāva sekmīgi cīnīties pret ģermanizāciju, sevišķi ekonomiskajā un kultūras jomā. Līdz pat Pirmajam pasaules karam lauksaimniecības attīstībā Pozene bija priekšā citām poļu apdzīvotajām teritorijām Krievijas un Austrijas sastāvā.

Taču reformas bija sāpīgas nabadzīgajiem zemniekiem, kuri nespēja izpirkt savu zemi un pakāpeniski tika no tās padzīti. Gandrīz pilnībā izzuda zemnieku sīksaimniecības. Tiesa, šo padzīšanas procesu nevajadzētu vienkāršot kā izraidīšanu uz četrām debess pusēm. Bieži, zemniekam atstājot savu agrāko saimniecību, viņiem tika ierādīts cita dzīves vieta~--- savs ``stūris'' (\pltxti{kąt}), iedalīts zemes gabals dārzeņu un kartupeļu audzēšanai un radītas citas atkarības formas no muižnieka. Lai nepaliktu bez jumta virs galvas, šādi zemnieki pēc pirmā aicinājuma ieradās darbā muižā. Ja šie cilvēki mēģināja nodarboties ar amatniecību un tirdzniecību, viņi sastapās ar profesionālu amatnieku un tirgotāju konkurenci, kuru vairākums bija vācieši.

Feodālisma apstākļos gadsimtu gaitā zemnieki, neizturējuši ekspluatācijas slogu, bēga no sava muižnieka. Muižnieki turpretī visiem līdzekļiem centās viņus noturēt, atrast bēgušos. Valsts aparāts palīdzēja muižniekiem notvert un atgūt savus dzimtbūšanas atkarībā esošos zemniekus. XIX gadsimta pirmajā pusē situācijas radikāli mainījās: muižnieki padzina daļu nabadzīgāko zemnieku no zemes, bet pēdējie savukārt pretojās, centās saglabāt savu saimniecību. Ir skaidrs, ka muižnieki padzīt zemniekus un pievienot viņu saimniecības saviem laukiem varēja tikai tad, ja bez citiem nosacījumiem (pārpalikušo zemnieku darba intensitātes pieauguma, tehnikas attīstības) bija pārliecināti, ka nepieciešamības gadījumā atradīs pietiekami daudz brīva darbaspēka, kuru varēs algot. Tātad, neraugoties uz feodālisma pastāvēšanu, jau eksistēja, lai arī ierobežots, brīva algota darbaspēka tirgus. Tas bija radies pateicoties zemnieku bēgšanai un klīšanai darba meklējumos. Kad šis darbaspēka tirgus jau bija pietiekami liels, ``liekos'' zemniekus varēja arī padzīt no viņu apstrādājamās zemes, tā savukārt atkal paplašinot algotā darbaspēka rezerves. Kartupeļu ražošanas izplatība deva iespēju iegūt produktu, ar kuru varēja pabarot zemnieku ģimenes no ievērojami mazāka zemes gabala. Tā muižnieki, pat neatņemot zemniekam viņa saimniecību, bet tikai samazinot tās platību, varēja pievienot savai muižai jaunas zemes platības.

No agrārās reformas īstenošanas laika sākās poļu zemnieku proletarizēto grupu izceļošana darba meklējumos uz Rūras ogļu ieguves uzņēmumiem, Reinas-Vestfāles metalurģiskajām un Berlīnes mašīnbūves rūpnīcām. Daudzi devās arī uz strauju attīstības ceļu uzsākušās Augšsilēzijas rūpniecības uzņēmumiem. Tā nežēlīgiem līdzekļiem tika veidota racionāla saimniekošanas sistēma.

Rūpniecības attīstība pašā Pozenē sastapās ar lielām grūtībām, jo bija jākonkurē ar daudz attīstītāko vācu apgabalu rūpniecību. Pozenes province toreiz piederēja pie visvājāk attīstītajiem Prūsijas apgabaliem. Tomēr Prūsijas varā esošajās poļu zemēs dzelzceļu celtniecība norisa straujāk un to garums bija lielāks nekā pārējās Polijas daļās. Taču satiksme tika nodrošināta pirmkārt ar Vācijas teritorijām. Pozenei nebija tiešas dzelzceļa stigas ar Varšavu. 1820.~gados Pozenes vadmala vēl tika izvesta uz Polijas karalisti, taču 30.~gados eksports gandrīz pilnībā apsīka. Konkurence, muitas robeža ar Krieviju noveda pie tā, ka tika slēgtas vairākas manufaktūras. Prūsijas un pēc tam arī Vācijas varas iestādes nebija ieinteresētas rūpniecības attīstībā poļu apdzīvotajos apgabalos, nepretojās rūpnieku, tai skaitā arī vācu, pārbraukšanai uz Polijas karalisti. Ar to Pozenes ekonomika saņēma smagu triecienu, ieguva vienpusīgāku agrāru raksturu. Šūšanas izstrādājumu ražošanā mašīnas sāka izmantot tikai 1859., apavu ražošana~--- 1861.~gadā. Puse no amatniekiem nelietoja algotu darbu. Tomēr nav apstrīdams, ka Prūsijā, bet vēlāk Vācijas impērijas (1871--1918) Prūsijas pavalstī Pozenes novads, neraugoties uz minētajām grūtībām, iedzīvotāju nacionālajām pretrunām, piedzīvoja ievērojamu saimniecisko augšupeju, tajā attīstījās moderns satiksmes tīkls, tika veikta plaša sabiedriskā un privātā celtniecība, ar industrializāciju tika vairākkārt kāpināta novada saimnieciskā ražība.

Kā smags slogs uz poļu iedzīvotājiem gūlās nepieciešamība dot rekrūšus Prūsijas armijai. Prūsijā militāro dienestu pildīja zemnieki un tie pilsētnieki, kuri nodarbojās ar lauksaimniecību. Šī kārtība tika attiecināta arī uz Prūsijā nonākušajām poļu zemēm, kuras agrāk tādu nepazina. Poļu zemnieki nevēlējās dienēt viņiem svešas valsts armijā, kura pie tam bija slavena ar savu nūjas disciplīnu. Pēc pirmās Žečpospolitas Polijas dalīšanas Prūsijas sastāvā nonākušie bēga atpakaļ uz pārpalikušo poļu-lietuviešu valsti. Pēc tās galīgas sadales šīs iespējas zuda.

\asterism

Sociālā ziņā Pozenes vāciešu vidū bija maza muižnieku grupa, jau pamatīgs pilsonības, arī amatnieku un zemnieku, slānis. Visstraujāk auga vācu inteliģences slānis. Īpaši palielinājās vācu ierēdņu skaits. Poļu ierēdņu jau bija maz. 1848.~gadā Pozenē no 700 ierēdņiem tikai 30 bija poļu tautības. Augstskolās izglītoto vāciešu īpatsvars pieauga arī citās nozarēs. Kopā ar ebrejiem vācieši sastādīja advokātu vairākumu. Puse no ārstiem bija vācieši. Auga vācu skolotāju skaits. Protestantu garīdznieki (1870.~gadā to bija 210) gandrīz bez izņēmuma bija vācieši. Katoļu garīdznieku vidū vāciešu bija 10--15\%. Tiesa, līdz 1870.~gadam vācu arhitektu, mākslinieku, literātu un aktieru skaits bija neliels~--- daži desmiti, kas bija izskaidrojams ar niecīgo pieprasījumu pēc mākslas.

1832.~gadā 185 muižas (gandrīz 22\% no visām) atradās vācu rokās. Vācu muižnieku vairākuma īpašumā galvenokārt bija no 200 līdz 1~000 ha lielas muižas. Parasti 1 cilvēkam piederēja 1 muiža, kas atviegloja tās pārvaldi. Vācu muižnieki Pozenes provincē reti piederēja kādai vecai aristokrātiskai dzimtai, kaut bija arī tādas. Piemēram, kopš 1819.~gada 25~641 ha piederēja firstiem \detxti{Thurn und Taxis}. Vairākums vācu muižu īpašnieku nebija dižciltīgie, bet gan agrāk bijuši augsti ierēdņi provinces pārvaldē ar zināšanām lauksaimniecībā. Tāpēc viņu apsaimniekošanas rezultāti parasti bija labāki nekā viņu poļu kaimiņiem. Vācu muižnieki ierīkoja solīdas saimniecības ēkas, centās paaugstināt ražas ar intensīvu saimniekošanu. Viņi pielietoja modernus mēslošanas līdzekļus, ieviesa jaunas kultūras, piemēram, kartupeļus, veica mājlopu selekciju, investēja produkcijas pārstrādē. Protams, bija arī izņēmumi, kad muižas īpašnieki interesējās tikai par ātrāku ienākumu saņemšanu, vai, neraugoties uz valsts palīdzību, nespēja pielāgoties konjunktūrai. Tika veidotas vācu lauksaimniecības biedrības. Pirmā radās 1826.~gadā. 1852.~gadā tās jau apvienojās biedrību savienībā. Līdz 1848.~gadam pastāvēja gan arī kopējas poļu un vācu organizācijas, arī lauksaimniecības biedrības.

Vāciešu vidū Pozenes provincē pats lielākais bija zemnieku slānis. Daļa no viņiem bija šai apgabalā apmetusies vēl pirms Žečpospolitas pirmās dalīšanas~--- XVIII gs. pirmajā pusē. Jaunākā grupa ieceļoja pēc 1815.~gada. Pozenes provincē bija salīdzinoši lēti nopērkama zeme, tāpēc šurp labprāt devās vācu zemnieki. Valdība viņus atbalstīja finansiāli. Brombergas pārvaldes iecirknī vēl pirms 1830.~gada tika ierīkotas 13 vācu kolonijas. Tomēr visu Pozenes zemnieku vidū vāciešu skaits nebija visai liels, sasniedza apmēram 1/3. Viņu ekonomiskais stāvoklis bija krietni labāks nekā poļu zemniekiem. Vācu zemniekiem piederēja pārsvarā vidējas un lielas saimniecības. Tam par cēloni bija kā labvēlīgie pārcelšanās noteikumi, tā pašu vācu pārceļotāju augstākais saimniekošanas līmenis. Ar valsts palīdzību tika veidotas izglītības biedrības, krājkases.

Pozenes provinces rietumu un ziemeļu pierobežas pilsētās dzīvoja ne tikai skaitliski, bet arī ekonomiski spēcīgs vācu pilsonības slānis. Pašā Pozenē vairāk nekā 2/3 pilsētas robežās esošas zemes atradās vāciešu rokās. Pēc īpašuma cenza pasīvās vēlēšanu tiesības bija tikai 115 poļiem un 315 vāciešiem. Taču vācu pilsonības spēks bija tikai relatīvs. Vācu pilsētu iedzīvotāju vidū lielāko daļu veidoja sīkpilsonība. Proletariāta īpatsvars pilsētnieku vidū bija neliels. Pat Pozenes pilsētā XIX gs. pirmajā pusē tas sasniedza augstākais 20\% no visiem iedzīvotājiem.

Poļu attieksme pret vāciešiem bija dažāda, tomēr vairākums juta līdzi mērķim atjaunot neatkarīgu Polijas valsti. Kad 1830.~gadā (un arī vēlāk 1863.~gadā) Krievijas piederošajā Polijas daļā sākās sacelšanās, daudzi poļu tautības Prūsijas pavalstnieki tajā piedalījās. Ap 12 tūkstošu Prūsijas poļu iestājās sacēlušos rindās Polijas karalistē. (Ir arī citi dati~--- ap 2~500 Prūsijas poļu piedalījušies cīņā pret Krievijas armiju.) Tie pārsvarā bija jauni cilvēki~--- amatnieku zeļļi, ģimnāzisti, dienas strādnieki. Sakauti viņi atkāpās uz Prūsijas teritoriju, caur Pozeni bēga uz Rietumiem. Taču jau 1830.~gada septembrī Pozenes ielās varēja atrast skrejlapas un plakātus ar draudiem vāciešiem. Tika atklāta gatavošanās uz poļu sacelšanos, kas, protams, nevarēja neizraisīt varas iestāžu pretsoļus.

Kaut Pozenes vāciešu lielākā daļa pret 1830.~gada poļu sacelšanos izturējās vienaldzīgi, prūšu ierēdņu acīs poļi vairs nebija ``uzticami pavalstnieki'' kā tas tika pausts agrāk, jo bija pierādījies, ka vietējo poļu mērķis nav uz mūžīgiem laikiem iekļauties Prūsijas sastāvā, bet viņi vēlas atjaunot Polijas valsti. Tas lika arī Prūsijai pievilkt grožus stingrāk. Savukārt Pozenes vāciešu pārstāvji pat devās kā brīvprātīgie uz Polijas karalisti, lai piedalītos poļu sacelšanās apspiešanā. 1830.~gada sacelšanās, tāpat kā Krievijā, arī Pozenē ievadīja izmaiņas t.s. ``poļu politikā'', pastiprinājās ģermanizācija. Salīdzinājumā ar Polijas karalisti, kur rusifikācija notika pakāpeniski, Prūsijā poļu administratīvās un izglītības iestādes tika likvidētas visai ātri. Kā norādījis vācu vēsturnieks M.~Aleksanders, iekļaušana vācu pārvaldes un tautsaimniecības sistēmā nesa vietējiem poļiem ekonomiskas priekšrocības, taču tas maz ietekmēja viņu apziņu, jo tiem nācās neatlaidīgi cīnīties par savas valodas un reliģijas saglabāšanu. Laikabiedriem nemanot Prūsija izaudzināja sev ienaidniekus, kuru nākamās paaudzes jau pēc Polijas valstiskuma atjaunošanas tos pašus ``nacionālos'' ieročus vērsa pret vāciešiem.

Par Pozenes virsprezidentu tika nozīmēts E.~Flotvels. Viņa amata laiku: 1830.--1841.~gadu Pozenes novadam veltītajā literatūrā sauc par \strong{``Flotvela ēru''}. Viņa oficiālais uzdevums bija lauzt poļu muižniecības pārmērīgo ietekmi lielhercogistē, bet faktiskais~--- pilnībā to sapludināt ar Prūsijas karalisti. Lielhercogiste tika pārvērsta par parastu Prūsijas provinci. Lai to panāktu, E.~Flotvels lika priekšā, pirmkārt, uzpirkt poļu muižniecībai piederošo zemi. To izdarīt nevajadzēja būt grūti, jo poļu šļahta bija iestigusi parādos. Otrkārt, bija jālikvidē poļu valodas pasniegšana vidējās un augstākajās mācību iestādēs. Treškārt, bija jāpakļauj valsts kontrolei katoļu garīdznieku izglītošana. Ceturtkārt, viņš ieteica poļu rekrūšus norīkot dienestā vācu pulkos. E.~Flotvels arī mēģināja nosaukto programmu īstenot. 1832.~gadā par iestāžu darba valodu oficiāli tika izsludināta vācu valoda, tikai zemākie garīdznieki, lauku un pilsētu ierēdņi drīkstēja sarakstīties poliski. 1833.~gadā apgabalu (\detxti{Kreis}) pārstāvju sapulcēm tika atņemtas tiesības izvirzīt kandidātus uz augstāko ierēdņu~--- landrātu amatiem. Ja pirms tam tie lielākoties bija poļi, tagad to vietas ieņēma vācieši. Šai pašā gadā tika nodibināts fonds, kurš iepirka par parādiem pārdodamos poļu muižnieku zemes īpašumus un atkal pārdeva tos nu jau vāciešiem. 1834.~gadā tiesas Pozenē tika pielīdzinātas pārējām Prūsijas tiesām. 1836.~gadā norisa pašvaldību reforma, poļu muižniecība un garīdzniecība zaudēja vairākas priekšrocības. Varas iestādes nonāca konfliktā ar katoļu baznīcu. 1838.~gadā Gnēzenes-Pozenes arhibīskaps M.~Duņins izdeva rīkojumu viņam pakļautajiem garīdzniekiem slēgt laulības starp dažādu konfesiju piederīgajiem tikai tad, ja tie deva solījumu bērnus audzināt katoļu garā. Tas varētu vest pie katoļu un protestantu laulībās augušo bērnu polonizācijas, tāpēc pret viņu tika uzsākts kriminālprocess. Arhibīskaps nevēlējās piekāpties, tika arestēts, notiesāts un ieslodzīts uz pusgadu cietumā. No šī laika Pozenē gan uzturējās un darbojās poļu intelektuāļi, bet galvenās nacionālās cīņas izvērtās blakus~--- Polijas karalistē. Pretpoļu rīcībai no Prūsijas puses vajadzēja stiprināt tās draudzību ar Krieviju.

``Flotvela ēras'' laikā veiktos pasākumus nevarēja nosaukt par gudru politiku. Tās mērķis~--- uz pretestību noskaņotās šļahtas vietā piesaistīt poļu zemniekus un pilsētniekus Prūsijas valstij palika nesasniegts. Kad 1840.~gada jūnijā par Prūsijas karali kļuva Fridrichs Vilhelms IV, lai gūtu Pozenes poļu muižniecības un garīdzniecības atbalstu, viņš vēlējās panākt izlīgumu ar šiem slāņiem. Jau 1840.~gadā arhibīskaps M.~Duņins ar triumfu atgriezās amatā, neatceļot savu rīkojumu par jauktajām laulībām. Tai pat gadā karalis paziņoja amnestiju 1830.~gada sacelšanās dalībniekiem. Drīzumā~--- 1841.~gada martā no amata tika atsaukts arī E.~Flotvels. 1842.~gadā Prūsijas Kultūras ministrijā tika radīta Katoļu nodaļa, kura galvenokārt nodarbojās ar poļu skolu jautājumu.

Relatīvās samierināšanās periodā poļu patrioti mēģināja savu slepeno organizāciju centrus, kas gatavoja sacelšanos, pārcelt uz Pozeni. Tomēr saspringuma atslābums poļu un vācu attiecībās bija tikai īslaicīgs. Prūsijas policija saņēma ziņās par poļu pagrīdes darbību, kura gan Pozenā nebija diez cik sekmīga.

Kad 1846.~gadā Pozenē ieradušies poļu revolucionāru pārstāvji no Polijas karalistes un emigranti no Parīzes, lai aģitētu par nacionālu sacelšanos, tie nesaņēma atbalstu no vietējiem poļu zemniekiem, kuri atcerējušies poļu šļahtas nodarītās pārestības. Vācu literatūrā aprakstīts gadījums, kad kāds poļu dižciltīgais ieradies ciema krogā, licis ieliet tur sanākušajiem zemniekiem šņabi un uzaicinājis tos tvert ieročus cīņai par Polijas brīvību. Uz to piecēlies sirms zemnieks un ar vārdiem ``\pltxti{Panie, dziękuję za waszą wolność}'' (``Kungs, pateicos par jūsu brīvību'') atpogājis kreklu un parādījis pletnes pēdas, kuras bija palikušas uz viņa muguras vēl no poļu ``brīvības'' laikiem. Saprotams, ka šādu ``brīvību'' zemnieki nevēlējās.

1846.~gada sākumā Pozenē tika arestēts poļu pagrīdes organizācijas vadītājs 1830.~gada sacelšanās dalībnieks L.~Mieroslavskis un ap 500 sazvērestības dalībnieku. No 1847.~gada augusta līdz decembrim 254 sazvērnieki tika tiesāti atklātā procesā Berlīnē. Apsūdzētie ar to ieguva atklātu tribīni, kur varēja izklāstīt savus mērķus. Viņi veikli saistīja Polijas un Vācijas jautājumus, jo abu tautu priekšā stāvēja jautājums, kā tās nonāks pie savas nacionālas valsts? Šādas runas vācu sabiedrībā guva pozitīvu atbalsi. Kā norādījis vācu vēsturnieks E.~Meijers, vācu liberāļi simpatizēja poļiem tāpēc, ka baidījās no tā, ka Krievija necietīs Vācijā liberālas reformas un poļi krievu iebrukuma gadījumā varētu būt vācu sabiedrotie. Arī tiesneši nepalika kurli pret izskanējušajiem argumentiem. Vairāk nekā puse apsūdzēto tika attaisnota. Pārpalikušie saņēma vairāk vai mazāk stingrus cietumsodus. Tikai astoņiem, tostarp arī L.~Mieroslavskim, piesprieda augstāko~--- nāves sodu. Tiesa, neviena dzīvība netika atņemta. 1848.~gada revolūcijas laikā 20.~martā berlīnieši viņus atbrīvoja no cietuma un tie ar triumfu varēja atgriezties Pozenē. Uz Berlīni tika nosūtīts pateicības vēstījums, kurā bija izteikta cerība, ka nāk laiks, kad ``brīvās Vācijas drošībai tiks uzcelta neatkarīgā Polija kā priekšnocietinājums pret aziātu spiedienu''.

Kad 1848.~gadā revolūcijas vilnis vēlās pār Eiropu, tās notikumos piedalījās arī poļi. Ā.~Mickevičs Romā nodibināja poļu leģionu, lai atbalstītu itāļu brīvības cīnītājus. Bādenē sacēlušos pusē 1849.~gadā cīnījās vairāk nekā 300 poļu virsnieku un karavīru. Otrs 1830.~gada sacelšanās dalībnieks ģenerālis J.~Bems vadīja Vīnes sacēlušos pret Austrijas ķeizara armiju. Pēc Vīnes sacelšanās sakāves viņš pievienojās revolucionāriem Ungārijā, kur L.~Košuta vadībā jau karoja ap 3~000 poļu leģionāru. Pēc sacelšanās sakāves J.~Bems bēga uz Krievijas ienaidnieces Turcijas teritoriju, pieņēma tur islāmu, vadīja turku karaspēku pret sacēlušamies arābiem, mira kā turku ģenerālis. J.~Bema piemiņa dzīvo Ungārijā viņa bijušā adjutanta, dzejnieka Š.~Petefī dzejā. Budapeštā ir uzcelts J.~Bema piemineklis. Interesanti, ka \strong{skanēja arī šādu poļu aktivitāti nosodošas balsis}. Piemēram, poļu publicists J.~Janovskis pauda poļu egocentrismu, 1848.~gadā uzstājoties par poļu asiņu ``taupīšanu'' cīniņā par ``svešām'' lietām, kalpošanu Polijai turpretī uzskatot par ``kalpošanu cilvēcei''.

Daļa Eiropas revolucionāru cerēja, ka tiešu poļu piemērs iejūsminās cīņai pārējas tautas. Vācu revolucionārais dzejnieks G.~Hervegs uzrakstīja dzejoli, kurā poļu patriotu vārdā griezās pie Eiropas tautām:

\vspace{1.5em}

\noindent
\begin{minipage}{0.6\textwidth}
\detxti{
``An dich, Europa, richten wir die Frage:\\
Veläβt du uns zum zweitenmal?''}
\end{minipage}
\hspace{1em}
\begin{minipage}{0.45\textwidth}
``Tev, Eiropa, mēs jautājam:\\
Vai atstāsi mūs otru reiz?''
\end{minipage}

\vspace{1.5em}

% page 92


Pēc Francijas karaļa Luija Filipa I atkāpšanās arī Vācijā aktivizējās revolucionārie spēki, izvirzījās iniciatīva izveidot Visvācijas parlamentu. Tā sagatavošanas gaitā 1848.~gada 31.~martā tika sasaukts t.s. ``Visvācijas priekšparlaments'' (\detxti{Vorparlament}) Frankfurtē. Tā dalībnieki aicināja noturēt brīvas parlamenta vēlēšanas, izteicās arī par Polijas apvienošanos. Priekšparlaments pasludināja Polijas dalīšanas par ``kaunpilnu netaisnību'' un ``vācu tautas svētu pienākumu veicināt Polijas atjaunošanu''.

Tāpat kā citās valstīs, arī Vācijā revolucionāri prasīja Polijas neatkarības atjaunošanu, lai tā sagrautu ``Svēto savienību''. Sociālists un viens no t.s. ``zinātniskā'' komunisma pamatlicējiem K.~Markss atzina Polijas nacionālās atbrīvošanas nepieciešamību, jo, pēc viņa domām, demokrātija Polijā nebija iespējama bez feodālo tiesību iznīcināšanas, bez agrāras kustības, kura pārvērstu feodāli atkarīgos zemniekus par brīviem īpašniekiem, bet agrārā revolūcija nebija iespējama bez vienlaicīgas nacionālās patstāvības izcīnīšanas. K.~Markss Polijas atbrīvošanu uzskatīja par pašas Vācijas atbrīvošanas priekšnoteikumu no patriarhāli feodālā absolūtisma.

Taču krievu vēsturnieks N.~Uļjanovs, kurš darbojās emigrācijā Rietumos, parādījis arī citu t.s. ``marksisma klasiķu'' nostājas interpretāciju. Viņi atbalstīja poļu revolucionāro kustību tāpēc, ka tā bija galvenokārt vērsta pret Krieviju, kura savukārt šai laikā Eiropā uzstājās kā Vīnes kongresā radītās starptautisko attiecību sistēmas garants. Pret Krieviju bija vai nu jāorganizē Eiropas valstu karš vai no tās jānorobežojas ar neatkarīgu Polijas valsti.

K.~Markss un F.~Engelss apzinājās, ka revolucionāru lomā Polijā pārsvarā uzstājas muižnieki-feodāļi, kuri tikai vārdos atzina sociālo atbrīvošanos. Taču šai gadījumā viņus maz interesēja, kas virza nacionālās atbrīvošanās kustību~--- demokrāti vai aristokrāti-muižnieki. Par 1830.~gada poļu sacelšanos K.~Markss un F.~Engelss rakstīja: ``Sauklis ``Lai dzīvo Polija!'' pats par sevi nozīmēja: nāvi Svētajai savienībai, nāvi Krievijas, Prūsijas un Austrijas militārajam despotismam, nāvi mongoļu kundzībai pār mūsdienu sabiedrību!'' Pie tam viņi atbalstīja poļu nacionālās atbrīvošanās kustību galvenokārt Polijas karalistē, noklusējot par Pozeni, kurai bija jāpaliek vāciskai.

Taču Polijas valstiskuma atjaunošana nevarēja būt nekas vairāk kā revolucionāru nodoms. Polijas sadali taču veica tās pašas trīs valstis: Prūsija, Austrija, Krievija, no kurām bija atkarīgs, vai tiks panākta Vācijas vienotība, bet Polijas neatkarības atjaunošanas plāns neglābjami pārvērta tās par iecerētās nacionālās apvienošanās pretiniecēm. Tā, Vācijas impērijas radīšana bija iespējama tikai Krievijas neitralitātes apstākļos, bet to varēja nodrošināt tikai atsakoties no Polijas patstāvības atjaunošanas. Tāpēc drīz ``marksisma klasiķi'' bija vīlušies savās cerībās. 1851.~gada 23.~maijā, sakarā ar izveidojušos starptautisko situāciju, kur Francija, Itālija un Polija bija ieinteresētas Vācijas sadrumstalotībā, un tai, kā uzskatīja F.~Engelss, izņemot Ungāriju, varēja būt tikai viens sabiedrotais~--- Krievija, kurā būtu uzvarējusi zemnieku revolūcija, viņš uzrakstīja garu vēstuli K.~Marksam, kurā izklāstīja savas pārdomas par poļiem, kā ``demoralizētu nāciju'' (\frtxti{nation foutue}): ``Viņus nākas lietot tikai kā līdzekli, un tikai līdz tam laikam, kamēr pati Krievija nepiedzīvos agrāru revolūciju. No šī brīža Polija zaudē jebkādas tiesības uz pastāvēšanu''. Attieksme pret poļiem F.~Engelsam bija radusies ļoti kritiska: ``Poļi vēsturē nekad nav darījuši neko citu, kā tikai tēlojuši drosmīgu un nebēdnīgu muļķību''. ``Poļos ir nemirstīga tieksme uz ķildām bez mazākā iemesla.'' Un visbeidzot: ``nevar atrast nevienu brīdi, kad Polija, kaut pret Krieviju, sekmīgi pārstāvētu progresu vai vispār paveiktu ko tādu, kam būtu vēsturiska nozīme. Pretēji tai Krievija tiešām iemieso progresu attiecībā pret Austrumiem.'' F.~Engelss Krievijā atrada daudz vairāk civilizācijas, izglītības un rūpniecības attīstības elementu nekā ``šļahtiski-miegainajā Polijā''. ``Nekad Polija neprata asimilēt nacionālā ziņā svešus elementus. Vācieši poļu pilsētās ir un paliek vācieši. Bet katrs Krievijas vācietis jau otrā paaudzē ir dzīvs piemērs tam, kā Krievija prot rusificēt vāciešus un ebrejus. Pat ebrejiem tur izaug slāvu vaigu kauli.'' `` Ceturtdaļa Polijas runā lietuviski, ceturtdaļa~--- rutēniski, neliela daļa~--- pa pusei krievu valodā, bet pati poļu daļa par aptuveni trešdaļu ir ģermanizēta''. F.~Engelss izteica pārliecību, ka ``Krievijā agrārā revolūcija notiks agrāk nekā Polijā, kā krievu nacionālā rakstura, tā arī vairāk attīstīto buržuāzisko elementu rezultātā. Ko gan nozīmē Varšava un Krakova pret Pēterburgu, Maskavu, Odesu utt.!''. Par F.~Engelsa ``pragmatismu'' liecināja konstatējums: ``Nācijai, kura var uzstādīt maksimāli 20~000~--- 30~000 cilvēku [domāts, karavīru~--- V.Š.] nevar būt balss tiesību. Bet Polija, acīmredzot, daudz vairāk nespēs uzstādīt.'' Tikpat revolucionāri-``pragmātisks'' bija secinājums: ``ar aizsardzības ieganstu paņemt no poļiem rietumos visu ko vien var, ieņemt viņu cietokšņus, īpaši Pozeni, ar vāciešiem, ļaut viņiem saimniekot, sūtīt viņus ugunī, aprīt viņu produktus, barot viņus ar Rīgas un Odesas solījumiem, bet gadījumā, ja izdotos iesaistīt kustībā krievus, apvienoties ar tiem un piespiest poļus piekāpties.'' Acīmredzot, nav īpaši jāsaka, ka F.~Engelss šeit pirmkārt rūpējās par Vācijas interesēm, kura viņam bija galvenā progresa un revolūcijas nesēja Eiropā. K.~Markss tāpat uzskatīja, ka Prūsijas Poliju, ar pilsētām, kuras apdzīvo vācieši, nav jāatdod ``tautai, kura līdz šim vēl nav pierādījusi savu spēju izkļūt no pusfeodālā dzīves veida, kurš balstās uz lauku iedzīvotāju nebrīvi.''

1848.~gada revolūcijas laikā Vācijā, nemieri (\pltxti{powstanie wiełopolskie}) notika arī Pozenē. Tur kā poļu pārstāvniecība radās Nacionālā komiteja (\pltxti{Komitet Narodowy}). Uz vietām poļu komitejas atņēma varu prūšu administrācijai. Prūsijas karalis Fridrichs Vilhelms IV 1848.~gada 24.~martā apsolīja ``nacionālu reorganizāciju'' vēl ar poļu vairākumu esošajā Pozenes lielhercogistē.

28.~martā Pozenē ieradās L.~Mieroslavskis un sāka organizēt brīvprātīgas poļu vienības (tajās sapulcējās ap 9~000 galvenokārt izkaptīm bruņotu vīru), kuru dalībnieki cerēja uz kopīgu vācu un poļu karu pret Krieviju. Pēc dažu poļu revolucionāru plāniem poļu korpusam no Pozenes bija jāiebrūk Polijas karalistē, lai ar to izraisītu Prūsijas un Krievijas karu. Prūsijas varas iestādes piesardzīgi 3.~aprīlī izsludināja aplenkuma stāvokli, sākotnēji uzsāka ar L.~Mieroslavski pārrunas, taču kad viņš pieprasīja nākamajai Polijas valstij arī Pozeni, stāvoklis saasinājās. Pozenes vācieši, kuri provinces robežapgabalos veidoja vairākumu, nevēlējās ietilpt poliskā valstiskā veidojumā, pieprasīja sadalīt provinci pēc nacionālās piederības principa.

14.~un 26.~aprīlī tika izdoti divi Prūsijas valdības rīkojumi. Atbilstoši tiem Pozenes lielhercogiste bija jāreorganizē. Vienā tās daļā tika solīts ierīkot vairāk vai mazāk autonomu poļu pašpārvaldi, tajā būtu garantēta izglītība, lietvedība un tiesas poļu valodā, ierēdņi varētu būt poļi. Toties lielākā, lielhercogistes rietumu daļa ar Pozenes pilsētu tiktu iekļauta Vīnes konferences radītajā Vācu savienībā (1815--1866). Poļi tam pretojās. Tā kā Pozenes lielhercogiste līdz tam gan bija piederīga Prūsijai, bet ne Vācu savienībai, šo rīkojumu saņemšana izsauca bruņotas sadursmes starp abām pusēm. 29.~aprīlī Prūsijas armija vērsās pret saformētajām poļu vienībām. Notika vairākas sīkas sadursmes. Nacionālā komiteja pašlikvidējās. Lai gan divās sadursmēs L.~Mieroslavskis uzvarēja, pret labi apmācīto prūšu armiju poļiem nebija tikpat kā nekādu izredžu. Nedēļas laikā poļu sacēlušos sakāva. 6.~maijā (citi dati~--- 9.maijā) viņi bija spiesti kapitulēt, taču ne tikai ierindas poļu kaujinieku, bet arī to vadoņi varēja mierīgi atgriezties dzīves vietās. L.~Mieroslavskis gan tika saņemts gūstā, bet 1848.~gada jūnijā pēc franču diplomātu iejaukšanās viņu atbrīvoja un jau nākamajā gadā viņš komandēja revolucionāru vienības Sicīlijā un Bādenē (Vācijā). Kaut poļu aktivitātes tika apspiestas ar bruņotu spēku, vācu vēsturnieki uzsver, ka par sacelšanās ātro sakāvi jāpateicas ne tik daudz prūšu varai, kura šai brīdī bija vāja, bet gan tam, ka poļu zemnieki nevēlējās ar savu ādu maksāt par poļu šļahtiču vēlmju piepildīšanu. Sacēlušos sagrāve 1848.~gada maijā gan nenozīmēja, ka uzreiz sākās strauja poļu ģermanizācija. Taču arī valdības aprīlī izdotie rīkojumi netika atcelti.

Frankfurtes parlaments (\detxti{Frankfurter Nationalversammlung}~--- pirmais visas Vācijas parlaments, kurš darbojās no 1848.~gada 18.~maija līdz 1849.~gada 31.~maijam) tika iesaistīts nacionālajos konfliktos, tai skaitā Polijas jautājuma izskatīšanā. Debates par poļu jautājumu notika 1848.~gada jūlijā. Centrālais bija jautājums, vai jaunajā Vācijas impērijā, kas tika projektēta, jāietilpst visai Pozenes provincei vai tikai tai tās daļai, kuru pārsvarā apdzīvoja vācieši. Tikai neliela deputātu daļa izteicās par atteikšanos no Pozenes poļu daļas. Tā sadursmē ar dzīves realitāti sāka mainīties vācu nacionālās idejas kā brīvības, neatkarības un vienotības idejas raksturs. Ar lielu balsu vairākumu Nacionālā sapulce pieņēma lēmumu par Pozenes rietumdaļas iekļaušanu Vācijas impērijā, bet no idejas atjaunot neatkarīgu Poliju atteicās.

Izklāstītie notikumi saasināja attiecības starp poļiem un vāciešiem Pozenē, parādīja, ka vācu vidū sajūsma par poļu cīņu par brīvību un pret Krieviju bija noplakusi. Vietējie vācu iedzīvotāji vērsās pret reformu plāniem.

Kopš 1848.~gada Berlīnē oficiāli runāja ne vairs par Pozenes lielhercogisti, bet Pozenes provinci, kaut arī vēl pēc Vācijas apvienošanās (1871) tās imperators lietoja arī Pozenes lielhercoga titulu. Prūsijas iekšējā organizācijā arī vairs nelietoja jēdzienu ``Prūsijas poļi'' ,bet gan ``poļu prūši''. Vienai novada daļai solītā autonomija tā arī nekad netika sagaidīta. Turpmāk vietējās saimnieciskās organizācijas tika veidotas jau pēc nacionālā principa: vāciešiem~--- savas un poļiem~--- savas. Tāpat kā Galīcijā, arī Pozenē vairs nebija cerību izraisīt visas tautas bruņotu brīvības cīņu. Izsīka arī Pozenes poļu emigrantu enerģija.

\strong{Ģermanizācijas politika} Pozenē tika izvērsta arvien noteiktāk. ``Izlīgums ar poļiem,'' jau 1848.~gada 20.~aprīlī O.~v.~Bismarks rakstīja avīzē ``\detxti{Magdeburger Zeitung}'' (``Magdeburgas Avīze'') - ``ir neiespējams; tos pilnībā iznīcināt ir nehumāni un turklāt neiespējami, ja vien paaudzes par to nenodarbosies''. Tika slēgtas poļu avīzes, aizliegtas poļu organizācijas. Kopš 1852.~gada dažādi oficiāli rīkojumi un norādījumi tika izplatīti tikai vācu valodā. Tomēr vēl XIX~gadsimta vidū pastāvēja laba vācu un poļu sadarbība izglītības jomā. Kopīgiem vācu un poļu spēkiem 1853.~gadā izdevās panākt reālskolas atvēršanu Pozenē. Abu tautu pārstāvji kopīgi apmeklēja sporta biedrības, piemēram, kopš 1838.~gada eksistējošo šaha klubu, 1860.~gadā dibināto vingrošanas biedrību. Taču 50.~gados Pozenes provincē palika vairs tikai trīs poļu ģimnāzijas, 1856.~gadā poļu ģimnāzijām tika pavēlēts latīņu klasiķus jau no jaunākajām klasēm tulkot tikai vācu valodā. 1858.~gadā visās pilsētas skolās visi mācību priekšmetu, izņemot dziedāšanu un poļu valodu, bija jāpasniedz vāciski. 1867.~gadā visās poļu sākumskolās ieviesa obligātu vācu valodas apmācību. Poļu valoda pamazām ieņēma arvien nenozīmīgāku lomu. Kā atzīmē vācu vēsturnieki, vācu liberālā opozīcija gan juta līdzi poļu patriotiem, bet rezultātu tam nebija. Turpretī auga nepatika pret poļu kaimiņiem. Vācu rakstnieka G.~Freitāga 1854.~gadā izdotajā romānā ``\detxti{Soll und Haben}'' (Vajadzēt un būt), kurš līdz gadsimta beigām piedzīvoja 40~izdevumu, kā tipiskas poļu īpašības tika nosauktas nesaimnieciskums, nolaidība, viltība.

1857.~gadā no pavisam 1~440 muižām Pozenes provincē 536 (jau 37\%) atradās vācu rokās. Par šo straujo pieaugumu bija jāpateicas valsts politikai. Valsts iestādes gan Berlīnē, gan Pozenē uzskatīja par savu uzdevumu palielināt vācu īpašumus provincē. Tomēr krasi noskaņotie vācu nacionālisti arī tad nebija mērā ar it kā nenoteikto valdības politiku. 1861.~gadā kādam parlamenta deputātam domātā vēstulē minētais G.~Freitāgs rakstīja: ``Mani tagad nodarbina Pozenes apsaimniekošanas un kolonizācijas projekts un es vēlos Berlīnē par to aģitēt, lai ar poļiem ātri tiktu galā. Šī ilgstošā vājuma apkaunojums kļūst pārāk liels. Ja valdībai nav tam spēka, mums ir jāmēģina to sagādāt privāti.'' Jāatzīmē, ka daudzi poļu muižnieki netika vaļā no agrākajiem parādiem, nespēja atteikties no agrāk ierastā dzīves stila, pielāgoties jaunajiem modernajiem saimniekošanas paņēmieniem un bija spiesti muižas pārdot pat bez īpaša spiediena.

Nākamais poļu un vācu attiecību saasinājuma periods sākās ar poļu nemieriem blakus esošajā Polijas karalistē 19.~gadsimta 60.~gados. Prūsijas premjerministrs O.~f.~Bismarks baidījās, ka sacelšanās Polijas karalistē varētu izplatīties arī uz Pozeni un tāpēc Prūsija 1863.~gada 27.~janvārī (8.~februārī) noslēdza ar Krieviju tā saucamo Anvenslēbena konvenciju (\detxti{Die Alvenslebensche Konvention}), nosauktu pēc Prūsijas pārstāvja sarunās ģenerāļa G.~f.~Alvenslēbena vārda, kura atļāva abām valstīm vajāt sacēlušos arī tad, ja tie bēga uz otras valsts teritoriju. Vienošanās nāca par labu Krievijas karaspēkam, kurš varēja sekot poļiem arī Prūsijas teritorijā. Taču literatūrā ir ziņas, ka 1863.--1864.~gadā vairāk poļu sacelšanās dalībnieku Prūsijā gāja bojā no prūšu karavīru, nekā no poļiem sekojošo krievu karavīru lodēm. Arī Prūsijā poļu patrioti tika tiesāti par palīdzību sacēlušamies Polijas karalistē. Kā uzskatīja poļu vēsturnieks A.~Žeļagovskis, šī konvencija bija O.~f.~Bismarka ``meistarstiķis'', kurš uz gadu desmitiem noteica apstākļus Eiropas austrumos, jo padarīja Aleksandram II neiespējamu jebkādu samierināšanās politiku ar poļiem un nodrošināja Prūsijai rīcības brīvību pret Austriju un Franciju.

Joprojām gan daļa Pozenes vāciešu ne tikai ar nepatiku skatījās uz Krievijas un arī Prūsijas varas iestāžu antipolisko darbību, bet pat atbalstīja sacēlušos, šujot tiem uniformas, palīdzot transportēt ieročus. Prūsijā tika runāts par Polijas valsts atjaunošanas nepieciešamību. Taču poļu kustībā, īpaši zemākajos slāņos, parādījās arī antivāciskas izpausmes, tāpēc pēc 1863.~gada poļu sacelšanās sakāves vāciešos auga antipoliskie noskaņojumi. Tomēr ikdienā dažādu etnisko grupu kopējā dzīve lielākoties norisa mierīgi, pat draudzīgi. Poļu neapmierinātība pārsvarā bija vērsta pret vācu iestādēm, bet ne vācu tautības līdziedzīvotājiem. Protams, bija arī izņēmumi.

Ģermanizācijas politika vēl pastiprinājās, kad XIX gadsimta otrajā pusē mainījās Prūsijas un citu valstu spēku samērs. Antipoliskie noskaņojumi bija arī rezultāts Prūsijas militārajiem panākumiem karos pret Dāniju un Austriju un centieniem pēc Vācijas vienotības pieauguma. Īpaši karos pret Austriju (1866) un Franciju (1870/1871) tika radīti priekšnoteikumi Vācijas impērijas radīšanai ar Prūsiju priekšgalā. Kaut poļu deputāti Prūsijas landtāgā un Vācijas reihstāgā protestēja pret viņu zemju iekļaušanu Ziemeļvācijas valstu savienībā (1866) un Vācijas impērijas sastāvā (1871), Pozenes province kopā ar Prūsiju kļuva par šo valstisko veidojumu sastāvdaļu. Vācijas iedzīvotāju vairākums bija sajūsmā par vienotas Vācijas impērijas nodibināšanu, ar neizpratni un aizvainojumu vērsās pret vietējiem poļiem, kuri ne tikai neizrādīja šādu sajūsmu, bet prasīja autonomiju un sapņoja par savu valsti. Vācieši tagad arī redzēja, ka Vācijai nav ko baidīties no Krievijas, arī tāpēc ātri pieauga nepatika pret dumpīgajiem poļiem. Arī 1871.~gada Vācijas Konstitūcija runāja tikai par ``vācu tautu'', neparedzot nekādas mazākumtautību tiesības.

Vācu uzvara karā pār Franciju, Vācijas impērijas izveide nostiprināja Prūsiju, ļāva tai aktivizēt ģermanizācijas politiku Pozenē. Tiesa, Vācijas impērijas nodibināšana nesa arī noteiktu tiesību paplašināšanos Pozenes poļiem. Atšķirībā no Prūsijas landtāga triju kārtu vēlēšanām, Vācijas reihstāga vēlēšanās varēja piedalīties visi pieaugušie vīrieši. Poļu deputātu skaits tajā svārstījās no 13 līdz 20. Gandrīz visos jautājumos, izņemot balsošanu 1893.~gadā par armijas palielināšanu, viņi kopā ar pārējo Vācijas mazākumtautību pārstāvjiem (dāņiem, elzasiešiem) balsoja kopā ar opozīciju. Panākt savu priekšlikumu ievērošanu viņi, protams, nevarēja, taču uzskatīja par vajadzīgu demonstrēt savu naidīgumu valdībai.

Vācu vēsturnieks H.~f.~Zizevics atzīmējis, ka periodā no 1862.~gada, kad O.~f.~Bismarks kļuva par Prūsijas Ministru padomes priekšsēdētāju, līdz pat 1916.~gadam Prūsijas un Vācijas politikā pret poļiem pastāvēja divi strāvojumi, kuri nepārtraukti mēģināja viens otru pārvarēt un iegūt vairākumu parlamentā. H.~f.~Zitzevics tos nosaucis par administratīvi-likumīgo virzienu un politisko virzienu.

Pirmais prasīja nekādā gadījumā nepārkāpt valsts likumus. Poļiem bija jākļūst par labiem Vācijas pilsoņiem, bet likumiem un citiem rīkojumiem bija jāpalīdz viņiem tādiem kļūt.

Otrais bāzējās uz sociāldarviniskiem priekšstatiem. Vācietība reprezentēja pilnvērtīgo, valdošo rasi, slāvi~--- nepilnvērtīgo rasi, pār kuru bija vajadzīga aizbildniecība un vadība. Attiecīgi aizbildņi varēja noteikt aizbildināmo tiesības un pienākumus.

Autoram gan jāsaka, ka otrais virziens bija redzamāks, darbojās aktīvāk. Augstākās personas Vācijas impērijā bija noskaņotas antipoliski.

Tā kanclers O.~Bismarks poļus konsekventi sauca par ``impērijas ienaidniekiem''. Pēc viņa vārdiem ``\dots{}tikai tas, kurš nepavisam nepazīst poļus, var šaubīties par to, ka viņi paliks zvērināti mūsu ienaidnieki\dots{}''

Stiprākais ģermanizācijas līdzeklis Pozenes provincē bija kolonizācija. Pēc vērtējumiem Pozenes provincē pēc tās pievienošanas Prūsijai bija no 200~000 līdz 300~000 jeb 25 līdz 35\% vāciešu. Līdz XIX gadsimta 60.~gadu sākumam \strong{vācu iedzīvotāju skaits} arvien \strong{pieauga}. Pieaugums sastādīja ap 300~000 cilvēku un vāciešu daļa visu provinces iedzīvotāju vidū palielinājās par 5 līdz 8\%. Tā, sakarā ar Pozenes cietokšņa izbūvi no 1828. līdz 1834.~gadam ieceļoja ap 1~000 vācu strādnieku. 1846--1848.~gada notikumi gan piebremzēja vāciešu ieceļošanu, taču 50.~gados tā atkal turpinājās. 1850.~gadā Prūsijā noslēdzās zemes piešķiršana zemniekiem. Zemnieki no dažādiem Vācijas novadiem Pozenē varēja iegādāties zemi ar vislabākajiem nosacījumiem. Līdz 1870.~gadam ievērojami pieauga vācu zemes īpašnieku slānis.

Taču jau līdz 1870.~gadam varēja konstatēt arī vāciešu izceļošanu no provinces. No 1816. līdz 1825.~gadam auga lētu tekstilizstrādājumu ievedums Pozenē no Silēzijas un Saksijas, kas audēju amatu noveda smagā krīzē, jo ar blakus esošo Krieviju pastāvēja muitas robeža. Daudzi vācu audēji pameta Pozeni un pārvācās uz poļu apdzīvotajiem apgabaliem Krievijā. 30. un 40.~gados sakarā ar varas iestāžu neiecietību no Pozenes uz Austrāliju izceļoja grupa t.s. ``vecluterāņu'' (šādi sauca luterāņus, kas uzstājās pret 1827.~gadā notikušo jaunas baznīcas liturģijas ieviešanu pēc Prūsijas parauga), bet lielākais izceļotāju skaits devās uz ASV. Kopš XIX gadsimta vidus vācu izceļotāji no Pozenes pārsvarā devās uz Vācijas rietumu rūpnieciskajiem apgabaliem. Līdz 60.~gadiem vāciešu pieplūdums Pozenes provincē bija lielāks nekā atplūdums, no 1824. līdz 1870.~gadam vācu ieceļotāju skaits par 31~000 pārsniedza izceļotāju skaitu. Pozenes pārvaldes iecirknī vācieši sastādīja 30 līdz 35\%, bet Brombergas (\pltxti{Bydgoszcz}) iecirknī gandrīz 50\%.

Tiesa, pret prūšu ierēdņu sniegtajiem datiem nākas izturēties ar neuzticību. 1860.~gadā viņi paziņoja, ka 44,4\% no 1,4~miljoniem Pozenes provinces iedzīvotāju ir vācieši. Taču sabiedrībā šis rezultāts tika pamatoti apšaubīts. Pēc konfesiju statistikas 1864.~gadā, pat ja visus ebrejus pieskaita vāciešiem, arī tad to īpatsvars sasniedza knapi 43\%. Kad 1871.~gadā vēlreiz notika uzskaite, to daļa bija samazinājusies līdz 41\%. Pēc citiem datiem, vāciešu un tiem pieskaitīto ebreju Pozenes provincē 1830.~gadā bija 40\%, 1860.~gadā~--- 45\%, bet 1910.~gadā~--- tikai 38\%, turpretī poļu~--- attiecīgi 60, 55 un 62\%.

Pēc 1861.~gada datiem gandrīz 40\% vāciešu poļu apgabalos dzīvoja pilsētās. Īpaši Pozenes pilsētā strauji auga vācu militāristu, ierēdņu un strādnieku skaits, vācieši tur kļuva par stiprāko tautību un palika tādi līdz 1870.~gadam. Pēc tam pārsvaru guva tādi faktori kā augstāka dzimstība poļu kā katoļu vidū un vācu tautības iedzīvotāju lielākā nekā poļu bēgšana no austrumiem (vācu \detxti{Ostflucht}, poļu \pltxti{ucieczka ze wschodu}), izceļojot uz Vācijas industriāli attīstītākajiem apgabaliem. Berlīnē atskanēja uztrauktas balsis: kā gan Pozenes provinci var ģermanizēt, ja šeit mītošo vāciešu skaits samazinās? Kā var pamatot, ka visur jārunā vāciski, ja to skaits, kuriem tā bija dzimtā valoda, samazinās? Gala secinājums, kuru izdarīja vācu birokrāti-nacionālisti bija: ir jāpaātrina poļu ģermanizācija.

Prūsijas konservatīvie, kuri neko negribēja dzirdēt par tautību izlīgumu, sauca to par [mazākuma] tautību reiboni (\detxti{Nationalitätenschwindel}), uzskatīja, ka ``poļu problēmai'' pirmkārt ir agrārpolitisks raksturs. Izturīgie un pieticīgie poļu sezonas strādnieki nodrošināja Vācijas centrālo un austrumu novadu muižu rentabilitāti. Pie tam, līdz Pirmajam pasaules karam augot Vācijas bruņotajiem spēkiem, neraugoties uz polisko izcelsmi un katolisko ticību, divvalodīgie Prūsijas poļu pavalstnieki lieti noderēja armijas papildināšanai. Vietējo vāciešu vairākums vairs nevarēja iedomāties Pozenes provinces atdalīšanos no Prūsijas un Vācijas. Tikai vācu liberāļi vēl piekrita provinces autonomijai Prūsijas sastāvā. Turpretī poļus autonomija vairs neapmierināja pat kā minimālā programma. Ja sākotnēji konflikts aprobežojās ar abu tautību elitēm, tālāk tas vērsās plašumā. Zemākajos slāņos izplatījās negatīvi priekšstati par otras tautas pārstāvjiem. Vācieši smējās par ``poļu saimniecību'' (\pltxti{polnische Wirtschaft}, t.i.~--- nesaimnieciskumu), poļi vāciešus lamāja par švābiem (vācu \detxti{Schwaben}~--- vācieši, runājoši īpašā dienvidvācu dialektā, kuri dzīvo Švābijā~--- \detxti{Schwabenland}) un ``kartupeļu rijējiem''. Kontakti sāka aprobežoties tikai ar oficiālo jomu.

Vācijas kancleram un Prūsijas premjerministram O.~f.~Bismarkam poļu nepakļāvība ļoti nepatika. Savās runās viņš atkal un atkal norādīja, ka Pozenes un Rietumprūsijas provincēm ir nepieciešama sasaiste ar Austrumprūsiju, Pomerāniju un Silēziju ne tikai stratēģisku, bet arī ekonomisku u.~c. apsvērumu dēļ. Pēc viņa uzskatiem poļiem bija jākļūst par poļu valodā runājošiem prūšiem, taču tam bija vajadzīgs, lai viņi mācītos arī vācu valodu, jo tikai tās zināšanas varētu viņiem dot visas Prūsijas pavalstnieku priekšrocības. O.~f.~Bismarks bija norūpējies par poļu nacionālismu, rakstīja, ka Polijas valsts atjaunošana nozīmētu automātisku sabiedrotā radīšanu jebkuram Prūsijas ienaidniekam. Kā cilvēkam viņam esot simpātijas pret poļiem, bet no vācu nacionālā viedokļa neesot citas izejas, kā tikai tos iznīdēt. Īpaši kanclers bija neapmierināts ar katoļu baznīcas priesteru aģitāciju pret Prūsiju. Viņš uzskatīja, ka katoļu baznīca var radīt Vācijai grūtības ārpolitikā, jo Austroungārijā katoļu reliģija bija valdošā un no šā viedokļa Prūsijas poļi varētu gribēt pievienoties tai. Blakus valodu politikai izvirzījās arī otrs ģermanizācijas līdzeklis: cīņa pret katolicismu.

Izvēršoties t.s. kultūrkampfam (vācu \detxti{Kulturkampf}~--- O.~f.~Bismarka un viņam pieslējušos vācu nacionālliberāļu cīņa pret katolisko Centra partijas un katoļu baznīcas ietekmi Vācijā. Latviešu valodā par to sīkāk var izlasīt autora grāmatā ``Vācijas impērija. 1871--1918''), jācieš bija arī poļu garīdzniecībai un draudzēm. O.~f.~Bismarks savā darbā ``Domas un atmiņas'' (``\detxti{Gedanken und Erringerungen}'', 1890) kultūrkampfam veltīto nodaļu sāka ar vārdiem: ``Kultūrkampfa sākumu man pārsvarā noteica tā poliskā puse''. 1872.--1874.~gadā visās ģimnāzijās, kur līdz tam mācības norisa poļu valodā, par vienīgo kļuva vācu valoda, poļu valodu pasniedza tikai kā mācību priekšmetu, bet visās zemākajās skolās poļu valodu varēja izmantot tikai reliģijas mācības stundās. Ja līdz šim poļi prūšu skolas Pozenē un Rietumprūsijā uzskatīja par kultūras veicināšanas līdzekli, tad tagad tās tika vērtētas kā apdraudējums savai tautībai. Atbildes reakcija bija poļu privātskolu skaita straujš pieaugums.

Kā vēlāk atzina pats O.~f.~Bismarks, kultūrkampfs bija ``vairāk vērsts pret polonismu, nekā pret katolicismu''. 1872.~gadā tika izdots Prūsijas likums par skolu uzraudzību, kas poļu garīdzniecībai atņēma skolu uzraudzības tiesības. Pret to protestēja Romas pāvests. Gnēzenes-Pozenes arhibīskaps M.~Ļedohovskis aicināja ticīgos nepakļauties valsts rīkojumiem. Viņam par vairākiem likumu pārkāpumiem pietiesāja naudas sodu 80~000 marku apmērā. Arhibīskaps nemaksāja, tad viņam kā pirmajam no Prūsijas bīskapiem piesprieda 2 gadus cietumsoda. Kad viņš arī tad nenolika savas pilnvaras, valsts reliģisko lietu tiesa paziņoja, ka viņš ir no amata atstādināts. Kad 1876.~gadā arhibīskapu atbrīvoja no aresta, pāvests viņu jau bija iecēlis par kardinālu un viņš devās uz Romu. Apcietināti tika arī apmēram 100 citu garīdznieki, daudzu vietas draudzēs palika neaizpildītas. Taču kultūrkampfam negatīva ietekme uz poļu pašapziņu bija tikai Austrumprūsijas Mazūru (poļu \pltxti{Mazurzy}, vācu \detxti{Masuren}) rajonā, kur vietējie luterticīgie iedzīvotāji gan runāja vienā no poļu valodas dialektiem, bet bez poļu pašapziņas padevās arvien lielākai vāciskuma ietekmei, nošķīrās no poļu tautas. Mazūri bieži sevi neuzskatīja nedz par vāciešiem, nedz poļiem, bet par poļu prūšiem. Pozenes novādā šī vācu ietekme bija minimāla. Katrs varas iestāžu konflikts ar katoļu baznīcu stiprināja saikni ``polis~--- tas ir katolis''. Poļu vēsturnieki atzīmē, ka kultūrkampfs pat veicināja poļu kultūras un ekonomiskās darbības pieņemšanos spēkā. Ticības, valodas un kultūras aizsardzība pret to veidoja poļu nacionālo pašapziņu. Tiesa, poļu publicists R.~Šimaņskis 1870.~gadā ar bažām atzīmēja, ka gan Pozenē, gan Silēzijā mantīgie slāņi un inteliģence, kam bija galvenā loma sabiedriskajās attiecībās, pārsvarā sastāv no ``nepoļu elementa''. Tā cēloņus R.~Šimaņskis saskatīja poļu muižnieku šķiriskajā egoismā, kuri nerūpējās par poļu buržuāzijas interešu aizsardzību. Tomēr 1873.~gadā, kad attiecības starp poļiem un vāciešiem kļuva arvien sliktākas, poļu lielmuižnieks M.~Jackovskis, uzskatot, ka poļu stāvokli, viņu nacionālo pašapziņu var nostiprināt ar sociāliem un kultūrizglītības pasākumiem, nodibināja pirmo poļu zemnieku biedrību. Līdz 1880.~gadam tādu bija jau ap 120. Biedrības sniedza saviem biedriem tehnisku palīdzību, deva kredītus, stiprināja poļu tradīcijas un parašas, pozitīvi ietekmēja arī tikumus. Bet, galvenais, tās ļāva cīnīties pret Prūsijas valdības antipoliskajiem pasākumiem, poļiem piederošās zemes izpārdošanu un nonākšanu vācu rokās. Tika dibinātas arī dažādas kultūras un izglītības biedrības.

1876.~gadā tika izdots likums, ar kuru valsts un pašvaldību iestādēs, tiesās vienīgā atļautā kļuva vācu valoda. Kopš 1815.~gada praktizētā divvalodība ar to tika pilnībā atcelta. Tāpēc poļi Vācijas impērijā jutās kā otrās šķiras cilvēki.

O.~f.~Bismarka cīņa pret katoļu baznīcu ilga līdz 1880.~gadam, kad bija redzams, ka tā ir beigusies neveiksmīgi, dienas kārtībā ir izvirzījušās citas nopietnas problēmas. Taču Vācijas kanclera vienošanās ar Romas pāvestu nenozīmēja, ka valdības pretpoliskā politika vājinātos. Izmainījās tikai tās taktika: ``poļu jautājums'' tika nodalīts no ``katolicisma jautājuma''. Taču, kā tagad atzinuši vācu vēsturnieki, sekmīgais asimilācijas process, kurš palēnām norisa kopš 1815.~gada, XIX gadsimta otrajā pusē tika negaidīti pārtraukts. Kā konstatējis vācu vēsturnieks M.~Broskats, ``beznodoma ģermanizācija'' pēc Vīnes kongresa bija saimnieciskā un sociālā progresa rezultāts. Tā vietā XIX gadsimta otrajā pusē sākās valsts realizēta poļu ierobežošanas politika. Tas izsauca poļu kopības sajūtu (\pltxti{społeczeństwo}), kas bija izplatīta visos slāņos un radīja ``valsti valstī''.

Pēc valodu statistikas datiem 1880.~gadā Pozenes provincē dzīvoja 1,7 miljoni iedzīvotāju, no tiem ap 705~000 labi runāja vāciski, nepilns miljons runāja poliski. 1885.~gadā kopējais iedzīvotāju skaits bija pieaudzis par gandrīz 13~000, taču vāciešu skaits bija samazinājies par ap 5~000, toties poļu skaits pieaudzis par 18~000. Tas bija noticis tāpēc, ka no Vācijas austrumu apgabaliem, ne tikai no Pozenes, arvien vairāk vāciešu pārcēlās uz Berlīni, bet īpaši Vācijas rietumiem: rūpniecības apgabaliem pie Reinas un Rūrā. Turpretī daudzi poļi no Polijas Austroungārijas un Krievijas daļām labprāt pārcēlās uz Pozeni vai Rietumprūsiju, jo atrada šeit labvēlīgus apstākļus peļņai.

Atbildot tam, O.~f.~Bismarks izvērsa cīņu pret cilvēkiem, kuriem nebija Vācijas pavalstniecības. Pēc 1880.~gada tautas skaitīšanas Pozenes provincē vien dzīvoja pāri par 10~000 nevāciešu, kuri nebija Prūsijas un Vācijas pavalstnieki. 1885.~gadā O.~f.~Bismarks nolēma izsūtīt visus nenaturalizējušos iebraucējus. Tā kā to vidū bija arī ap 1/3 ebreju, šai politikai ne tikai antipoliska, bet arī antiebrejiska (antisemītiska) virzība. Kopā no Vācijas īsā laikā bija jādeportē 30~000 cilvēku, no tiem 5~200 no Pozenes provinces. Taču plāni sastapa plašu atbalsi un pretestību kā iekšzemē, tā ārzemēs. O.~f.~Bismarks tomēr aizstāvēja savu kursu, lai veicinātu vācietību Pozenes un Rietumprūsijas provincēs. Kopš 1885.~gada ar Prūsijas valdības lēmumu no Vācijas izraidīja 32~tūkstošus cilvēku (galvenokārt poļus, bet 1/3 bija ebreji). Lielākoties tie bija cilvēki, kuri jau gadu desmitus bija dzīvojuši Prūsijā, atraduši tur savu otro dzimteni. Ar 1885.~gadu jauniem iebraucējiem no Krievijas, Austroungārijas ceļš uz Pozeni bija liegts. Tiesa, pēc vairākiem gadiem, ar 1890.~gadu, kad t.s. poļu aizsprosts \detxti{(Polensperre}) tika atcelts, jo muižniekiem vajadzēja darba spēku, Prūsijā atkal ieplūda tūkstoši poļu sezonas strādnieku. To skaits pārsniedza 80.~gados izraidīto skaitu. Viņu stāvoklis gan palika beztiesisks. Pēc mazākā vietējā ierēdņa, bieži arī vācu muižas īpašnieka mājiena viņus varēja izraidīt pāri robežai uz pastāvīgo dzīves vietu. Kā norādīja vācu sociologs M.~Vēbers, strādnieku ieceļošana un izceļošana bija cīņas līdzekļi latentā cīņā starp darbu un īpašumu. Izceļošana bija pielīdzināma latentam streikam, bet poļu ievešana~--- atbilstošs cīņas līdzeklis pret to.

1886.~gada aprīlī parlaments pieņēma ``Likumu par vācu pārceļošanas veicināšanu uz Rietumprūsiju un Pozeni'' (``\detxti{Gesetz betreffend die Beförderung der deutschen Ansiedlung in den Provinzen Westpreuβen und Pozen}''), ar kuru piešķīra finanšu līdzekļus poļu muižnieku zemju uzpirkšanai, lai piešķirtu tās vācu zemnieku un strādnieku pārceļotājiem uz minētajām provincēm. Tika nodibināta karaliskā Prūsijas pārceļošanas komisija (\detxti{Königlich-preuβischen Ansiedlungskomission}) ar sēdekli Pozenē. Līdz 1914.~gadam ap 0,5~miljonu ha zemju iepirkšanai tika iztērēti ap 500~miljonu marku. Rezultātā vācu iedzīvotāju skaits pieauga par 22~000 ģimenēm (citi dati~--- ap 20~000) jeb ap 150~000 cilvēku, taču ap ¼ no viņiem jau bija Pozenes vai Rietumprūsijas iedzīvotāji. Gala rezultātā vācieši no nosauktajiem apgabaliem vairāk izceļoja uz Vācijas rietumu rajoniem, nekā tajos ieceļoja, bet pārceļošanas komisija vairāk iepirka vācu, nevis poļu muižniekiem piederošo zemi.

Ja citās Vācijas provincēs vietējās kopienas varēja līdzrunāt skolotāju izvēlē, Pozenē tā bija pilnīgi valsts rokās. 1887.~gadā no skolām pilnībā padzina poļu valodu. Uzstājoties Prūsijas parlamentā, O.~f.~Bismarks 1887.~gada 28.~janvārī no vienas puses paziņoja, ka pasākumus pret poļiem izsaukusi pašu Pozenes iedzīvotāju~--- poļu izturēšanās, ka 45~gados nav izdevies panākt poļu dižciltīgo labvēlību Prūsijas valsts idejai utt., bet turpat arī apgalvoja: ``Mēs garantējam poļu zemniekiem, uzticamiem Prūsijas pavalstniekiem, ka mēs nevēlamies pret viņu valodu izturēties naidīgi, mēs gribam vienīgi dot viņiem iespēju apgūt vācu valodu''. Lozunga ``Skaldi un valdi'' realizācijai noderēja arī administratīvā reforma, izdalot atsevišķus apgabalus no tiem, kuros dzīvoja pārsvarā poliski runājoši iedzīvotāji. No 1887.~gada līdz 1918.~gadam Pozenes provincē bija 42 apgabali (\detxti{Kreis})~--- par 15 vairāk nekā 1815.~gadā.

Pēc O.~f.~Bismarka atstādināšanas (var piebilst, ka jau minētais poļu vēsturnieks J.~Feldmans 1924.~gadā rakstīja, ka droši vien nav cita vārda, kurš poļu ausīs skanētu tik drūmi, kā Vācijas ``dzelzs kanclera'' O.~f.~Bismarka vārds) kanclera L.~f.~Kaprivi laikā antipoliskais kurss tika mīkstināts, jo jaunajam valdības vadītājam vajadzēja poļu deputātu atbalstu parlamentā. 1891.~gadā skolās atkal atļāva mācīt poļu valodu. Vietējos cilvēkus iecēla par Pozenes virsprezidentiem (vietējās pārvaldes vadītājiem), poļu tautības prelāts kļuva par Gnēzenes-Pozenes arhibīskapu. Novadā tika atļauts iebraukt poļiem no citām valstīm~--- jo lauksaimniecībai trūka strādnieku. Poļu deputāti tāpēc atbalstīja likumus, kuri paredzēja Vācijas bruņoto spēku palielināšanu, ar to rādot, ka viņi nav valsts ienaidnieki. Taču antipoļu politikas mīkstināšana bija īslaicīga.

Poļu deputāti bija pieļāvuši kļūdu, cerot uz vācu ``pīrāgu'', drīz atkal viņi ieraudzīja ``pātagu''. Sakarā ar savu uzņemšanu Tornā (Toruņā) ķeizars Vilhelms II 1894.~gada 22.~septembrī paziņoja: ``Man ir kļuvis zināms, ka poļu līdzpilsoņi šeit neuzvedas tā, kā no viņiem to sagaida un var vēlēties. Viņi var sev atgādināt, ka tie tāpat kā vācieši pilnā mērā varēs paļauties uz manu gādību un līdzjūtību tikai tādā gadījumā, ja bez ierunām jutīsies kā Prūsijas pavalstnieki\dots{}''

Atkal sākās vācu valodas uzspiešana, poļu valodas izskaušana no skolām. 1894.~gadā nodibinājās ``Savienība vācietības (\detxti{Deutschtums}) veicināšanai Ostmarkā''. Palielinājās fonds vācu pārceļotāju atbalstam. Pozenes pilsētā 1900.~gadā poļi sastādīja vairs tikai 55\% iedzīvotāju. No 1900.~gada arī reliģijas mācību visās skolās bija jāmāca tikai vācu valodā.

Arī liberāli, demokrātiski noskaņotie vācieši atbalstīja centienus ģermanizēt poļus. Tika uzskatīts, ka Pozene un Rietumprūsija ar Prūsiju un Vāciju jau tā saaugušas kopā, ka šī saikne kļuvusi nesaraujama uz mūžīgiem laikiem. Pat vācu sociāldemokrāti, XIX gadsimta beigās uzskatīja Polijas neatkarību par neiespējamu. Partijas valdes loceklis I.~Auers 1898.~gadā kolēģei no Polijas R.~Luksemburgai paziņoja: ``\detxti{Wir alle fünf im Parteivorstand betrachten die Unabhängigkeit Polens als einen Unsinn, eine Phantasie''; ``Man kann ``den polnischen Arbeitern keinen größeren Gefallen tun, als sie zu germanisieren, aber man darf es den Leuten nicht sagen.''} (``Mēs visi pieci partijas valdes locekļi Polijas neatkarību uzskatām par neprātu, fantāziju''; ``Nevar izdarīt poļu strādniekiem lielāku pakalpojumu kā viņus ģermanizēt, taču to nedrīkst šiem ļaudīm sacīt.'')

Tomēr ģermnizatoru cerētie rezultāti izpalika. Kaut līdz 1914.~gadam uz Vācijas austrumu provincēm bija pārcēlušies ap 120~000 vāciešu, viņu īpatsvars turējās pastāvīgi pie 38\% robežas. Toties poļu iedzīvotāju īpatsvars, neraugoties uz to nemitīgo izceļšanu, no 62,3\% 1887.~gadā pieauga uz 67,8\% 1905.~gadā. Pozenes provincē izauga moderna, pašapzinīga poļu paaudze, kura bija pārdzīvojusi nacionālu konkurenci un mācījusies no tās atteikšanos no kompromisiem. Valsts pasākumi vācu īpatsvara paaugstināšanai, īpaši poļiem piederošās zemes izpirkšana un piešķiršana vāciešiem, nevarēja būtiski ietekmēt iedzīvotāju nacionālo sastāvu, tikai saasināja nacionālpolitisko konfliktu. Piemēram, kāda poļu avīze 1901.~gadā sludināja: ``Neviens polis nedrīkst precēties ar vācieti vai citu svešu meiteni, tas ir nāves grēks \citespace{} Labāk, ja mūsu meitenes paliek neprecētas līdz nāvei, nekā viņas apprecas ar vācieti.''

Mūsdienās vācu vēstures literatūrā ir parasts Prūsijas-Vācijas politiku ķeizaru Vilhelma I un Vilhelma II un īpaši O.F.~Bismarka laikā uzskatīt kā neizdevušos un vērtēt to negatīvi. Nav noliedzams, ka tās mērķi netika sasniegti. Taču H.f.~Zitzevics norādījis, ka šai politikai bija arī ievērojami panākumi, īpaši kulturālajā un saimnieciskajā jomā. Vispārējais izglītības līmenis (1815.~gadā tikai 20\% skolas vecuma bērnu apmeklēja skolas, XIX gadsimta beigās tas, ka visi bērni apmeklēja skolas, bija pats par sevi saprotams), kā arī pilsētnieku un zemnieku, kā arī visu vidusslāņu saimnieciskais stāvoklis pārdzīvoja tādu attīstību, kādu nesasniedza Krievijai un Austroungārijai piederošajos poļu apgabalos. Piemēram skat. \ref{tab:table1}. tabulu.

\noindent
\begin{table}[h!]
\caption{Lauksaimniecības kultūru ražība hektolitros Pozenes provincē un Polijas karalistē pirms Pirmā pasaules kara} \label{tab:table1}
\begin{tabularx}{\linewidth}{|p{5cm}|p{1.5cm}|p{3.5cm}|}
\hline
\strong{Lauksaimniecības kultūra} & \strong{Pozene} & \strong{Polijas karaliste} \\
\hline
Kvieši & 22,0 & 11 \\
\hline
Rudzi & 18,3 & 10 \\
\hline
Mieži & 43,4 & 11 \\
\hline
Auzas & 22,6 & 9 \\
\hline
Kartupeļi & 156,0 & 62 \\
\hline
Cukurbietes & 305,0 & 198 \\
\hline
\end{tabularx}
\end{table}

% page 101


Taču šie panākumi nestiprināja vācu pozīcijas provincē. Vācu iedzīvotāju skaits samazinājās, poļu vidū auga nacionālistiski noskaņojumi. No poļu valodā runājošiem Prūsijas pavalstniekiem tapa poļi ar nacionālo pašapziņu.

Lai nu kā, Pozenē intensīvais poļu zemnieku-preču ražotāju, kuri iekļāvās tirgus ekonomikas mehānismā, veidošanās process, kā likās, radīja priekšnoteikumus zemnieku pilsoniski-demokrātiskas kustības izveidei. Taču ģermanizācijas politika un poļu zemju kolonizācija atbīdīja sociālās pretrunas otrajā plānā, tuvināja dažādus poļu slāņus, vienoja tos nacionālā cīņā pret ģermanizācijas draudiem. Šādos apstākļos Pozenes poļu zemnieki nonāca aktīvākās poļu sabiedrības daļas~--- mantīgo slāņu, vispirms jau nacionāldemokrātu partijas (par to skat. Polijas karalistei veltītajā apakšnodaļā), ietekmē. Līdz pat 1918.~gadam šeit nepastāvēja atsevišķa zemniecības partija.

Tomēr salīdzinājumā ar kaimiņos esošo Polijas karalisti poļu mazākuma stāvoklis Pozenes provincē bija krietni labāks. Lai citējam tikai tur iznākošā laikraksta ``\pltxti{Oredownik}'' (``Aizstāvis'') 1900.~gada 26.~janvāra numuru: ``Ko mūsu tautieši karalistē (Krievijas Polijā) dara slepus, mēs zem prūšu sceptera darām atklāti, izmantojot Prūsijas Konstitūciju, likumus un valsts iestādes. Visur mēs tiecamies un strādājam pie tā, lai uzturētu spēkā mūsu īpašo nacionālo stāvokli un mūsu raksturu saskaņotu ar mūsu pienākumu pret valsti, kurai mēs piederam. \citespace{} Šeit Prūsijā mums ir tiesības uz īpašu nacionālo stāvokli. \citespace{} Mēs zem prūšu sceptera veicam iekšējo darbu, atklāti un legāli stiprinot un attīstot mūsu tautiskumu, tikai pateicoties tam, ka Prūsijas Konstitūcija mums dod tiesisku pamatu šim iekšējam nacionālajam darbam.''

Pozenes \strong{poļu politiskās dzīves} spektrā ilgstoši vadošo vietu saglabāja konservatīvie elementi, kuri arī pārstāvēja vietējos poļus Vācijas reihstāgā, taču ar laiku to lomu sāka apstrīdēt nacionālistiskās grupas, kas darbojās nacionāldemokrātu (par tiem skat. apakšnodaļu par Polijas karalisti) ietekmē. Par šo slāņu preses orgānu kļuva avīze ``\pltxti{Oredownik}'' (``Aizstāvis'').

80.~gados radās pirmie strādnieku pulciņi, 90.~gados Polijas ``prūšu'' daļa sociālistiskajā kustībā darbojās ap 400 cilvēku. 1893.~gadā arī Pozenē (pēc Polijas karalistes) tika nodibināta Poļu sociālistiskā partija Prūsijā (poļu \pltxti{Polska Partia Socjalistyczna Zaboru Pruskiego}, vācu \detxti{Polnische Sozialistische Partei in Preußen}). Vācijas Sociāldemokrātija (\detxti{Sozialdemokratische Partei Deutschlands}) to uzskatīja par savu sastāvdaļu. Turpretī poļu sociālisti uzskatīja savu partiju par pilnībā autonomu. Attīstījās šīs partijas sakari ar Polijas Sociālistisko partiju (PPS) Polijas karalistē.

80.~gadu otrajā pusē un 90.~gados pastāvošajai zemnieku kustībai bija pārsvarā ekonomisks raksturs. Salīdzinājumā ar citās valstīs esošajiem poļu apgabaliem Pozenē politisko partiju ietekme bija neliela.

Īpaša tēma ir jautājums par \strong{ebreju stāvokli} Pozenes provincē. No vienas puses, ebreji, kļuvuši par Prūsijas pavalstniekiem, tagad varēja bez ierobežojumiem pārvietoties uz Berlīni, Breslavu (vācu \detxti{Breslau}, poļu \pltxti{Wrocław}) un citām Prūsijas, vēlāk Vācijas pilsētām. Tas pavēra viņiem jaunas iespējas. No otras puses, apkārtējie iedzīvotāji labprāt redzētu viņu asimilāciju. Taču tā bija saistīta ar viņu emancipāciju, vienlīdzīgu tiesību iegūšanu, ko citu tautu pārstāvji visbiežāk nelabprāt atzina. Pozenes lielhercogistē 1825.~gadā dzīvoja ap 65~000 ebreju (No 1~032~000 iedzīvotāju tas sastādīja 6,3\%). 1843.~gadā viņu skaits palielinājās līdz 80~000. Lielākoties viņi mita lielākajās pilsētās (Pozenē viņu bija 22\% iedzīvotāju), tad mazākajās, pavisam maz laukos. 63,3\% ebreju nodarbojās ar tirdzniecību, naudas maiņu, uzņēmējdarbību, 34\%~--- ar amatniecību.

1833.~gadā visus Pozenes ebrejus ar likumu sadalīja divās daļās: ``naturalizētajos'' un ``nenaturalizētajos''~--- provincē tikai pieciešamajos. Lai naturalizētos, ebrejam bija jāpierāda, ka viņam ir nevainojama reputācija, jāuzņemas saistības visās sabiedriskās vietās lietot vācu valodu, jāpieņem uzvārds, viņam bija jābūt pastāvīgai dzīves vietai Pozenē kopš 1815.~gada, pietiekošiem eksistences līdzekļiem: profesijai vai zemes īpašumam 2~000 tāleru vērtībā, vai arī kapitālam 5~000 tāleru vērtībā. Naturalizāciju varēja sasniegt arī ar ``patriotisku rīcību''. Pieciešamajiem bija jāievēro daudzi ierobežojumi: viņi drīkstēja precēties tikai pēc 24 gadu sasniegšanas, tikai ārpus pilsētām viņiem privātīpašumā vai nomā drīkstēja būt sava saimniecība, viņiem bija aizliegts būt par pauniniekiem, krogus atļauts turēt tikai pilsētās utt. Abas kategorijas tika piespiedu kārtā apvienotas t.s. ``izraēļu korporācijā'',~--- ebreju kopienā. Līdz 1848.~gadam tikai niecīgs skaits~--- 5,5\% ebreju kļuva par ``naturalizētajiem''.

Ebreju asimilācijas nodoms nebija īstenojams bez viņu izglītošanas. Prūsijas varas iestādes ilgstoši šaubījās~--- ierīkot viņiem atsevišķas mācību iestādes, vai ļaut apmeklēt vispārējās. Galu galā ebrejiem atļāva gan apmeklēt kristiešu skolas, gan uzturēt privātskolas ar valsts ierēdņu ``pārbaudītiem'' skolotājiem. 1845.~gadā uz visiem ebrejiem attiecināja kara klausību. Ebreju kopienai lika nodarboties tikai ar savām garīgajām, reliģiskajām lietām, taču viņiem palika slēgta pieeja nodarbei ar izglītību un vispār kultūru. 1850.~gadā paziņoja, ka ebreji nedrīkst kļūt par tiesnešiem, jo ``viņi nedrīkst pieņemt kristiešu zvērestu''. Liela daļa ebreju bija sašutuši. Viens no viņu rabīniem 1848.~gadā paziņoja: ``Mēs esam un gribam būt tikai vācieši! Mēs nevēlamies nekādu citu tēvzemi kā vācu. Tikai pēc ticības mēs esam izraelīti, visā citā mēs piederam valstij, kurā dzīvojam''. Taču ilgstoši liegt formālu līdztiesību vairs nevarēja. 1869.~gadā Ziemeļvācijas savienības parlamentā pieņēma likumu par ebreju vienlīdzību tiesībās ar pārējiem līdziedzīvotājiem. 1871.~gadā ar Vācijas impērijas nodibināšanos likums tika attiecināts uz visu valsti. Taču pēc tam sākās izceļošanas vilnis, kurš būtiski samazināja ebreju skaitu Pozenes provincē. 1861.~gadā viņu Pozenes provincē bija 74~000~ (5\%), 1880.~gadā~--- 57~000 (3,3\%), 1890.~gadā 44~000 (2,5\%), 1910.~gadā~--- 26~500 (1,3\%). Izceļojot ebreji meklēja lielākas iespējas savai saimnieciskajai darbībai. Pēc 1870.~gada proletārisko ebreju izceļotāju vietā arvien vairāk bija pilsonisko. Attīstoties poļu uzņēmēju darbībai, poļu akciju sabiedrībām, poļu uzņēmēji, konkurējot ar vāciešiem, arvien vairāk izspieda ebrejus. Pie tam darbojās arī citi faktori. No 70.~gadiem daudzi ebreji izceļoja tāpēc, ka Pozenē nebija universitātes, citu augstskolu. Bet ebreju vidū tieksme pēc izglītības bija lielāka nekā kristīgo iedzīvotāju vidū.

Arī ebreju pensionāri (pareizāk~--- rantjē) labprātāk savu mūža nogali pavadīja Rietumu pilsētās, arī Berlīnē, kur bija mazāk antisemītisma izpausmju. Pozenē valdošais kastu gars daudziem saindēja dzīvi (Vācu virsnieki, ierēdņi, uzņēmēji un skolotāji dzīvoja nošķirtu dzīvi no citiem vāciešu slāņiem). Tas nenozīmē, ka starp atsevišķiem ebrejiem un vāciešiem vai poļiem personīgā līmeni nepastāvēja draudzīgas attiecības. Taču poļu un vācu attieksme pret ``savu'' ebreju, piemēram, maiznieku vai kurpnieku, varēja ievērojami atšķirties no attieksmes pret svešu šīs tautības piederīgo.

Daļas iedzīvotāju nepatiku izraisīja ebreju nevēlēšanās nostāties kādas nacionālās vai politiskās grupas atbalstītāju rindās. XIX gadsimta beigās un XX gadsimta sākumā īpašu gan vācu, gan poļu kritiku izsauca cionistu uzskatu sludināšana par ebreju valsts dibināšanas Palestīnā nepieciešamību.

\subsection{Polijas karaliste (Vislas novads) Krievijas varā}

Vīnes kongresā par Polijas karalisti (\pltxti{Królestwo Polskie}, \rutxti{Царство Польское}) pasludināto Polijas austrumu daļu sarunu valodā, ņemot vērā Vīnes kongresa lomu tās izveidē, dēvēja arī par ``kongresa Poliju'' (\pltxti{Królestwo Kongresowe}) un ``kongresovku'' (\pltxti{Kongresówka}). Tajā ietilpa apmēram 1/7 agrākās Žečpospolitas teritorijas ar 1/5 iedzīvotāju. Karaliste dalījās 8~vojevodistēs: Augustovas (\pltxti{Augustowska}), Kališas (\pltxti{Kaliska}), Krakovas (\pltxti{Krakowskie}, pati Krakovas pilsēta tajā neietilpa, bet vojevodistes nosaukums deva mājienu, ka ar laiku arī pilsēta varētu pievienoties tai), Ļubļinas (\pltxti{Lubelskie}), Mazovijas (\pltxti{Mazowiecki}), Plockas (\pltxti{Płockie}), Podļesjes (\pltxti{Podlaskie}), Sandomiras (\pltxti{Sandomierskie}).

Runājot par bijušās Polijas apgabaliem Krievijas varā, noteikti jāievēro, ka Polija XIX gadsimtā ietilpa Krievijas impērijas rietumu nomalē. Iedzīvotāju blīvums un skaits šeit bija krietni lielāks nekā Iekškrievijā. Viņu sociāli ekonomiskās un kultūras attīstības līmenis redzami pārsniedza ne tikai citu~--- austrumu, dienvidu, ziemeļu~--- valsts nomaļu, bet arī Krievijas centrālo apgabalu iedzīvotāju attīstības līmeni. 1816.~gadā karalistē mita ap 3,3~miljoni iedzīvotāju. No tiem pēc valodas (ko šai gadījumā var uzskatīt par diezgan precīzu nacionālā sastāva atspoguļotāju) 75\% bija poļi, 10\% ebreji, 7,5\% vācieši, 5\% lietuvieši un 2,5\% ukraiņi. Pēc ticības 83,5\% bija katoļu, 2,5\% uniātu,10\% jūdaistu un 3,75\% protestantu. Pēc citiem datiem ap 73\% iedzīvotāju bija zemnieki, 12\% sīkpilsoņi, 7,5\% šļahtič, 7,5\% beztiesīgie ebreji. (Dati par ebrejiem ir atšķirīgi, jo viņi izvairījās no tautas skaitīšanas).

Vēl pirms Polijas otrās dalīšanas 1791.~gadā ar imperatores Katrīnas II dekrētu Krievijas teritorijā tika izsludināta nometinājuma josla (\rutxti{Черта оседлости}), kurā drīkstēja dzīvot ebreji. (Citur drīkstēja apmesties tikai kristītie ebreji, kā arī jūdaisma ticību saglabājušie 1.~ģildes tirgoņi, personas ar augstāko izglītību, reģistrētās prostitūtas, bijušās militārpersonas u.c.). Pēc otrās Polijas dalīšanas tajā ietilpa arī Latgale. Pēc trešās Žečpospolitas dalīšanas nometinājuma joslā ietilpa arī Viļņas un Grodņas guberņas, kur bija daudz ebreju. Nometinājuma josla aptvēra speciāli noteiktas pilsētas un miestiņus, jo ebrejiem laukos arī bija aizliegts dzīvot. Galīgu juridisku noformējumu nometinājuma josla ieguva ar 1804.~gadā izdotajiem ``Noteikumiem ebreju iekārtošanai'' (``\rutxti{Положение об устройстве евреев}''), kas uzskaitīja tās guberņas un teritorijas, kurās ebreji drīkstēja apmesties un tirgot. Pēc Vīnes kongresa ebreju nometinājuma joslā nonāca arī visas Polijas karalistes guberņas.

Mūsdienu krievu sociālās vēstures pētnieks B.~Mironovs ir izdalījis vairākus Krievijas politikas principus tās inkorporētajās nacionālajās nomalēs.

\begin{enumerate}
\item Krievija ilgstoši tajās saglabāja to administratīvo kārtību, vietējos likumus un iestādes, zemes īpašuma attiecības, ticību, valodu un kultūru, kas pastāvēja pirms to pievienošanas Krievijai. Pagāja vairākas desmitgades, līdz pakāpeniski Viskrievijas kārtība tika attiecināta arī uz nomalēm. Cara valdība plaši sadarbojās ar vietējām feodālajām elitēm, kuras saņēma tādas pat tiesības kā krievu muižnieki. Tas centrālajai varai atviegloja jauno teritoriju pārvaldīšanu.

\item Nekrieviem tika radītas dažas priekšrocības salīdzinājumā ar krieviem. Tautas, kuras nepazina dzimtbūšanas attiecības, ar pievienošanu Krievijai tā arī neuzzināja, kas tā tāda. Ar nodokļu sistēmas palīdzību, kur nekrievu tautas maksāja mazākus nodokļus, tika radīts stāvoklis, ka nacionālajās nomalēs dzīvojošo nekrievu materiālais stāvoklis bija relatīvi labāks nekā krievu stāvoklis. (To atspoguļoja tāds apkopojošs labklājības rādītājs kā vidējais augums. Tā 1874.--1883.~gadā armijā iesaucamo vidējais augums no 28 lielkrievu guberņām bija 1~620~mm, bet no 22 pārējām guberņām~--- 1~627~mm). Teiktais nenozīmē, ka nekrievi neizjuta carisma spaidus, taču krieviem šai ziņā bija dažas ``priekšrocības''.

\item Etniskie kritēriji nebija noteicošie virzībā pa sociālajām kāpnēm. Krievijas politiskā, militārā, kultūras un zinātnes elite bija daudznacionāla. Uzticība tronim, dižciltība un profesionālisms tika vērtēti augstāk kā etniskā vai konfesionālā piederība. Ierēdņu vidū 1850.~gadā nekrievu bija 16\%, augstākās birokrātijas sastāvā 1853.~gadā nekrievu bija 32,7\%, 1917.~gadā~--- 11,8\%. Armijas virsnieku vidū 1912.~gadā nepareizticīgo, tātad cittautiešu, bija 11\%.
\end{enumerate}

Kopumā šo principu darbību varēja vērot arī Polijas karalistē, taču ar vienu būtisku papildinājumu, uz kuru norādījis arī pats B.~Mironovs. Ja Krievijai pievienotās tautas demonstrēja lojalitāti centrālajai varai, tad to zināmā autonomija palielinājās (kā Somijai XIX gadsimtā), ja tās izrādīja naidīgumu vai separātismu~--- tā tika atņemta. Polijas karaliste bija visspilgtākais piemērs, kā notikumi attīstījās pēc otrā varianta.

Īpaši jāuzsver, ka politisko procesu intensitāte Polijas karalistē bija neizmērojami augstāka nekā citās nomalēs. Ekonomiski atrašanās milzīgās impērijas sastāvā bija izdevīga Polijas karalistes rūpniecības attīstībai. Karalistes tautsaimniecība attīstījās krietni straujāk nekā visas Krievijas saimniecība kopumā. Taču pēc diviem nesekmīgiem sacelšanās mēģinājumiem poļiem tika atņemtas daudzas politiskās tiesības, ieviesti daudzi ierobežojumi. Tas, tāpat kā atmiņas par Žečpospolitas slaveno pagātni un arī konfesionālās atšķirības, poļu tautā izraisīja pastāvīgu neapmierinātību par atrašanos Krievijas impērijas sastāvā.

Poļu sabiedrība dažādi vērtēja situāciju un reaģēja uz laikmeta izaicinājumiem. Jau XVIII gadsimta beigās poļu dzejnieks K.~Kozmjans par poļu dzīvi Krievijas rietumu guberņās rakstīja: ``No zināma skatu punkta mēs šeit dzīvojam labāk nekā republikas [domāta Žečpospolita~--- V.Š.] laikā; mēs lielā mērā saglabājam to, ko mums deva dzimtene. Mums tagad nenākas baidīties no Umaņas slaktiņa (\pltxti{Rzeź humańska,}~--- 1768.~gadā sacēlušos ukraiņu zemnieku veikta Umaņas pilsētas un tās apkārtnes iedzīvotāju masu slepkavošana, kurā pēc dažādiem datiem gāja bojā no 12 līdz 20 tūkstošu ebreju, poļu un ukraiņu-uniātu) atkārtošanās; kaut Polijas vairs nav, mēs dzīvojam Polijā, un mēs~--- poļi''. Taču vairākums poļu šļahtiču sapņoja par nedalītas savas varas atjaunošanu neatkarīgā Polijā.

Sākotnēji Krievijas poļu zemes pārvaldīja Pagaidu augstākā padome, radīta 1813.~gada 1.~martā, Polijā ienākot Krievijas karaspēkam. Taču Pagaidu padome nebija vienīgā pārvaldes iestāde. 1814. gada beigās par poļu armijas virspavēlnieku tika iecelts Krievijas cara brālis lielkņazs Konstantīns, kurš faktiski pildīja vietvalža funkcijas. Pēdējās Vīnes kongresa dienās~--- 1815.~gada 22.~maijā tika parakstīti ``Polijas karalistes Konstitūcijas pamati''. Dokumentā bija uzsvērta Polijas karalistes saistība personālā ūnijā ar Krieviju, kā arī tas, ka tās Konstitūcijai jābalstās uz vēl pirms trešās Žečpospolitas dalīšanas 1791.~gadā 3.~maijā pieņemtās Konstitūcijas principiem.

Jaunās \strong{Polijas karalistes Konstitūcijas} sagatavošanā aktīvi piedalījās Krievijas imperatoram Aleksandram I tuvu stāvošais bijušais Krievijas ārlietu ministrs (1804--1806) kņazs Ā.~Čartorijskis, grāfs L.~Plāters un citi poļu magnāti. Taču tā paredzēja pārāk lielas tiesības nacionālajam Seimam, un Aleksandrs I lika tās sašaurināt, vienlaikus paplašinot monarha tiesības. Ā.~Čartorijskim Aleksandrs I gan to skaidroja ar negatīvo noskaņojumu pret poļiem Krievijas sabiedrībā: ``Poļu armijas uzturēšanās veids pie mums, laupīšanas Smoļenskā un Maskavā, visas valsts nopostīšana ir atdzīvinājušas agrāko naidu''. Tomēr jaunā Polijas karalistes Konstitūcijas noteiktā iekārta bija liberālāka nekā savulaik Varšavas hercogistē pastāvējusī.

Pēc labojumiem 1815.~gada 17.novembrī Aleksandrs I Konstitūciju parakstīja. Tā atzina Polijas karalistes mūžīgu saistību ar Krievijas impēriju, abu ārējo politiku realizēja Ārlietu ministrija Pēterburgā. Katoļu reliģija bija pielīdzināta pārējām, taču atradās īpašā valdības aizbildniecībā.

Pēc Konstitūcijas izpildvaru īstenoja karalis (vienlaikus Krievijas imperators), likumdošanas varu dalot ar Seimu. Karalim piederēja likumdošanas iniciatīva un \latxti{veto} tiesības. Viņš varēja pasludināt karu, slēgt līgumus, iecelt visus ierēdņus.

Pasīvās vēlēšanu tiesības tika piešķirtas visiem (izņemot ebrejus, kuriem politisko tiesību nebija), kas maksāja ne mazāk kā 100~zlotu tiešo nodokļu un bija sasnieguši 30 gadu vecumu. Aktīvās vēlēšanu tiesības baudīja šļahtiči~--- zemes īpašnieki no 21~gada vecuma, kā arī garīdznieki, skolotāji, amatnieku darbnīcu īpašnieki, nomnieki un tirgotāji, kuru īpašumā bija preces 10~tūkstošu zlotu vērtībā. Kopā vēlēšanu tiesības ieguva ap 3\% no iedzīvotājiem. (Salīdzinājumam, Francijā 1820.~gadā vēlēšanu tiesības bija tikai 80~000, bet Polijas karalistē~--- 100~000 iedzīvotāju. Tas gan bija saistīts nevis ar imperatora demokrātismu, bet ar muižnieku lielāku īpatsvaru Polijā nekā Francijā.) Seimā pastāvēja divas palātas. Augšpalātu (\pltxti{Izba Senatorska}) nozīmēja monarhs, apakšpalātā (\pltxti{Izba Posolska}) bija 128 pakāpeniskās vēlēšanās ievēlēti deputāti. 77~--- no dižciltīgajiem, 51~--- no pilsētu un lauku kopienām.

Pārvaldes priekšā atradās imperatora iecelts vietvaldis. Pirmais vietvaldis bija jau minētais ģenerālis J.~Zaijončeks~--- bijušais jakobīnis, kurš bija dienējis Napoleona armijā, taču pēc tam uzticīgi kalpoja Krievijai. Vietvaldim palīgā tika nozīmēta Valsts padome (\pltxti{Rada Stanu}), un Administratīvā (jeb Pārvaldes) padome (\pltxti{Rada Administracyjna}). Valsts padome, sastāvoša no ministriem, padomes locekļiem, un monarha īpaši uzaicinātām personām, sastādīja likumprojektus, apsprieda atsevišķu ministriju ziņojumus, risināja juridiskus strīdus u.tml. Administratīvajā padomē, kuras rokās atradās izpildvara, ietilpa ministri un atkal monarha īpaši uzaicinātas personas. Tika nodibinātas piecas komisijas (ministrijas): Kara, Tieslietu, Iekšlietu, Finanšu, Ticības lietu un Tautas izglītības. Ministri parasti nāca no poļu aristokrātijas vidus. Karalistes teritorija tika sadalīta astoņās vojevodistēs: Augustovas, Kališas, Krakovas, Ļubļinas, Mazovijas, Plockas, Radomas un Sandomežas (\pltxti{Województwo augustowska, kaliskie krakowskie, lubelskie, mazowieckie, płockie, radomske, sandomierskie}).

Krievijas armijas sastāvā atradās Poļu korpuss, kurš saglabāja savu formu. Miera laikā pastāvēja 30--35~000 vīru armija, to drīkstēja izmantot dzimtenes aizsardzībai tikai karalistes robežās un tās lielums bija atkarīgs no karalistes budžeta. Izdevumi armijas uzturēšanai sastādīja lielu budžeta daļu. Polijas armijas apbruņojumu pilnībā ražoja Krievijas fabrikās. Saglabājās arī poļu ordeņi: Baltā ērgļa (\pltxti{Order Orła Białego}), Svētā Staņislava (\pltxti{Order Świętego Stanisława}) un Kara nopelnu (\latxti{Virtuti Militari}). Par kara ministru un poļu armijas virspavēlnieku tika nozīmēts jau minētais cara brālis lielkņazs Konstantīns, ne visai psihiski nosvērts cilvēks, kurš iejaucās arī citu ministru kompetencēs, jo ģenerālis J.~Zaijončeks bija vāja rakstura cilvēks, nevēlējās ķildoties ar lielkņazu. Lielkņazs pat apguva poļu valodu, kaut karalistē šļahtas ikdienas sarunu valoda bija franču. Kad Aleksandrs I 1819.~gadā Konstantīnam pakļāva arī ārpus Polijas karalistes esošo Krievijas rietumu guberņu karaspēku, poļos pat radās utopiskas cerības, ka varētu notikt ``atkalapvienošanās'' ar šiem kādreiz Žečpospolitas sastāvā esošajiem apgabaliem.

Konstitūcijā tika deklarēta personas neaizskaramība, ticības un preses brīvība, tiesu neatkarība, poļu valoda tika atzīta par oficiālu karalistes valodu, kuru lietoja arī administrācija, tiesas un armija. Visi valsts amati tika piešķirti poļiem. Katram nākamajam Krievijas imperatoram bija atsevišķi jākronējas ar Polijas karalistes kroni, jāzvēr uzticība Polijas Konstitūcijai. Neraugoties uz Aleksandra I ieviestajiem ierobežojumiem, kas pavēra plašas iespējas Konstitūcijas garantiju vājināšanai, Polijas Konstitūcija tobrīd bija viena no pašām liberālākajām Eiropā. 1818.~gadā imperators Aleksandrs~I, personīgi atklājot Seima sēdi, paziņoja: ``Iepriekšējā zemes organizācija man ļāva ieviest to, kuru es jums dāvāju, iedarbinot liberālas iestādes. Šīs pēdējās vienmēr būs manu rūpju objekts, un es ar Dieva palīgu ceru izplatīt to ietekmi uz visām zemēm, kuras ar Dieva gribu dotas man pārvaldē.'' Seims pieņēma visus likumprojektus, kurus piedāvāja imperators, izņemot likumprojektu par civillaulību. Aleksandrs~I bija apmierināts ar tā darbību.

Pirmajos Polijas karalistes pastāvēšanas gados poļu varas iestādes baudīja iespēju noteikt karalistes sociāli ekonomiskās politikas virzību, reāli izmantoja administratīvo varu un naudas resursus. Taču samērā liberālās Konstitūcijas ieviešana drīz nonāca pretrunā ar Krievijas imperatora patvaldniecisko varu, viņa valdības un galma tieksmi tādu realizēt arī Polijas karalistē.

Pēc 1826.~gada īpašs vietvaldis karalistē vairs netika iecelts, tā pienākumus pildīja lielkņazs Konstantīns. Viņš neievēroja Karalistes Konstitūciju, iedzīvotāju tiesības. Tas izsauca lielu vietējo iedzīvotāju neapmierinātību. Tomēr lielkņazs Konstantīns vēlāk kļuva iecietīgāks, to uzskatīja par viņa ``polonozācijas'' sekām, jo viņu ietekmēja morganiskā sieva~--- poliete I.~Grudzinska, kura ieguva kņazienes Lovičas titulu. Precību ar viņu dēļ lielkņazs bija spiests atteikties no tiesībām uz Krievijas troņa mantošanu.

Polijas karalistē izmaiņas notika arī baznīcas organizācijā. 1818.~gadā pāvests noteica 8 katoļu bīskapiju un 1 uniātu bīskapijas robežas. Varšavas bīskaps kļuva par Polijas karalistes primasu (latīņu~--- \latxti{primas}~--- pirmajā vietā esošs, Romas katoļu baznīcā valsts galvenā bīskapa goda apzīmējums). Uniātu baznīcai piederēja 287 draudžu baznīcas ar 220~000 ticīgajiem, 4 klosteri ar 25 mūkiem un neliela seminārija.

Kopumā politiskā iekārta Polijas karalistē sākotnēji bija liberālāka nekā pārējā Krievijas impērijas daļā. Taču drīzumā cariskā administrācija sāka būtiski ierobežot konstitucionālās tiesības un brīvības. 1819.~gadā tika ieviesta cenzūra, 1821.~gadā bija spiests atstāt dienestu Polijas karalistes izglītības ministrs S.~Potockis, kura laikā tika atvērta Varšavas universitāte un ievērojami pieauga skolu skaits. 1825.~gadā, gatavojoties Seima sasaukšanai, tika pieņemts Konstitūcijas ``papildus pants'', kurš atcēla Seima sēžu atklātumu.

Žečpospolitas otrā un trešā dalīšana pārtrauca XVIII gadsimta beigu augšupejošo poļu \strong{ekonomikas} attīstības procesu, kara darbība poļu zemēs XVIII gadsimta sākumā atnesa sev līdzi lielus materiālus zaudējumus Polijai. Taču ekonomiskās attīstības rezultāts~--- feodāli dzimtbūtniecisko attiecību pagrimums noteica arī turpmāko Polijas karalistes ekonomisko attīstību. Sākotnēji ekonomikas jomā Polijas karaliste bija gandrīz neatkarīga no Krievijas. Saglabājās naudas vienība zlots (\pltxti{złoty}). Taču finanšu stāvoklis pirmajos gados bija smags. Karalistei bija jāmaksā agrāk izveidojušies parādi Austrijai un Prūsijai. Pirmajos gados karaspēka uzturēšanai Krievija piešķīra karalistei avansu, taču no 1817.~gada armijas uzturēšana pilnībā gūlās uz karalistes budžetu, bet jau izmaksātais avanss tika pievienots parādiem. Armijas pavēlnieks lielkņazs Konstantīns pieprasīja arvien lielākus izdevumus tās labā. Karalistes iestādes, ievācot nodevu parādus par iepriekšējiem gadiem, tos ieskaitīja tekošajā budžetā. Nolīdzināt iepriekšējo budžetu deficītu nekādi neizdevās. 1821.~gadā Aleksandrs I pat piedraudēja poļiem atņemt finansiāli-ekonomisko autonomiju, ja tie nespēs tā organizēt valsts kasi, lai uzturētu savu armiju.

Ekonomiskie procesi pirmajos Karalistes pastāvēšanas piecpadsmit gados ietvēra sevī divas strāvas. No vienas puses, tie balstījās uz Apgaismības laikmeta, Četrgadu Seima un Varšavas hercogistes laikā veikto reformu pamatiem. No otras puses daudzas norises ekonomikā iedīglī jau nesa jaunas kvalitatīvas izmaiņas, kuras sagatavoja pāreju uz jaunu kapitālistisko attīstības fāzi. Šim periodam bija raksturīga jaunu ražošanas formu~--- fabriku (no latīņu ~\latxti{fabrica}~--- darbnīca, rūpniecības uzņēmumu ar lielāku nekā darbnīcā strādnieku skaitu, parasti ražojošs preces nevis pēc pasūtījuma, bet tirgum), jaunu ražošanas attiecību~--- algota darba, jaunu sociālo grupu~--- algoto strādnieku, buržuāzijas veidošanās, veco ražošanas attiecību transformāciju jaunajās, kapitālistiskajās.

Ja XVI--XVIII~gadsimtā Žečpospolita Rietumeiropas valstīm piegādāja galvenokārt labību un kokmateriālus, XIX gadsimta sākuma kari izpostīja pastāvējušo darba dalīšanas sistēmu. Rūpniecības attīstība Rietumeiropā veda pie labības tirgus paplašināšanās un muižu izaugsmes uz zemnieku zemes rēķina Austrumeiropā. 1815.~gadā Polijas karaliste bija vēl agrāra zeme, 80\% iedzīvotāju dzīvoja no lauksaimniecībā gūtajiem ienākumiem, 70\% no nacionālā ienākuma deva zemes īpašums. Pilsētas bija attīstītas vāji. Lielākā bija Varšava ar vairāk nekā 100~000 iedzīvotāju. Pavisam Polijas karalistē bija ap 480 pilsētu, taču tikai 59 no tām iedzīvotāju bija no 3 līdz 10 tūkstošiem. 1816.~gadā pilsētās dzīvoja 19,4\% visu iedzīvotāju.

Varšavas hercogistē 1808.~gadā ieviestais Napoleona kodekss saglabāja savu spēku arī Polijas karalistē. Zeme atradās muižnieku, valsts un zemnieku īpašumā. Darbojās 1807.~gada Varšavas hercogistes Konstitūcijas pants par zemnieku atbrīvošanu no dzimtbūšanas.

Parasti muižniekiem nepiederēja mājlopi un zemes apstrādei nepieciešamais inventārs, viņu zemi klaušu kārtībā apstrādāja zemnieki ar savu inventāru. Muižnieki maz lietoja algotu darbu. No muižniekiem tikai mazai daļai (ap 4~000) piederēja lielas muižu saimniecības~--- latifundijas (latīņu \latxti{lātus} ``plašs'' + \latxti{fundus} ``ferma, nekustams īpašums''). No laukos dzīvojošajiem vairāk nekā 40~000 šļahtiču ap 4~100 nomāja muižas, vairāk nekā 32~000 sīko šļahtiču vai nu nebija savas zemes vai tie bija nelieli zemes gabali. Šie sīkie šļahtiči pēc ekonomiskā stāvokļa bija tuvi zemniekiem, tikai nepildīja feodālās saistības. Poļu vēsturnieks V.~Smoļenskis, kurš pētījis sīko šļahtiču dzīvi, rakstīja, ka viņu īpašumā parasti bija vairāki sīki skrejgabaliņi, un tāpēc tie bija atkarīgi no kaimiņiem, nespēja patstāvīgi saimniekot uz savas zemes.

Valstij piederēja t.s. valsts jeb nacionālās muižas, kuras bija izveidotas no bijušajām karaļu, klosteru, kā arī franču maršaliem un ģenerāļiem atsavinātajām zemēm.

Zemnieku zeme parasti vēl neatradās to pilnīgā privātīpašumā, bet gan bija aplikta ar dažādām feodālajām saistībām. (Klaušas sasniedza ne mazāk, bet atsevišķos gadījumos arī vairāk nekā 3 līdz 4 dienas nedēļā.) 1815.~gada Konstitūcija gan apstiprināja jau ar 1807.~gada Konstitūciju un Napoleona kodeksu pasludinātās zemnieku tiesības uz brīvību, bet muižnieku tiesības uz zemi, taču punkts par zemnieku tiesībām lielākoties bija fiktīvs, jo no 1818.~gada par vietējo tiesu iestāžu, kur nonāca zemnieku sūdzības, vadītājiem kļuva muižnieki. Pret zemniekiem, kuri atteicās pildīt feodālās saistības, vajadzības gadījumā tika sūtīts karaspēks. Vēl XIX gadsimta 20.~gados Polijas karalistē klaušu zemnieki sastādīja 51\% no visiem zemniekiem. Zemnieku pusnaturālo saimniecību pārvēršanās par preču ražotājām norisa visai gausi.

Sākotnējā kapitāla uzkrāšanās Polijas karalistē bija raksturīga ar lielāku nekā citās Eiropas valstīs muižnieku lomu šai procesā, ievērojamu valsts lomu, kas veicināja kapitāla uzkrāšanos lielo zemes īpašnieku un dažādo topošās buržuāzijas grupu rokās, ārējo avotu izpalikšanā. Sākotnēji izšķiroša loma kapitāla uzkrāšanā bija muižniekiem un valstij, mazāka~--- tirgotājiem un augļotājiem. Kā uzsvērusi krievu vēsturniece L.~Obušenkova XIX gadsimta pirmajā trešdaļā Karalistē sākotnējā kapitāla uzkrāšana rūpniecībā atpalika no tā uzkrāšanas tirdzniecībā, bet pēdējā~--- no tā uzkrāšanas muižnieku rokās lauksaimniecībā. Folverki (no vācu \detxti{Vorwerk}, poļu \pltxti{folwark}) jeb lopu muižas, pusmuižas kļuva par preču lielražošanas centriem, pārvērtās par puskapitālistiskiem, pusfeodāliem uzņēmumiem. Kapitāla uzkrāšana lauksaimniecībā nozīmēja galvenokārt tā uzkrāšanu to muižnieku rokās, kuru saimniecības sāka evolucionēt pa kapitālisma ceļu. Notika tas galvenokārt divos veidos~--- padzenot zemniekus no viņu apstrādātās zemes, lai to pievienotu muižai, un izpārdodot valsts muižas. Līdz 1830.~gadam bija pārdotas ap 150 valsts muižas, kas veda pie muižnieku zemes īpašuma pieauguma. Tiesa, bija vairāki gadījumi, kad valsts muižas pārdeva arī zemniekiem. Tad viņi ieguva šo zemi pilnā privātīpašumā. Muižnieki kopumā zemi zemniekiem nevēlējās pārdot. Tiesa, var pieminēt gadījumu, kad jau agrāk minētais poļu filozofs, rakstnieks un mecenāts S.~Stašičs pirms nāves novēlēja savas zemes sadalīt zemniekiem. Šis eksperiments kļuva par savdabīgu pieminekli XVIII gadsimta otrajā pusē tā arī neīstenotajiem sociālajiem ideāliem. Taču tas bija izņēmuma gadījums.

1827.~gadā ap 30\% no visiem lauku iedzīvotājiem bija bezzemnieki. Ja tiem pieskaita vēl sīku zemes gabalu turētājus, no kuriem to ģimenes nevarēja pārtikt, šis skaitlis pieauga līdz 50\%. Bezzemnieki tika nodarbināti muižās kā algota darba strādnieki.

Poļu historiogrāfijā ir izteikti divi galvenie viedokļi par kapitālistisko attiecību iespiešanos Polijas karalistes lauksaimniecībā. Profesors A.~Grodeks uzskatīja, ka XVIII gadsimta beigās un XIX gadsimta pirmajā pusē zemnieku padzīšanai no zemes nebija nekā kopīga ar kapitāla pirmatnējo uzkrāšanu, ka zemnieku noslāņošanās bija ar feodālu, nevis kapitālistisku raksturu. Tomēr vairākums poļu vēsturnieku, kuri pētījuši agrārās problēmas Polijas karalistē, tam nepiekrīt. Viņi muižnieku realizēto zemnieku padzīšanu no zemes, viņu zemes pievienošanu muižām saista ar kapitālistisko attiecību veidošanos lauksaimniecībā.

Norisa izmaiņas lauksaimnieciskās ražošanas struktūrā. Pieauga mājlopu skaits, palielinājās sugas lopu audzēšanas gadījumi. Tā, ja 1822.~gadā karalistē bija 1~527~tūkstošu smalkvilnas aitu, tad 1827.~gadā~--- jau 2~477~tūkstoši. Kopumā tomēr lopkopība atradās zemā attīstības līmenī. 1827.~gadā uz vienu lauku iedzīvotāju bija 0,129~zirgi un 0,463~ragulopi. Vasarās mājlopus izveda ganībās, bet ziemās baroja ar salmiem. Sienu zirgiem un vēršiem deva tikai smagāko lauku darbu periodos.

Viens no svarīgākajiem lauksaimniecības attīstības faktoriem bija kartupeļu audzēšana, kam grūdienu deva 1816.--1817.~gadu neražas. Kartupeļu un lopbarības kultūru iekļaušanu sējas apritē veicināja pāreju no trīslauku uz daudzlauku sistēmu, jaunas lauksaimniecības tehnikas izmantošanu. Lielus ienākumus muižniekiem deva degvīna dedzināšana. 1829.~gadā tika uzcelta pirmā cukurfabrika.

Tomēr līdz 1830.~gadam kapitālisma attīstība lauksaimniecībā tikai vēl sākās. Galvenā feodālās rentes forma bija klaušas. Straujāk kapitālisma virzienā evolucionēja muižu saimniecības Polijas karalistes rietumos.

Viens no kapitāla uzkrāšanas avotiem bija tirgotāju un augļotāju kapitāla uzkrāšana un pārvēršanās par rūpniecisko kapitālu. Pirmajos Polijas karalistes pastāvēšanas gados sākotnējā kapitāla uzkrāšanās procesu negatīvi ietekmēja pilsētu un vietējā tirgotāju kapitāla vājums. Lieltirgotāji koncentrējās galvenokārt Varšavā, citās pilsētās pārsvarā bija sīktirgotāji, augļotāji un pārpircēji.

Vīnes kongresā bija pieņemts lēmums, ka starp poļu zemēm, kuras pēc dalīšanām bija nonākušas dažādu valstu sastāvā, jāpastāv brīvai tirdzniecībai, taču šis lēmums tika ievērots tikai pirmajos gados. Pēc Polijas karalistes nodibināšanas tās ekonomiskajai attīstībai pavērās divi ceļi: 1) frītrēderisma (angļu \entxti{free trade}~—brīvā tirdzniecība, politika, kas atbalsta brīvu preču un pakalpojumu kustību pāri administratīvajām robežām un valsts neiejaukšanos privātražošanas sfērā) un 2) protekcionisma (no latīņu \latxti{protectio}~--- aizbildniecība, aizsardzība; iekšējā tirgus aizsardzības politika pret konkurenci ar zināmu ierobežojumu: importa un eksporta nodevu, subsīdiju u.c. līdzekļiem). Attīstības ceļa izvēlē izšķiroša bija valsts loma.

Frītrēderisma politiku aizstāvēja Polijas karalistes iekšlietu ministrs grāfs T.~Mostovskis. Viņš cerēja izmantot poļu zemju ģeogrāfisko stāvokli un atjaunot Polijas kā starpnieka lomu tirdzniecībā starp Rietumeiropu un Tuvējiem austrumiem. Tāpēc viņš uzstājās kā Varšavas gadatirgu iniciators. Par to noturēšanu divreiz gadā~--- pavasarī un rudenī~--- vietvaldis paziņoja 1817.~gada februārī. Gadatirgi tika uzskatīti par līdzekli kā likvidēt budžeta deficītu. 1818.~gadā uz gadatirgiem ieradās vairāk nekā 200 ārzemju tirgotāju. Brīvās tirdzniecības politikas realizācijā bija ieinteresēti arī muižu īpašnieki kā labības pārdevēji un kā ar muitas nodevu neapliktu ārzemju preču pircēji. 1816.--1819.~gadā Polijas karalistē pastāvēja brīvās tirdzniecības sistēma. Muitas tarifs vadījās tikai no valsts kases fiskālajām interesēm.

Taču nacionālās tautsaimniecības attīstībai tolaik vairāk atbilstoša bija protekcionisma politika. Enerģiski pret brīvās tirdzniecības politiku iestājās Polijas tirgotāji. Īpaši viņi iebilda pret privilēģiju piešķiršanu ārzemju tirgotājiem Varšavas gadatirgos.

1820.~gadā Polijas karaliste tika iekļauta Krievijas impērijas muitas robežās. Faktiski cara valdība neiebilda, ka arī poļi pāriet uz šādu sistēmu. Tādejādi kā Krievijā, tā Polijas karalistē tika ieviesta vienota muitas sistēma. Muitas robežas atcelšana starp Polijas karalisti un Krieviju radīja noieta tirgu poļu precēm, Tās bija kvalitatīvākas nekā dzimtbūtnieciskajā Krievijā izgatavotās, sāka strauji iekarot Krievijas (daļēji arī Ķīnas) tirgu. Taču, no otras puses, Krievijas varas iestādes pārliecinājās, ka Polijas karalistes iekļaušana vienotā muitas sistēmā un vienotas muitas robežas noteikšana Krievijai nav izdevīga. Krievijas rūpniecībai nebija izdevīga konkurence ar rūpniecības ražojumiem, kurus varēja ievest bez muitas no Austrijas un vācu valstiņām vispirms Polijas karalistē, bet no turienes arī Krievijā. Turpretī rūpniecības attīstība pašā Polijas karalistē Krievijai zināmā mērā bija izdevīga, jo tā sākotnēji netraucēja Krievijas rūpniecības attīstību. Krievijas tirgus pat bija ieinteresēts saņemt dažus poļu tekstilrūpniecības izstrādājumus. Muitas nodevas lielumu noteica krievu ierēdņi, pirmkārt ievērojot Krievijas rūpniecības intereses, kuras tika noskaidrotas pastāvīgos strīdos.

1820.--1821.~gadā pirmo vietu Polijas karalistes tirdzniecībā ieņēma Prūsija, otro un trešo~--- Krievija un Austrija. Karalistei ārējā tirdzniecībā bija negatīvs balanss, vairāk preču tika ievests nekā izvests. Tā bija pārpludināta ar ārzemēs ražotajām tekstila un metālapstrādes precēm. 1820.~gadā karalistē tika pārdots ārzemēs ražoto audumu par 24~miljoniem zlotu, bet vietējo, poļu ražoto~--- par 1 miljonu zlotu, ārzemēs ražotais metāls un tā izstrādājumi tika pārdots par 3~miljoniem zlotu, bet poļu metāls un ražojumi~--- par 707~tūkstošiem zlotu. Tādejādi zuda apstākļi tekstil un metālrūpniecības attīstībai pašā Polijas karalistē.

Polijas karalistes valdības politika strauji mainījās, kad par tās finanšu ministru kļuva F.~Druckis-Ļubeckis, kurš bija konsekvents protekcionisma piekritējs, un ar jauna muitas tarifa ieviešanu Krievijā 1822.~gada martā, kurš Polijas karalistei atjaunoja muitas autonomiju. Karalistē tika ieviests savs protekcionistisks muitas tarifs, kurš noteica augstas muitas nodevas kā no ārzemēm ievedāmajai lauksaimniecības produkcijai, tā rūpniecības precēm un greznumpriekšmetiem. Toties tirdzniecībā starp karalisti un Krieviju muitas nodevas bija krietni mazākas. Izejvielu un neapstrādātu izstrādājumu tirdzniecībā starp tām nodevas vispār tika atceltas.

Šis tarifs aizsargāja veidojošos vietējo rūpniecību no lēto austriešu un vācu preču konkurences, bet muitas nodevas kļuva par vienu no galvenajiem budžeta ienākumu avotiem. F.~Druckim-Ļubeckim izdevās faktiski aizliegt vilnas rūpniecības izstrādājumu, īpaši vadmalas audumu, ievešanu karalistē un izmantot lielās austrumu tirgus iespējas poļu vilnas tekstilizstrādājumu pārdošanai. Poļu tirgotāji un rūpnieki guva ievērojamu peļņu no muitas tarifu samazināšanas ar Krieviju. Tā, 1828.--1830.~gadā par vadmalas ievešanu Krievijā pēc atvieglotā tarifa viņi samaksāja 1,4~miljonus rbļ. muitas nodevu. Ja tiktu lietots parastais ārējās tirdzniecības, bet nevis atvieglotais, Polijas karalistei piemērojamais muitas tarifs, viņiem nāktos samaksāt nodevās 74,54 miljonu rbļ.

Pāreja uz protekcionismu gan izraisīja t.s. muitas karus ar Prūsiju (1823--1825) un Austriju (1824--1828). Šo valstu (īpaši Austrijas) daļa Polijas karalistes tirdzniecībā samazinājās, toties auga tirdzniecība ar Krieviju. Tiesa, Polijas karalistes tirdzniecības apjoms ar rūpnieciski attīstīto Prūsiju joprojām bija lielāks nekā ar Krieviju. 1826.~gadā pirmo reizi Polijas karalistes tirdzniecības bilance ar Krieviju kļuva aktīva. Muitas sistēma, kad uz robežas ar Prūsiju un Austriju muitas nodevas bija augstas, bet ar Krieviju~--- zemas, Polijas uzņēmējiem bija izdevīga, deva papildus priekšrocības. Kā raksta jau minētā krievu vēsturniece L.~Obušenkova, abu valstu: Krievijas un Polijas karalistes muitas tarifi veicināja lauksaimniecības un tekstilrūpniecības attīstību karalistē. Liela nozīme bija arī brīvam poļu preču tranzītam uz Krievijas ostām pie Baltijas jūras, arī ārzemju preču ievešanai Polijas karalistē caur Liepāju, Rīgu un Ventspili. Mūsdienu poļu vēsturnieki atzīst, ka no Krievijas puses tai laikā nebija Polijas ekonomiskas ekspluatācijas.

F.~Druckis-Ļubeckis pilnveidoja arī valsts monopoltiesību sistēmu. Vislielākos ienākumus deva tabakas, pēc tam sāls monopols. Valsts savas monopoltiesības pārdeva atsevišķiem tirgotājiem, kuri par tām maksāja lielu naudu. Tā 1816.--1821.~gadā valsts ienākumi no sāls monopoltirdzniecības sastādīja 5 miljonus zlotu, bet, kad F.~Druckis-Ļubeckis ķērās pie tās reorganizācijas,~--- 14~miljonus zlotu.

Tiesa, šļahtiči bija nemierā ar 20.~gados realizēto nodokļu politiku. Absolūtos skaitļos tie gan nelikās lieli. Salīdzinājumam, ja Polijas karalistē 1830.~gadā vienam iedzīvotājam gadā vidēji nodokļos bija jānomaksā 26,25~franču franki, tad Francijā~--- 33, bet Anglijā~--- 65~franki. Te gan jāņem vērā, ka ekonomiskās attīstības rādītājos karaliste ievērojami atpalika no nosauktajām Rietumu valstīm.

Ja no XIX gadsimta beigām par rūpniecības attīstību poļu vēsturnieku vidū ilgstoši norisa strīdi, vai tā nav mākslīgi ienesta Polijas karalistē, tad mūsdienās viennozīmīgi valda priekšstats, ka tai bija savi iekšējie priekšnoteikumi. Poļu zemēs tradicionālās rūpniecības nozares bija tekstilrūpniecība, kalnrūpniecība un metalurģija.

Pirmās, vēl pie muižām dibināmās manufaktūras Polijā parādījās jau XVIII gadsimta sākumā. Gadsimta beigās~--- 80. un 90.~gados parādījās jau karaļa un pilsētnieku īpašumā esošas manufaktūras. Tomēr rūpniecības attīstības manufaktūru periods kā tāds Polijas karalistē sākās XIX gadsimta 20.~gados, bet noslēdzās 40. vai, pēc citiem vērtējumiem, 50.~gados. Tad Polijas karalistē rūpniecības attīstība varēja balstīties uz jau pastāvošajām tradīcijām.

Taču vietējo iedzīvotāju vidū bija daudz mazāk preču patērētāju nekā Rietumeiropas valstīs. Pēc pētnieku domām, diezgan ievērojama pirktspēja bija tikai 10\% Polijas karalistes iedzīvotāju: lielajiem zemes īpašniekiem, muižu nomniekiem, augstākās garīdzniecības, ierēdniecības un virsniecības pārstāvjiem, nelielai lielāko tirgotāju grupai. Bez tam no tirgus bija atkarīgi ierēdņi un kalpotāji, brīvo profesiju pārstāvji, algotā darba strādnieki, pilsētnieku zemākie slāņi. Zemnieki pirka lauksaimniecības inventāru, apģērbu, sadzīves priekšmetus. Liela loma preču apgrozībā bija valsts iepirkumiem gan armijas, gan citām vajadzībām (ceļu un kanālu, sabiedrisko ēku celtniecībai iepērkamajiem būvmateriāliem, instrumentiem u.c.).

Valsts veicināja rūpniecības attīstību ar kvalificētu speciālistu piesaisti. Katrs ārzemju (galvenokārt vācu) amatnieks vai zemnieks, kurš pārcēlās uz Polijas karalisti, tika uz sešiem gadiem atbrīvots no visiem nodokļiem, karaklausības, nemaksāja muitu par savas kustamās mantas pārvešanu. Dati par iebraucēju skaitu no 1815 līdz 1830.~gadam gan ir atšķirīgi~--- pētnieki to skaitu vērtē robežās no 20 līdz pat 300 tūkstošiem. Atbraucēju vidū lielākoties bija dažādu specialitāšu tekstilamatnieki, to vidū bija daudz sīkuzņēmēju, kuri izjuta finansiālas grūtības dzimtenē, arī zeļļu bez darba rīkiem.

Liela nozīme rūpniecības attīstībā bija satiksmes ceļu attīstībai~--- šoseju un ūdens ceļu celtniecībai. Ar transporta sistēmas reorganizāciju nodarbojās vietējās poļu varas iestādes, tādēļ pirmajā vietā tika stādītas vietējās tautsaimniecības un tirdzniecības attīstības intereses. 1824.~gadā tika aprēķinātas izmaksas ap 10~t (precīzi~--- 98~280 kg) akmeņogļu piegādei no Dombrovas uz Varšavu. Šī daudzuma ogļu ieguve izmaksāja ap 1 tūkstoti zlotu, bet transportēšana~--- 3 tūkstošus, t.i.~--- bija trīs reizes dārgāka. Tāpēc saprotams, ka transporta problēma uztrauca kā karalistes vietējās, tā centrālās varas iestādes. Pirmkārt ceļu attīstībā bija ieinteresēta valdošā šķira. Taču to būvniecība prasīja lielus kapitālieguldījumus, kuri vai nu vispār neatmaksājās vai atmaksājās ļoti ilgā periodā. Tāpēc izdevumus uzņēmās valsts. Lai atvieglotu ekonomiskos sakarus starp augošajiem rūpniecības centriem tika būvēti ceļi un kanāli. 1829.~gadā Polijas banka piešķīra kredītus bruģētu ceļu~--- t.s. traktu izbūvei starp lielākajām pilsētām. Tika atjaunoti arī vecie ceļi. Augošā poļu buržuāzija pamatīgi nopelnīja no ceļu būvniecības. Pelnīja arī muižniecība. Blakus nospraustajām ceļu stigām esošajās muižās tika pirkti būvmateriāli, arī klaušinieku darbaspēks, kuri nebūt necentās panākt labāko darba kvalitāti. Tomēr valsts nodrošināja savas stratēģiskās intereses, muižnieki ieguva labus ceļus savas produkcijas nogādei tirgos, uzņēmēji guva peļņu, akumulēja kapitālu. Līdz 1830.~gadam tika uzbūvēti 1~030 km jaunu bruģētu ceļu, nerēķinot vecos, pastāvīgi atjaunojamos traktus. Tā rezultātā tika izveidota regulāra satiksme starp rūpniecības centriem, lauku novadiem un upju ostām.

Ja XVIII gadsimta beigās magnāti ceļoja karietēs, turpretī pilsētnieki savās tirdzniecības, bet ierēdņi dienesta darīšanās, šļahtiču atvases uz mācību iestādēm devās speciāli salīgta zemnieka vai ebreju sīktirgotāja ratos, tad ar ceļu izbūvi saistījās jauna transporta veida~--- diližansa rašanās. Tas pastāvīgi kursēja starp lielākajām pilsētām, bija salīdzinoši lēts, pārvadāja pasažierus, pastu, nelielus dārgu preču daudzumus.

Kuģojamās upes tika atbrīvotas no aizgruvumiem, aizsprostiem, dzirnavām. Tika būvēti kanāli. No tiem ievērojamākais bija Augustovas kanāls, kura būvei bija ne tikai ekonomisks, bet arī politisks aspekts. Muižnieki tā centās atbrīvoties no Prūsijas atkarības, kuras valdījumā atradās Vislas grīva. Kanāls savienoja Vislu ar Nemunu un caur Mēmeli (Klaipēdu) deva poļu precēm izeju uz jūru. Tā atklāšana notika 1839.~gadā. Tā garums sniedzās pāri 100 kilometriem. Bija izbūvētas 18 slūžas, daudz aizsprostu un tiltu.

1815.--1830.~gadā galvenā rūpniecības nozare bija tekstilrūpniecība. 1815.--1830.~gados jau bija vērojama pāreja no preču sīkražošanas pie manufaktūrām un pirmajām fabrikām. Visas tās nozares attīstījās uz algotā darbaspēka izmantošanas pamata. Savukārt tekstilnozares iekšienē pirmā vieta piederēja vilnas audumu ražošanai, otrā~--- linaudekla, bet kokvilnas audumu ražošana aizsākās tikai XIX gadsimta 20.~gados. Vājā poļu mašīnbūvniecība nespēja piegādāt nepieciešamās iekārtas. Polijā darbojošies tekstilrūpnieki mašīnas parasti ieveda no Anglijas. Kad tajā darbojās aizliegums uz to pārdošanu, imports notika kontrabandas ceļā caur Beļģiju. Tā mašīnu imports veicināja ražošanas spēku attīstību telstilrūpniecībā un celtniecībā, toties bremzēja mašīnbūvniecībā.

Vilnas audumu, galvenokārt vadmalas, produkcija rada noietu kā iekšējā, tā arī ārējā tirgū. Liels atspaids tekstilražošanai bija vadmalas valsts iepirkumi armijas mundieriem. Arī zemnieki nēsāja rupjvilnas vadmalas apģērbu. Poļu vadmala tika arī izvesta uz Krievijas guberņām un Ķīnu. Sakarā ar to arī attīstījās jau minētā aitkopība, tikai XIX gadsimta otrajā pusē aitu vilna no Austrālijas izspieda Eiropā ražoto.

Valsts iestādes palīdzēja ražotājiem sagādāt izejvielas, ierīkoja to noliktavas, subsidēja jaunu ražošanas iekārtu iegādi ārzemēs (ja 1812.~gadā darbojās tikai 100 aužamās stelles, tad 1830.~gadā~--- jau 5 tūkstoši), regulēja attiecības dažādu tekstilrūpniecības apakšnozaru vidū. Manufaktūru izveide radīja pamatu tālākai pārejai uz fabriku dibināšanu, pirmās no tām radās jau 20.~gadu beigās.

Attīstīt kaut cik plaši linaudekla ražošanu neizdevās. Armijas apgādē ar linaudeklu priekšroka tika dota augstākas kvalitātes audumam no Saksijas. No XIX gadsimta 20.~gadiem samazinājās linu audzēšana un linaudekla darināšana, to izspieda kokvilnas audumi. Kokvilnas auduma ražošana bija pati jaunākā tekstilnozare poļu zemēs. Pirmā kokvilnas audumus ražojošā fabrika tika uzbūvēta 1820.~gadā pie Varšavas. Pāreja uz kokvilnas kā izejvielas izmantošanu bija raksturīga kapitālistiskajai ražošanai, šī produkcija varēja būt sliktākas kvalitātes, bet tās priekšrocība bija lētums. Daudzie no savas saimniecības padzītie un ``stūrī'' iedzītie zemnieki bija slikti pircēji, jo tiem trūka naudas, taču tādu bija jau simti tūkstošu un pirka viņi lētāko kokvilnas audumu. Tieši kokvilnas audumi bija domāti iekšējam tirgum un pieprasījums pēc tiem izrādījās stabils. Kaut cara valdība, aizsargājot krievu ražotāju intereses, aizliedza ievest krievu guberņās poļu ražotos kokvilnas audumus un to izstrādājumus, iekšējā tirgus pieprasījums pēc lētiem audumiem bija tik liels, ka to ražošana attīstījās ļoti strauji, no 1825. līdz 1830.~gadam pieauga gandrīz 5 reizes. Toreizējā Mazovijas vojevodistē (\pltxti{Województwo Mazowieckie}) radās vairāki kokvilnas ražošanas centri. Ievērojamākais no tiem bija Lodza (\pltxti{Łódź}). 1820.~gadā ar lielkņaza Konstantīna rīkojumu tā tika pasludināta par fabriku pilsētu. Tai laikā Lodzā bija 112 namu ar 799 iedzīvotājiem. Taču jau 1829.~gadā Lodza bija kļuvusi par īstu pilsētu ar 4~273 iedzīvotājiem.

Poļu vēsturnieki raksta, ka Lodzas kā jauna tekstilrūpniecības centra uzplaukums sākās jau no XIX gadsimta 20.~gadiem, kad šeit attīstījās vadmalas un kokvilnas audumu ražotnes. Lodzas uzņēmēju ražojumus lielā daudzumā uzpirka Krievijas armijas apgādes dienests. Karalistes valdība parasti uzticējās lieluzņēmējiem un atbalstīja tos arī finansiāli. Sīkajiem uzņēmējiem bija grūti bez ārējas palīdzības paplašināt savu biznesu. Vācu vēsturnieki gan uzskata, ka Lodzas apgabala jeb, kā tas kādreiz tika saukts, ``Polijas Mančesteras'' stabila augšupeja sākās tikai XIX gadsimta vidū. Vācu zinātniskajā literatūrā uzsvērts, ka Lodzas rūpniecības īpatnība bija tā, ka šeit tekstilnozarē strādāja galvenokārt vācieši, kurus šurp bija aicinājuši poļu muižnieki. Vairākas pilsētiņas ar Lodzu priekšgalā XIX gadsimta pirmajā pusē apdzīvoja galvenokārt vācieši, kuri līdz 90\% bija luterāņi. Tikai kopš 1848.~gada fabriku pilsētās tika atļauts apmesties ebrejiem un turpmāk viņu skaits līdz XIX gs. beigām auga tāpat kā vāciešu skaits. Turpretī poļu tautsaimniecības vēsturnieks V.~Kula uzsvēris, ka Lodzas tekstilrūpniecība mantoja Lielpolijas un Silēzijas vadmalas ražošanas tradīcijas. Lodzas apgabalā ienāca emigranti no Lielpolijas, Ļubušas zemes (poļu \pltxti{Lubusz, ziemia lubuska}, vācu \detxti{Land Lebus}~--- vēsturiska novads Polijā un Vācijā abos Oderas upes krastos) un Lejas Silēzijas, lielā mērā ar poliskām etniskajām saknēm. Tas lielā mērā noteica ātro vācu strādnieku polonizāciju Lodzā.

Arī kalnrūpniecība un metalurģiskā rūpniecība nebija jaunas nozares. Poļu zemēs jau izsenis ieguva derīgos izrakteņus. Tie bija koncentrēti galvenokārt divu~--- Sandomiras un Krakovas vojevodistu teritorijā. Īpaši Dombrovas (\pltxti{Dąbrowa Górnicza}) apkārtnē atradās bagātas rūdu un ogļu iegulas. Dzelzs ieguvei bija liela loma jau senatnē, taču īpaši ievērojami pieprasījums pieauga līdz ar dažādu dzelzs mašīnu ražošanas sākumu. Jau XVIII gadsimta beigās Žečpospolitā darbojās 165 kalnrūpniecības uzņēmumi, taču XIX~gadsimta sākumā, neraugoties uz izjūtamo dzelzs trūkumu, dzelzs manufaktūras panīka sakarā ar nepietiekamo pieprasījumu pēc to produkcijas. Feodālisma laikmetā manufaktūras ražoja galvenokārt pirmās nepieciešamības preces vai pusfabrikātus (piemēram, dzelzs loksnes), no kuriem tālāk sīkākas darbnīcas izgatavoja gala produktu. Tālāk manufaktūras, pārtopot par rūpnīcām, no pusfabrikātu ražošanas pārgāja uz ražošanas līdzekļu ražošanu topošajiem spirta brūžiem, alus darītavām, cukurfabrikām, metalurģiskajām un tekstilrūpnīcām. Polijas karalistes valdība aizliedza jebkādas rūdas pārdošanu ārzemēm, jo tas atbalstītu konkurējošo (īpaši Saksijas) rūpniecību. 1816.~gadā tika panākta vienošanās par poļu metalurģijas ražojumu ievešanu Krievijā bez muitas. Tika atjauninātas vietējo uzņēmumu iekārtas. 1824.~gadā Polijas karalistē darbojās 37 dzelzsrūdas raktuves, 9 domnas, 6 liešanas fabrikas, 32 puddlinga krāsnis, kurās no čuguna ieguva dzelzi, 2 lokšņu dzelzs izgatavošanas cehi un vairāk nekā 100 sīkas darbnīcas.

Atsevišķi jāpasaka par akmeņogļu ieguvi. XVIII gadsimta beigās tās galvenokārt ieguva Austrijai piederošajos Javoznas (\pltxti{Jaworzno}) un Ščakovas (\pltxti{Szczakowa}~--- tur 1767.~gadā tika izveidota pirmā ogļu šahta Polijā) rajonos. Taču to ieguve, kaut ogļu slāņi atradās tuvu zemes virsmai un tika izmantots klaušu darbs, izrādījās neizdevīga lielo transporta izdevumu dēļ. Apstākļos, kad mežu vēl bija daudz un metalurģiskā tehnika vēl iztika ar kokoglēm, šahtas drīz tika pamestas. Taču, kad metalurģijas attīstība noteica akmeņogļu nepieciešamību lielākas temperatūras radīšanai domnās lai novērstu nevēlamus piejaukumus metālā, akmeņogles uzvarēja kokogles. Vispirms tās sāka plaši lietot metālapstrādē, cinka iegūšanā un kuģniecībā, vēlāk arī melnajā metalurģijā, uz dzelzceļa un ēku apsildīšanā. Vācu kapitālisti arī pirka oglēm bagātos zemes gabalus, taču nevis lai tos izstrādātu, bet gan lai neļautu tos izstrādāt citiem un tā mazinātu konkurenci Vācijā iegūtajām oglēm.

Par ievērojamu rūpnieciskās ražošanas centru attīstījās Varšava. 1825.~gadā tajā tika uzskaitītas 5~808 fabrikas un amatnieku darbnīcas. Polijas karalistē 1815.--1830.~gadā gūtie ekonomiskie panākumi apliecināja poļu nācijas dzīvotspēju, kaut arī tās etniskā teritorija bija saskaldīta.

Uz tautsaimniecības attīstības fona mainījās \strong{noskaņojums sabiedrībā}.

Sākotnēji ziņa par Polijas karalistes izveidi Vislas krastos tika saņemta ar eiforiju. Imperatoru Aleksandru I poļu intelektuāļi gandrīz uzreiz pasludināja par ``nācijas glābēju'' (\pltxti{zbawcą narodu}). Pateicība viņam kā jaunajam Polijas monarham tika izteikta dažādos veidos: par viņu tika sacerētas dzejas un dziedājumi, aizlūgts Dievs, viņa vārdā nosauca jaundzimušos bērnus, pie sienām tika izkārti litografēti imperatora~--- labdara portreti. Patriotiskā aizgrābtība aculieciniekos atsauca atmiņā līdzīgas ainas 1806.~gada decembrī, kad Napoleons iegāja no prūšiem atkarotajā Varšavā. Tiesa, francūži toreiz smagi piekrāpa radušās cerības, radot tikai necilo Varšavas hercogisti. Taču tagad poļi bija guvuši ko vairāk~--- Polijas vārds atgriezās politiskajās kartēs, troni ieņēma nevis kāds parasts hercogs, bet karalis!

Protams, jau no paša sākuma netrūka arī neapmierināto. Vienus tik ļoti bija savaldzinājusi Napoleona personība, ka viņi vēl ilgi nespēja atteikties no šī valdzinājuma. Citi, jau lielākā skaitā esošie, sūkstījās, ka Polijas karaliste teritorijas ziņā ir mazāka nekā bija Varšavas hercogiste, un to nekādi nevarēja salīdzināt ar Žečpospolitas plašajām teritorijām. Trešie ar aizdomām uzlūkoja dažus Aleksandra I praktiskos soļus.

Taču šī neapmierinātība, šaubas bija gandrīz nemanāmas izglītotās poļu sabiedrības laimes sajūtas priekšā: atkal pastāv Polija, atkal tā ir karaliste ar savu Konstitūciju, kura atsaucas uz 1791.~gada pamatlikumu, viss pārējais~--- sīkumi, kam nav vērts piešķirt īpašu nozīmi. Palika nepamanīts, ka Aleksandra I patvaldnieciskās reformas lielā mērā tikai izrietēja no cara varas tradīcijām. Neraugoties uz to, ka imperators pats deva poļiem Konstitūciju, viņš to neuzskatīja par sev saistošu dokumentu. Viņš arī drīz saprata, ka divu varas sistēmu: patvaldības un konstitucionālas monarhijas samierināšana nav viegla.

Zināmā autonomija, liberālā politiskā iekārta, progresīvāka nekā pašu krievu apdzīvotajā Krievijas daļā tomēr neapmierināja ievērojamu poļu tautas daļu. Tūkstošgadu valstiskuma tradīcija, daudziem sasniegumiem bagātā vēsture, katolicisms, zināma pārākuma apziņa pār iekarotājiem neļāva poļiem samierināties ar suverenitātes zaudēšanu. Reliģija sniedza poļiem drošu atbalstu nacionālajās brīvības cīņās. Viņi akcentēja savu konfesionālo piederību, tā lieku reizi uzsverot savu savdabību, pretstatot to apspiedējiem. (Tiesa, šis nacionālās atbrīvošanās kustības katoliskais raksturs neļāva tai gūt plašāku atbalstu pareizticīgajā krievu sabiedrībā.)

No otras puses, kā uzskata krievu vēsturnieks B.~Mironovs, pēc 1815.~gada Krievijas forsētā valstiskuma izbūve Polijā radīja stāvokli, kad šis valstiskuma radītājs~--- Krievija~--- poļiem vairs nebija vajadzīgs un viņi centās no tā tikt vaļā. Tiesa, viņu neapmierinātība bija virzīta ne tik daudz pret Aleksandru I personīgi, kā pret viņa ieceltajiem ierēdņiem: vietvaldi, kara ministru un poļu armijas virspavēlnieku, (vēlāk, pēc J.~Zaijončeka nāves arī vietvaldi) lielkņazu Konstantīnu, citiem ministriem.

Tālredzīgākie Krievijas valsts darbinieki paredzēja, ka poļu neapmierinātība var kārtējo reizi beigties ar sacelšanos.

Kad 1818.~gadā Varšavā notika svinības sakarā ar Seima atklāšanu divu krievu ģenerāļu starpā notika zīmīga saruna. I.~Paskevičs pajautāja grāfam A.~Ostermanam, kādas tam būs sekas. Pēdējais nedomājot atbildēja: ``Pēc desmit gadiem Tu ar savu divīziju dosies pret viņiem triecienā.'' Paredzētājs kļūdījās tikai par 3 gadiem.

Neapmierinātība auga arī Polijas karalistes armijā. Tā bija neliela, bet no Napoleona karu laika palikušo virsnieku skaits~--- salīdzinoši liels. Tāpēc izredzes sekmīgi kāpt pa karjeras kāpnēm tiem bija visai nelielas. Tas bija viens no cēloņiem kāpēc auga revolucionāri noskaņojumi virsnieku vidē. Lielu neapmierinātību poļu virsniekos izsauca lielkņaza Konstantīna nievājošā izturēšanās pret viņiem. Pirmajos četros gados gandrīz 50 virsnieku beidza dzīvi pašnāvībā, nevarot paciest aizvainojumus no Konstantīna puses. Daudzi virsnieki, kam bija citi iztikas avoti, atstāja dienestu. Virsnieku sašutumu izsauca arī nelikumības un Konstitūcijas neievērošanas gadījumi.

Par poļu aristokrātiju šai laikā poļu vēsturnieki V.~Rostockis un V.~Zajevskis raksta, ka tā pēc saviem politiskajiem nodomiem dalījās divās daļās. Veco aristokrātu dzimtu, kuras carisms bija atstūmis no valsts pārvaldes, pārstāvji bija noskaņoti opozicionāri pret carismu. Šīs daļas jūtu un interešu redzamākais izteicējs bija Ā.~Čartorijskis. Otra daļa poļu aristokrātijas, saistīta ar elementiem, kuri ar saviem kapitāliem piedalījās rūpniecības attīstībā, augstākā birokrātija (kā F.~Druckis-Ļubeckis) un armijā atstātie poļu ģenerāļi bija visstingrākie sacelšanās pret imperatoru pretinieki, perspektīvas tālākai attīstībai saskatīja vienīgi savienībā ar carismu. Abas daļas rīkojās atbilstoši saviem uzskatiem.

Nacionālās neatkarības kustības dalībnieki sākotnēji izmantoja tādu organizācijas formu kā masonu jeb brīvmūrnieku ložas. Jau Varšavas hercogistē (1807--1813) darbojās vairākas organizācijas, radušās uz masonu kustības pamata (masonu kustība ir saskaņā ar brālības principiem veidota organizācija, noteiktā mērā arī vispasaules savienība, kuras dalībniekus vieno kopīgi morāles, ētikas, filosofijas, metafizikas principi un ideāli). Tikai Polijas karalistes teritorijā 1815.~gadā pastāvēja 13 ložas, 1819.~gadā Polijas karalistē radās nacionālā masonu kustība (\pltxti{Wolnomularstwo Narodowe}), kuras priekšgalā stāvēja Polijas karalistes armijas virsnieks V.~Lukasiņskis. Līdz 1821.~gadam ložu skaits pieauga līdz 21. (Poļu politiķis un vēstures pētnieks Č.~Vicehs pat apgalvoja, ka kopā Krievijai pakļautajā Polijas daļā darbojās 47 masonu ložas ar ap 5~000 locekļu.) 1821.~gada 1.~maijā masonu pārstāvji sapulcējās mežā Varšavas pievārtē un nodibināja Patriotisko biedrību (\pltxti{Towarzystwo Patriotyczne}). Tā bija ievērojamākā no XIX gadsimta 20.~gadu sākumā galvenokārt no šļahtičiem veidojamajām slepenajām biedrībām, kuras stādīja uzdevumu izcīnīt Polijas neatkarību. Jau 1822.~gadā notika pirmie aresti. Arī V.~Lukasiņskis tika apcietināts un, pavadījis cietumā 46~gadus, tur arī nomira, bet imperators Aleksandrs~I pavēlēja visas masonu ložas slēgt.

Patriotiskajai biedrībai 1822.~gadā izveidojās kontakti ar krievu dekabristu organizācijām. Iniciatīvu parādīja krievu dekabristu ``Dienvidu biedrība'' (``\rutxti{Южное общество}''), kura piedāvāja kopīgu sacelšanos pret carismu, republikas nodibināšanu. ``Dienvidu biedrības'' vadītājs P.~Pestelis uzrakstīja programmatisku dokumentu ``Krievu tiesa'' (``\rutxti{Русская правда}''), kuras galvenos atzinumus 1823.~gadā atbalstīja ``Dienvidu biedrība''. Tajā bija teikts: ``Polijas patstāvības atjaunošanai jānotiek uz tādiem pamatiem un tādos apstākļos, kuri pilnā mērā nodrošinātu Krieviju nākamībā pret visdažādāko darbību, kura varētu būt pretēja tās drošībai vai pilnīgam tās mieram''.

1825.~gada sākumā notika Patriotiskās biedrības un krievu dekabristu ``Dienvidu biedrības'' (``\rutxti{Южное общество}'') pārstāvju sarunas. Jau sarunu sākumā poļu pārstāvji paziņoja savu vēlmi par Polijas atjaunošanu agrākajās robežās, ieskaitot Volīniju, kā arī baltkrievu un lietuviešu zemes. Tāda jautājuma nostādne nesolīja panākumus sarunām. P.~Pestelis izteica domu, ka vistaisnīgākā būtu strīdīgo teritoriju iedzīvotāju aptauja: ``lai paši izvēlas, kādai tautai vēlas piederēt''. Sarunu noslēgumā tomēr tika panākta vienošanās par vienlaicīgu kopēju bruņotu uzstāšanos. Tikšanās notika arī turpmāk. Taču negaidītā Aleksandra I nāve pārsteidza dekabristus nesagatavotus un viņu rīkotā sacelšanās neizdevās. Arī kopīga uzstāšanas ar poļu sazvērniekiem nenotika, jo viņi situāciju Eiropā vērtēja kā bezcerīgu, gribēja nogaidīt sacelšanās Krievijā rezultātus un tikai tad rīkoties. Vēlāk viens no 1830.~gada sacelšanās ievērojamiem darbiniekiem M.~Mohnackis nosodīja Patriotiskās biedrības neizlēmību: ``Mēs palaidām garām, gandrīz noraidījām labvēlīgu apstākli, kurš ne tik drīz atkal radīsies''.

1825.~gadā Krievijas un Polijas karalistes troni ieņēma Nikolajs I. Viņš solīja saglabāt Polijas karalistes konstitucionālās iestādes, kaut sarakstē ar brāli Konstantīnu neslēpa savu riebumu pret konstitucionālo iekārtu. Nikolajam I jau no bērnības bija ieaudzināta antipātija pret poļiem un viņš to nekad neslēpa. 1826.~gadā pēc vietvalža J.~Zaijončeka nāves viņa vietā jauns netika nozīmēts, bet tā pilnvaras nododot Administratīvajai padomei. Nikolajs I arī kategoriski nepiekrita poļu šļahtiču plāniem panākt lietuviešu, baltkrievu un ukraiņu apdzīvoto guberņu iekļaušanu Polijas karalistē. Lieta tā, ka 1817. un 1819.~gadā ar imperatora Aleksandra I dekrētiem lielkņaza Konstantīna kā Polijas karalistes armijas pavēlnieka vara tika izplatīta arī uz Krievijas Ziemeļrietumu novada guberņām, kur tika izveidots Lietuvas atsevišķais korpuss (\rutxti{Литовский отдельный корпус}), daļa no kura bija pastāvīgi izvietota Polijas karalistes teritorijā. Pat administratīvajās un civillietās vietējās iestādes sāka griezties pie Konstantīna Varšavā. Izglītības jomā izvērsās polonizācija. 1825.~gadā, neilgi pirms savas nāves apmeklējot Poliju, Aleksandrs I sarunās ar lielkņazu Konstantīnu apstiprinājis viņam savu nodomu pievienot Lietuvu Polijas karalistei. Turpretī Nikolaja I nodomi bija pavisam pretēji, viņš vēlējās Krievijas rietumu guberņas administratīvā, reliģiskā un nacionālā ziņā unificēt ar pārējām impērijas guberņām. Arī lielkņaza Konstantīna vara jaunajam imperatoram šķita pārlieku liela, uz tās rēķina viņš vēlējās nostiprināt monarha autoritāti. Nikolajs I, nākot tronī, uzsāka Lietuvas civilās administrācijas, kā arī Lietuvas atsevišķā korpusa rusifikāciju. (1827.~gadā korpusu sāka veidot no Grodņas, Minskas, Volīnijas rekrūšiem, bet iesaucamos no Viļņas un Podolijas guberņām nosūtīja uz impērijas iekšieni, nomainot Lietuvas korpusā ar krievu Pleskavas un Tveras guberņas rekrūšiem. 1831.~gadā Lietuvas korpuss tika pārdēvēts par 6.~korpusu.) Sākās varas iestāžu spiediens pret uniātu baznīcu ar mērķi pievienot to pareizticīgajai.

Iespējams, ka lielkņazs Konstantīns, nākot tronī Nikolajam I, savu varu Varšavā, tiesības uz mūžu pārvaldīt Polijas karalisti un daļēji arī Krievijas Ziemeļrietumu novadu uzskatīja kā kompensāciju par Krievijas troņa atdošanu jaunākajam brālim. Lielkņazs Konstantīns kļuva par sava veida starpnieku starp Krievijas imperatoru un Polijas karalistes sabiedrību. Ar šādu starpniecību nemierā bija abas puses.

Dekabristu lietas izmeklēšanas gaitā 1826.~gadā tika arestēti arī atsevišķi poļu Patriotiskās biedrības dalībnieki. Pret 8 no tiem tika celta apsūdzība. Lietu nodeva Seima tiesas izskatīšanai. Pretēji imperatora vēlmei 1828.~gada pavasarī tiesa noraidīja apvainojumus valsts nodevībā un atstāja spēkā vienīgi apsūdzības par piederību slepenām biedrībām, pieņemot salīdzinoši mīkstu spriedumu. Visbargākais sods~--- 3 gadi ieslodzījumā 1828.~gadā tika piespriests tikai vienam cilvēkam. Tomēr imperators spriedumu apstiprināja. Taču visi tiesājamie tika nogādāti Pēterburgā, it kā lai konfrontētu ar krievu sazvērniekiem.

Krievu~--- turku kara laikā (1828--1829) Nikolajs I vēlējās izmantot arī Polijas karalistes armiju, taču pret to enerģiski iebilda un savu panāca lielkņazs Konstantīns. Tā kā sākotnēji kara iznākums bija neskaidrs, Nikolajs I nevēlējas uztraukt poļu sabiedrību un sarežģīt situāciju Polijā. (Pēc Varnas ieņemšanas kā savas labvēlības apliecinājumu viņš ziedoja Varšavai divpadsmit lielgabalus, saņemtus kā trofejas, nedaudz vēlāk Varšavas katedrālei vairākus turkiem atņemtus karogus.)

Varšavā Nikolajs I kronējās gan tikai 1829.~gadā. Tiesa, viņš bija atvedis līdzi 11 gadus veco troņmantnieku Aleksandru (nākamo Aleksandru II), kurš tērpās poļu gvardes strēlnieka mundierī un tekoši runāja poliski. Arī pats Nikolajs I centās iekarot poļu augstmaņu, bet sevišķi virsnieku, simpātijas, bez konvoja pastaigājās pa pilsētu. Tomēr pirms aizbraukšanas Nikolajs I parakstīja dekrētu par jaunu senatoru iecelšanu, kas bija pretrunā ar 1815.~gada Konstitūciju, kas ierobežoja to skaitu. Tas izsauca poļu nepatiku. Valdīja neapmierinātība ar Krievijas kundzību, tika gatavotas sazvērestības. Tagad poļu sabiedriskie darbinieki centās uzsvērt Aleksandra I nopelnus, tā vēloties ietekmēt Nikolaju I turpināt viņa brāļa realizēto kursu attiecībās ar poļiem. Katru reizi, kad Nikolajam I tika atgādināts par Polijas karalistes Konstitūcijas pārkāpumiem, tika pieminēta arī Aleksandra I labvēlīgā attieksme pret poļiem. Kā rakstīja poļu vēsturnieks S.~Smoļenskis, tauta nejuta pret Nikolaju I pateicību kā pret Aleksandru I, tajā auga nepatika un naids pret Polijas Konstitūcijas pārkāpēju.

Mūsdienu poļu vēsturnieki M.~Timovskis, J.~Kenevičs un J.~Holcers ir atzinuši: ``No mūsdienu pozīcijām, pēc pusotra gadsimta, var konstatēt, ka lielos vilcienos piecpadsmitgade ``pēc Vīnes kongresa'' bija pati labvēlīgākā no poļu nacionālo interešu viedokļa visā periodā no 1795. līdz 1918.~gadam (jo Varšavas hercogiste bija tikai epizode, kura maksāja milzīgus cilvēku un materiālos upurus, saistītus ar pastāvīgiem kariem)''. Diemžēl šie vēsturnieki izvairās izsvērt, kam un cik liela atbildība bija jānes par nosauktā posma pārtraukšanu, un vai tai (pārtraukšanai) bija citas un kādas alternatīvas. Autors no savas puses var tikai teikt, ka bez skaidri redzamās Krievijas carisma vainas, noteikti jārunā arī par poļu radikāļu atbildību savas tautas priekšā par visiem tālākajiem notikumiem, poļu tautas ciešanām.

\strong{1830.--1831.~gadā} notika \strong{poļu sacelšanās}, pazīstama arī kā Novembra sacelšanās (\pltxti{Powstanie listopadowe}). (Baltkrievu autors A.~Tarass to sauc par 1830.--1831.~gada poļu-krievu karu). Šā gada augustā Varšavā nonāca ziņa par sacelšanos Parīzē un Burbonu (franču \frtxti{Bourbon}~--- dinastija Eiropā) dinastijas gāšanu. Sekoja baumas, ka Krievijas imperators Nikolajs I revolūcijas apspiešanai Francijā vēlas sūtīt krievu un poļu spēkus. Patiesībā Nikolajs I gan tikai paziņoja Konstantīnam, lai viņš veic pasākumus krievu karaspēka izmetināšanai, kurš caur Poliju dosies apspiest revolūciju Beļģijā. Poļu sazvērnieki, baidoties, ka arī karalistes armija var tikt izvesta no tās teritorijas, forsēja sacelšanās sākumu, nolēma carismam pretstatīt tautas atbrīvošanās centienus.

Polijas karalistē sazvērnieki varēja rēķināties ar ap 10~000 poļu karavīru pret 7~000 krievu, pie kam daudzi no pēdējiem bija dzimuši un auguši bijušajās Polijas teritorijās. 1830.~gada sacelšanās faktiski bija vienīgā no visām poļu bruņotajām akcijām, kurai bija kaut niecīgas izredzes uz panākumiem. Polijai bija lai maza, tomēr sava armija, kura guva virkni uzvaru cīņā pret Krievijas karaspēku.

Sacelšanās dalībniekiem nebija kādas izstrādātas rīcības un mērķu programmas. Šļahtiču apziņā sacelšanās ideja pati par sevi bija tik diža, ka aizēnoja visas citas sociāli-politiskās problēmas, kuru uzstādīšana, skarot šļahtiču kārtas interese, varēja būtiski mazināt viņu plašo aizrautību ar nacionālās atbrīvošanās ideju.

1830.~gada 17.(29.)~novembrī pusotra desmita bruņotu sazvērnieku, kursanti no Varšavas apakšvirsnieku skolas P.~Visocka vadībā iebruka lielkņaza Konstantīna rezidencē Belvedēras pilī (\pltxti{pałac Belwederski}) Varšavā. Taču kamerdīners (vācu \detxti{Kammerdiener}~--- istabas kalpotājs) paguva lielkņazu brīdināt par briesmām, viņš spēja noslēpties un tā izglābās, sacēlušies vienīgi nogalināja kādu krievu ģenerāli, ievainoja dažus galminiekus un apkalpotājus. Pēc tam pilsētas ielās viņi vēl nogalināja arī vairākus Krievijai uzticīgus poļu ģenerāļus, tai skaitā kara ministru M.~Hauki, kurš bija karojis gan vēl T.~Kostjuško, gan Napoleona vadībā, bet tagad atteicās izpildīt revolucionāru prasību un nostāties sacēlušos pusē, griezās pie tiem ar runu, kurā viņu darbošanos nosauca par neprātīgu un aicināja doties mājup. Vēlāk izrādījās, ka ministra krūtis viņa sievas un bērnu klātbūtnē bija cauršautas 19 vietās.

Imperators Nikolajs I vēlāk lika nogalinātajiem poļu ģenerāļiem, kuri labāk krita no sacēlušos rokas, nekā tiem pievienojās, uzcelt lielu obelisku, izrādīja īpašas rūpes par viņu ģimenēm. Tā, M.~Haukes meita Jūlija kļuva par Krievijas ķeizarienes galma dāmu, apprecēja Hessenes princi Aleksandru, viņu pēcteči saradojās ar daudziem Eiropas valdnieku namiem.

Turpretī mūsdienās Polijā 29.~novembris tiek atzīmēts kā ievērojams Polijas vēstures notikums. Šai dienā karaskolu beidzējiem svinīgi tiek piešķirtas virsnieku pakāpes.

Krievu publicists un vēsturnieks N.~Bergs rakstīja, ka, ja sacelšanās nebūtu krievus pārsteigusi pēkšņi, ``uguni varētu apdzēst tai pat mirklī, un [1830.~gada] 30.~novembra rītā daudzi iedzīvotāji pat nezinātu, ka tika gatavots kaut kāds nopietns sprādziens''. Arī poļu vēsturnieks V.~Smoļenskis uzskatīja, ka lielkņazam Konstantīnam pietika karaspēka, lai apspiestu sacelšanos pašā tās iedīglī. ``Sacelšanās izdzīvoja tikai pateicoties Konstantīna neaktivitātei, kurš, pastaigājoties krievu karaspēka ierindas priekšā Ujazadovska alejā (\pltxti{Aleje Ujazdowskie}), nepārtraukti atkārtoja, ka pašiem poļiem ir jānomāc viņu tautiešu radītās nekārtības''. Iespējams, ka tas bija pārāk optimistisks stāvokļa vērtējums. Poļu armijas daļas nostājās sacēlušos pusē un kopā ar pilsētniekiem ieņēma ieroču arsenāla un krievu ulānu (viegli bruņotas kavalērijas paveids) pulka kazarmu ēkas. Drīz visa Varšava atradās jau sacēlušos rokās. Poļu armiju atbalstīja iedzīvotāji. Sākusies kā militāra apvērsuma mēģinājums, kustība izvērsās par tautas sacelšanos. Krievu garnizons Varšavā neizrādīja pienācīgu pretošanos un pilsētu atstāja. Lielkņazs Konstantīns paziņoja, ka ``katra izlieta asins lāse tikai sabojās lietu'' un ar krievu daļām uzsāka atkāpšanos uz Krievijas robežu.

Varšavā izveidojās Pagaidu valdība, kura, kustības aktīvistu spiesta, savukārt par diktatoru nozīmēja Napoleona karu dalībnieku ģenerāli J.~Hlopicki. Sabiedrības radikālā daļa diktatūru sagaidīja ar prieku. Tā uzskatīja, ka J.~Hlopickis izdarīs to, ko nedarīja Pagaidu valdība: pasludinās sakaru saraušanu ar Nikolaju~I, aicinās Lietuvas un Krievijas provinces uz sacelšanos un sāks cīņu par neatkarīgu Poliju. Taču ģenerālis, kurš neticēja sacelšanās uzvarai, centās atrisināt konfliktu sarunu ceļā, nosūtot sūtni pie Nikolaja I. Pagaidu valdība kā noteikumus, lai Polijas karaliste paliktu imperatora pakļautībā, izvirzīja prasības, lai Nikolajs I ievēro 1815.~gada Konstitūciju, to izplata arī uz lietuviešu, baltkrievu un ukraiņu zemēm, izsludina amnestiju, apņemas nesūtīt krievu armiju Polijā.

Populārā izklāstā šo noteikumu izpilde nozīmētu virknes pamatos nepoļu apdzīvotu teritoriju pievienošanu Polijas karalistei. Jākonstatē, ka poļu nacionālās atbrīvošanās cīņa norisa reizē ar viņu cīņu par svešu tautu pakļaušanu. Poļi kādreiz šais zemēs bija valdījuši, taču nonākot pašiem citas valsts pakļautībā, nebija iemācījušies cienīt arī citu tautu brīvību. Tas, ka poļu nacionālās apziņas formēšanās bija gājusi tālāk kā ukraiņiem un baltkrieviem, nedeva pirmajiem tiesības ignorēt gadsimtu gaitā daudzskaitlīgās sacelšanās skaidri parādīto šo tautu nevēlēšanos pakļauties poļu paniem.

Nikolajs I apsolīja vienīgi amnestiju sacelšanās pārtraukšanas gadījumā. Krievijā šai laikā nebija organizēta spēka, kurš varētu sniegt poļu sacelšanās dalībniekiem reālu palīdzību.

Decembrī sanāca Seims, kurš nosodīja carisma noziegumus, bet sacelšanos pasludināja par visas poļu tautas lietu. Taču imperators pavēlēja atjaunot agrākās varas struktūras, bet poļu armiju koncentrēt Pinskas rajonā karalistes austrumos. Saasinājās uz samierināšanos orientētā diktatora attiecības ar citiem līderiem un 1831.~gada 18.~janvārī J.~Hlopickis atteicās no diktatora pilnvarām. Seims par diktatoru iecēla vēl vienu T.~Kostjuško un Napoleona karu dalībnieku M.~Radzivillu, kurš gan arī drīz no amata atteicās.

25.~janvārī Seimam kļuva zināms imperatora brīdinājums, ka pirmais lielgabala šāviens no poļu puses nozīmēs Polijas karalistes patstāvības beigas. Atbildei uz to Seims bez balsošanas pieņēma lēmumu gāzt Nikolaju I no Polijas karalistes troņa, atzīstot to par vakantu. Tas bija līdzvērtīgi kara pieteikumam.

Dažas dienas vēlāk~--- 28.~janvārī Seims izveidoja Nacionālo valdību (\pltxti{Rzad narodowy}) ar kņazu Ā.~Čartorijski priekšgalā. Jau minētais V.~Zajevskis uzskata, ka Ā.~Čartorijskis un viņa piekritēji pēc sacelšanās sākuma nolēma nostāties tās priekšgalā, lai piešķirtu tai mērenu, konservatīvu raksturu. Visus spēkus viņi veltīja, lai izmantotu sacelšanos izdevīgu sarunu vešanai ar imperatoru, taču nevēlējās piekrist bezierunu kapitulācijai. Nacionālajā valdībā ietilpa arī demokrātisko aprindu pārstāvji, kā Viļņas universitātes docētājs vēsturnieks J.~Lelevels. Viņš nedaudz izmainīja jau XVIII gadsimta beigās (1795), kad poļu šļahta bez panākumiem mēģināja iesaistīt savā cīņā ukraiņu un baltkrievu dzimtcilvēkus, radušos lozungu ``\pltxti{Za naszą i waszą wolność}'' (Par mūsu un jūsu brīvību). Uz sacēlušos karoga tagad bija rakstīts ``\pltxti{W imię Boga za Naszą i Waszą Wolność}'' (Dieva vārdā par mūsu un jūsu brīvību). Šādā variantā tas arī kļuva starptautiski populārs.

3.~februārī Seims apstiprināja zvērestu, kuru pats arī pirmais nodeva, ka sacelšanās mērķis ir ``tās neatkarības, tās pakāpes starp tautām, kurai Radītājs iecerējis poļu tautu, iegūšana''. Ar īpašu manifestu Seims mudināja tautu nodot šo zvērestu.

Uz Parīzi, Vīni un citām Eiropas galvaspilsētām tika nosūtīti aģenti, kuru uzdevums bija rosināt šo valstu valdības iejaukties Polijas notikumos. Kņazs Ā.~Čartorijskis ar brāļa K.~Čartorijska starpniecību saistījās ar Austrijas galmu, piedāvājot Polijas kroni vienam no austriešu erchercogiem, ja vien Austrija atbalstīs sacelšanās mērķus. Sākotnēji Vīnes galms ar apmierinājumu uzņēma ziņas par sacelšanos Polijā, taču, kad tā radikalizējās un par vienu no līderim kļuva J.~Lelevels, Austrija ne tikai neatbalstīja sacelšanos, bet arī noslēdza savu robežu, traucējot poļu sakarus ar rietumu valstīm. Kaut, pēc poļu vēsturnieka Š.~Aškenazi vērtējuma, Prūsija bija ieinteresēta Polijas karalistes sabrukumā, kas radītu iespēju iegūt jaunas teritorijas (tiesa, šo secinājumu minētais zinātnieks nepamatoja un šādas vēlmes noteikti bija pretrunā ar Prūsijas drošības un stabilitātes interesēm), Prūsija tāpat aizvēra savu robežu, noslēdza ar Krieviju traktātu, apņemoties uz poļu robežas nostādīt 60~tūkstošu vīru lielu korpusu, gatavu ieņemt Varšavu. Francijā sabiedrība sacelšanos Polijā uztvēra ar saucieniem ``Palīgā poļiem!''. 1831.~gada 13.~februārī pūlis Parīzē apmētāja akmeņiem Krievijas sūtņa namu, izsitot tam logus. Kopš 1831.~gada februāra par Francijas bruņotu iesaistīšanos cīņa dumpīgo poļu pusē pret Nikolaju I dedzīgi iestājās un vāca iemaksas sacelšanās vajadzībām Polijas sacelšanās atbalsta komitejas priekšsēdētājs marķīzs M.~De~Lafajets. Taču šo sajūsmu neatbalstīja Jūlija revolūcijas rezultātā tronī kāpušais karalis Ludvigs-Filips. Tikpat nesekmīgi bija poļu sūtņu pūliņi Anglijā. Nekādu būtisku palīdzību no ārzemēm poļu sacēlušies nesaņēma.

Polijas sadalei, poļu, baltkrievu, ukraiņu, baltiešu zemju iekļaušanai Krievijas sastāvā acīmredzot vajadzēja pēdējo nostiprināt. Bet jebkura Krievijas nostiprināšanās Rietumos parasti tika uzskatīta par tiešu izaicinājumu savai drošībai. Tāpēc arī pārliecība, ka ''Eiropa mūs aizsargās'', stiprināja poļu virsslāņa apņēmību cīnīties par Žečpospolitas atjaunošanu un iespēju apspiest nepoļu iedzīvotājus. Poļu sacelšanos vadoņi neņēma vērā tikai vienu: lielvalstīm nav mūžīgu ienaidnieku, tām ir tikai mūžīgas intereses. Ja tās nesakrita ar poļu interesēm, pēdējiem uz kādu palīdzību bija velti cerēt.

Revolucionārāk noskaņotie poļu nacionālās kustības elementi no ``Patriotu biedrības'' (\pltxti{Towarzystwo Patriotyczne}) centās sacelšanās īstenošanā iesaistīt zemniekus. Tikai zemnieku atbrīvošana no klaušām varētu nostādīt tos sacelšanās pusē. Taču likumprojektu par klaušu nomaiņu ar nodevām Seims noraidīja, jo tas izsauktu neapmierinātību Prūsijā un Austrijā, apdraudētu arī pašu poļu šļahtiču, kuri atbalstīja sacelšanos, vairākuma intereses. Tāpēc šai virzienā arī praktiski nekas netika darīts. Ar to sacelšanās jau sākotnēji bija nolemta neveiksmei. (Tiesa, krievu vēsturnieks A.~Pogodins uzskatīja, ka, ja Seims 1830.~gadā pat pasludinātu zemnieku brīvlaišanu, zemnieki diezin vai sekotu tam.) Plašas zemnieku masas, īpaši ukraiņu un baltkrievu zemnieki, sacelšanos neatbalstīja.

1831.~gada pavasarī poļu armijā bija ap 79~tūkstoši cilvēku ar 158~lielgabaliem. Tajā karoja arī sievietes, piemēram, Līksnā pie Daugavpils dzimusī grāfiene E.~Plātere. Par viņu stāsta leģendas, viņa esot līdzinājusies Francijas nacionālajai varonei Žannai d’Arkai, viņas izveidotā un vadītā vienība esot iedvesusi šausmas ienaidniekam. Viņai veltīts arī Ā.~Mickeviča dzejolis ``Pulkveža nāve'' \pltxti{(``Śmierć pułkownika'').} Tiesa, sacelšanās liecinieks prelāts, bīskaps, Varšavas garīgās akadēmijas rektors P.~Butkevičs, būdams jau Parīzē, uzrakstīja memuārus, kuros rādīja, kā E.~Plātere, pieķerta intīmos sakaros ar pakļautajiem virsniekiem, kļuva par izzobošanas objektu, pēc sakāvēm pameta savu vienību likteņa varā, bēguļoja un pirms 1831.~gada 23.~decembrī iestājušās nāves nožēloja savu iepriekšējo rīcību. Iespējams gan, ka minētais autors vadījās no citiem, nevis patiesības noskaidrošanas motīviem.

Šai laikā radās viena no vispopulārākajām poļu patriotiskajām dziesmām, 1830.~gada Novembra sacelšanās muzikālais simbols ``1831.~gada (lai atšķirtu no 1905.~gada tāda pat nosaukuma dziesmas) Varšavjanka'' (``\pltxti{Warszawianka 1831 roku}''). Parīzes dzejnieks Ž.~Delavins 1831.~gada februārī sacelšanās Varšavā iespaidā franču valodā pēc ``Marseljēzes'' (``\frtxti{La Marseillais''}~--- Francijas himna) parauga uzrakstīja dzejoli ``\frtxti{La Varsovienne''}. Pēc mēneša tas nonāca Varšavā, tika pārtulkots, publicēts, Varšavas operas teātra diriģents un komponists K.~Kurpinskis tam sacerēja melodiju. 1831.~gada 5.~aprīlī dziesma tika pirmo reizi publiski nodziedāta operas izrādes laikā un publika to uzņēma ar entuziasmu. Turpmākajos gadu desmitos dziesma kļuva par vienu no poļu nacionālās atbrīvošanās kustības simboliem. Līdz 1926.~gadam, kad ``Dombrovska mazurka'' kļuva par oficiālo Polijas himnu, ``1831.~gada Varšavjanka'' bija viena no piecām kandidātēm uz himnas vietu. Tā bija arī viena no populārākajām 1944.~gada Varšavas sacelšanās dziesmām.

1831.~gada februārī feldmaršala I.~Dibiča, pēc tam feldmaršala I.~Paskeviča vadītā Krievijas armija, kurā bija ap 115~000 karavīru, neraugoties uz holēras epidēmiju (tā aiznesa vairāk nekā 27 tūkstošu krievu karavīru dzīvību, par tās upuriem kļuva arī feldmaršals I.~Dibičs un lielkņazs Konstantīns), sakāva poļus, kuru spēki sniedzās līdz 55 tūkstošiem. (Tāpat kā 1794.~gada sacelšanās laikā arī 1831.~gadā ap 180~--- 200 dažādu tautību krievu karavīru un virsnieku pārgāja poļu sacēlušos pusē). 26.~septembrī, tai pat dienā, kad pirms gadiem notika Borodinas kauja, tika ieņemta Varšava. Sakarā ar to A.~Puškins uzrakstīja dzejoli ``Borodinas gadadienā'', kurā bija rindas:

\vspace{1.5em}

\noindent
\begin{minipage}{0.55\textwidth}
\rutxti{
Сбылось~--- и в день Бородина\\
Вновь наши вторглись знамена\\
В проломы падшей вновь Варшавы;\\
И Польша, как бегущий полк,\\
Во прах бросает стяг кровавый~---\\
И бунт раздавленный умолк.}
\end{minipage}
\hspace{1em}
\begin{minipage}{0.5\textwidth}
Ir noticis~--- Borodinas dienā\\
Jau atkal slejas mūsu karogi\\
Jau atkal kritušajā Varšavā;\\
Un Polija, kā bēgošs pulks,\\
Met pīšļos asiņaino standartu~---\\
Un dumpis nomākts klusē.
\end{minipage}

\vspace{1.5em}

% page 121


Taču pirms Varšavas ieņemšanas, kā liecina kņazienes N.~Goļicinas atmiņas, sacēlušies izrēķinājās ar sagūstītajiem krievu karavīriem un vairākiem galminiekiem. Kņaziene to nosauca ``par bojā ejošās nācijas pēdējo ārprāta lēkmi''. Nākas tikai konstatēt, ka visdažādākajos laikos patriotisms un cīņas griba ne vienmēr tiek apliecināta atklātā kaujas laukā, bet dažkārt arī vairāk nekā apšaubāmā veidā. Nebija izņēmums arī 1830.--1831.~gada poļu sacelšanās. Divas dienas ilgajā Varšavas ieņemšanā krievu karaspēks vēl zaudēja ap 10 tūkstošu cilvēku, poļu~--- līdz 11 tūkstošu. Trīs tūkstoši tika saņemti gūstā. Tika iegūti arī 132 lielgabali.

Pēc tam sacelšanās turpinājās mazāk nekā mēnesi. Daļa poļu karaspēka: ap 15 tūkstošu cilvēku ar 42 lielgabaliem pārgāja Austrijas un ap 20 tūkstošu ar 96 lielgabaliem~--- Prūsijas robežu, kur tika internēta. Vēl gan notika mēģinājums atjaunot bruņotu cīņu. 1833.~gadā Krievijas impērijas robežu šķērsoja vairākas sīkas poļu revolucionāru-emigrantu vienības J.~Zaļivska vadībā. Izraisīt sacelšanos neizdevās, ekspedīcija tika sakauta, daļu tās dalībnieku sagūstīja, pakāra, nošāva vai nosūtīja katorgā uz Sibīriju. Pats J.~Zaļivskis bēga uz Austriju, tika tur arestēts un notiesāts uz 20~gadiem cietumā. Pēc amnestijas 1848.~gadā viņš devās uz Franciju, kur arī mira.

I.~Paskevičs par sacelšanās apspiešanu saņēma Varšavas kņaza (\rutxti{князь Варшавский}) titulu, kļuva arī par Polijas karalistes vietvaldi (1832--1856). Kā raksta Polijas vēstures pētnieks krievu vēsturnieks G.~Matvejevs, ar šo laiku Krievija sāka uzskatīt, ka tā ir iekarojusi Polijas karalisti, nevis ieguvusi to uz Vīnes kongresa (1815) lēmumu pamata. Tiesa, pēc sacelšanās sakāves 1831.~gada beigās iznāca imperatora rīkojums, ar ko sacelšanās dalībnieki tika amnestēti, izņemot notikumu dalībniekus Varšavā 1830.~gada novembrī, Seima deputātus, kuri bija balsojuši par Polijas karalistes troņa atņemšanu Nikolajam I, un poļu virsniekus, aizbēgušus uz ārzemēm.

Nikolajs I 1830.--1831.~gada sacelšanos uzskatīja par poļu nodevību, poļu nacionālās jūtas~--- par nepiepildāmu fantāzijas augli un savus ārpolitiskajos plānos vadījās no Krievijas valdošās Romanovu dinastijas interesēm. 1831.~gada vasarā viņš pat nopietni apsvēra plānu apmainīt ar Prūsiju un Austriju to valdījumos nonākušās poļu zemes. (Par trim Polijas karalistes rietumu vojevodistēm viņš no Prūsijas vēlējās saņemt Toruņu, Klaipēdu un Nemunas grīvu, bet par Krakovas vojevodisti no Austrijas Tarnopoles apriņķi.) Tomēr acīmredzot galu galā imperators saprata, ka šāda apmaiņa bija pārāk sarežģīta un tāpēc nerealizējama un no tās atteicās.

Reizē varēja runāt par Aleksandra I realizētās ``mīkstās'', samērā liberālās politikas neveiksmi Polijas karalistē. Krievu vēsturnieks B.~Mironovs par pēc Napoleona kariem XIX gadsimta sākumā izveidojošos situāciju Polijā Krievijas impērijas sastāvā ir norādījis: ``Polija sniedz piemēru, kā liela autonomija, liberāla politiskā iekārta, kas bija daudz progresīvāka salīdzinājumā ar iekārtu pašā Krievijā, neradīja apmierinājumu iekarotās tautas prātos. Kā to var izskaidrot? No vienas puses, tūkstoš gadu vecā valstiskuma tradīcija, ar daudziem sasniegumiem iezīmēta vēsture, katolicisms, pārākuma sajūta pār uzvarētājiem neļāva poļu tautai samierināties ar suverenitātes zaudēšanu. No otras puses, kā man šķiet, ka Krievijas forsētā valstiskuma celtniecība Polijā 1815.~gadā radīja tādu situāciju, ka Krievija~--- Polijas valstiskuma radītāja~--- poļiem kļuva praktiski nevajadzīga pēc tam, kad jaunā Polijas valsts bija izveidota. Poļi steidzās atbrīvoties no nevajadzīgas aizbildniecības.''

Kā pamatoti atzinis vācu vēsturnieks E.~Meijers, sacelšanās dalībnieki, to uzsākot, faktiski paši lika uz spēles visus tos sasniegumus, kuri bija panākti pusotrā gadu desmitā pēc Vīnes kongresa un gandrīz visu zaudēja.

Eiropā sabiedrība līdzjūtīgi pieņēma poļu neatkarības cīnītājus. Francijas karaļa Luija Filipa valdībā neliedza viņiem patvērumu. Morālu atbalstu viņiem sniedza Polijas atbalsta komiteja, kuru vadīja ģenerālis M.~Lafajets. Tiesa, drīzumā pēc imperatora Nikolaja I lūguma Luisa Filipa valdība aizliedza darbību, vērstu pret Krieviju. Tāpēc, piemēram, J.~Lelevels bija spiests pamest Franciju un apmesties Briselē.

Arī Polijai blakus esošajā Vācijā, sabiedriskā doma juta līdzi poļu brīvības centieniem. Radās ap 100~poļu draugu biedrību (\detxti{Polenfreundvereine}), kuras gan vēlāk to politizācijas dēļ tika aizliegtas. Poļu emigrantus sagaidīja silta uzņemšana. R.~Vāgners 1836.~gadā uzrakstīja uvertīru ``\detxti{Polonia''} (``Polija''), tapa vairāku citu vācu komponistu ``poļu dziesmas'', virkne vācu dzejnieku veltīja dzejoļus poļu brīvības cīņām. Tikai daži no vācu dzejniekiem, kā H.~Heine un E.~Arndts, spēja būt arī kritiski. 40.~gados gan Vācijā ``sajūsma par poļiem'' noplaka. Acīmredzot to ietekmēja tas, ka no ap 50~000 pēc sacelšanās no dzimtenes bēgušajiem tās dalībniekiem pēc cara amnestijas pasludināšanas ap 40~000 atgriezās Polijas karalistē, tā netieši atzīstot, ka nemaz tik ``nepanesami'' apstākļi tajā nebija.

Ebreju izcelsmes poļu vēsturnieks Š.~Aškenazi īpaši uzsvēris, ka tieši poļu cerību sabrukums uz Lietuvas [faktiski daļas Krievijas Ziemeļrietumu un Dienvidrietumu novadu~--- V.Š.] pievienošanu Polijas karalistei Nikolaja I valdīšanas laikā bija galvenais cēlonis, kāpēc sākās 1830.~gada poļu sacelšanās. Viņš rakstīja, ka ``Novembra revolūcija bija pirmkārt Polijas karš ar Krieviju Lietuvas dēļ. Otrs cēlonis, kaut tas bija skaidrāk saredzams, jo atradās sabiedriskās apziņas virspusē, bija ar samērā mazāku nozīmi: tas bija konstitucionālais jautājums, pastāvīga [Polijas karalistes] konstitucionālā statusa pārkāpšana.'' Vēsturnieks arī atzīmējis, ka sava nozīme bija arī ortšķirīgiem cēloņiem, kā pretrunām starp valdošā nama brāļiem Nikolaju un Konstantīnu, krievu--turku karam, kurš solīja labas izredzes uz sacelšanās izdošanos, revolūcijai Rietumos, kura deva signālu sprādzienam.

Nav pilnīgu datu par sacelšanās dalībnieku sociālo stāvokli. Taču ziņu par 16~tūkstošu Lietuvas, Baltkrievijas un Ukrainas teritorijās cara valdības represētajiem sacelšanās dalībniekiem analīze rāda, ka to vidū ap 33\% piederēja pie nodevu maksātāju, pirmkārt zemnieku, slāņa, gandrīz 50\%~--- šļahtiču kārtai, ap 5\%~--- garīdzniecībai. Kaut nav datu par pašu poļu zemēm, tomēr esošās ziņas ļauj izdarīt zināmus secinājumus par sacelšanos kopumā. Statistiskais materiāls rāda, ka sacelšanās kustībā bija divi pamatelementi: pirmkārt, pārsvarā šļahtiču vestā nacionālās atbrīvošanās kustība; otrkārt, zemnieku antifeodālā cīņa. Abas attīstījās paralēli, taču ne sinhroni. Abas tās stiprināja viena otru, kaut pilnīgas saskaņas starp tām gandrīz nekur nebija. Krievijas rietumu guberņu zemnieki bija lietuvieši, baltkrievi, ukraiņi. Daļa no viņiem sacēlušos formējumu sastāvā atradās piespiedu kārtā, pēc savu muižnieku pavēles. Tā daļa, kura apzināti stājās sacēlušos rindās, sekoja saviem mērķiem, kuri bieži atšķīrās no sacēlušos valdības pasludinātajiem. Daudzi zemnieki, īpaši Lietuvā, uzstājās ne tik daudz pret cara valdību, cik pret klaušu sistēmu, saviem, pārsvarā poļu, muižniekiem.

Pēc sacelšanās poļu emigrācijas lielākā daļa, sastāvoša no vidējiem un zemākajiem šļahtiskās un buržuāziskās inteliģences slāņiem, nāca pie secinājuma, ka kopumā Polijas krišanas galvenais cēlonis bija šļahtas anarhistiskā kundzība, bet 1830.--1831.~gada sacelšanās sakāves cēlonis~--- sacelšanās vadošo slāņu konservatīvisms, bailes iesaistīt cīņā iedzīvotāju pamatmasas~--- zemniecību. Tā, viens no bijušajiem Patriotiskās biedrības kreisā spārna vadītājiem T.~Krepovickis, vēlāk emigrācijā analizējot 1830.--1831.~gada sacelšanās mācības, šļahtičus~--- muižu īpašniekus sauca par ``Tēvzemes un cilvēces nodevējiem'', apgalvojot, ka tās pārstāvji ``nebija nedz dižas idejas, nedz revolucionāras domas iedvesmoti''.

1848.~gadā, pieminot 1830./1831.~gada notikumus Polijā, F.~Engelss konstatēja, ka Polijas valdošā šķira izrādījās tikpat egoistiska, aprobežota un bailīga likumdošanas sapulcē, cik pilna entuziasma un vīrišķības tā bija kaujas laukā. Pēc F.~Engelsa vērtējuma ``1830.~gada sacelšanās nebija ne nacionāla revolūcija (tā atstāja aiz borta trīs ceturtdaļas Polijas), ne sociāla vai politiska revolūcija, tā neko neizmainīja tautas iekšējā stāvoklī, tā bija konservatīva revolūcija''. Arī poļu vēsturnieki V.~Bortnovskis, J.~Dutkevičs, V.~Zajevskis u.c. XX gadsimta 50.--60.~gados uzsvēra, ka sacēlušos kreisais spārns, kurš sapratis, ka sacelšanās var uzvarēt, tikai ejot pa radikālu politisku un sociālu reformu ceļu, izrādījās pārāk vājš, tāpēc vara nonāca aristokrātisko grupējumu rokās, kuri nevēlējās nekādas reformas un karadarbības jēgu saskatīja tikai cīņā ar carisko Krieviju par ukraiņu, baltkrievu un lietuviešu zemēm, kādreiz piederējušām Žečpospolitai. Pēc autora domām, nevajadzētu noliegt sacelšanās nacionālās atbrīvošanās un arī politisko raksturu, taču citādi ar nelielām iebildēm F.~Engelsa bargajam spriedumam var piekrist. Sacelšanās cieta sakāvi, jo tās dalībnieki nespēja izvirzīt skaidru un tālejošu nacionālu un sociālu programmu.

Padomju literatūrā tika izplatīti spriedumi par minētās sacelšanās starptautisko nozīmi, jo tā veicināja revolucionārās kustības attīstību daudzās Eiropas valstīs, to sveica dekabristi un krievu jaunākās paaudzes revolucionāri. Šeit gan jāatgādina pašu marksistu atzinums, ka katrā revolūcijā galvenie ir iekšējie cēloņi. Polijas pēc F.~Engelsa vārdiem ``konservatīvā revolūcija'' kalpot par paraugu attīstītākajām Eiropas zemēm varēja gandrīz vienīgi ar savu faktu, nevis programmu un taktiku.

Mūsdienu krievu publicists M.~Muravjovs uzsvēris: ``Vēlme sagrābt ``atņemto novadu'' (tā poļi sauca Krievijas impērijas rietumu guberņas, kuras ietilpa Žečpospolitas sastāvā līdz tās sadalei) kļuva par uzmācīgu poļu ``nacionālo ideju''. Loģika šajos centienos bija vienkārša: Polija savās etnogrāfiskajās robežās, praktiski sakrītošās ar ``kongresovku'', [Publicists šeit nav korekts, poļu etnogrāfiskās robežas ievērojami pārsniedza t.s. ``kongresa Polijas'' ietvarus, par ko jau augstāk runāts~--- V.Š.] iegūstot neatkarību, būs maza un neietekmīga valsts. Toties, paplašinot robežas uz austrumiem, Polija atkal kļūs par lielvalsti. Tāpēc visas poļu sacelšanās pret Krieviju bija ar nacionālas sagrābšanas raksturu. To apspiešana no Krievijas puses bija nepieciešama pašaizsardzība.'' Publicists gan nepamatoti nostājies cariskās Krievijas režīma, kurš vairāk rūpējās par valsts rietumos dzīvojošo tautu ekspluatāciju, nevis aizsardzību pret poļu kundzību, apoloģēta lomā, taču viņa norāde uz faktiski Polijas šļahtas interesēs esošā mērķa~--- citu etnosu apdzīvoto teritoriju iekļaušanas Polijas valstī pacelšanu ``nacionālās idejas'' augstumos ir pelnījusi ievērību. Kādreizējā Žečpospolitas ``vēsturiskā diženuma'' atjaunošanas ideja vēl mūsdienās darbojas kontrproduktīvi Polijas attīstības ceļu meklējumos.

Pēc 1830.~gada sacelšanās sakāves \strong{Krievija izvērsa cīņu pret} valsts rietumos dzīvojošo tautu \strong{polonizāciju}. Poļu literatūrā jēdziens ``polonizācija'' tiek traktēts kā ``pašpolonizācija'', kā cilvēku brīvprātīga izvēle. Jāievēro, ka Žečpospolita nebija nacionāla valsts mūsdienu izpratnē, nāciju veidošanās laikmetā Polijas valsts nepastāvēja un poļu rokās nebija valstisku līdzekļu asimilācijas realizēšanai. Taču XIX gadsimta sākumā nedrīkst arī pārāk zemu novērtēt tādu reliģiozu, kultūras un sociālu institūtu kā katoļu baznīcas, oficiālās skolas, vietējās administrācijas (līdz 1863--1864.~gada sacelšanās laikam), muižas ietekmi uz nepoļu iedzīvotājiem poliskuma izplatībā pat apstākļos, kad nepastāvēja Polijas valsts. Protams, jāuzsver arī, ka Krievijas iestāžu vēršanās pret polonizāciju reizē nozīmēja virzību uz rusifikāciju.

Ievērojamākajiem sacelšanās dalībniekiem~--- emigrantiem, kā Ā.~Čartorijskim un J.~Lelevelam, aizmuguriski tika piespriests nāves sods. Šļahtičiem~--- sacelšanās dalībniekiem karalistē tika atsavinātas ap 3~000 muižu (apmēram 1/10 no lielsaimniecībām).

1833.~gada septembrī Krievija un Austrija Minhengrecā (vācu \detxti{Münchengrätz}, čehu \cstxti{Mnichovo Hradiště)} noslēdzot līgumu par kopēju politiku pret Turciju, saskaņoja arī konvenciju par \latxti{status quo} ievērošanu poļu zemēs un politisko noziedznieku savstarpēju izdošanu. Pēc mēneša tai pievienojās arī Prūsija. Konvencija faktiski deva trīspusējas garantijas tai kārtībai, kura tika radīta Krievijai piederošajā Polijas daļā pēc 1830.--1831.~gada sacelšanās.

Pat vēl vairāk kā Polijas karalisti carisma sodu politika skāra bijušās Lietuvas lielkņazistes teritorijas. Poliskās izglītības iestādes tika rusificētas, Viļņas universitāte 1832.~gadā slēgta. Daļa poļu šļahtiču Lietuvā un vēlāk arī tagadējās Ukrainas un Baltkrievijas teritorijā zaudēja savas dižciltīgo tiesības, arī muižas, tika pazemināti līdz viensētniekiem (\rutxti{однодворцы}), kuri, kaut bija personīgi brīvi, bet maksāja valstij nodevas. Sākusies jau XIX gadsimta 20.~gados dokumentu, kas apliecinātu šļahtiču dižciltīgo izcelsmi, pārbaude stipri ieilga un faktiski nebija pa spēkam smagnējai Krievijas birokrātiskajai mašīnai. Tomēr tā veicināja muižnieku norobežošanos no saviem nabagajiem kārtas brāļiem. Tiesa, lielākā daļa muižu palika to agrāko īpašnieku rokās, carisms turpināja aizstāvēt muižniecības intereses. 1836.~gadā Nikolajs I poļu muižniecības tiesības un privilēģijas pielīdzināja krievu muižniecības kā kārtas tiesībām.

Jautājumu par poļu ierēdņu izmantošanu Krievijas impērijas dienestā noteica 1852.~gada 14.~novembrī imperatora apstiprināts nolikums. Tajā bija uzskaitītas 28~guberņas, kuru iestādēs varēja strādāt poļu jaunieši. Šai uzskaitījumā nebija Pēterburgas guberņas, Baltijas, Besarābijas, toreizējās Mazkrievijas (Ukrainas), Kaukāza, Sibīrijas apgabalu. No tām guberņām, kuras robežojās ar Rietumu novadu, sarakstā bija tikai divas~--- Pleskavas un Smoļenskas guberņas.

Daļu poļu aktīvistu izsūtīja no Polijas karalistes un nometināja citās Krievijas impērijas guberņās, arī Kaukāzā un Sibīrijā. (Pēc vācu vēsturnieka H.v.~Zitzevica datiem tika izsūtīts ap 45~000 poļu šļahtiču.) Taču nevajadzētu pārspīlēt viņu ciešanas. Kā raksta mūsdienu poļu vēsturniece V.~Sļivovska, vairumā gadījumu pat uz katorgas darbiem notiesātie poļi, izņemot pirmās dienas, faktiski neizcieta piespriesto sodu. Katordznieku jūgu patiesībā nesa tikai kriminālnoziedznieki. Lieta tā, ka visā Sibīrijā turīgajiem tirgoņiem, vietējās administrācijas pārstāvjiem bija vajadzīgi ārsti, skolotāji, kuri spētu sagatavot viņu bērnus iestājai ģimnāzijā vai augstskolā, audzinātāji, kuri varētu mācīt bērniem mūziku, dejas, glezniecību. To visu varēja veikt izsūtītie poļi. Galvaspilsētas bija tālu, uz vietas visu lēma ģenerālgubernatori un tiem pakļautie ierēdņi. Poļus labprāt ņēma darbā dažādās kancelejās ``papīra darbiem''. Ar laiku daži kļuva pat par zelta raktuvju administratoriem. Oficiālajās atskaitēs viņi visi nesa tiem uzlikto sodu, ir zināms, ka Demidovu (\rutxti{Демидовы}~--- bagāta krievu uzņēmēju dzimta) rūpnīcā viņiem bija izgatavotas arī speciālas, viegli noņemamas važas~--- speciāli revizoru atbraukšanas gadījumiem. Pēc 1856.~gada amnestijas avīze ``\rutxti{Сибирь}'' (``Sibīrija'') ar nožēlu rakstīja par to, ka gandrīz visi šeit mitušie poļi aizbraukuši.

1832.~gada 14.~februārī Polijas 1815.~gada Konstitūcijas vietā tika ievests Organiskais statuts (poļu \pltxti{Statut Organiczny Królestwa Polskiego}, krievu \rutxti{Органический статут Царства Польского}). Tiesa, krievu vēsturnieks A.~Korņilovs uzsvēra, ka tas pilnībā tā arī nekad netika ieviests, jo poļu revolucionāri atkal gatavojās sacelties. Tomēr faktiski Polijas karaliste pārvērtās par vienkāršu Krievijas provinci, neatņemamu Krievijas impērijas sastāvdaļu. Īpaša kronēšana ar tās kroni vairs nebija paredzēta. Līdz 1856.~gadam Krievijas Polijā darbojās kara stāvoklis. Seims tika likvidēts. Saglabājās tikai dažas autonomas iestādes. Pārvaldes padomes locekļi vairs nebija poļi. Padomes kompetence tika aprobežota ar reliģijas, kultūras un tirdzniecības lietām. Tikai līdz 1841.~gadam saglabājās Valsts padome kā konsultatīva iestāde pie Krievijas Valsts padomes (1861.~gadā, sākoties revolucionāram pacēlumam, to gan atkal atjaunoja). 1832.~gadā poļu valūta~--- zlots tika nomainīta ar Krievijas rubli. Uz monētām, kuras tika kaltas 1832--1842 gadā, bija uzraksts «\rutxti{15 коп.}~--- \pltxti{1 zloty}» (15 kapeikas~--- 1 zlots).

Poļu armija tika likvidēta, tās vietā ieviesta rekrūšu ņemšana Krievijas armijā. Polijā tika izvietota stipras Krievijas armijas daļas. 1832.--1834.~gadā augstienē pie Varšavas pēc imperatora Nikolaja I pavēles 1832.--1834.~gadā tika uzbūvēts cietoksnis, kurš izmaksāja tolaik milzīgu summu~--- 11 miljonus rubļu. To kā sodu par sacelšanos bija jāsedz Varšavas pilsētai un tagadējās Polijas Nacionālās bankas priekštecei. Miera laikā šai Varšavas citadelē atradās 5~000 karavīru. Cietokšņa lielgabali bija notēmēti uz pilsētu. Imperators draudēja ar tiem nolīdzināt Varšavu no zemes virsas mazākās nepakļaušanās gadījumā. Cietoksnis tika izmantots arī kā ieslodzījuma vieta. Tā cellēs varēja turēt līdz 3~000 ieslodzīto. Citadelē tika izpildīti arī nāves sodi. Apmeklējot 1835.~gadā pilsētu, Nikolajs I atteicās pieņemt pilsētnieku delegāciju ar padevības apliecinājumiem, paziņojot, ka tā viņš atbrīvo tos no nepieciešamības melot.

Tiesa, Nikolajs I tā arī nespēja realizēt savu nodomu radīt Polijas karalistē spēcīgu krievu ierēdniecības slāni. Neraugoties uz aicinājumiem kalpot tēvijai Polijā, citiem imperatora gribas izpaudumiem, krievi tā īsti nespēja iedzīvoties poļu sabiedrībā, bija pārstāvēti galvenokārt Krievijas armijas sastāvā.

Turpretī Pēterburgā gan pastāvēja daudzskaitlīga poļu kolonija, kura kļuva par poļu-krievu sabiedrisko sakaru uzturēšanas centru.

Ir jāuzver, ka nebūt ne visi poļu izcelsmes aristokrāti, šļahtiči bija nelokāmi Krievijas valdīšanas Polijā pretinieki. Daudzi poļu muižniecības pārstāvji uzticīgi kalpoja Krievijas imperatoram un Polijas karalim. XIX~gadsimta 50.~gados Pēterburgas ierēdniecībā poļu īpatsvars sastādīja 6\%. To, ka daudzi poļi ieguva izglītību un izcilu stāvokli sabiedrībā, pateicoties dzīvei Krievijā, patriotiski noskaņotie poļi ilgstoši nevēlējās redzēt. Viņi uzskatīja, ka, piemēram, Pēterburgā, pozitīvi vērtējams polis varēja atrasties vai nu kā ieslodzītais, kā T.~Kostjuško, vai kā troni zaudējis monarhs, kā S.~Poņatovskis. Patiesībā Pēterburgā pastāvīgi dzīvoja poļi~--- Krievijas Valsts padomes locekļi (vēlāk, jau 20.~gadsimtā, arī Krievijas Valsts domes deputāti), poļu aristokrātijas pārstāvji, galma titulu nesēji, gvardes pulku virsnieki un Kadetu korpusa audzēkņi, tātad cilvēki, kuri jau bija dzīvē ko sasnieguši vai arī gatavojās kāpt pa karjeras kāpnēm.

No izcilā poļu dzejnieka Ā.~Mickeviča biogrāfijas Polijā parasti atceras viņa ieslodzījumu Viļņā, izsūtīšanu uz Krieviju un gadus, pavadītus emigrācijā. Taču vairāk nekā gadu viņš nodzīvoja Pēterburgā, apmeklējot vietējos salonus, uzturot kontaktus ar krievu sabiedriskajiem un kultūras darbiniekiem. Kaut Varšava bija visai dinamisks kultūras centrs, kurš saistīja arī daudzus krievus, tomēr salīdzinājumā ar Pēterburgu visu XIX gadsimtu tā bija kultūras province. Poļu inteliģenti, kuri visai sarkastiski atsaucās par visu krievisko, ar pasaules literatūras sasniegumiem bieži iepazinās, pateicoties labiem un lētiem to tulkojumiem krievu valodā, kuri tika iespiesti impērijas galvaspilsētā. Kā noskaidrojis poļu zinātnieks L.~Baziļevs, no XVIII gadsimta beigām līdz Pirmā pasaules kara sākumam Pēterburgā vien īsāku vai garāku laiku (no dažām dienām līdz visai savai dzīvei) pabija apmēram 1/4 miljons poļu. Tiek uzskatīts, ka XIX gadsimta 40.~gados katrs trešais no Pēterburgas universitātes 750 studentiem varēja būt polis no Lietuvas vai Polijas karalistes. Viņi uzskatīja sevi par Krievijas monarha pavalstniekiem, realizēja viņa impērijas politiku, darbojās apzinīgi un uzcītīgi, saņēma par to apbalvojumus un paaugstinājumus amatos, tai pat laikā nezaudējot savu poļu apziņu un ieinteresētību poļu stāvokļa uzlabošanā, bieži palīdzot poļu iestādēm gan finansiāli, gan ar savu ietekmi. XIX~gadsimta vidū 3\% no visiem Krievijas ierēdņiem bija poļi.

Samērā daudz šļahtiču dienēja Krievijas armijā. 1862.~gadā Krievijas armijas virsnieku vidū pareizticīgo bija 69,37\%, katoļu~--- 20,06\% un protestantu~--- 9,33\%. Statistika pēc tautībām netika vesta, taču katoļi gandrīz visi bija poļi, bet protestanti~--- vācieši. Tātad katrs piektais imperatora armijas virsnieks bija polis, kaut Krievijas iedzīvotāju vidū poļi sastādīja mazāk nekā 6\%. Kā raksta poļu vēsturnieki, nevar noliegt, ka daļa poļu patriotu dienestu valstu~--- Polijas pakļāvēju~--- armijās uzskatīja par iespēju apgūt karavīra prasmes, kuras pēc tam varētu izmantot nacionālās lietas labā. Virkne poļu izcelsmes virsnieku~--- Z.~Serakovskis, Z.~Padļevskis, R.~Trauguts u.c.~--- savas militārās zināšanas un prasmes vēlāk lika lietā pret Krieviju 1863.~gada sacelšanās laikā, par ko pēc sacelšanās sakāves arī tika pakārti vai nošauti.

Kā poļu revolucionāra dzīves gājuma piemēru var uzskatīt Z.~Serakovska biogrāfiju. Viņš bija dzimis nabadzīga šļahtiča ģimenē Volīnijā, mācoties Pēterburgas universitātē, iesaistījās revolucionāra studentu pulciņa darbībā, tika arestēts un nosūtīts karadienestā. Pēc astoņiem dienesta gadiem viņš saņēma virsnieka pakāpi, atgriezās Pēterburgā, beidza Ģenerālštāba akadēmiju, dienēja Kara ministrijā. Tās uzdevumā pabija komandējumos ārzemēs, kur iepazinās ar Dž.~Garibaldi, poļu emigrantiem, krievu revolucionāro demokrātu A.~Hercenu. Krievijā viņš uzturēja sakarus ar revolucionārajiem demokrātiem N.~Černiševski un N.~Dobroļubovu. 1859.--1860.~gadā Z.~Serakovska vadībā tika izveidots Krievijas Ģenerālštābā dienošo poļu virsnieku pulciņš, kura darbībā piedalījās arī poļu civilierēdņi un studenti. Pēc sacelšanās Polijā sākuma viņš palūdza atvaļinājumu ārzemju apmeklējumam, bet pats devās uz Lietuvu, kur kļuva par vienu no sacelšanās vadītājiem.

Taču bija arī pretēji, uzticīga dienesta Krievijai piemēri.

Lai nosaucam dažus no tiem. S.~Grabovskis XVIII~gadsimta beigās un XIX~gadsimta sākumā piedalījās vairākos poļu karos pret Krieviju, pabija arī izsūtījumā Sibīrijā, pēc apžēlošanas Napoleona armijas sastāvā atkal iesaistījās karagājienā pret Krieviju, līdz 1813.~gadā kaujā pie Leipcigas krita krievu gūstā. Kopš 1815.~gada viņš dienēja Polijas karalistē, bija tuvs lielkņazam Konstantīnam. 1825.~gadā dekabristu sacelšanās laikā, būdams Pēterburgā, spēja nodemonstrēt savu uzticību Nikolajam I, kļuva par Polijas karalistes valsts sekretāru. 1830.~gada sacelšanās laikā S.~Grabovskis to neatbalstīja, bet kļuva par Krievijas ieceltās Polijas Pagaidu valdības un Krievijas Valsts padomes locekli.

F.~Druckis-Lubeckis piedzima Pēterburgā un, neraugoties uz poļu izcelsmi, palika vienmēr uzticīgs Krievijas monarham un iestājās par Polijas un Krievijas vienošanos. Viņš piedalījās A.~Suvorova vadītajos karagājienos uz Itāliju un Šveici. Napoleona karagājiena uz Krieviju laikā F.~Druckis-Lubeckis dienēja Krievijā un bija atbildīgs par tās armijas apgādi. 1813.--1815.~gadā bija Krievijas okupētās Varšavas hercogistes Pagaidu augstākās padomes loceklis, 1821.~gadā kļuva par Polijas karalistes finanšu ministru, centās paplašināt karalistes autonomiju. 1830.~gada sacelšanās laikā viņš veda Nikolajam I sacēlušos prasības, taču pēc atbildes nogādāšanas Varšavā atgriezās Pēterburgā, ar to saglabājot imperatora uzticību. 1832.~gadā F.~Druckis-Lubeckis tika nozīmēts par Krievijas Valsts padomes locekli.

Ļoti daudzos XIX gadsimtā Krievijas vestajos karos drosmīgi cīnījās ģenerālis Ā.~Ževuzskis. Kad sākās 1830.~gada poļu sacelšanās, viņš, neraugoties uz savu polisko izcelsmi, saglabāja uzticību Krievijas imperatoram un aktīvi piedalījās sacelšanās apspiešanā, par ko tika apbalvots ar zelta zobenu, uz kura bija uzraksts ``Par drosmi'', un vairākiem ordeņiem. Arī 1863.--1864.~gadā Ā.~Ževuzskis, būdams Kijevas kara apgabala pavēlnieks, palika atstatus no sacelšanās Polijā.

Tādejādi nevajadzētu uzskatīt, ka jebkurš polis patriotisma vārdā bija gatavs cīnīties pret Poliju pakļāvušajām valstīm, aizmirstot par visu citu. Daļa no viņiem spēja saskatīt, ka Polijas karaliste sacelšanās rezultātā ne tikai zaudēja to uzplaukumu, kuru tā sasniedza ap F.~Drucki-Ļubecki saliedēto darbinieku ekonomiskās politikas rezultātā, bet arī savu Konstitūciju un vismaz nominālo patstāvību, kļuva par Krievijas provinci, kaut arī zināma autonomija vēl saglabājās. Kā rakstīja poļu vēsturnieks O.~Haleckis, Polijas īpašais valstiskais statuss tika nomainīts ar ``simulētu autonomiju''.

1837.~gadā poļu vojevodistes pielīdzināja Krievijas guberņām, kuru priekšgalā atradās iecelti gubernatori. Skolu lietas tika nodotas tieši Tautas Izglītības ministrijas Pēterburgā pārziņā. Tika slēgta Varšavas un Viļņas universitātes. (Interesanti, ka Nikolajs I, apzinoties katoļu reliģijas nepieciešamību kārtības uzturēšanai, šo augstskolu teoloģijas fakultātes saglabāja kā atsevišķas mācību iestādes.) Īpaši otrās slēgšana negatīvi ietekmēja arī lietuviešu un baltkrievu sabiedriski politisko un nacionāli kulturālo dzīvi. Samazinājās ģimnāziju skaits. Tika ieviesta stingra cenzūra. Bija aizliegts ne tikai drukāt poļu dzejnieku un citu kultūras darbinieku darbus, bet arī pieminēt viņu vārdus.

Vēloties mazināt poļu ietekmi uz citām slāvu tautām, Krievijas valdība centās ierobežot katoļu bīskapiju skaitu. Saskaņā ar draudžu skaita krišanos samazinājās arī katoļu baznīcu skaits. Katoļu draudzi varēja nodibināt, ja tā aptvēra 100 ēku un 400 ticīgo. Pēc 1830.--1831.~gada sacelšanās valdība spēra soļus, lai vēl vairāk ierobežotu katoļu garīdznieku, kuru vairākums netieši bija atbalstījuši sacelšanos, ietekmi. Viņiem aizliedza atstāt savas draudzes bez speciālas atļaujas. Baznīcu zemes īpašumus konfiscēja. Pieauga kā pareizticīgās, tā katoļu baznīcas atkarība no valsts. 1841.--1843.~gadā tās nonāca valsts apgādībā. 1842.~gadā garīdzniekiem noteica algas. Ar laiku arvien redzamāka kļuva valsts labvēlība tieši pareizticīgajai baznīcai. 1839.~gadā Polockā (krievu \rutxti{Полоцк}, baltkrievu \rutxti{Полацк}) notika uniātu baznīcas koncils, kas nolēma pievienot uniātu draudzes pareizticīgo baznīcai. Sākās aktīva uniātu draudžu pievienošana pareizticīgajām.

Arī Polijas karalistei blakus esošajās guberņās (Grodņas, Kauņas, Minskas, Viļņas, Vitebskas) tika ievadīti rusifikācijas pasākumi. Pēc Krievijas piemēra Prūsija un Austrija, kuras atbalstīja Krievijas pasākumus pēc sacelšanās apspiešanas, tagad varēja arī pašas daudz asāk vērsties pret poļu nacionālajām prasībām. Tā Prūsijas varas iestādes izdeva Krievijai lielu daļu poļu karavīru, kuri pēc sacelšanās bija nonākuši tās teritorijā. Poļi~--- Prūsijas pavalstnieki, kuri bija karojuši sacēlušos rindās, tika neklātienē notiesāti.

Ar poļu 1830.--1831.~gada sacelšanās apspiešanu notikušais pavērsiens Krievijas impērijas politikā Polijā izpaudās arī Krievijas sabiedriskajā dzīvē. 1831.~gadā tapa A.~Puškina dzejolis ``\rutxti{Клеветникам России}'' (``Krievijas apmelotājiem''). Lūk, fragments no tā:

\vspace{1.5em}

\noindent
\begin{minipage}{0.55\textwidth}
\pltxti{Уже давно между собою\\
Враждуют эти племена;\\
Не раз клонилась под грозою,\\
То их, то наша сторона.\\
Кто устоит в неравном споре:\\
Кичливый лях иль верный росс?\\
Славянские ль ручьи\\
сольются в русском море?\\
Оно ль иссякнет? Вот вопрос.}
\end{minipage}
\hspace{1em}
\begin{minipage}{0.6\textwidth}
Sen sens ir naids\\
Starp ciltīm šīm;\\
Ne reizi vien ir nācies vētrā liekties,\\
Gan viņējai, gan mūsu pusei.\\
Kurš uzvarēs šai strīdā nevienlīdzīgā:\\
Vai lepnais poļu pans vai uzticamais krievs?\\
Vai krievu jūrā\\
slāvu strauti saplūdīs?\\
Vai izsīks tā? Lūk, jautājums.
\end{minipage}

\vspace{1.5em}

% page 128




Dzejolī atklājās dzejnieka domu gaita: gadsimtiem ilgās cīņās Krievija bija pieveikusi Poliju. Tāpēc Krievijas pavalstniekiem arī turpmāk bija jāpaliek ``uzticamiem krieviem''~--- sava cara kalpiem, bet ``lepnie poļu pani'', kas priekšroku deva savām feodālajām privilēģijām, faktiski bija zaudējuši savu tautu, atdodot to svešzemnieku (prūšu un austriešu) kundzībā, no kā bija tikai viens glābiņš~--- ``ieplūst krievu jūrā''.

Nebija nejaušība arī tā, ka 1836.~gadā pirmuzvedumu piedzīvoja krievu komponista M.~Gļinkas opera ``Dzīvību par caru'' (\rutxti{Жизнь за царя}), vēlākais nosaukums~--- ``Ivans Susaņins'' (\rutxti{Иван Сусанин}). Operas darbība notiek 1613.~gadā, kad ``juku laikos'' ieslīgušajā Krievijā tronī nāca Romanovu dinastijas dibinātājs Mihails. Tā vēstīja par krietno krievu zemnieku I.~Susaņinu, kurš, samaksājot ar savu dzīvību, izglāba caru no poļu iebrucējiem. Jau 1862.~gadā krievu-ukraiņu vēsturnieks N.~Kostomarovs šo faktu apstrīdēja. Arī mūsdienu literatūrā norādīts, ka leģendu neapstiprina zinātniski dati, tomēr Krievijā tā joprojām tiek izmantota patriotisma audzināšanā. M.~Gļinkas estētiskās koncepcijas pamatā bija varonīgo krievu un nelietīgo poļu pretnostatījums, ko atspoguļo gan divi kori~--- poļu un krievu, gan divi kontrastējoši scenogrāfijas un mūzikas stili. Poļus raksturo dižmanība, viņi dzied tikai korī un dejo ceremoniālas masu dejas polonēzes, mazurkas un krakovjaka ritmā. Turpretī krievi dzied vai nu jaukas tautasdziesmas, vai romantiskas ārijas. Pēc I.~Susaņina nogalināšanas epiloga kulminācija ir aina ar Mihaila Romanova triumfālo ierašanos Maskavā Sarkanajā laukumā. Ar M.~Gļinkas darbu cariskajā Krievijā atklāja ikvienu operas sezonu Maskavā un Sankt-Pēterburgā. To 1866.~gadā iestudēja Prāgā čehu valodā, 1878.~gadā~--- Rīgā latviešu valodā un 1899.~gadā~--- Pozenes Vācu teātrī Prūsijā. Taču nekad to neuzveda Varšavā vai Krakovā.

Sakarā ar sacelšanās apspiešanu Polijā Krievijas sabiedrībā iezīmējās negatīvas tendences. Ne bez pamata Nikolajam I tuvi cilvēki sūdzējās, ka ``poļu revolūcija aizturēja Krievijas attīstību par 30~gadiem''. Krievijas slavofīli uz Poliju raudzījās visai aizdomīgi, cīnītājus par Polijas neatkarību vērtēja kā Krievijas un visu slāvu tautu vienotības ienaidniekus. Arī vēsturnieks A.~Pogodins uzskatīja, ka Krievija, kura jau bija izkļuvusi no Austrijas un tās ārlietu ministra, viena no Svētās savienības arhitektiem K.~Meterniha ietekmes un tuvinājās Anglijai, atkal tika iesviesta Svētās savienības apkampienos. Pēc poļu sacelšanās Nikolajs I pulcēja ap sevi konservatīvi noskaņotus cilvēkus. Šajā laikā, kā rakstīja vēsturnieks, ``Krievijā gāja bojā Puškins,~Ļermontovs, Gogolis, apklusa jebkura brīvdomība, zēla korupcija''.

Pašā Polijas karalistē pēc 1830--1831.~gada sacelšanās nacionālās atbrīvošanās kustības veidols ievērojamā mērā demokratizējās. No 1835.līdz 1845.~gadam kustībā ievērojami auga tā slāņa īpatsvars, kuru poļu vēsturnieki visbiežāk sauc par inteliģenci: sīkie ierēdņi, ārsti, skolotāji, pārvaldnieki, t.s. brīvo profesiju pārstāvji, studenti. Tas bija tipisks pārejas perioda slānis. Tā atvērtība un augošā loma padarīja garīgo darbu pievilcīgu kā nabadzībā grimstošajai šļahtiču daļai, kura centās pretoties savai noslīkšanai sabiedrības padibenēs, kā arī pilsētniekiem, kuriem izglītības iegūšana solīja vilinošu perspektīvu uzkāpt pa sabiedriskās hierarhijas kāpnēm. Runājot par poļu inteliģenci kopumā, jāuzsver, ka materiāli nodrošināto slānis tās rindās bija ļoti plāns, pārsvarā bija grupas, kuras knapi varēja iztikt, nebija pārliecinātas par savu rītdienu, saistīja savu nākotni ar politiskām pārmaiņām.

Pēc dažiem datiem no 1835. līdz 1841.~gadam šī inteliģences slāņa īpatsvars sastādīja 43\%, bet no 1842.~līdz 1845.~gadam~--- 61\% no visiem atbrīvošanas kustības dalībniekiem. Kustība bija ļoti raiba pēc sava sastāva, arī tās ideoloģija bija pretrunīga. Taču nacionālo prasību jomā visas politiskās grupas un partijas prasīja atjaunot Poliju 1772.~gada robežās. Tomēr arī šeit vienotība bija pārsvarā ārēja. Vieni gribēja atjaunot veco Žečpospolitu, citi tās teritorijā cerēja nodibināt demokrātisku republiku, kurai būtu jāveic radikāli sociāli pārveidojumi, bet pēc tam jādod ukraiņiem, baltkrieviem un lietuviešiem pašiem lemt savu likteni.

Poļu zemēs darbojās virkne nelegālu organizāciju.

40.~gadu sākumā Varšavā darbojās konspiratīva organizācija, saistīta ar publicistu E.~Dembovski, kurš minēts jau sakarā ar notikumiem Galīcijā. Lietas izmeklēšanas gaitā tā ieguva nosaukumu ``Demokrātiskā biedrība''. Nosaukums nebija nejaušs. Biedrības programmatiskie dokumenti nav saglabājušies, taču tās dalībnieku izteikumi ļauj spriest, ka par savu mērķi tā stādīja Polijas valstiskuma atjaunošanu, sociāli-ekonomisko attiecību pārveidi.

Vienu no konspiratīvajām poļu organizācijām, kura pastāvēja 1840.--1844.~gadā, vadīja zemnieku izcelsmes katoļu garīdznieks P.~Scegennijs. No tiem šīs organizācijas biedriem, kuri tika saukti pie atbildības par pretvalstisku darbību, apmēram puse bija zemnieki, starp pārējiem bija divi muižnieku dēli, daži muižu rentnieki un amatnieki, pārējie~--- sīki ierēdņi, kā arī skolotāji, privātkalpotāji, ģimnāzisti, studenti.

Organizācijas politiskā programma saturēja prasības pēc tautvaldības un visu līmeņu pārvades iestāžu ievēlējamības. Nākotnes valstiskumu P.~Scegennijs un viņa līdzgaitnieki iedomājās kā daudznacionālu republikānisku federāciju, kurā vadošā vieta piederētu poļu tautai. Federācijas robežas bija paredzēts noteikt pa Oderas, Nisas, Daugavas un Dņepras upēm. Tika uzskatīts, ka federācijai pievienosies arī Austrijas impērijā dzīvojošie slāvi.

P.~Scegennija un viņa līdzgaitnieku sociālā programma paredzēja feodāli-dzimtbūtnieciskās kārtības, kārtu, nacionālās un reliģiskās nelīdztiesības likvidēšanu, zemes pasludināšanu par visas tautas īpašumu, izdalot lietošanā ar mantošanas tiesībām saimniecības tiem zemniekiem, kuri tās apstrādāja. Organizācijas dalībnieki bija pārliecināti, ka viņu prasību īstenošana radīs sabiedrību, kurā visi būs laimīgi, un uz mūžīgiem laikiem valdīs brīvība, vienlīdzība un brālība. Faktiski šī programma zem utopiskā sociālisma formas izteica darbaļaužu, pirmkārt zemnieku, intereses buržuāziskās revolūcijas gaitā. Tā atspoguļoja tās izmaiņas, kuras notika poļu atbrīvošanās kustībā. Pat cara vietvaldis I.~Paskēvičs par 40.~gadu konspiratoriem rakstīja, ka viņi ``stāda mērķi ne tikai gāzt krievu valdību Polijā, bet reizē iznīcināt arī poļu šļahtičus-muižniekus.''

Poļu muižniecības augšslāņi bija vislabāk piemērojušies valdošajai kārtībai. Krievu vēsturnieks N.~Bergs rakstīja, ka vietvaldis I.~Paškevičs, kurš 1856.~gada sākumā šķīrās no dzīves, ``atstāja Poliju pilnīgi mierīgu \citespace{} Aristokrāti apspieda visu, bija sava veida vietvalža līdzvaldnieki. Ko tikai toreiz nevarēja poļu tiesās ietekmīgs poļu aristokrāts, īpaši ievērojot poļu ierēdņu neparasto korumpētību! Ar aristokrātu palīdzību Paškēvičs darbināja visas savas atsperes \citespace{} viņš uzskatīja šādu pārvaldes metodi ne tikai par labāko, bet arī par neizbēgamu, ievērojot vispārējo lietu stāvokli impērijā un toreizējos vispārējos uzskatus.'' Pēc tā paša N.~Berga domām nākamais vietvaldis kņazs M.~Gorčakovs aristokrātiem deva pārāk lielu vaļu un neizrādīja pietiekamu stingrību pret poļiem.

Taču daudzi šļahtiči turpināja pretestību carismam, devās emigrācijā. Pēc poļu publicista un vēsturnieka L.~Vasiļevska datiem pēc 1831.~gada emigrācijā atradās ap 70~tūkstošu sacelšanās dalībnieku un atbalstītāju. Pat neraugoties uz cara izsludināto amnestiju, daļa 1830.~gada sacelšanās dalībnieku, turpināja dzīvot \strong{emigrācijā} (galvenokārt Šveicē un Francijā), ko poļu vēsturiskajā literatūrā sauc par ``lielo''. Emigrācijā dzīvoja dzejnieks Ā.~Mickevičs, komponists F.~Šopēns u.c. Pēc citiem, vācu vēsturnieku datiem Eiropā darbojās līdz 10~000 poļu emigrantu. 1847.~gadā tikai Francijā vien bija 4~790, bet Anglijā ap 800 iebraukušo no Polijas. Poļu historiogrāfija pirmkārt norāda uz šīs emigrācijas garīgo nozīmīgumu, bet ne tās skaitu. Emigranti bija galvenokārt virsnieki un inteliģences pārstāvji, cēlušies no šļahtičiem, daļēji arī no pilsonības. Poļu nacionālās kustības smaguma centrs pārvietojās uz ārzemēm, īpaši Franciju. Lielā emigrācija padziļināja poļu patriotisma antikrievisko, šovinistisko raksturu. Eiropas demokrātiskās sabiedrības atbalsts, liberāļu simpātijas daudzos emigrantos stiprināja apziņu, ka poļiem lemts kā savulaik Jēzum Kristum kļūt par mesiju apspiesto tautu atbrīvošanās kustībā.

Emigranti, kaut arī dalījās aristokrātiskajā un demokrātiskajā spārnā, lielākoties bija gatavi piedalīties katrā Eiropas valstīs notiekošā revolucionārā pasākumā. Taču drīzumā viņi arī organizatoriski nodalījās divās grupās: t.s. ``baltajos'' (mērenajos) un ``sarkanajos'' (demokrātos). Abas grupas vēlējās panākt Polijas valstisku neatkarību, taču tās sasniegšanas ceļus redzēja savādāk.

Vislabāk materiāli nodrošināts un vispopulārākais ārvalstu valstsvīru vidū bija \strong{``balto''} līderis, jau vairākkārt minētais kņazs Ā.~Čartorijskis, kurš vadīja emigrācijas konservatīvo spārnu. Tas pulcējās ap viņa rezidenci \frtxti{Hôtel Lambert} (grezna savrupmāja, celta XVII~gadsimtā, kuru nopirka Ā.~Čartorijska sieva). 1843.~gada martā Ā.~Čartorijska vadībā konstituējās ``\pltxti{Towarzystwo Monarchicne Fundatorow i Przyjaciόl Trzeciego Maja''} (Monarhistiskā 3.~maija [domāta 1791.~gada 3.~maija Konstitūcija] dibinātāju un draugu biedrība)~--- konservatīvo vadošais politiskais kodols. Pēc viņu domām Polijas jautājumu varēja risināt tikai visas Eiropas krīzes gadījumā. Šīs politikas būtību izteica Ā.~Mickeviča vārdi: ``\pltxti{O wojnę powszechną za wolność ludόw / prosimy się Panie}'' (``Vispārēju karu par tautu brīvību / lūdzam Tev, Kungs''). Poļu sacelšanās būtu dienas kārtībā, tikai pastāvot labvēlīgiem apstākļiem. Līdz tam par galveno uzdevumu tika uzskatīts Eiropas lielvalstu valdošajās aprindās uzturēt poļu lietai labvēlīgu noskaņojumu. Padomju vēsturnieki rakstīja, ka ``baltie'' un \frtxti{Hôtel Lambert} aprindas vēlējās, lai sacelšanās Polijā ``gruzdētu'', kas rādītu Rietumeiropas valstu valdībām, ka poļu tauta vēl ir dzīva un aizstāv savas nacionālās un reliģiskās tiesības. Ā.~Čartorijska vadībā faktiski tika uzturētas neoficiālas diplomātiskas attiecības ar Lielbritānijas, Francijas un Turcijas valdībām. Reizē tika mēģināts turpināt poļu leģionu tradīciju. Taču mēģinājumi to nodibināt Beļģijā, Portugālē, Ēģiptē un Turcijā beidzās nesekmīgi. 1834.~gadā ap 1~000 poļu iestājās franču Ārzemnieku leģionā, taču jau nākamajā gadā pārstrukturēšanas rezultātā tā vienības zaudēja ``nacionālo'' raksturu.

No otras puses, ``baltie'' emigranti centās nodalīt jautājumu par nacionālo atbrīvošanos no jautājuma par sociālo problēmu revolucionāru atrisināšanu. Ā.~Čartorijska piekritēji izvirzīja lozungu: ``Vispirms būt, pēc tam~--- kā būt''. Visi vēlākie notikumi rādīja, ka šāda pieeja tikai traucēja atrisināt jautājumu ``būt vai nebūt''.

\strong{``Sarkanie''} jeb nosacīti ``demokrāti'' skaita ziņā bija pārsvarā, taču baudīja mazāku starptautisko atbalstu. Viņu pirmā organizācija bija 1831.~gadā Parīzē nodibinātā ``\pltxti{Komitet Narodowy Patriotyczne}'' (``Patriotiskā nacionālā komiteja'') jau minētā vēsturnieka un politiķa J.~Lelevela vadībā. Viņš norādīja, ka pastāv divas Eiropas; tautas masu Eiropa, kura cīnās pret feodālismu un despotismu, un reakcionāro valdību Eiropa, un aicināja visus poļus sadarboties ar pirmo.

No poļu emigrantu organizācijām skaitliski pati lielākā bija ``\pltxti{Towarzystwo Demokratyczne Polskie}'' (``Poļu demokrātiskā biedrība'', 1832--1862). Tai bija savi pārstāvji Polijā. 1836.~gadā tā izdeva savu pamatprogrammu~--- ``\pltxti{Wielki Manifest}'' (``Dižais manifests''), kurš pasludināja ``tautas revolūcijas'' lozungu, ko varētu īstenot galvenokārt zemniecības spēkiem, bet saistībā ar visas Eiropas revolūciju, feodālo pienākumu likvidēšanu un zemniekiem iedalīto zemes platību nodošanu bez atlīdzības to īpašumā. Tā sludināja šķiru mieru poļu nācijas iekšienē. Tiktu likvidētas visas privilēģijas, ieviesta brīvība un vienlīdzība, atgūtas Žečpospolitas 1772.~gada robežas. Taču reizē deklarācijā bija izcelta Polijas īpašā loma Eiropā. ``Polija pagātnē aizsargāja Rietumus pret barbariskajiem tatāru, turku un ``moskaļu'' iebrukumiem. Polija gāja bojā tāpēc, ka tad, kad Rietumos cilvēku brīvības doma pasludināja karu vecajai kārtībai, par kuras aizstāvībai iestājās krievu despotisms, Polija, pildot savu vēsturisko misiju, iesaistījās cīņā pret šo spēku, taču tika uzvarēta. Eiropas glābšana tika atlikta.'' No sacītā tika secināts, ka Polijas glābšana nav tikai pašas Polijas, bet visas cilvēces lieta. Deklarācija rādīja, ka tās autori Polijas misiju uzskatīja būt par sargu Eiropas Austrumos. Tatāru un turku briesmas nebija vairs aktuālas, tātad Eiropu bija jāglābj no ``moskaļiem''.

Manifesta vājā vieta bija cerības uz brīvprātīgu šļahtas atteikšanos no savām privilēģijām. Tas aicināja griezties pie šļahtas ``ne ar ercenģeļa zobenu, bet ar tēvzemes vēstures grāmatu rokās''. Protams, aicinājums no tā izskanēšanas brīža bija nolemts neveiksmei. Jau 1846.~gada sacelšanās Krakovā parādīja, ka nav iespējams samierināt šļahtiču--muižnieku un zemnieku intereses. Taču ``Poļu demokrātiskā biedrība'' neatteicās no savām ilūzijām, vienīgi aktivizēja kārtu solidaritātes idejas paušanu, ``nācijas vienības'' vārdā. Tai pat laikā arī organizācijas iekšienē pastāvēja domstarpības, daļa biedru bija liberāļi vai mēreni demokrāti, citi~--- sociālisti-utopisti.

Emigrācijas revolucionārais spārns vienojās organizācijā ``\pltxti{Lud Polski}'' (``Polijas tauta'', 1835--1846, darbojās galvenokārt Lielbritānijas teritorijā), kura uz laiku pat izstājās no Demokrātiskās biedrības un iestājās par feodālisma, kārtu nevienlīdzības, zemes privātīpašuma likvidēšanu zemnieku sacelšanās ceļā. Polijas valstiskuma atjaunošana tika saistīta ar tādas iekārtas, kur tikai priviliģēto saujiņa izmantoja ``visu kopīgo zemi un tās augļus'' likvidēšanu, sludināta cīņa par demokrātiju, kura padarīs visus poļus ``bez vārda, dzimtas un reliģijas atšķirības'' brīvus un laimīgus. Poļu sociālisti-utopisti, kaut pieslējās t.s. ``kristīgā sociālisma'' ievirzei, atšķirībā no šo ideju atbalstītājiem Rietumos aicināja iznīcināt feodālismu vardarbības ceļā. Tomēr 1846.~gada Krakovas revolūcijas mācība jau 1848.~gadā lika demokrātiem darboties ar liberāļiem vienotās organizācijās.

Šajā laikā poļu vidū joprojām bija populārs lozungs ``\pltxti{Za naszą i waszą wolność}'' (Par mūsu un jūsu brīvību), kurš aicināja arī krievu revolucionārus uz kopīgu cīnu pret carismu. Toties Rietumu revolucionāri gandrīz nespēja iedomāties kopīgu krievu un poļu cīņu pret carismu, kas tai pavērtu jaunas perspektīvas. Tā, F.~Engelss 1851.~gada 23.~maijā rakstīja: ``Ja krievus var pamudināt uz aktīvu rīcību, tad jāveido savienība ar viņiem un jāpiespiež poļus piekāpties''.

Kā jau teikts, 1846--1848.~gadā Polijā atkal notika sacelšanās mēģinājumi, poļi piedalījās sacelšanās Austrijā, Ungārijā, taču Polijas karalistē nemieru gandrīz nebija.

Pēc Krievijas imperatora Nikolaja I nāves un Aleksandra II nākšanas tronī 1856.~gadā sākās zināma liberalizācijas fāze carisma politikā Polijā. Taču, kad imperators Aleksandrs II, ar kuru visas Eiropas liberāļi saistīja lielas cerības, 1856.~gadā ieradās Varšavā un poļu sabiedrības pārstāvji iesniedza viņam petīciju ar lūgumu atjaunot Polijas autonomiju un pārtraukt ierēdņu patvaļu, viņa atbilde bija kā auksta ūdens šalts: ``Jūs esat tuvi manai sirdij tāpat kā somi un citi Krievijas pavalstnieki, taču es nevēlos, lai tā kārtība, ko noteica mans tēvs, tiktu mainīta. Tāpēc, nekādu ilūziju, kungi! Es pratīšu apstādināt to centienus, kuri iedomāsies pakļauties sapņiem. Es pratīšu rīkoties tā, ka šie sapņi nepārkļūs pār sapņotāju iztēles robežu. Polijas laime slēpjas tās pilnīgā saplūšanā ar manas Impērijas tautām. Viss, ko paveica mans tēvs, ir paveikts labi.'' Sabiedrībā īpaši tika uzsvērti viņa vārdi: ``\frtxti{Point de rệveries, Messiers}'' (franču: Nekādu ilūziju, kungi!) Tiesa, jau nākamajā dienā pēc citētās runas Aleksandrs II parakstīja manifestu par politieslodzīto poļu amnestiju un atļauju poļu emigrantiem atgriezties dzimtenē. Tiesības atgriezties no ārzemēm ieguva gan tikai ``grēkus'' nožēlojušie, kuri saņēma attiecīgu apliecinājumu no Krievijas konsulārajām iestādēm. Četru gadu laikā Polijā atgriezās ap deviņi tūkstoši izsūtīto un emigrantu. Pēc N.~Berga vērtējuma tā ``sarkano'' pulciņi Polijas karalistē ieguva vadītājus, kuri ieradās no Sibīrijas un Eiropas, bet ``sarkano'' ietekmē aktivizējas arī ``balto'' pulciņi. Tika atcelts kara stāvoklis, Varšavā atvērta Medicīniski-ķirurģiskā akadēmija, dažādas svētdienas skolas un arodskolas. Tika mīkstināta cenzūra. Varēja izdot agrāk aizliegtu rakstnieku darbus. Piemēram, Varšavā izdeva dažus 1855.~gadā mirušā Ā.~Mickeviča darbus. Iznāca jaunas avīzes un žurnāli. Atļāva kultūrizglītības un ekonomisko biedrību darbību.

Starptautiskajā līmenī Polijas jautājums neoficiāli tika skarts miera konference pēc Krievijai neveiksmīgā Krimas kara. Anglija un Francija samierinājās ar Aleksandra II īstenoto agrāk sodīto poļu amnestiju, citas prasības neizvirzīja.

\strong{Saimniecība Polijas karalistē XIX gs. vidū}, neraugoties uz nacionālo apspiešanu, attīstījās ļoti sekmīgi, straujiem tempiem. Krievijas tirgus pavēra Polijas tautsaimniecībai lielas iespējas. Polijas karaliste piesaistīja amatniekus ar izdevīgajiem apmešanās noteikumiem un atbrīvošanu no nodokļiem.

Jau 1847.~gadā tika atklāta tvaikoņu kustība pa Vislu. Kopš XIX gadsimta 40.~gadiem Polijas karalistē tika būvēts dzelzceļš. Tieši Polijā pēc dzelzceļa līnijas Pēterburga~--- Carskoe selo būves 23~km garumā kā otrā Krievijā tika būvēta līnija no Varšavas līdz Austrijas robežai 308~km garumā. To uzbūvēja 1845.--1848.~gadā. Šis ceļš savienoja Varšavu ar Dombrovas ogļu ieguves baseinu, kas veicināja strauju ogļraktuvju attīstību. 1848.~gadā Varšavu savienoja ar Vīni. Kaut, vadoties no stratēģiskiem apsvērumiem, tālākā dzelzceļa celtniecība notika diezgan lēni un tā tīkls, salīdzinot ar citām Eiropas zemēm, bija samērā rets, tomēr Krievija tirgus deva lieliskas attīstības iespējas tekstilrūpniecībai Lodzas apgabalā un smagajai rūpniecībai Augšsilēzijā. 1862.~gadā tika pabeigta Pēterburgas--Varšavas un Skernevices (\pltxti{Skierniowice})~--- Brombergas (poļu \pltxti{Bydgoszcz}, vācu \detxti{Bromberg)} dzelzceļa līnija. 1871.~gadā Varšavu dzelzceļš savienoja ar Maskavu.

Uzņēmēji, kuri apņēmās uzbūvēt attiecīgos dzelzceļa posmus, muižnieki, kuri iedzīvojās no sliktas kvalitātes pārtikas pārdošanas strādnieku brigādēm, kuras virzījās pa jaunbūvējamo stigu, pat akcionāri, kuri, cerot uz peļņu, riskēja nopirkt dažas dzelzceļa sabiedrības akcijas, nevarēja pilnā mērā apjaust to labumu, ko jaunais transporta veids nesīs visai zemei. Parasti visi vēlējās iedzīvoties bagātībā uzreiz, vēl celtniecības laikā. Tāpēc notika vairāk vai mazāk fiktīvi celtniecības kompāniju bankroti, bija vērojama zema darbu kvalitāte, kura parasti atklājās tikai tad, kad pa jaunajiem uzbērumiem un sliežu ceļiem sāka kursēt vilcieni. Sliedes tika vestas no Anglijas, vagoni~--- no Vācijas. Tomēr pāris desmitgažu laikā notika milzu pārmaiņas. Dzelzceļš, kuram, kā likās, ienākumi bija jāsaņem galvenokārt no pasažieru transportēšanas, negaidīti sāka nest nesalīdzināmi lielāku peļņu no preču pārvadāšanas. Dzelzceļš tuvināja dažādus Polijas novadus, pat esošus dažādu valstu atkarībā.

Dzelzceļu celtniecībai Polijas karalistē gan bija ievērojams trūkums~--- tā tomēr nespēja pilnībā apmierināt rūpniecības vajadzības. Ar Prūsijai piederošo Polijas daļu Varšava bija savienota tikai četros punktos, Austrijai piederošo~--- tikai vienā. Vīnes, Brombergas un Lodzas līnijām bija Rietumeiropai parastais (šaurais) sliežu platums, pārējām~--- pēc Krievijas parauga~--- platais. Vienota standarta trūkums apgrūtināja pārvadājumus pat pašas Polijas karalistes robežās.

\strong{Buržuāzijas} jeb pilsonības veidošanās gaitā Polijas karalistē bija arī savas specifiskas īpatnības, ko noteica kapitāla sākotnējās uzkrāšanas process, kam lielā mērā bija raksturīga kapitāla gūšana no avotiem, kuri nebija saistīti ar pievienotās vērtības ražošanu. Jauno, kapitālistisko iekārtu simbolizējošā lielburžuāzija poļu zemēs veidoja mazskaitlīgu slāni un lielā mērā vēl izmantoja kapitāla sākotnējās (nekapitālistiskās) uzkrāšanas metodes. Tā Polijas karalistē XIX gadsimta pirmajā pusē vislielākie kapitāli piederēja nelielai komersantu grupai, kura atpirka no valsts dažādus tai piederošus monopolus~--- sāls, tabakas u.c. pārdošanas tiesības. Netiešo nodokļu iegūšanas tiesību atpirkšana, valsts pasūtījumi, valdības dotācijas, valsts uzņēmumu iegūšana uz izdevīgiem noteikumiem nomā~--- tādi bija visdrošākie līdzekļi straujai tikšanai pie bagātības. To zaudēšana draudēja ar gandrīz neizbēgamu bankrotu. Tas viss spieda buržuāziju censties iegūt cara valdības, īpaši finanšu iestāžu korumpēto ierēdņu labvēlību. Tas noteica arī šīs buržuāzijas daļas mazo interesi par nacionālās atbrīvošanās lietu. Desmitgadēs pēc Polijas dalīšanas poļu nacionālo varoņu rindās neizvirzījās neviens komersants vai rūpnieks, kuriem~--- bieži pamatoti~--- bija augļotāju un spekulantu reputācija. Tā kā ārpus agrārā sektora bija izjūtams kapitālu trūkums, komersantu un uzņēmēju vidū izvirzījās liela grupa muižnieku, kuri gan nereti rūpes par saviem uzņēmumiem uzticēja pārvaldniekiem, nākušiem no tirdzniecības un banku sektora.

Īpašu noti buržuāzijas un arī pilsētu iedzīvotāju sastāva izveidē ienesa ebreji, kuri, kā jau minēts, pēc senas tradīcijas bija juridiski nepilntiesīgi. Ebrejiem tiesības nepiešķīra pat samērā liberālās Varšavas hercogistes un Polijas karalistes Konstitūcijas. Gluži pretēji, XIX gadsimta 20.~gados tika pastiprināti aizliegumi viņiem dzīvot lauku apvidos, tāpēc 60.~gadu vidū ebreji jau sastādīja gandrīz pusi karalistes pilsētu iedzīvotāju. Tikai 1862.~gadā pēc A.~Veļepoļska, (par viņu tālāk tiks runāts plašāk), kurš vēlējās veicināt ``ebreju izcelsmes poļu'' veidošanos, iniciatīvas tika atcelti ierobežojumi ebrejiem, saistīti ar dzīves vietas izvēli, dažādu amatu ieņemšanu, papildus nodokļi tiem. Tomēr, neraugoties uz nodalīšanu no poļiem gan juridiskiem, gan ar kultūras un sadzīves šķēršļiem, ebreji ļoti aktīvi piedalījās Polijas pilsētu un miestu saimnieciskajā dzīvē. Amatniecībā, tirdzniecībā, kredīta operācijās viņu loma bija visai ievērojama, dažkārt pat noteicošā.

\strong{Strādniecība} jeb rūpnieciskais proletariāts galvenokārt veidojās no poļu un citu tautību amatniekiem, pilsētu plebsa, zemniekiem, daļas deklasētās šļahtas. Strādnieku darba un dzīves apstākļi bija neapmierinoši. Jau 1823.~gadā Varšavā notika pirmais streiks. 1848.~gadā neapmierinātība parādījās jau plašās strādnieku aprindās. Par strādnieku apziņu var spriest pēc tā, ka vēl 60.~gadu sākumā Lodzas audēji, tāpat kā angļu luditi, (angļu \entxti{luddites}~--- no viņu vadoņa N.~Luda angļu strādnieku grupa, kura XIX gadsimta sākumā, protestējot pret izmaiņām savā stāvoklī, kuras nesa rūpniecības apvērsums, bieži iznīcināja mašīnas un citas iekārtas) uz bezdarbu atbildēja ar mašīnu iznīcināšanu.

Attīstību bremzēja valsts muižu zemnieku klaušu darba izmantošana. XIX~gadsimta pirmajā pusē lai nodibinātu fabriku kapitālistam, kaut retāk nekā 18.gadsimtā, tomēr bieži vien vajadzēja iegādāties muižu ar tās izejvielu un, galvenais, darbaspēka rezervēm. (Jāuzsver, ka ne Prūsijai, ne Austrijai piederošajās poļu zemēs rūpniecībā nepastāvēja valsts uzņēmumi, kuri izmantotu klaušu darbu. Galīcijā XIX gadsimta pirmajā pusē klaušu darbs tika izmantots tikai muižniekiem piederošajās manufaktūrās.) Klaušu darba izmantošana uzņēmumos strauji samazinājās 50.~gados, kad lielākā daļa muižu pārgāja no klaušām uz činšu un auga algoto strādnieku skaits. Zemnieku noslāņošanās nodrošināja pastāvīgu darbaspēka pieplūdumu rūpniecībai. Beidzot kalnrūpniecībā un metalurģijā tika atļauts iesaistīties arī privātkapitālam.

1845.~gadā karalistē bija 46~tūkstoši strādnieku. No Varšavas un Lodzas fabrikām tikai dažās strādnieku skaits sasniedza vairākus simtus. Pārējās strādnieku bija krietni vien mazāk. Rūpniecība saražoja produkciju 9,8~miljonu rubļu vērtībā. Turpretī amatniecība, ar kuru 1842.--1843.~gadā nodarbojās 65,5~tūkstoši cilvēku, deva produkciju 6,2~miljonu rubļu vērtībā. Turpmākajos gados notika lielas izmaiņas. 1860.~gadā strādnieku skaits rūpniecībā pieauga līdz 75~tūkstošiem, viņu saražotās produkcijas vērtība līdz 32~miljoniem rubļu. 1864.~gadā 78~tūkstoši strādnieku ražoja produkciju 50~miljonu rubļu vērtībā. Tātad 20.~gados rūpnieciskās produkcijas vērtība bija pieaugusi 20~reizes. Ar amatniecību 1860.~gadā nodarbojās 91~tūkstotis cilvēku, kas ražoja produkciju par vairāk nekā 18~miljoniem rubļu.

Tālāk attīstījās tekstilrūpniecība. Vācu vēsturnieki gan uzsver, ka vadmalas aušana pēc uzplaukuma 30.~gados pēkšņi piedzīvoja strauju kritumu. Cēloņi tam bija meklējami daudzu aušanas uzņēmumu izvietojumā Viduspolijā, kas radīja lielu konkurenci; bez tam tika ievests daudz ārzemēs ražota kokvilnas auduma. Mehanizācija kokvilnas audumu ražošanā noveda īpaši pie vilnas audumu sīkražošanas panīkuma. Visvairāk tomēr traucējusi 1831.~gadā ierīkotā muitas robeža starp Poliju un Krieviju. Padomju vēsturnieks A.~Manusevičs gan norādīja, ka vienkāršākais muitas robežas pārvarēšanas līdzeklis bija uz Krievijas tirgu strādājošās tekstilražošanas pārnešana uz kaimiņos esošo Belostokas (\pltxti{Białystok}) apgabalu, kurš pēc trešās Žečpospolitas dalīšanas atradās Prūsijas, bet pēc Tilzītes miera (1807)~--- Krievijas impērijas sastāvā, un kurp bez grūtībām varēja piegādāt izejvielas no Polijas. Otrs muitas traucēkļu pārvarēšanas līdzeklis bija mašīnu ieviešana un darba intensitātes palielināšana, tāpēc 1831.~gada muitas tarifs gala rezultātā tikai veicināja rūpniecības attīstību Polijas karalistē. 1847.--1849.~gadā tika izstrādāts un 1850.~gadā apstiprināts vienots muitas tarifs Krievijai un Polijas karalistei, iekšējā muitas robeža ar to 1851.~gadā tika likvidēta (tāpēc arī vācu vēsturnieki XIX gadsimta vidu uzsver kā robežšķirtni). Polijas karalistes loma Krievijas tirdzniecībā īpaši pieauga Krimas kara laikā, kad Krievijas tirgotāji, iepirkuši preces Rietumos, Krievijas jūras blokādes dēļ bija spiesti tās ievest pa sauszemes ceļu caur Poliju.

50.--60.~gados Polijā, īpaši tekstilrūpniecībā, norisa rūpniecības apvērsums. Kapitālistiskā kokvilnas auduma ražošanas manufaktūra ap gadsimta vidu pārvērtās par mehanizētu fabriku. Mehanizēts tika viss tehnoloģiskais process: sākumā vērpšana (1840), pēc tam arī aušana (1864). 1860.~gadā tekstilrūpniecībā darbojās 360~uzņēmumu ar 36~tūkstošiem strādnieku. Uzplaukumu veicināja kā iekšējā tirgus augšana, tā arī poļu preču ieveduma pieaugums Krievijā, īpaši pēc muitas robežas likvidēšanas starp Krieviju un Polijas karalisti. Stāvokli vilnas audumu ražošanā tā gan neglāba, toties veicināja ārzemju investīciju (galvenokārt vācu un ebreju kapitāla) pieplūdumu Lodzā. Auga arī pati pilsēta. XIX un XX gs. mijā tajā mita jau 400~000 iedzīvotāju. Blakus kokvilnas audumu ražošanai Lodzā ražoja arī zīdu, attīstījās arī lenšu, trikotāžas izstrādājumu ražošana.

Svarīga nozare Polijas karalistē bija kalnrūpniecība. To savās rokās turēja valsts, privātnoma un ārzemju kapitāla izmantošana kalnrūpniecībā bija aizliegti. Plaši sāka pielietot tvaika mašīnas, mehanizēt atsevišķus izejvielu ieguves un ražošanas procesus. Jau XIX gadsimta 20.~gados radās pirmās mašīnbūves fabrikas, kurās ražoja iekārtas tekstilrūpniecības, kalnrūpniecības, metalurģiskajām un metalapstrādes rūpnīcām. Progress bija tik ievērojams, ka jau 20.~gadu beigās poļu mašīnbūvniecības uzņēmumi varēja izturēt konkurenci arī ārējos tirgos. 1833.~gadā kalnrūpniecība un metalurģija pārgāja Polijas bankas, bet 1843.~gadā~--- valsts kases pārziņā. Pastāvēja uzņēmumi gan ar vecu tehnoloģiju, kuri lietoja kokogles un ūdens enerģiju, gan celti jauni (Dombrovas baseinā), kuri kā kurināmo izmantoja akmeņogles.

% TODO: Нужен заголовок таблицы!

\noindent
\begin{table}[h!]
\caption{Polijas karalistē saražotās rūpnieciskās produkcijas aptuvenā kopīgā vērtība (miljonos rubļu)} \label{tab:table2}
\begin{tabularx}{\linewidth}{|p{4cm}|p{4cm}|}
\hline
1845.~g. & 9,8 \\
\hline
1857.~g. & 21,3 \\
\hline
1864.~g. & 50,0 \\
\hline
\end{tabularx}
\end{table}

% page 136


Kaut tautsaimniecība attīstījās, tomēr karaliste vēl joprojām palika agrāra zeme ar muižniecisko saimniecību pārsvaru. No 40.~gadu beigām līdz 1863.~gadam Polijas karaliste bija vienīgā no poļu zemēm, kur vēl saglabājās zemniecības feodālā atkarība. Zemnieku apstrādājamā zeme skaitījās muižnieku īpašums un atradās tikai zemnieku lietošanā. Galvenie ieguvēji bija muižnieki, kuri centās atsavināt zemnieku saimniecību zemi, nemazinot to pienākumus. Ja XIX gadsimta otrajā gadu desmitā zemnieku zeme vēl sastādīja 58\% no visas lauksaimniecībā izmantojamās zemes, tad 50.~gados tās bija palicis vairs tikai ap 32\%. (Pēc citiem datiem, 50.~gadu beigās 60\% lauksaimniecībā izmantojamo zemju bija muižu un t.s. folverku rīcībā.) Jāpiebilst, ka muižu rīcībā bija ne tikai lielākā, bet arī labākā, auglīgākā, izdevīgāk izvietotā zemes daļa, jau agrāk vai šai laikā grupēta lielos masīvos. Šais saimniecībās bija vērojams arī lielāks ražības pieaugums. Ja 1840.~gadā salīdzinājumā ar iesēto sēklu graudaugu raža bija 3,18~reizes lielāka, tad 1861.~gadā~--- jau 4,9~reizes. Tā kā pieauga arī sējumu platība, graudu kopprodukcija minētajos gados pieauga par 177\%. Palielinājās arī citu lauksaimniecības nozare produktivitāte. Ja 1822.~gadā Polijas karalistē bija 1,5~miljoni aitu, tad 1861.~gadā~--- jau 3,7~miljoni. 1839.~gadā bija 1,17~miljonu liellopu, bet 1861.~gadā~--- 2,07~miljoni. 1839.~gadā karalistē bija 442~tūkstoši zirgu, bet 1861.~gadā~--- 605~tūkstoši.

XIX gadsimta pirmajā pusē Polijas karalistes lauksaimniecībā arvien vairāk ekstensīvo saimniecības metožu vietā tika pielietotas intensīvās. Tas pirmkārt nozīmēja pāreju no trīslauku sistēmas uz daudzlauku, ievērojot racionālu augu seku. Pāreja uz daudzlauku sistēmu nebija iespējama bez racionālas lopkopības. Tā prasīja lopbarības bāzes paplašināšanu, kas nebija iedomājama bez pārejas uz zaļbarību, kuras sagāde savukārt prasīja pāreju uz daudzlauku sistēmu. Tas viss kopā prasīja kapitālieguldījumu palielināšanu: dzīvā un lietiskā inventāra iegādi, kura līdz tam muižām, apstrādājamām ar zemnieku arkliem un vēršiem, nebija. Muižniekiem piederoša dzīvā un lietiskā inventāra aprūpei bija vajadzīgs pastāvīgs personāls, kvalificētāks un atbildīgāks nekā zemnieki-klaušinieki. Tehnisko kultūru ieviešana augu sekā radīja lielu darbaspēka pieprasījumu. Šeit strādnieki varēja būt nekvalificēti un fiziski mazspējīgi, taču viņu bija vajadzīgs daudz un viņu darbam bija jābūt lētam. Lopkopības attīstība un tehnisko kultūru audzēšana pavēra agrāk nebijušas iespējas folverkos izmantot sieviešu un bērnu darbu. Tehnisko kultūru un lauksaimniecības produktu pārstrādes rūpnīcas (piemēram, tikai sezonas laikā strādājošās cukurfabrikas) izmainīja sezonas darbaspēka pieprasījumu.

Pakāpeniski muižas vērtību noteica ne vairs tās platības lielums, bet tajā investēto kapitālu daudzums, kurš, saprotams, arī nodrošināja tās ražību. Šļahtiči-sīkmuižnieki centās izvairīties no ``liekiem'' kapitālieguldījumiem, toties lielmuižnieki jau meklēja jaunas investīciju iespējas. Ne katram zemes īpašniekam pāreja uz kapitālistiskām ražošanas metodēm bija pa spēkam. Ekonomiski spēcīgākajiem tā bija vieglāka. Vājākie folverki, nespēdami izturēt konkurenci, izmisīgi cīnījās par pastāvēšanu, izmantojot tradicionālās feodālās metodes. Turpretī lielākās, spēcīgākās muižnieku saimniecības enerģiski cīnījās par ārējiem tirgiem. Muižnieku-kapitālistu akciju sabiedrības centās panākt produktu izvešanu vismaz pēc pirmās to pārstrādes (izvest miltus, nevis graudus), tātad~--- par attiecīgi augstāku cenu. Tiesa, vēl joprojām visai izdevīga bija neapstrādātu baļķu pārdošana. Uz Angliju, Vāciju tika vesti poļu muižnieku mežos augušie koki, kuru nociršana un pludināšana izmaksāja lēti, bet kuru cenas ārzemju tirgos sakarā ar strauji augošo kuģu būvniecību un celtniecību cēlās galvu reibinošos tempos. Tāds stāvoklis XIX gadsimtā saglabājās vēl ilgi, nesot muižniekiem ienākumus: vieniem greznībai, citiem~--- ražošanas attīstībai.

Auga pieprasījums pēc lauksaimniecības mašīnām. 1833.~gadā nodibinājās pirmie nelielie lauksaimniecības mašīnu ražošanas uzņēmumi. 1840.~gados to skaits pieauga. 30.~gados sāka audzēt cukurbietes un radās pirmās cukurfabrikas. 1849.--1850.~gadā pastāvēja jau 33 tādi uzņēmumi. 50.~gadu sākumā eksistēja arī ap 3~000 spirta brūžu. Zemes apstrādes intensifikācija noteica ražīguma pieaugumu. No XIX gadsimta 30. līdz 50.~gadiem tas palielinājās apmēram pusotras reizes. Pieauga arī lauksaimniecības produkcijas ražošana uz vienu iedzīvotāju.

Pateicoties kapitālistisko saimniecības metožu pielietošanai, ievērojami pieauga muižnieku saimniecībās ražotā preču produkcijas daļa. Graudus un vilnu tās audzēja pārdošanai, kartrupeļus~--- šņabja, bietes~--- cukura ražošanai. Taču, no vienas puses, folverki ražoja masu produkciju tirgum, bet no otras~--- tās ražošanu nodrošināja ar klaušu darbu. Ļoti lēni norisa klaušu nomaiņa ar algotu darbu. Pēc 1846.~gada datiem renti jeb činšu maksāja 1/3~visu zemnieku saimniecību, pārējās vēl joprojām pildīja klaušas. No 3,3~miljoniem Polijas karalistes zemnieku 1859.~gadā ap 1,34~miljonus sastādīja bezzemnieki, kas veidoja 40,5\% no visiem zemniekiem.

Zemniecības pamatmasu stāvokļa pasliktināšanās rezultātā radās stagnācija iedzīvotāju pieaugumā. 1846.~gadā Polijas karalistē bija 4~867~tūkstoši iedzīvotāju, bet 1859.~gadā to skaits saruka līdz 4~764~tūkstošiem. Ebreju izcelsmes poļu ekonomists un vēsturnieks S.~Kempners norādīja: ``Tas bija iedzīvotāju zemnieciskās daļas, kura gandrīz vai mira badā, pauperizācijas rezultāts.''

Taču ekonomiskie panākumi nespēja nolīdzināt politiskās un sociālās pretrunas. Baidoties, ka zemnieku sacelšanās no Galīcijas var pārsviesties arī uz Polijas karalisti, 1846.~gadā Nikolajs I izdeva dekrētu, ar kuru aizliedza muižniekiem zemniekus izdzīt no viņu lietojumā esošajām zemēm vai samazināt to apjomu, solīja uzraudzīt činša līgumu slēgšanu. Poļu vēsturnieks V.~Grebeņskis komentēja tālāko: ``Iekšlietu komisiju, realizējot šo dekrētu, vadīja zemnieku naida pret šļahtu pamodināšanas tendence un tā sēja zemnieku iedzīvotājos sabiedriskās cīņas sēklas. Pārņemot savās rokās tiesības izšķirt strīdus starp muižniekiem un zemniekiem, komisija apzināti radīja dažādus sarežģījumus, kuri paplašināja bezdibeni starp zemnieka būdu un muižas centru.'' Šeit nu jāatzīst, ka cariskā administrācija nekādi nevarēja vadīties no šķiru cīņas saasināšanas motīva. Cita lieta, ka izpildot dekrētu, tika radītas neērtības šļahtičiem zemnieku ekspluatācijā. Svarīgākais vēsturnieka~--- šļahtas interešu izteicēja rakstītajā ir atzinums par ``bezdibeni starp zemnieka būdu un muižas centru'', un tas bija radies jau sen pirms Nikolaja I dekrēta.

1857.~gadā cars atļāva radīt Zemkopības biedrību (\pltxti{Towarzystwo Rolnicze}), ko vadīja grāfs A.~Zamoiskis. Biedrība atspoguļoja mantīgās šļahtas intereses, kura vēlējās mīkstināt politisko režīmu Polijas karalistē, atjaunot tās autonomiju, pievienot tai lietuviešu, ukraiņu un baltkrievu apdzīvotās zemes, zemniecības jautājumā centās panākt klaušu atcelšanu un nomaiņu ar činšu. Šādas prasības atbalstīja arī pilsētnieki, jo to izpilde veicinātu tautsaimniecības attīstību. Grāfs A.~Zamoiskis vispār uzskatīja, ka Polijai ir izdevīgāk būt vienā valstī ar Krieviju, jo Polijai atbrīvojoties, Krievija atkal mēģinātu to pakļaut, kas liktu Polijai tērēt milzīgus līdzekļus aizsardzībai. Dažu gadu laikā biedrība izauga, tajā iestājās vairāki tūkstoši cilvēku, strādāja pastāvīgas komisijas. Biedrība ieguva ne tikai ekonomisku, bet arī lielu politisku nozīmi, daļēji kļuva par parlamenta aizvietotāju. Poļu cerību uzplaukumu veicināja arī sabiedriskās aktivitātes Krievijā, jaunu reformu gatavošana. Poļu vidū radās kustība, vērsta uz sadarbību ar Krieviju, īpaši sociālekonomiskos jautājumos.

Taču 50.~gadu beigās Krievijā izveidojās revolucionāra situācija. Poļu sabiedrībā uzskatīja, ka briestošie satricinājumi Krievijā rada labvēlīgu augsni poļu mērķiem. Itālijā sākās karš par zemes atbrīvošanu no austriešu kundzības. Itāļiem palīdzēja Francija, kura bija ieinteresēta Austrijas novājināšanā. Francijas imperators Napoleons III pasludināja t.s. ``nacionālo principu''~--- nacionālo neatkarību. Poļu vidū Napoleons III ieguva milzu popularitāti.

Arī Krievijas impērijas sastāvā ieejošajās poļu zemēs 1859.~gadā pastiprinājās pret carismu vērsti noskaņojumi: tika nēsāti tradicionālie poļu apģērbi, lietotas dažādas rotas lietas ar nacionālajām emblēmām, demonstratīvi svinētas vēsturiskas gadadienas tādiem notikumiem, kā, piemēram, Ļubļinas ūnija (1569), 3.~Maija Konstitūcijas (1791) pieņemšana, Novembra (1830) sacelšanās. Tika dziedātas patriotiskas dziesmas, sarīkoti ``kaķu koncerti'' zem carismam uzticīgo poļu sabiedrības pārstāvju logiem.

1860.~gadā Polijā sākās pirmās masu manifestācijas. Oktobrī triju Poliju sadalījušo valstu monarhu tikšanās laikā Varšavā patrioti izvērsa aģitāciju par tikšanās boikotēšanu. Aleksandram II iebraucot pilsētā, tās ielas kļuva gluži tukšas. 20.~oktobrī operas teātrī pirms izrādes sākuma cara ložu aplēja ar sērskābi, no galerijas tika nomesti flakoni ar smakojošu šķidrumu, izplatījās tāds smārds, ka sapulcējusies publika bija spiesta atstāt skatītāju zāli.

1861.~gada sākumā Polija gaidīja divus svarīgus notikumus. 21.~februārī bija jāsākas ``Zemkopības biedrības'' kārtējam kongresam, kuram vajadzēja pieņemt svarīgus lēmumus zemnieku jautājumā, un 25.~februārī apritēja 30.~gadadiena kopš kaujas pie Grohovas (\pltxti{Bitwa pod Olszynką Grochowską}) 1831.~gada poļu sacelšanās laikā. (Tajā pēc poļu datiem krita 7~000 poļu un 10~000 krievu, bet pēc krievu datiem 12~000 poļu un 9~400 krievu.) Viens no poļu emigrācijas vadoņiem L.~Mieroslavskis, kurš 60.~gadu sākumā mēreno poļu demokrātu, Ā.~Čartorijska pretinieku vidū bija ieguvis lielu ietekmi, no Parīzes pat atsūtīja direktīvu Grohovas kaujas gadadienā sākt sacelšanos. Tiesa, izrādījās, ka viņa aprēķini par iespējamo Francijas palīdzību ir nereāli un viņš pats atteicās no sava plāna.

1861.~gada 24.~februārī ``Zemkopības biedrības'' pieņemtajā rezolūcijā tika izteikta tikai vēlme visur nomainīt klaušas ar činšu (atgādinājumam: poļu \pltxti{czynsz}, vācu \detxti{Zins}~--- procents, no latīņu. \latxti{Census}, Žežpospolitā tā sauca nodevas graudā jeb naudā, kuras zemnieki maksāja par zemes lietošanu muižniekam), bet pēc tam gadu desmitu laikā ļaut zemniekiem izpirkties brīvībā un kļūt par zemes īpašniekiem. (Līdz 1859.~gadam uz činšu bija pārgājušas 41\% zemnieku saimniecību, taču daudzās no tām zemnieki nodevas maksāja, reizē pildot arī klaušas. Vairākumā gadījumu pāreju uz činšu pavadīja daļas zemnieku saimniecību zemes ``nogriešana'', apmaiņā pret sliktākas kvalitātes zemi.)

25.~februārī poļu revolucionāru pulciņi organizēja vairākus tūkstošus dalībnieku lielu demonstrāciju sakarā ar kaujas pie Grohovas gadadienu, kurai vajadzēja ne tikai demonstrēt sabiedrības noskaņojumu, bet arī ietekmēt ``Zemkopības biedrību'' radikālākas darbības virzienā. Speciāli izsūtīts žandarmu puseskadrons demonstrāciju izklīdināja nagaikām un zobeniem, arestēja daudzus tās dalībniekus. Kārtība tika atjaunota, taču ne uz ilgu laiku. Pēc divām dienām 27.~februārī notika nākamā demonstrācija. Demonstranti, sastapuši ceļā karaspēka aizsprostu, to apmētāja akmeņiem. Karaspēks atbildēja ar zalvi, rezultātā pieci cilvēki tika nogalināti, bija arī ievainotie. Demonstranti nogalināto ķermeņus vadāja pa pilsētu un rādīja iedzīvotājiem. No poļu buržuāzijas un inteliģences pārstāvjiem sastādīta komiteja ieradās pie Krievijas vietvalža kņaza M.~Gorčakova un lūdza izmeklēt incidentu, ļaut organizēt upuru apbedīšanu. Lūgumi tika apmierināti. Apbedīšana norisa 2.~martā un izvērtās ap 100~000 cilvēku demonstrācijā. Kristīgajā ceremonijā piedalījās arī ebreju garīdzniecība savos nacionālajos tērpos. Uz ielām nebija nedz kareivju, nedz policijas. ``Zemkopības biedrības'' vadība nosūtīja caram vēstījumu, saskaņotu ar buržuāziski-inteliģentisko komiteju, kurā paziņoja, ka zeme nevarēs attīstīties, kamēr poļu tautas nacionālie principi netiks īstenoti visās sabiedriskās dzīves jomās. Imperators Aleksandrs II vēstījumu pieņēma, savā vēstulē vietvaldim apsolot īstenot dažas reformas.

Stāvokli Polijā īpaši saasināja dzimtbūšanas atcelšana Krievijā (parakstīta 1861.~gada 19.~februārī, izsludināta 5.~martā). Agrārā reforma Krievijā neattiecās uz Polijas karalisti, bet poļu zemnieki cerēja, ka arī viņi varēs saņemt zemi. Gandrīz vairākas nedēļas pēc februāra demonstrācijām Varšavā valdīja diezgan liela sapulču un vārda brīvība. Demokrātiskie pulciņi gaidīja norādījumus no L.~Mieroslavska, taču, kā rāda poļu vēsturnieks S.~Keņēvičs, viņš šai laikā tikai ieteica paplašināt kustību uz austrumiem~--- uz ukraiņu, baltkrievu un lietuviešu zemēm, prasot lai vietējie šļahtiči piešķirtu zemniekiem zemi uz labvēlīgākiem nosacījumiem, nekā to solīja valdība, bet pašā Polijā nesteigties, iet kopā ar šļahtu un pretoties sociālās revolūcijas tendencēm.

Sabiedrības neapmierinātības pieaugums piespieda cara valdību uzsākt \strong{reformas} arī \strong{Polijā}. 1861.~gada 27.~martā tika publicēti rīkojumi par Polijas karalistes Valsts padomes atjaunošanu, tika pavērtas plašākas iespējas poļu valodas un kultūras apguvei skolās. Tika radīta ticības un izglītības lietu komisija ar lielmuižnieku marķīzu A.~Veļepoļski priekšgalā. Viņa mērķis bija Polijas pārveidošana sadarbībā ar Krieviju un pakāpeniska Polijas autonomijas atjaunošana Organiskā statuta robežās. A.~Veļepoļskis bija pārliecināts, ka ar vardarbību poļi nespēs atbrīvoties no Krievijas kundzības un, ka vispār nevis Krievija un krievi, bet gan vācieši ir galvenie poļu ienaidnieki. Viņš nepretendēja uz vecās Žečpospolitas bijušajiem austrumu apgabaliem, uzskatot, ka tas būtu pilnīgi nereāli. Šais apgabalos tobrīd dzīvoja ap 10 miljoniem iedzīvotāju, no kuriem tikai nedaudz vairāk par miljonu bija poļi. A.~Veļepoļska reformu programma paredzēja pārvaldes repolonizāciju, izglītības reformu, agrārā un ebreju jautājumu atrisināšanu. Vismaz divos pirmajos no nosauktajiem virzieniem iezīmējās panākumi. Satiksmes un pasta jautājumos karaliste kļuva neatkarīga no Krievijas. 1861.~gadā tika nodibinātas ap 300~poļu skolas. Tomēr poļu sabiedrībā A.~Veļepoļskim bija maz piekritēju. Pat tie poļu aristokrātijas pārstāvji, kuri bija apmierināti ar cariskās Krievijas sociālo politiku un saistījuši savas intereses ar Krieviju, centās panākt visu agrāk Polijai piederējušo ukraiņu, baltkrievu un lietuviešu apdzīvoto apgabalu apvienošanu ar Polijas karalisti vienā autonomā valstī ar īpašu Konstitūciju. Arī šo apgabalu poļu iedzīvotāji juta līdzi tautiešiem rietumos un pauda vēlmi apvienot Lietuvu, Baltkrieviju un Labā krasta Ukrainu ar Polijas karalisti.

Uzsāktie pārkārtojumi radīja nepamatotas cerības poļu sabiedrībā. Īpaši jaunatne, pārvērtējot savus spēkus, cerēja, ka varēs uzspiest Krievijai arvien jaunus piekāpšanās soļus. Daļa poļu virsslāņu pārstāvju, kuri pēc saviem sociāli-ekonomiskajiem uzskatiem ļoti maz atšķīrās no krievu liberāļiem un pat konservatoriem, atšķirībā no krievu virsslāņiem, bija gatavi uz bruņotu cīņu pret patvaldību, lai radītu muižnieciski-buržuāzisku Poliju. Reizē jāuzsver, ka viņi nebija ar nepārvaramu sienu atdalīti arī no pilsoniskās nacionālās kustības, no plebejiskajiem revolucionāriem. Atsevišķi publicisti ir izteikuši viedokli, ka labvēlīgas notikumu attīstības gadījumā nacionālā kustība varēja savīties ar sociālo.

1861.~gada 6.~aprīlī varas iestādes atlaida ``Zemkopības biedrību'', tā vājinot to poļu slāņu pozīcijas, kuri gribēja izvairīties no sacelšanās. Sākās poļu pilsoņu manifestācijas. 8.~aprīlī atkal radikāļu organizēto demonstrāciju Varšavā ar šāvieniem izklīdināja karaspēks, nogalinot vairāk nekā 100 cilvēku. Atskanēja kara trauksmes signāls un seši lielgabala šāvieni pavēstīja varšaviešiem, ka pilsētā tiek ievests kara stāvoklis. Karaspēks ieņēma svarīgākās vietas. Taču nekāda sacelšanās vēl nesākās, jo nebija kam nostāties patriotiskās kustības priekšgalā. Virkne slepenu organizāciju tika sagrautas.

Laukos tai pat laikā zemnieki ``Zemkopības biedrības'' lēmumu ar ieteikumu nomainīt klaušas ar čiņšu uztvēra kā reālu varas iestāžu rīkojumu. Kad laukos nonāca ziņas par reformu Krievijā, daudzu ciemu zemnieki nolēma atteikties no klaušu pildīšanas. Aprīļa vidū kustība aptvēra 176~ciemus, bet mēneša beigās jau 377~muižas, 954~ciemus. Pavisam līdz maija vidum pretošanās kustībā muižniekiem iesaistījās vairāk nekā 22~tūkstošu zemnieku sētu ar vairāk nekā 180~tūkstošiem cilvēku, kas sastādīja 15 līdz 20\% visu klaušinieku.

1861.~gada 2.~oktobrī Krievijas vietvaldis Polijas karalistē ģenerālis K.~Lamberts, kurš nomainīja maijā mirušo M.~Gorčakovu, centās rīkoties bez vardarbības, tomēr, atbildot uz virkni nemieru Polijas pilsētās, bija spiests izsludināt visā karalistē kara stāvokli. Pēc dienas~--- 3.~oktobrī, lielas ļaužu masas Varšavā piepildīja baznīcas, tur notika demonstratīvs aizlūgums par nacionālo varoni T.~Kostjuško, tika dziedātas patriotiskas himnas. Ar nolūku arestēt visus no baznīcām iznākušos vīriešus, tās aplenca un bloķēja krievu karaspēks. Taču pēc dievkalpojumu beigām neviens baznīcas neatstāja. Pagāja diena, iestājās nakts. Tad tika dots rīkojums karaspēkam ielauzties baznīcās. Vienā no tām karavīri neatrada nevienu cilvēku, visi bija paguvuši pazust pa sānejām. Pārējās baznīcās pēc dažādām ziņām tika arestēti no 1~500 līdz 3~000 cilvēku, kuri izrādīja pretošanos. Tiesa, pēc K.~Lamberta pavēles lielāko daļu arestēto drīz atbrīvoja. (Tāda nekonsekvence izsauca Varšavas ģenerālgubernatora A.~Hercencveiga protestu. Notika asa izskaidrošanās un, lai izbēgtu sodam par dalību klasiskā divkaujā, tika pieņemts tās ``amerikāņu'' variants~--- izlozēts, kam dzīve jābeidz pašnāvībā. Nepaveicās ģenerālgubernatoram, viņš nošāvās.) Protestējot pret iebrukumu baznīcās, katoļu garīdzniecība uz laiku slēdza visus Varšavas dievnamus. Tie nedarbojās līdz pat 1862.~gada sākumam. K.~Lamberts tika atsaukts no vietvalža amata. 2.~novembrī no sava amata atteicās un uz Pēterburgu devās A.~Veļepoļskis. Tur ilgstošās pārrunās viņam izdevās pārliecināt Aleksandru II turpināt ``kontrolētas reformas''.

Tomēr ziemā Varšavā uz ielām tika izvietotas karaspēka daļas, notika pilsētnieku atbruņošana. Konfiscēja vairāk nekā septiņus tūkstošus medību bišu, daudz citu ieroču. Taču pagrīdē turpinājās poļu patriotu organizēšanās, veidojās dažādas komitejas.

Svarīgi, ka manifestācijās gandrīz nepiedalījās viens no poļu sabiedrības pamatslāņiem~--- zemniecība. Poļu konservatīvais vēsturnieks un notikumu aculiecinieks V.~Pšiborovskis par lauku baznīcās rīkotajām patriotiskajām manifestācijām rakstīja: ``Ļoti bieži gadījās, ka zemnieki, padzirduši aizliegtu dziesmu, devās laukā no baznīcas un bariem metās bēgt, it kā viņus biedētu tās šļahtiskās Polijas rēgs, par kuru ``kungi'' lūdza dievu.'' Minētais vēsturnieks atstāstīja faktu par Sandomiras vojevodistē 1862.~gada septembra vidū organizētu lielu patriotisku pasākumu ar dievlūgšanu. Kad ksendzs pēc uzbudinoša sprediķa pieprasīja lai sapulcējušies paceļ roku kā zīmi zvērestam, ka viņi cīnīsies par tēvzemi līdz pēdējai asins lāsei, iestājās klusums, kuru pārtrauca sauciens: ``Lai neviens neceļ roku!''~--- un zemnieki pūlī devās projām no baznīcas.

Arī Polijas karalistes sabiedrībā pēc emigrācijas parauga 1861.~gada beigās noformējās divas galvenās politiskās nometnes~--- \strong{``baltie''} (konservatīvie, arī daļa liberāļu) un \strong{``sarkanie''} (radikālie demokrāti).

``Baltie'' daļēji nāca no ``Zemkopības biedrības'', izteica zemes īpašnieku~--- šļahtas un lielpilsonības intereses, atbalstīja pasīvas opozīcijas taktiku. Sociālajā sfērā ``baltie'' iestājās par feodālo attiecību likvidēšanas pēc Prūsijas parauga~--- pēc iespējas mazāk zemi atstājot zemniekiem. A.~Veļepoļskim neizdevās piesaistīt ``baltos'' savam kursam, jo pēc to domām viņš pārāk maz uzmanības veltīja agrāk Žečpospolitai piederošo apgabalu pievienošanai Polijas karalistei. ``Baltie'' 1862.~gada decembrī izveidoja savu ``direktorātu'', kurā galvenokārt darbojās bijušās ``Zemkopības biedrības'' locekļi.

``Sarkanie'' galvenokārt veidojās no sīkburžuāziskajiem~--- ierēdņu, pilsētnieku, studentu u.c.~--- slāņiem, iestājās par bruņotas sacelšanās sagatavošanu, lai tādā ceļā atjaunotu Polijas neatkarību. Sarkano paraugs bija Itālijas apvienošanās Dž.~Garibaldi un K.~Kavūra vadībā. ``Sarkano'' lozungs skanēja: ``\ittxti{Polonija farà da se}'' (itāļu ``Polijai pašai sevi jāatbrīvo''). ``Sarkano'' labējais spārns, tai skaitā L.~Mieroslavskis, pirmajā vietā stādot Polijas atjaunošanu 1772.~gada robežās, pietiekami nenovērtēja sociālās problēmas. Kreisais, revolucionāri-demokrātiskais ``sarkano'' spārns: Z.~Serakovskis, J.~Dombrovskis, K.~Kaļinovskis, A.~Mackevičs atzina lietuviešu, baltkrievu, ukraiņu tiesības uz pašnoteikšanos, sociālajā sfērā iestājās par radikālākām reformām. Vienoja abas nometnes vēlme atjaunot Žečpospolitu agrākajās robežās.

Situācija bija nestabila, ``rūgšana'' tautā turpinājās. Baznīcu darbības atjaunošana 1862.~gada pavasarī neuzlaboja stāvokli. Varšavā faktiski turpināja eksistēt kara stāvoklis. Šādos apstākļos Aleksandrs II atkal mēģināja arī savos poļu valdījumos atgriezties pie piekāpšanās politikas. Aprīlī tika apžēloti daudzi politiskie ieslodzītie. Polijas karaliste ieguva diezgan plašu autonomiju ar savu pārvaldi. Maijā par vietvaldi nozīmēja ar saviem liberālajiem uzskatiem pazīstamo cara brāli lielkņazu Konstantīnu Nikolajeviču (nejaukt ar Aleksandra I brāli Konstantīnu Pavloviču). Tiesa, savā instrukcijā brālim Aleksandrs II rakstīja, ka Polijas karalistei ``vienmēr ir jāpaliek Krievijas īpašumā'' un viņš ``nekādā gadījumā'' nepieļaus lai tā iegūtu Konstitūciju un savu armiju. Augstākais, ko tai pakāpeniski varētu piešķirt~--- zināmā mērā atsevišķu pārvaldi.

Būdams dedzīgs Aleksandra II realizējamo reformu piekritējs, lielkņazs Konstantīns Polijas karalistē nesekmīgi centās realizēt samierniecisku politiku. Lielkņazs plaši izmantoja viņam piešķirtās apžēlošanas tiesības, līdz 1862.~gada septembrim no 499~notiesātajiem viņš piedeva 289. Par vietvalža palīgu civillietās kļuva A.~Veļepoļskis. Civillietu nodošanai poļa rokās vajadzēja simbolizēt imperatora labvēlību poļiem. A.~Veļepoļskis tūlīt publicēja jau iepriekš sagatavotus rīkojumus par reformām.

Pirmkārt, 1862.~gada 16.~maijā tika izdots rīkojums par klaušu nomaiņu no 1.~oktobra ar čiņšu. Nedz muižnieki, nedz zemnieki nevarēja no tā atteikties. Tas tika noteikts zemāks kā agrāk, taču zemnieki zaudēja servitūtus~--- tiesības izmantot agrāk kopīpašumā, bet tagad muižnieku īpašumā pārgājušās pļavas, mežus, ūdeņus. Rīkojums neskāra citus zemnieku pienākumus muižnieku un valsts priekšā, nabagākos zemniekus muižnieki tāpat kā iepriekš varēja padzīt no zemes. Tā rīkojums kopumā atbilda muižnieku interesēm, kuri pat pieaugošās zemnieku kustības apstākļos nevēlējās šķirties no savām feodālajām tiesībām, bet neapmierināja zemniekus. Antifeodāli nemieri aptvēra simtiem ciemu. Zemnieki vēlējās panākt, lai klaušu atcelšana notiktu bez atlīdzības un apstrādātā zeme nonāktu viņu īpašumā.

Otrkārt, jūnijā tika izdoti rīkojumi par Polijas karalistes Valsts padomes atjaunošanu un vietējās pašvaldības iestāžu~--- vēlētu pilsētu, apriņķu un guberņu padomju izveidi. Kaut radikāļi aicināja boikotēt šo padomju vēlēšanas, tās notika, pilsētās tika ievēlēti buržuāzijas, apriņķos~--- muižnieku pārstāvji. Manifestāciju kustība no Polijas karalistes pārsviedās arī uz Lietuvu, Baltkrieviju, Labā krasta Ukrainu. Lietuvā jau 1861.~gada augustā varas iestādes bija spiestas izsludināt kara stāvokli.

Treškārt, tika atcelti ierobežojumi ebrejiem dzīves vietas izvēlē, privātīpašuma iegādē, sabiedrisko pienākumu izpildē. Tika atcelti īpašie nodokļi ebrejiem. Saglabājās gan arī daži ierobežojumi~--- iegādāties zemi, kur tika pildītas klaušas, ieņemt ciemu vecāko (\pltxti{wójt}) amatus. Atceļot daudzos ierobežojumus ebrejiem, rīkojums bija solis uz priekšu nacionālā jautājuma risināšanā. Tiesa, netika piešķirtas kādas politiskās brīvības.

Ceturtkārt, īpaši svarīgs bija rīkojums, veltīts izglītības sfērai. Pilsētās, miestos un lielos ciemos tika dibinātas valsts sākumskolas ar bezmaksas apmācību poļu valodā. Tika atļauts dibināt jaunas poļu ģimnāzijas. Mācību maksa tika pazemināta. Uz Kara medicīniskās akadēmijas bāzes kā \pltxti{Szkoła Główna Warszawska} (Varšavas Galvenā skola) faktiski tika atjaunota 1832.~gadā slēgtā Varšavas universitāte. Galvenajā skolā darbojās 4 nodaļas: vēstures un filoloģijas, fizikas un matemātikas, juridiskā un medicīniskā. Salīdzinājumā ar iepriekšējo izglītības sistēmu, tas bija ievērojams progress.

Bez jau norādītā, tika veikti vēl daži soļi režīma daļējai mīkstināšanai. Tā, tika likvidēts ģenerālgubernatora amats, atlaisti daži sabiedrības visneieredzētākie ierēdņi.

Taču poļu neatkarības centieni nenorima. Jebkuras Krievijas piekāpšanās poļu patstāvības centieniem nāca par vēlu un šķita tiem pārāk niecīgas, tās izsauca poļos tikai vēl lielāku neapmierinātību ar pastāvošo stāvokli. Radikāļi, cenšoties traucēt reformu īstenošanu, kas varētu mazināt atbalstu revolucionāru atbalstu tautā, izvērsa teroru. Poļu revolucionāri darbojās, īstenojot lozungu ``Jo sliktāk [tautai]~--- jo labāk [revolūcijai]'', ar teroru centās izprovocēt bargus atbildes soļus, lai tos izvirzītu kā apvainojumu saviem politiskajiem pretiniekiem. Kā raksta vācu vēsturnieks E.~Meijers, ``sarkanie'' rīkoja masu pasākumus nolūkā izsaukt sadursmes ar Krievijas armijas daļām un provocēt asins izliešanu. Vēsturnieks gan nedod pierādījumus, ka tiešu tādi bija revolucionārpatriotu nodomi, bet notikumi attīstījās tieši tādā virzienā un saprātīgi cilvēki to varēja paredzēt. Diemžēl, radikāļi savu mērķu sasniegšanai ir gatavi upurēt ne tikai savas, bet arī daudzu citu cilvēku dzīvības.

Pēc vietvalža Konstantīna ierašanās Varšavā pret viņu notika atentāts, taču lielkņazs tika tikai viegli ievainots. Arī pret A.~Veļepoļski tika veikti divi neveiksmīgi atentāti. Radikālpatrioti tā centās nepieļaut viņu plāniem neatbilstošas reformas, kuras varētu nomierināt tautu. Visus notvertos teroristus 1862.~gada augustā pakāra Varšavas citadelē. Taču atentātu politiskais mērķis bija sasniegts, revolucionāru provokācijas izraisīja sabiedrībā sašutumu nevis pret slepkavības mēģinājumu realizētājiem, bet to sodītājiem. Izvērsās dažādu aizlūgumu, dievkalpojumu ar politisku ievirzi noturēšana, pieauga sacelšanos gatavojošo ietekme.

Gan Rietumu sabiedriskā doma, gan poļu, gan arī padomju vēsturnieki tolaik Polijā notiekošos procesus traktēja kā plašu demokrātisku kustību pret krievu cara tirāniju ar diviem nesaraujami saistītiem uzdevumiem: sociālo~--- zemnieku atbrīvošanu un nacionālo~--- visas tautas atbrīvošanu. Turpretī mūsdienu krievu vēsturnieks A.~Širokorads pievērsis uzmanību tam, ka 60.~gadu sākums bija Aleksandra II reformu izvēršanās posms Krievijā, norisa zemnieku brīvlaišana, tika gatavota zemstu ieviešana, tiesu u.c. reformas. Cita lieta, ka krievu revolucionāri prasīja vēl radikālākus pasākumus. Turpretī poļu nemiernieki pēc A.~Širokarada domām nestādīja mērķi veikt kādas demokrātiskas vai ekonomiskas reformas, bet, gluži pretēji, vairākums t.s. ``izglītoto poļu'' ar sašutumu uzņēma Aleksandra II 1861.~gada manifestu par zemnieku atbrīvošanu no dzimtbūšanas. Poļu sabiedrības galvenais mērķis palika Polijas atjaunošana ``no jūras līdz jūrai''. 1863.~gada sacelšanās viņa vērtējumā bija ``vienīgi panu un ksendzu inspirēta no augšas''. Abiem viedokļiem var atrast argumentus.

Var ilgi diskutēt par to, kā reformas, Aleksandra II un jaunā vietvalža lielkņaza Konstantīna piekāpšanās poļu prasību priekšā ietekmēja stāvokli, vai tā veicināja vai kavēja jaunas poļu sacelšanās sākumu. Piemēram, konservatīvais krievu publicists kņazs V~Meščerskis poļu sacelšanās izraisīšanā vainoja ne tikai revolucionāro demokrātu A.~Hercenu, bet arī liberālo izdevēju A.~Krajevski, Krievijas iekšlietu ministru P.~Valujevu, ietekmīgo galmā Sanktpēterburgas ģenerālgubernatoru kņazu A.~Suvorovu, Polijas karalistes vietvaldi lielkņazu Konstantīnu, kuri, koķetējot ar poļiem, parādījuši nepieļaujamu vājumu. Būtībā visa krievu konservatīvā prese lielkņaza Konstantīna politikā saskatīja pārlieku piekāpību poļu priekšā un nedaudz vēlāk~--- 1863.~gada rudenī viņš nolika vietvalža pilnvaras. Arī mūsdienu krievu vēstures literatūrā ir sastopams viedoklis, ka Krievijas varas iestādes pārāk bieži lietoja ``pīrāga'' metodi~--- periodiski notika politisko noziedznieku amnestijas un vispār šīm iestādēm trūka nepieciešamās stingrības, bet sazvērnieki to uztvēra kā vājuma pazīmi.

Atbildot uz Aleksandra II gatavības uz sadarbību apliecinājumiem, uz Konstantīna iniciatīvām, ap trīssimt sabiedrības pārstāvju, ``balto'' piekritēju, 1862.~gada jūlijā griezās ar lūgumu pie A.~Zamoiska paziņot lielkņazam poļu prasības: pieņemt liberālu, Polijas autonomiju garantējošu satversmi, radīt pašu poļu armiju un pievienot tai atdalītos apgabalus. ``Kā poļi mēs varam atbalstīt valdību tikai tad, kad tā kļūs par poļu valdību, un kad visi apgabali, kas sastāda mūsu dzimteni, tiks apvienoti un varēs lietot Konstitūciju un brīvas iestādes''~--- tā tika izteikti ``balto'' uzskati. Tikai uz šādiem noteikumiem viņi vēlējās sadarboties ar Krieviju. Citiem vārdiem, tika prasīta ne tikai demokrātijas ieviešana, bet arī citu tautību iedzīvotāju apdzīvoto apgabalu pievienošana Polijas karalistei. A.~Zamoiska iesniegumā bija arī vārdi: ``\pltxti{Žądać nie mame prawę, a prosić o coś nie chemy}'' (``Mums nav tiesību ko prasīt, bet lūgt mēs nevēlamies''), kurus varēja arī tulkot vismaz kā brīdinājumu, ja ne draudus. Sakarā ar sabiedrības radikālisma pieaugumu Aleksandrs II izsauca A.~Zamoiski uz Pēterburgu. Kad A.~Zamoiskis Aleksandram II apstiprināja, ka ``tikai Lietuvas un atņemto provinču pievienošana karalistei apmierinās poļus'', viņu ``palūdza'' uz laiku doties uz ārzemēm.

No 1862.~gada vasaras sacelšanās gatavošanu vadīja Centrālā Nacionālā Komiteja (\pltxti{Centralny Komitet Narodowy}), kuru ``sarkanie'' bija izveidojuši 1861.~gada oktobrī (gan zem cita nosaukuma, tie mainījās). Viens no tās locekļiem bija Krievijas ģenerālštāba poļu virsnieku pulciņa dalībnieks J.~Dombrovskis. 1862.~gada februārī viņš ieradās Varšavā, uzsāka nelegālu darbību, izvirzījās par vienu no nacionālās kustības vadītājiem.

Jūlijā Centrālā Nacionālā Komiteja publicēja ``instrukciju'', kura faktiski saturēja sacelšanās programmu. Par organizācijas mērķi tika paziņota Polijas neatkarības atjaunošana 1772.~gada robežās, kas dotu visiem pilsoņiem ``pilnīgu brīvību un vienlīdzību likuma priekšā'', bet tautām, dzīvojošām valstī, to ``tiesību cienīšana''. Tika sludināta ``brālība starp šķirām'' un zemes piešķiršana zemniekiem ar valdības atlīdzību muižniekiem.

Ar to 1863.~gada sacelšanās, no vienas puses, atšķirībā no sacelšanās 1830.--1831.~gadā nacionālās atbrīvošanās lietu saistīja ar pārmaiņām īpašuma un sociālajā sfērā. No otras puses, no šļahtiču un pilsonības aprindām nākušie sacelšanās dalībnieki tomēr nespēja piedāvāt pietiekami dziļas izmaiņas agrārajās attiecībās, likvidējot muižnieku zemes īpašumu, kas varētu piesaistīt viņiem plašas zemnieku masas.

Mūsdienu krievu dzejnieks un publicists, žurnāla ``\rutxti{Наш современник}'' (``Mūsu laikabiedrs'') galvenais redaktors S.~Kuņajevs, kurš no krievu nacionālista pozīcijām polemizē ar poļu publicistiem, poļu šļahtas neapmierinātības cēloni saskata tās lielajā īpatsvarā kopējā iedzīvotāju skaitā, kad Krievijā tikai katrs divsimtais tās Eiropas daļas iedzīvotājs bijis muižnieks, bet Polijā katrs desmitais, un Krievijas imperatora veiktajā zemnieku atbrīvošanā no dzimtbūšanas. Patiesībā Polijas karalistē dzimtbūšana bija atcelta jau XIX gs. sākumā. Runa varētu būt tikai par to, ka daļa poļu muižnieku zaudēja varu pār zemniekiem ārpus Polijas karalistes esošajās Krievijas rietumu guberņās, bet pašā Polijas karalistē muižnieki varēja būt neapmierināti tikai ar klaušu aizstāšanu ar renti. Tikai sacelšanās rezultātā Polijas karalistē Krievijas imperators izdeva dekrētu par zemes piešķiršanu zemniekiem. Šis piemērs rāda, kā strīdos tiek aizmirsta vēsturiskā precizitāte.

Interesantus spriedumus ir izteicis poļu ekonomikas vēsturnieks V.~Kula. Parasti par vienu no topošās buržuāzijas antifeodālo noskaņojumu cēloņiem uzskata tās neapmierinātību ar pastāvošo kārtību, kad feodāļi turēja zemniekus dzimtbūtnieciskā atkarībā un kā vienīgie izmantoja to darbaspēku. Analizējot situāciju, V.~Kula norādījis, ka XIX gadsimta pirmajā pusē Polijas karalistē darba roku trūkumu izjuta nevis jaunā buržuāzija, bet gan muižniecība. Tāpēc arī buržuāzijai nebija pamata būt neapmierinātai ar pastāvošo šauro, bet tās vajadzības tomēr apmierinošo darbaspēka tirgu. Abas šķiras~--- gan muižniecība, gan buržuāzija izmantoja tām garantētās darbaspēka ekspluatācijas sfēras. Tāpat buržuāzija jau varēja ieguldīt savus kapitālus zemes iegādē, saimniekojot uz tās ar kapitālistiskām metodēm. Tāpēc arī poļu buržuāzija nebija ieinteresēta izvirzīt konsekventus antifeodālus lozungus, iesaistīties cīņā pret feodālismu. Arī šļahtas noslāņošanās process vēl nebija beidzies. Daļa no tās, zaudējusi savu īpašumu, bija pārgājusi citu slāņu rindās, taču daļa vēl joprojām visiem spēkiem centās saimniekot savās muižas ar vecajām feodālajām metodēm, jo tai nebija izvēles. Savu līdzekļu saimniecības kapitālistiskai pārveidošanai šai šļahtas daļai nebija, kredītu saņemt sakarā ar lielajiem parādiem tā nevarēja, pārdot daļu īpašumu, lai iegūtu līdzekļus,~--- arī nē, jo šie īpašumu jau tā bija nelieli. Ceļš uz kapitālismu šai šļahtas daļai draudēja ar izputēšanu. Tāpēc tā, neatkarīgi no saviem politiskajiem uzskatiem, sociālajos jautājumos, īpaši agrārajā, bija viens no konservatīvākajiem spēkiem.

Padomju vēsturnieks V.~Zaicevs savukārt izdalījis divas zemniecības daļas: 1) uz valsts un 2) uz privātīpašnieku zemes saimniekojošas. Valsts īpašumā esošajās zemēs kā tās pārvaldītājs uzstājās vietējā administrācija. Tāpēc valsts zemnieku sociālie centieni (pirmkārt pēc zemes iegūšanas) sakrita ar nacionālo atbrīvošanās kustību, bija vērsti pret kopējo ienaidnieku~--- Krievijas impēriju. Citāda situācija bija privātajā īpašumā esošajās saimniecībās. Šeit zemnieku sociālās intereses sadūrās ar zemes īpašnieku interesēm. Pēdējie, jau attīstošos preču-naudas attiecību apstākļos centās iegūt no sava īpašuma maksimālu peļņu un savās muižās paaugstināja zemnieku ekspluatāciju. Tāpēc privātajos īpašumos pastāvēja stipras sociālās pretrunas, kuras, no vienas puses, kavēja zemnieku nacionālās pašapziņas rašanos, bet, no otras, stiprināja viņos monarhistiskas ilūzijas.

Kā pārliecinoši rāda poļu vēsturnieks S.~Keņevičs, lai piesaistītu nacionālajai kustībai zemniecību, bija jārealizē radikāla agrārā reforma. Muižniekiem zemnieku labā bija jāziedo daļa savas zemes, arī baznīcas institūcijām jāatsakās no saviem zemes valdījumiem. Tikai tādos apstākļos bija iespējama vienota nacionāla fronte. Zemes piešķiršana zemniekiem radītu tajos nacionālas jūtas un veidotu sacelšanās kustībai plašu masu pamatu. Taču kā rādīja gan iepriekšējo, gan šīs sacelšanās pieredze, muižnieki nebija spējīgi pat uz nelieliem upuriem savu tautiešu--zemnieku labā. Sacēlušos agrārā programma stādīja uzdevumu veidot ``brālību starp šķirām''. Praksē tas nozīmēja to, ka nacionālā sacelšanās kustība pakļāva zemnieku intereses muižnieku interesēm. Dabīgi, ka tas neapmierināja zemniekus.

Arī nostādnes nacionālajā jautājumā neveicināja plašu cittautiešu masu atbalstu sacelšanās kustībai. Poļu sabiedrisko darbinieku demokrātisma pārbaudei sava veida lakmusa papīrs bija attieksme pret lietuviešu, ukraiņu un baltkrievu tautām. Jā, Centrālās Nacionālās Komitejas programmā gan tika runāts par ar Poliju apvienoto tautu ``tiesību cienīšanu''. Taču tāpat tajā gāja runa par to, ka sacēlušos mērķis ir atjaunot Poliju 1772.~gada robežās, tātad~--- iekļaujot tajā arī lietuviešus, ukraiņus, baltkrievus un daļēji arī latviešus. Šīs prasības izpilde nozīmētu ukraiņu un latviešu apdzīvoto teritoriju sadali divās daļās, bet lietuviešu un baltkrievu zemju pilnīgu iekļaušanu Polijā. Tātad apvienojot vienu (poļu) tautu, tiktu dalītas un pakļautas poļiem citas tautas.

Par agrākās Žečpospolitas robežu atjaunošanu, šo tautu apdzīvoto teritoriju pievienošanu Polijai vairāk vai mazāk apzināti iestājās vairākums poļu patriotu. Ā.~Čartorijska piekritēji t.s. austrumu nomaļu (\pltxti{Kresy Wschodnie}) iedzīvotājus uzskatīja par ``vienotas poļu tautas'' daļu. L.~Mieroslavskis un viņa piekritēji tāpat noklusēja kārtu un nacionālo apspiešanu ``kresos'' un uzskatīja lietuviešus, ukraiņus un baltkrievus tikai par poļu nācijas ``atzarojumiem'', uzstājās pret kādu ``īpašu'' tiesību piešķiršanu tiem. ``Vienotas tautas teoriju'' atbalstīja Ā.~Mickēvičs un J.~Lelevels (kurš nākotnē redzēja tikai demokrātiskās Polijas pilsoņus, kuri runās poliski, ukrainiski, lietuviski), kaut arī viņi 1772.~gada robežas prasīja jaunai, pārveidotai Polijas valstij, centās apvienot visas tautības pret apspiedējiem. Emigrantu organizācija ``\pltxti{Lud Polski}'' (``Polijas tauta'') 1772.~gada robežās vēlējās radīt līdztiesīgu tautu valsti, uz šī pamata jau bija jāizveidojas vienotai nācijai no poļiem, lietuviešiem, ukraiņiem un baltkrieviem. Atsevišķas poļu emigrantu grupas runāja par piekāpšanos lietuviešiem un ukraiņiem valodas un kultūras jomā, citas~--- par zināmas autonomijas piešķiršanu. Tādejādi radikālākās poļu aprindas 1772.~gada robežu jautājumu saistīja ar sabiedrisko attiecību demokratizāciju šajās teritorijās. Taču poļu revolucionāru vidū bija arī tādi: P.~Scegennijs, B.~Švarce, J.~Dombrovskis, kuri pret ``vienotas tautas teoriju'' izturējās visai skeptiski.

Loģiski, ka poļu patrioti domāja galvenokārt par to, kāda situācija pēc nacionālās atbrīvošanās kustības uzvaras izveidosies pašā Polijā, bet ne aiz tās robežām. To viņiem nevar pārmest. Taču reizē jākonstatē, ka cerība ar pašu spēkiem un ar Rietumeiropas palīdzību izcīnītu uzvaru veda gan pie pretinieku~--- triju Poliju sadalījušo lielvalstu spēku nenovērtēšanas, gan lietuviešu, ukraiņu un baltkrievu tautu nacionālās un sociālās atbrīvošanās centienu neievērošanas. Tikai poļu sacelšanās vadītājs Lietuvā Z.~Serakovskis izvirzīja ideju par Polijas, Ukrainas, Baltkrievijas un Lietuvas autonomiju pārveidotas demokrātiskas Krievijas sastāvā.

Poļu patriotu kreisais spārns gan panāca, ka Centrālā Nacionālā Komiteja (\pltxti{Centralny Komitet Narodowy}) 1862.~gada septembrī vēstulē avīzes ``\rutxti{Колокол}'' (``Zvans'', 1857--1867) redakcijai iekļāva vārdus: ``Mēs vēlamies atjaunot Poliju tās agrākajās robežās, atstājot tautām tajās dzīvojošām: lietuviešiem un rutēņiem brīvību palikt savienībā ar Poliju vai rīkoties ar savu likteni pēc savas gribas''. Centrālā Nacionālā Komiteja solīja veltīt visus spēkus lai piesaistītu ``\pltxti{kresus}'' kopīgai cīņai ar poļiem. Taču skaidras nostādnes, kā to izdarīt, nebija.

Sakarā ar samierniecisko pozīciju sociālajos jautājumos Centrālajā Nacionālajā Komitejā nebija arī vienotas nostājas, kāda valsts iekārta nākotnē tiks pasludināta. Augustā tika arestēts un Varšavas citadelē ieslodzīts J.~Dombrovskis. (Viņu notiesāja uz 15~gadiem katorgas darbos, taču 1864.~gada beigās viņam izdevās izbēgt no pārsūtīšanas cietuma Maskavā un doties uz ārzemēm.) Neraugoties uz komitejas locekļu pašu republikānisko pārliecību, tā savos dokumentos vairījās lietot vārdu ``republika'', jo baidījās zaudēt sabiedrības virsslāņu atbalstu. Tiesa, savukārt baidoties no tautas masu neapmierinātības, nekas netika teikts arī par monarhijas pasludināšanu.

Var secināt, ka neraugoties uz visiem poļu revolucionāru demokrātiskajiem, nacionālās atbrīvošanās centieniem, lielā mērā taisnība bija Krievijas ārzemju ticību garīgā departamenta (\rutxti{Департамент духовных дел иностранных исповеданий}) direktoram V.~Skripicinam, kurš vēstulē Krievijas kara ministram D.~Miļutinam rakstīja: ``Toreiz [Žečpospolitas laikā] poļu muižniecība bija kolektīva valdošā dinastija, bet tagad tā ir kolektīvs pretendents [uz varu], kurš tāpat kā visi pretendenti, nekad neatteiksies no savām zaudētajām tiesībām un patiesi nepakļausies nekādai augstākai varai, kura nenāktu no viņa paša.''

Septembrī Centrālā Nacionālā Komiteja publicēja paziņojumu, ka tā turpmāk darbosies kā ``nacionāla valdība'' un sāka veidot savu varas aparātu. Agrāko piecu guberņu vietā tika veidotas astoņas vojevodistes, uz vietām tika nozīmēti komisāri, kuriem bija jārūpējas par Komitejas rīkojumu izpildi. Varšavā tika organizēta slepena nacionālā policija, kura apsargāja Centrālo komiteju. Šis policijas rindās iestājās arī daudz oficiālās policijas ierēdņu, kuri par pēdējās darbību ziņoja sazvērniekiem.

Bez tam Centrālā Nacionālā Komiteja, lai traucētu poļu iesaukšanu Krievijas armijā, publicēja dekrētus, ar kuriem lika visiem poļiem~--- pilsētu, apriņķu un guberņu padomju locekļiem atlaist šīs iestādes, noteica nodokli, kurš bija jāmaksā visiem poļiem. Tas sastādīja 0,5\% no nekustāmā īpašuma vērtības un 5\% no ienākumiem. Ja pirmajam dekrētam vairums turīgo padomju locekļu nepakļāvās, vienīgi izsakot savu protestu pret nelikumīgo iesaukšanu, bet dažas padomes atteicās sūtīt savus pārstāvjus iesaukšanas komisijās, tad otrais dekrēts tika pildīts visai cerīgi. Nodokļa iekasētāji apstaigāja mājas un, saņēmuši vajadzīgo summu, atstāja numurētas kvītis. To numurus publicēja avīze, tā pierādot, ka naudu saņēma nacionālā valdība. Kā liecinājis laikabiedrs V.~Pšiborovskis, nodokli visi maksāja bez pretošanās un negriezās pie oficiālās varas pēc aizsardzības. Maksāja ne tikai sarkanie, bet arī baltie un pat daži cara iestāžu ierēdņi. Pašā Centrālajā Nacionālajā Komitejā, kuras sastāvs pastāvīgi mainījās, norisa strīdi par sacelšanās sākšanas laiku.

1862.~gada beigās konspiratīvā organizācija aptvēra ap 20~tūkstošus cilvēku. Visur tika dibinātas šūniņas triju cilvēku sastāvā, katrs tās dalībnieks zināja tikai savus līdzbiedrus un ``desmitnieku'', kas apgrūtināja sazvērestības atklāšanu. Tika izstrādāti vairāki sacelšanās plāna projekti, par tiem norisa strīdi organizācijas vadība, kuras locekļu sastāvs savstarpējās idejiskās cīņās bieži mainījās. ``Sarkano'' provinciālās komitejas darbojās arī Lietuvā un Labā krasta Ukrainā.

Centrālās Nacionālās Komitejas pārstāvji veda pārrunas ar krievu revolucionāriem. 1862.~gada septembrī Londonā notika sarunas ar carismam opozicionārās avīzes ``\rutxti{Колокол}'' izdevējiem A.~Hercenu un N.~Ogarevu. Viens no viņiem~--- A.~Hercens~--- jau 50.~gadu vidū izvērsa poļu neatkarības un krievu un poļu revolucionāru savienības nepieciešamības propagandu. Būdams krievu patriots, bet nosodīdams Krievijas valdības rīcību Polijā, viņš rakstīja: ``Mēs vēlamies Polijas neatkarību, jo mēs vēlamies Krievijas brīvību'', ``Polijas atbrīvošana, tai pieguļošo apgabalu un Krievijas atbrīvošana~--- nav šķiramas.'' Avīze ``\rutxti{Колокол}'' sludināja: ``Polijai pieder \emph{neatņemamas} pilnas tiesības uz valstisku patstāvību, \emph{neatkarīgu} no Krievijas.'' Reizē A.~Hercens noteikti noraidīja nacionālistiskos apgalvojumus par to, ka Polijas tiesības uz neatkarību ir saistītas ar ``vēsturiskajām'' tiesībām uz Žečpospolitas veco robežu atjaunošanu, ieskaitot ukraiņu, baltkrievu un lietuviešu zemes. Viņš rakstīja: ``Nevienu nevajag ne rusificēt, ne polonizēt \citespace{} nevienam nevajag traucēt runāt un domāt, mācīties un rakstīt kā viņš vēlas''. Tai laikā tā bija revolucionāra nacionālā jautājuma izpratne. A.~Hercens izteica ticību, ka poļu patrioti un krievu demokrātija varēs pilnībā samierināties, pauda cerību par poļu tuvināšanos slāvu pasaulei. Viņš prasīja, lai krievu karavīri zvērētu poļu sacelšanās gadījumā necelt ieročus pret Poliju, bet uzskatīja, ka Polijai jāizpērk ``sava atsvešinātība, savs rietumnieciskais aristokrātisms, uzticība pāvestam''. Tiesa, A.~Hercena izvirzītais slāvu tautu federācijas lozungs bija utopisks, tāpat kā viņa ideja par zemnieku kopienas sociālisma iespējamību. Taču lozungam bija demokrātisks saturs, jau tāpēc vien tas augstu vērtējams.

Pēterburgā 1862.~gada novembrī~--- decembrī norisa poļu revolucionāru sarunas ar slepenās revolucionārās biedrības ``Zeme un Brīvība'' (``\rutxti{Земля и воля}'', 1861--1864) Centrālo komiteju. Ar poļu revolucionāriem bija saistīti ``Krievu virsnieku komitejas'' (''\rutxti{Комитет русских офицеров}'', 1861--1863) Polijā pārstāvji.

Kā ievērības vērta jāuzsver poļu vēsturnieka S.~Keņēviča doma, ka šajā laikā poļu un krievu revolucionāru solidaritātes idejai nācās lauzt gadsimtu gaitā starp tautām uzcelto aizspriedumu, nelabvēlības, asu savstarpēju pretenziju sienu. Pret revolucionāru sadarbības piekritējiem tika izvirzīti pārmetumi par ``pārdošanos'' mūžsenajam ienaidniekam.

Rīkojās arī oficiālās varas iestādes. Lai izolētu gatavojamās sacelšanās potenciālos dalībniekus, tās pēc A.~Veļepoļska iniciatīvas iecerēja veikt 20--30~gadu vecu vīriešu rekrūšu iesaukumu armijā. Atšķirībā no nesen izsludinātajiem jaunajiem iesaukšanas noteikumiem, tam bija jānotiek nevis lozējot, bet pēc jau iepriekš sagatavotiem sarakstiem. Tajos bija ietverti, pēc dažādām ziņām no 10~000 līdz 12~000 poļu jauniešu, kuri bija zināmi varas iestādēm kā demonstrāciju un ielu nekārtību dalībnieki. Visiem bija skaidrs, kāds ir šādas iesaukšanas kārtības mērķis. Tai bija vai nu jāsagrauj revolucionāru organizācija, vai jāizsauc priekšlaicīga sacelšanās, kuru varētu viegli apspiest. Pat daudzi augstākie ierēdņi un militārpersonas, ieskaitot pašu vietvaldi lielkņazu Konstantīnu iebilda pret šādu provokāciju. Taču A.~Veļepoļskis panāca sava projekta apstiprināšanu Pēterburgā. ``Augonis ir nobriedis un to vajag uzšķērst,~--- paziņoja A.~Veļepoļskis saviem oponentiem.~--- Sacelšanos es apspiedīšu nedēļas laikā un tad varēšu pārvaldīt [karalisti].''

Paziņojums par priekšā stāvošo iesaukumu, kura datums netika atklāts, izsauca lielu satraukumu sacelšanās gatavotāju nometnē. Organizācija nebija gatava sākt sacelšanos, bet tās biedri, kuriem draudēja iesaukums, prasīja rīkoties. Lai nezaudētu lielu daļu atbalstītāju un autoritāti tautā, Centrālā Nacionālā komiteja paziņoja, ka tā aizstāvēs jauniešus pret iesaukšanu un ``pirms naidnieks pagūs veikt barbarisko rīcību, jau sitīs atmodas stunda''. Tādejādi faktiski tika atzīts, ka rekrūšu iesaukšana nozīmēs sacelšanās sākumu. Uz Parīzi tika nosūtīta cilvēku grupa ieroču iepirkšanai, virsnieki emigranti tika aicināti atgriezties dzimtenē.

Krievu revolucionāri vienprātīgi iestājās pret nekavējošos sacelšanās sākumu Polijas karalistē, jo Krievijā viņi vēl nebija gatavi sākt bruņotu cīņu pret carismu, atsevišķas poļu sacelšanās panākumiem neticēja, bet gan uzskatīja, ka tās sakāve tikai nostiprinās carismu un uz ilgu laiku aizturēs revolūciju Krievijā. Avīzes ``\rutxti{Колокол}'' izdevēji, griežoties pie krievu virsniekiem Polijā, brīdināja: ``Priekšlaicīgs sprādziens Poliju neatbrīvos, jūs pazudinās un noteikti apstādinās mūsu krievu lietu''. Krievu revolucionāri ieteica poļu biedriem sākt sacelšanos ne agrāk kā 1863.~gada pavasarī, kad Krievijā pēc dzimtbūšanas atcelšanas bija jābeidzas zemnieku ``pagaidu atkarības stāvoklim'' (\rutxti{временнообязанное состояние}) no muižniekiem un varēja gaidīt, ka zemnieki pieprasīs zemi. Tā tika cerēts apvienot revolucionāro kustību Krievijā un Polijā. Taču A.~Veļepoļskim izdevās uzspiest Centrālajai Nacionālajai komitejai kauju valdības izraudzītajā brīdī.

Faktiski 1862.~gadā, tāpat kā 1830.~gadā, lēmumu par sacelšanās sākšanu pieņēma neliela cilvēku grupa, kurai nebija konkrētas programmas un darbības plānu nākotnei. Taču poļu vēsturnieki uzsver, ka sacelšanās vēriens, tās ilgums liekot atmest pieņēmumus par sacēlušos vieglprātību un politiskā brieduma trūkumu. Nacionālās emocijas bijušas tik saspringtas, ka varējušas izlauzties jebkurā brīdī. Tas esot bijis vairāk goda, nekā politikas jautājums. Pēc autora domām, jautājumos, kur iet runa par tūkstošiem cilvēku dzīvībām un labklājību, revolucionāri, vai tie būtu jakobīņi, boļševiki vai poļu nacionālisti, diemžēl parasti izturas ar pārlieku pašpārliecinātību, padodas sava mesiānisma sajūtai.

Tostarp 23.~decembrī cara policija atklāja Centrālās Nacionālās komitejas galveno tipogrāfiju, arestēja tās locekli B.~Švarci. Pastāvēja aizdomas, ka ar viņu policijas rokās nonākuši arī Komitejas dokumenti. 24.~decembrī tika saņemta ziņa no Parīzes, ka arestēti visi turp izbraukušie iepirkt ieročus. Baidoties no jauniem arestiem, steidzami tika izveidots atkal jauns Centrālās Nacionālās Komitejas sastāvs, kurš apstiprināja iepriekšējo lēmumu par sacelšanās sākumu līdz ar rekrūšu iesaukšanu. Arestētais J.~Dombrovskis no cietuma atsūtīja plānu, kurš paredzēja 4~000 sacelšanās dalībnieku uzbrukumu Novogeorgijevskas cietoksnim (krievu \rutxti{Новогеоргиевская крепость}, poļu \pltxti{Twierdza Modlin}), tā arsenāla sagrābšanu un reizē sacelšanās sākšanos visā karalistē. Centrālā Komiteja plānu pieņēma un uzsāka pēdējos sacelšanās sagatavošanas pasākumus.

Centrālā Nacionālā komiteja gatavojās sākt sacelšanos 1863.~gada 14.~(26.)~janvārī. Taču varas iestādes nolēma pagrīdniekus apsteigt. Naktī uz 3.~(15.)~janvāri sākās rekrūšu iesaukums armijā, par kuru gan sacelšanās gatavotāji savlaicīgi uzzināja un brīdināja organizācijas biedrus. Apmēram tūkstotis jauniešu bēga no Varšavas, pulcējās mežos, organizēja pirmās sacēlušos vienības, apbruņotas ar izkaptīm, nažiem, daļēji medību bisēm. Tomēr simti pilsētā vēl palikušo, kas neslēpās mežā un kam nebija nodoma pretoties, tika sapulcināti citadelē. Šķita, ka iesaukšana ir izdevusies un A.~Veļepoļskis bija apmierināts.

Bija skaidrs, ka tuvākajās dienās piespiedu iesaukšana turpināsies. Centrālā Nacionālā komiteja nolēma sākt sacelšanos ātrāk nekā bija plānots~--- naktī no 10.~(22.). uz 11.~(23.)~janvāri.

Pirms paša sacelšanās sākuma 10.~janvārī Centrālā Nacionālā komiteja pasludināja sevi par \strong{Nacionālo Pagaidu valdību} (\pltxti{Tymczasowy Rząd Narodowy}), publicēja savus programmatiskos dokumentus: manifestu un agrāros dekrētus.

Manifestā tika pasludināts, ka Polija ``negrib un nevar'' bez pretestības pakļauties tai vardarbībai, ko pār to realizē krievu carisms~--- nelikumīgajai rekrūšu iesaukšanai. Visas kārtu un reliģiskās privilēģijas tika pasludinātas par atceltām, proklamēta ebreju vienlīdzība. Centrālā Nacionālā komiteja kā tagad vienīgā likumīgā poļu nacionālā valdība aicināja Polijas, Lietuvas un Krievijas tautas uz cīņu par atbrīvošanos. Noslēgumā dokumenta autori griezās pie krievu tautas. Centrālā Nacionālā komiteja paziņoja, ka tā nevaino krievu tautu par noziegumiem pret Poliju, jo tā pati cieš no carisma jūga, izteica cerību, ka tā neatbalstīs tirānu un brīdināja, ka pretējā gadījumā būs neizbēgams karš starp divām tautām.

Tika publicēti arī dekrēti par zemnieku apstrādājamās zemes platības nodošanu viņu rokās ar sekojošu kompensāciju zemes īpašniekiem no valsts kases, kā arī par neliela zemes gabala (3 morgu jeb ap 1,5 ha) piešķiršanu sacelšanās dalībniekiem~--- bezzemniekiem no nacionālajiem fondiem.

Vērtējot manifestu un agrāros dekrētus, padomju vēsturnieks M.~Misko uzsvēris, ka agrārajā jautājumā tie bija liels solis uz priekšu salīdzinājumā ar 1831.~gadu. Pirmkārt, tie pilnībā atcēla visus zemnieku feodālos pienākumus~--- gan klaušas, gan činšu. Otrkārt, zemniekiem tika piešķirtas visas to rīcībā esošās zemes (agrāk līdz 3 morgiem lielos zemes gabalus muižnieks varēja zemniekam atsavināt). Treškārt, reforma attiecās ne tikai uz muižnieku, bet arī uz valsts un baznīcas zemniekiem. Beidzot, ceturtkārt, tā klusu ciešot atzina servitūtu palikšanu zemnieku rokās. Kopumā sacēlušos izsludinātā reforma zemniekiem deva daudz vairāk nekā cara valdības un A.~Veļepoļska dekrēti. Iniciatīva bija sacēlušos rokās. Tomēr dekrētos nekas nebija teikts par zemes piešķiršanu bezzemniekiem, izņemot tos, kuri piedalīsies sacelšanās. Un arī tad zeme bija solīta tikai pēc sacelšanās beigām un tikai trīs morgu apmērā, kas bija nepietiekoši ģimenes uzturēšanai. Bez tam, dekrētos nekas nebija teikts par pēdējos gadu desmitos zemniekiem atsavināto zemju likteni, tātad tās paliktu muižnieku rokās. Beidzot, reforma paredzēja atlīdzību muižniekiem par zaudētajām zemēm, kas būtībā nozīmēja, ka zemniekiem (gan ar citu iedzīvotāju slāņu palīdzību) būtībā būs jāmaksā izpirkšanas maksa par savu zemi. Reformas īstenošanas princips nebija pietiekami skaidrs. To vajadzēja realizēt vietējām varas iestādēm un sacēlušos pavēlniecībai. Taču sacēlušos vienības sākumā izveidojās tikai dažviet un tā reformas īstenošana tika bremzēta. Reformas realizēšanā netika aicināti iesaistīties paši zemnieki. Arī priekšlikums uzdot reformu īstenot muižniekiem tika noraidīts, jo sacēlušies baidījās izsaukt to pretestību un vispārēju sajukumu valstī. Vēlāk sacēlušos militārajiem priekšniekiem tika uzdots nepieļaut sociālu apvērsumu. Zemnieku uzbrukumu gadījumā muižniekiem vai to mantai vainīgie bija jāsoda, viņu mājas jānodedzina. Tādejādi progresīvie dekrēti tikai daļēji atbilda sākušās sacelšanās izvēršanas veicināšanai.

Tas pats jāsaka par nacionālā jautājuma risināšanu. Centrālās Nacionālās Komitejas manifests neko neteica par ukraiņu, baltkrievu, lietuviešu pašnoteikšanās tiesībām, kuras taču Komiteja bija it kā atzinusi vēstulē žurnāla ``\rutxti{Колокол}'' redakcijai 1862.~gada septembrī. Tādejādi ``vienotās poļu nācijas'' teorija, mēģinājumi praksē notušēt nacionālās pretrunas, faktiski vienoja poļu demokrātus ar konservatīvajiem.

\strong{Sacelšanās} sākās nelabvēlīgos apstākļos. Atšķirībā no 1830.~gadā poļu rīcībā nebija labi apmācītas armijas. Sacelšanās dalībniekiem nebija nedz ieroču, nedz finanšu. Valstī bija savākts tikai ap 600~medību bišu. Organizācijas kasē bija ap 7,5 tūkstoši rubļu. Sacelšanās dalībnieki nebija militāri apmācīti. Zemnieki nebija gatavi sākt sacelšanos. Krievu revolucionāri savu sacelšanos plānoja vēlā pavasarī. Poļu revolucionāri uzsāka cīņu ziemā, kura neapmācītām vienībām radīja lielākus škēršļus kā labāk sagatavotajai krievu armijai. Tai Polijas karalistē bija ap 100~000 vīru ar 176 lielgabaliem.

Sacelšanās militāro vadību 18.~janvārī Centrālā Nacionālā komiteja uzticēja ģenerālim L.~Mieroslavskim, kurš tika pasludināts par diktatoru. Virkne vēsturnieku vērtē šo lēmumu kā kļūdainu, jo viņš savas personīgās intereses stādījis augstāk par nacionālajām, pēc sava rakstura neesot bijis spējīgs apvienot sacēlušos vadošo kodolu. 1861.~gada oktobrī viņš ar Itālijas valdības atļauju bija organizējis Dženovā (\pltxti{Genova}) virsnieku skolu (\pltxti{Polska Szkoła Wojskowa w Genui}) sacelšanās militārās vadības sagatavošanai. Tajā iestājās daudz emigrācijā esošu poļu jauniešu. Tiesa, drīz skolā sākās nesaskaņas sakarā ar L.~Mieroslavska dižmanību un viņš skolu atstāja, skaļi paužot savu neapmierinātību. Taču 1862.~gada jūnijā Krievija atzina jauno nacionālās atbrīvošanās kustības rezultātā izveidoto Itālijas valsti, viens no līguma nosacījumiem ar to bija poļu militārās skolas slēgšana. Itālija to arī lika poļiem izdarīt.

Naktī uz 11.~(23.)~janvāri cīņās iesaistījās ap 6~000 sacelšanās dalībnieku, sapulcinātu vairāk nekā 30~vienībās. Sajukums organizatoriskajā sacelšanās sagatavošanā noveda pie tā, ka iecerētā poļu un krievu revolucionāri noskaņoto virsnieku kopīgā uzstāšanās nenotika, tā vieta norisa 18 uzbrukumi Krievijas armijas garnizoniem. (Pavisam tie atradās 180~vietās.) Taču uzbrukumi lielākoties bija neveiksmīgi. To militārais efekts bija niecīgs, toties politiskās sekas~--- katastrofālas. Tā, mēģinājums ieņemt Plockas (\pltxti{Płock}) pilsētu, kur sacēlušies gribēja ierīkot savu centru, beidzās ar neveiksmi, ap 150 sacēlušos tika sagūstīti. Neviena guberņas pilsēta netika ieņemta. (Uz Varšavu, kur atradās liels krievu armijas garnizons, sākotnēji sacelšanās neizplatījās. Tas paglāba to no apšaudes ar lielgabaliem no citadeles. Cietoksnī atradās 555 lielgabali. Tāpēc nav brīnums, ka sacēlušies cietoksni ieņemt nevarēja. Revolucionāru organizācijas autoritāte tika satricināta. Jau pēc sacelšanās sākuma Varšavas citadeles garnizons tika pastiprināts līdz 16~000 vīriem.) Poļu sacēlušos uzbrukumos cara armija cieta niecīgus zaudējumus, taču tās personālsastāvā dzima naids pret uzbrucējiem. Kaut daļa krievu revolucionāri noskaņoto virsnieku pārgāja sacēlušos pusē, daži pat neveiksmīgi mēģināja aizvest sev līdzi uz sacēlušos nometnēm veselas karaspēka daļas, jau pirmās sacelšanās dienas parādīja, ka plašākas poļu un krievu sadarbības nodomi ir nereāli. Visi plāni par krievu revolucionāru vienību (družīnu) izveidi, kas dotos palīgā poļiem, izrādījās neizpildāmi, kaut kā atsevišķi cīnītāji sacēlušos pusē darbojās vairāki simti krievu, ukraiņu, baltkrievu un citu Krievijas tautu pārstāvju. (Šveices armijas virsnieks H.f.~Erlahs, kurš sacelšanās laikā kā novērotājs atradās sacēlušos rindās, pēc tās publicētā grāmatā atzīmēja, ka ``Starp ik dienas vienībā ienākušajiem brīvprātīgajiem vienmēr noteikts procents ir krievu bēgļu''. Vēlāk cariskās varas iestādes īpaši bargi sodīja poļu pusē pārgājušos Krievijas armijas karavīrus. No 183 Krievijas gūstā kritušajiem pārbēdzējiem 89 tika izpildīts nāves sods.)

Viens no krievu revolucionāru vadītājiem M.~Bakuņins, kurš jau no 40.~gadiem bija saistīts ar poļu revolucionāriem, ierosināja izveidot krievu leģionu, kurš cīnītos sacēlušos pusē. Ar 1863.~gada 9.~(21.)~februāri datētā vēstulē poļu revolucionāriem viņš rakstīja: ``Nav nekādu šaubu, ka zem lieliskā ``Zemes un brīvības'' karoga pastāvošs krievu nacionālais leģions radītu lielu iespaidu krievu armijā, kas virzīta ne tikai pret jums, bet pret visu Krieviju. Pats tā pastāvēšanas fakts būtu līdzvērtīgs vairākām uzvarētām kaujām. Par nelaimi, tagad šī projekta īstenošana ir kļuvusi daudz grūtāka, nekā, piemēram, pirms mēneša''. Viens no krievu virsniekiem--revolucionāriem Polijā A.~Potebņa gan vēl mēģināja gūt viena no populāriem sacēlušos vienību vadītājiem M.~Langeviča atbalstu krievu leģiona idejai, taču pēdējais vēsi izturējās gan pret poļu un krievu revolucionāru sadarbību, gan konkrēto ideju. A.~Potebņa drīz krita kaujā pret Krievijas karaspēku un ideja tā arī palika nerealizēta, krievu karavīru masveida pāreja sacēlušos pusē nenotika. Faktiski tā arī nevarēja notikt, jo sacēlušies poļi turpināja uz krieviem raudzīties kā ``barbariem''. ``Prom uz Āziju, Čingizhana [mongoļu karavadonis un valdnieks] pēcteči''~--- šie vārdi no kādas poļu nacionālās atbrīvošanās kustības laiku dziesmām ataino noturīgo poļu masu apziņas līmeņa stereotipu par Maskavas ``barbarijas'' civilizācijas līmeni.

L.~Mieroslavskis, šķērsojis pie Pozenes Polijas robežu, savāca ap 400--500 cilvēku lielu vienību, taču divās sadursmēs krievu karaspēks to izklīdināja, iegūstot trofejas, arī L.~Mieroslavska saraksti. Pats viņš bēga atpakaļ uz Parīzi, taču turpināja sevi uzskatīt par diktatoru.

``Balto'' vadība neticēja sacelšanās panākumiem, nejuta līdzi sacēlušamies un sākotnēji pat plānoja viņus publiski nosodīt. Viņi baidījās, ka L.~Mieroslavskis varētu piešķirt kustībai sociālas revolūcijas raksturu, virzīt zemnieku jautājumu uz sociālistisku risinājumu. Nevarot novērst sacelšanos, ``baltie'' nolēma tai pievienoties, lai pārņemtu tās vadību.

Izmantojot L.~Mieroslavska neveiksmes, ``baltie'' panāca, ka februāra beigās par diktatoru izsludināja M.~Langeviču, bet jau pēc nedēļas krievu karaspēka spiediena rezultātā viņš pārgāja Galīcijas robežu un Austrijas varas iestādes viņu internēja. No sacelšanās dalībnieku viedokļa ļaunākais bija tas, ka M.~Langevičs to izdarīja tikai sava štāba pavadībā un slepus no vienkāršajiem cīnītājiem. Visi bija sašutuši, un, saucot viņu par ``nodevēju'', arī paši devās pāri Austrijas robežai. Zemnieki sagūstīja un varas iestādēm nodeva ap 200~sacelšanās dalībnieku. Lielkņazs Konstantīns 1863.~gada 17.~februārī pat telegrafēja Aleksandram II: ``Zemnieki labākajā noskaņojumā, priecājas, ka viss ir beidzies''. Līdz šim nenorimst strīdi par M.~Langeviča rīcības motīviem. Parasti poļu vēsturnieki atzīst, ka viņš vēlējies Galīcijā rast papildspēkus un ieročus, bet daži autori viņam pārmet ``negodīgumu'', savukārt konservatīvais vēsturnieks V.~Pšiborovskis ir pievedis liecību, ka viņš vēlējies pārtraukt bezcerīgo sacelšanos.

Krievu karaspēks tika koncentrēts, tā izvietošanas vietu skaits samazinājās četras reizes. Tas deva iespēju sacelšanās dalībniekiem ieņemt plašas teritorijas, ieskaitot daudzas apriņķu pilsētas. Tomēr februārī Krievijas karaspēks uzsāka uzbrukumus lieliem spēkiem un sacēlušos stāvoklis pasliktinājās. Janvārī notika 58~karaspēka sadursmes ar sacelšanās dalībniekiem, februārī~--- 76. Visā Polijas karalistē atkal ieveda kara stāvokli, izsludināja, ka tie dumpinieki, kas kritīs gūstā ar ieročiem rokās, tiks tiesāti lauku kara tiesā pēc saīsinātas tiesvedības un sodi tiks izpildīti nekavējoties, visu sacelšanās dalībnieku īpašumiem uzlika sekvestru (rīcības ierobežojumus).

Tomēr 1863.~gada 31.~martā cars izdeva manifestu, kurā piedāvāja amnestiju visiem, kuri līdz 1.~maijam noliks ieročus. Manifestā Aleksandrs II arī solīja piešķirt karalistei vietējās pašpārvaldes tiesības. Carisms vēl neatteicās no piekāpšanās politikas. Taču tad pat aprīlī ``Centrālajā Nacionālajā komitejā'' vairākumu ieguvušie un maijā par Nacionālo valdību (\pltxti{Rząd narodowy}) to pārdēvējušie ``baltie'' piedāvājumu noraidīja. Tika cerēts, ka Rietumvalstis veiks intervenci pret Krieviju, lai palīdzētu poļiem. Tā kā Ā.~Čartorijskis bija jau miris un viņa vietu labējās emigrācijas vadībā ieņēma viņa dēls V.~Čartorijskis, viņš arī tika iecelts par sacēlušos galveno diplomātisko pārstāvi ārzemēs. Plaši izplatījās baumas par drīzu Francijas un Anglijas iejaukšanos. Tāpēc sacelšanās it kā varēja nevis degt pilnās liesmās, bet tikai pamazām gruzdēt, lai tikai saglabātu pamatu ārvalstu intervencei.

Aprīlī un maijā karalistē notika 127~bruņotas sadursmes. Tikmēr Krievijas karaspēks koncentrējās vairākos desmitos Polijas karalistes pilsētu. Vasaras mēnešos Polijas karalistē norisa 238 bruņotas sadursmes. Sacēlušos kara darbība pārvērtās par partizānu cīņām. Drīz kļuva redzams, ka poļi ar militārām metodēm uzvarēt nespēs. 1863.~gada maijā Polijas karalistē par vietvalža pienākumu izpildītāju, bet oktobrī~--- par vietvaldi tika iecelts ģenerālis F.~Bergs, vecs karotājs, tai skaitā arī 1831~.~gada poļu kampaņas dalībnieks. A.~Veļepoļskis jūlijā palūdza atvaļinājumu un devās uz ārzemēm, bet augustā tika atlaists no visiem amatiem. Tas arī nozīmēja, ka viņa politika bija piedzīvojusi sakāvi. Poļu sabiedrībā dažādi vērtēta viņa personība un politiskā pozīcija. Revolucionāri viņu nosodīja kā nacionālo interešu nodevēju, kurš stādījis uzdevumu poļu mantīgos slāņus samierināt ar carismu, saglabājot Krievijas kundzību poļu zemēs. Liberāļi A.~Veļepoļska realizētajā politikā saskatīja kreiso uzstāšanās rezultātā neizmantotu iespēju uzlabot Polijas stāvokli.

Ar jauna vietvalža nozīmēšanu iezīmējās Krievijas pāreja uz terora politiku Polijas karalistē. F.~Bergs ieveda stingru policijas un vietējās administrācijas uzraudzību. Sākās sacelšanās dalībnieku publiska sodīšana. Ja agrāk viņu nošaušana notika citadelē, tagad to rīkoja Varšavas laukumos. Tika pastiprināti policijas spēki. Varšavā agrāko tūkstoš policistu vietā darbojās 3 tūkstoši. Policijā iesauca regulārās armijas karavīrus un virsniekus. Tika aizliegta pārvietošanās naktīs, izbraukšana no pilsētas bez speciālas atļaujas. Naktīs pilsētu apsargāja daudzas kājnieku un jātnieku patruļas. Sākās poļu tautības ierēdņu nomaiņa ar speciāli atlasītiem un uz Poliju nosūtītiem krievu ierēdņiem. Tika ieviesta iedzīvotāju kolektīvā atbildība. Ja kādā ēkā atrada ieročus vai slapstošos sacelšanās dalībnieku, tās īpašnieku un iedzīvotājus tiesāja kara tiesa. Atentātu liecinieki, kuri nepalīdzēja izmeklēšanai notvert vainīgos, tika uzskatīti par līdzdalībniekiem un atbildēja pēc kara stāvokļa likumiem. Svarīgs iedzīvotāju ietekmēšanas līdzeklis bija to aplikšana ar kontribūcijām, naudas sodi un konfiskācijas. Plaši tika praktizētas sacelšanās dalībnieku muižu un citu īpašumu konfiskācijas.

Taču ``baltie'' turpināja izvairīties no radikālas agrārā jautājuma risināšanas, cerot uz Rietumvalstu palīdzību pret Krieviju, nevis enerģiski kopojot tautas spēkus. Arī materiālie resursi netika mobilizēti pietiekamā mērā. Mantīgo šķiru pārstāvji gan maksāja nacionālās nodevas un pirka nacionālā aizņēmuma obligācijas. Diezgan plaši līdzekļi ieplūda no ārzemēs dzīvojošajiem emigrantiem, kuri grupējās ap \pltxti{Hotel Lambert} un atbalstīja ``baltos''. Taču daudzi bagātnieki arī izvairījās no finansiālajām saistībām. Ar to mantīgās šķiras kopumā sacelšanās labā ziedoja daudz mazāku daļu savu līdzekļu, nekā vēlāk zaudēja no cariskās valdības uzliktajiem sodiem un kontribūcijām. Ja sākotnēji sacēlušies savāca 400 tūkstošus zlotu (atcerēsimies, 1 zlots bija līdzvērtīgs 15 kapeikām), tad 1863.~gada jūnijā Varšavā no Polijas karalistes galvenās kases tika nolaupīti vairāk nekā trīs miljoni rubļu, citur~--- vēl viens miljons. (Tiesa, no galvenās kases nolaupītajā summā tikai 0,5 miljoni bija skaidra nauda, citi līdzekļi bija vērtspapīros, kurus Nacionālā valdība tā arī nespēja realizēt). Bez tam sacēlušies pret ``kvītīm'' rekvizēja zirgus, pajūgus, apģērbu un pārtiku. Naudu vāca arī, pieprasot nomaksāt nodevas divus gadus uz priekšu, pat aplaupot iedzīvotājus.

Sagādātos līdzekļus izmantoja ieroču iepirkšanai ārzemēs un pašā Polijā. Pēc dažām ziņām sacelšanās laikā tika iepirkts ap 100~000 karabīņu. Taču nelaime bija, tā, ka daļu ieroču sacelšanās vadītāji turēja noliktavās, daļu konfiscēja prūšu un austriešu varas iestādes, pirms tie tika nogādāti uz Polijas karalisti, daļa nonāca cara varas iestāžu rokās, tā kā sacēlušies ieguva tikai mazu daļu. 1863.~gada jūlijā ar karabīnēm bija apbruņoti tikai ap 13~000 sacēlušos. Puse no sacelšanās dalībniekiem karoja, bruņoti ar izkaptīm un pīķiem.

Sacelšanās dalībnieku vienības sastāvēja no kājniekiem, nelielām kavalērijas daļām, tehniskajiem vadiem un visai nelielas dažās vienībās esošas artilērijas, kuru daļēji veidoja koka lielgabali. Pēc sociālā sastāva visdažādākā bija kavalērija, tomēr vairākumu tajā veidoja šļahtiči. Kājnieki bija strēlnieki un ``izkaptnieki'' (\pltxti{kosynierzy}), kuri galvenokārt nāca no zemnieku vidus. Gandrīz katrā vienībā bija ārsts un ksendzs.

Kaut 22.~janvāra agrārie dekrēti tika īstenoti, no 1863.~gada 1.~aprīļa tika pārtraukta činša iekasēšana, nacionālās varas iestādes brīdināja muižniekus nepieprasīt atcelto saistību izpildi un stingri sodīja likuma neievērotājus, tomēr, kaut daudzi poļu zemnieki piedalījās sacelšanās jau no pirmajām tās dienām, zemnieku pamatmasa palika pasīva. Tai pat laikā daudzi zemnieki uzstājās arī pret sacelšanās dalībniekiem, uzlūkojot tos par muižnieku interešu aizstāvjiem. Sacēlušies pirms tās sākuma nebija spējuši pierādīt zemniekiem savu mērķu tautisko, atbrīvošanās raksturu. Tās bija arī sekas šļahtiču privilēģijām un sacelšanās dalībnieku neieinteresētībai panākt vispārēju vienlīdzību. Agrārajai reformai netika veltīta vajadzīgā uzmanība. Nabadzīgākie zemnieki un bezzemes algādži turpināja palikt ārpus reformas, muižnieki prasīja no tiem agrāko saistību izpildi. Arī par zemes piešķiršanu sacelšanās dalībniekiem~--- solītajiem 3 morgiem~--- tika runāts reti. Šis solījums faktiski netika izpildīts. No otras puses, cara varas iestādes visādi centās nomelnot sacelšanās kustību kā muižniecisku un pret zemniekiem vērstu. Padomju vēsturnieks M.~Misko uzsvēris, ka zemnieku uzstāšanās pret sacelšanos tās pirmajās nedēļās jāuzskata nevis kā antinacionāla, bet kā sociāla, pret muižniekiem virzīta kustība.

Atbildot uz zemnieku uzstāšanos pret sacelšanās dalībniekiem--muižniekiem, sacēlušies pielietoja represijas. Daudzi zemnieki tika apvainoti spiegošanā, aģitācijā pret ``nacionālo'' valdību un nošauti vai pakārti. Ir grūti noteikt precīzu sacelšanās upuru skaitu, bet kā uzskatīja poļu profesors S.~Kenevičs, gada laikā Poļesjes (\pltxti{Polésie}) rajonā vien no sacēlušos rokām gāja bojā 200--300 zemnieku. Krievu vēsturnieks emigrācijā A.~Kersnovskis rakstīja, ka kopumā sacēlušies nogalinājuši līdz 5~000 mierīgo iedzīvotāju, galvenokārt etnisko poļu. Toties ir zināms tikai viens gadījums, kad pēc viena no sacēlušos vadītājiem Z.~Padļevska pavēles arī kāds muižnieks tika nošauts par atteikšanos sniegt palīdzību sacelšanās dalībniekiem un griešanos pie cara varas iestādēm ar lūgumu sniegt viņam militāru apsardzi.

Neraugoties uz vairākām neveiksmēm, 1863.~gada vasarā sacelšanās kustība aptvēra gandrīz visu Polijas karalisti, kā arī daļu Lietuvas, Baltkrievijas un Labā krasta Ukrainas teritorijas. Kopā kustībā piedalījās ap 200~000 cilvēku.

Jau 1862.~gada vasaras beigās izveidojās lietuviešu un baltkrievu demokrātu organizācija, kas uzturēja sakarus ar Z.~Serakovska pulciņu Pēterburgā. Redzamākie baltkrievu un lietuviešu demokrāti: publicists, dzejnieks K.~Kaļinovskis un garīdznieks (ksendzs) A.~Mackēvičs atbalstīja poļu atbrīvošanās kustību. Tā K.~Kaļinovskis uzskatīja, ka baltkrievu revolucionāriem ir jāorientējas uz Poliju, ka poļu atbrīvošanās kustībā varēs garantēt pašvaldību arī baltkrievu un lietuviešu apdzīvotajām teritorijām. Viņa devīze bija: ``Poļu lieta~--- tā ir mūsu lieta, tā ir brīvības lieta''. Arī A.~Mackēvičs cerības uz lietuviešu tautas atbrīvošanos saistīja ar cīņu pret carismu kopā ar poļu, krievu un baltkrievu tautām.

Pēc poļu sacelšanās sākuma 1863.~gada martā~--- aprīlī tā plaši izplatījās arī Lietuvā un Rietumbaltkrievijā. Lietuvā sacelšanās ieguva plašas zemnieku uzstāšanās raksturu, kura attīstījās agrāras revolūcijas virzienā. Martā beigās šeit militāro vadību uzņēmās Z.~Serakovskis. Baltkrievijā sacelšanās attīstījās vājāk. Vairākums sadursmju ar carisko armiju norisa Grodņas guberņā. Krievija veica pasākumus, lai sacelšanās neizplatītos tās rietumu guberņās. Šo guberņu zemnieki jau 1863.~gada pavasarī tika atbrīvoti no ``pagaidu atkarības'' stāvokļa un kļuva par pilntiesīgiem savas zemes saimniekiem, un viņu izpirkšanas maksājumi samazināti. Tas būtiski traucēja sacelšanās izplatību. Tomēr galvenais sacelšanos bremzējošais faktors bija tas, ka lietuviešu, baltkrievu un ukraiņu zemnieki neatbalstīja prasību atjaunot Žečpospolitu 1772.~gada robežās.

Var piebilst, ka arī jaunlatviešu izdotās ``Pēterburgas Avīzes'' nesimpatizēja poļiem, jo poļu kustības darbinieki bija muižnieki un garīdznieki, un paši poļu, bet īpaši leišu zemnieki tiem dažviet izrādīja pretestību. ``Pēterburgas Avīzes'' nosodīja poļu sacelšanos.

Piemēram, kad baltkrievu apdzīvotajos novados kara darbību uzsāka no poļu apgabaliem ieradušās vienības, tās šeit lielākoties ātri tika sakautas. Baltkrievu zemnieki dažkārt pat kopā ar cara armiju cīnījās pret K.~Kaļinovska vienībām. (Neraugoties uz to, viņš līdz pat savam arestam 1864.~gada janvārī turpināja atbalstīt poļu brīvības cīnītājus.) Arī Vladimiras--Volinskas (poļu \pltxti{Włodzimierz Wołyński}, krievu \rutxti{Владимир-Волынскuй}) rajonā ap pusotra tūkstoša zemnieku ar izkaptīm pievienojās krievu karaspēkam lai ``iztīrītu'' apkaimi no sacelšanās dalībniekiem. Kaut daudzi šļahtiču izcelsmes poļi arī juta līdzi apspiestajām ukraiņu, baltkrievu un lietuviešu tautām, pat piedalījās to nacionālās atbrīvošanās kustībā, poļu muižas šais zemēs, tāpat kā carisms, bija nacionāls un sociāls apspiedējs. Ukraiņu, baltkrievu un lietuviešu nāciju veidošanās norisinājās cīņā gan pret carismu, gan poļu muižu.

Līdzīgi notikumi norisa \strong{Latgalē}. Tur jau pirmajās sacelšanās dienās ieradās vairāki emisāri no Polijas un Lietuvas, kuri centās piesaistīt kustībai arī vietējos zemniekus, tāpat rīkojās vietējie muižnieki un mācītāji, aicinot uz cīņu par Polijas brīvību, aizmirstot naidu pret kungiem un ar tiem kopā dodoties cīņā pret ''moskaļiem'' (krieviem). Kā konstatējis poļu tautas vēsturi daudz pētījušais vēsturnieks Ē.~Jēkabsons, notika savdabīga sacensība par Latgales zemnieku simpātijām. Krievijas valdība 1863.~gada martā atļāva zemniekiem iepirkt zemi bez muižnieku piekrišanas un izsludināja likumu par sacelšanās dalībnieku īpašumu konfiskāciju. Savukārt Polijas Pagaidu valdība aizliedza muižniekiem prasīt, lai zemnieki pildītu klaušas. Taču muižnieki nevēlējās aizliegumam pakļauties. Rezultātā Polijas atjaunošanas ideju zemnieki saistīja ar panu (muižnieku) varas pastiprināšanos. 1863.~gada aprīlī Latgales poļu muižnieku vienība grāfa L.~Plātera vadībā uzbruka vāji apsargātam krievu armijas ieroču transportam ceļā no Daugavpils (vācu \detxti{Dünaburg}, poļu \pltxti{Dyneburg}, krievu \rutxti{Динабург}, \rutxti{Двинск}) uz Krāslavu (poļu \pltxti{Krasław}, krievu \rutxti{Креславка}). Krievijas varas iestādes savukārt aicināja zemniekus uzbrukt L.~Plātera vienībai, kurā darbojās arī vairāki īpaši viņu neieredzēti muižnieki. Zemnieki sakāva un sagūstīja poļu nodaļu. L.~Plāters tika ieslodzīts Dinaburgas cietoksnī, kur viņam arī tika piespriests un izpildīts nāves sods. Savukārt aktīvākajiem zemniekiem, kas bija piedalījušies sacelšanās dalībnieku vajāšanā, žandarmērija izmaksāja balvās 450~sudraba rubļus.

Samērā maz sacēlušos vidū bija ebreju. Ebreji varētu to atbalstīt, ja tās uzvara nestu viņu stāvokļa uzlabošanos. Taču nopietns šķērslis ebreju dalībai sacelšanās kustībā bija poļu šovinisms. Poļu turīgo slāņu vidū bija daudz cilvēku, kuri naidīgi izturējās pret ebrejiem un bija pret vienlīdzīgu tiesību piešķiršanu viņiem, jo baidījās no to konkurences. Tikai sākotnējā, manifestāciju periodā ebreji plaši piedalījās kustībā. Iziet uz ielas vai organizēt aizlūgumu sinagogā viņi varēja brīvi, daudz grūtāk bija iestāties sacēlušos vienībās, kuras komandēji nacionālistiski noskaņotie šļahtiči. Pēc dažām ziņām sacelšanās kustībā piedalījās tikai 100 līdz 200 ebreju. Vairāk viņu bija sacēlušos civilajās varas iestādēs. Pārējās ebreju masas parasti neatšķīrās no citiem līdzpilsoņiem: akurāti maksāja nacionālos nodokļus, sniedza sacelšanās organizācijām dažādus pakalpojumus.

XIX gadsimta 60.~gados atkal, tiesa, pēdējo reizi šī gadsimta vēsturē, \strong{Polijas jautājums nonāca Eiropas diplomātijas uzmanības centrā}. Romas pāvests lika baznīcās aizlūgt par cietēju Poliju, taču pirmkārt viņš rūpējās par katoļu baznīcas pozīciju stiprināšanu tajā. K.~Markss un F.~Engelss, vērtējot poļu sacelšanos pirmkārt no starptautiskā viedokļa, kļūdaini uzskatīja, ka ``Eiropā atkal ir plaši atklāta revolūciju ēra''. F.~Engelss rakstīja K.~Marksam: ``Poļi ir malači. Un ja viņi noturēsies līdz 15.~martam, tad visa Krievija sāks kustēties''. Un atkal tā bija kļūda. Masu kustības atplūdi Krievijā, sākušies 1861.~gadā, šajā attīstības etapā izrādījās vēsturiski neatgriezeniski.

Sarežģītāka bija pasaules \strong{lielvalstu valdību pozīcija}. Visvairāk ieinteresēta atbalstīt poļu tautas cīņu bija Francija. Neatkarīga Polija tai būtu izdevīga sabiedrotā ne tikai pret Prūsiju un Austriju, bet arī Krieviju, kura bija viens no galvenajiem Francijas sāncenšiem kontinentā. Napoleons III labprāt vēlētos pārdalīt Eiropas karti, apvienot savā ietekmē Itāliju, pievienot Francijai Reinas kreiso krastu. To nevarēja panākt bez Krievijas atbalsta. Meklējot šādu atbalstu, Napoleons solīja Krievijai atteikties no oficiālas palīdzības poļu kustībai. Tiesa, ievērojot savas valsts sabiedrisko domu, Napoleons III ieteica Krievijai veikt piekāpšanās poļu priekšā. Cerības uz Krievijas atbalstu Francijai nepiepildījās un Francijas imperators Napoleons~III, kurš bija velti cerējis uz Francijas tuvināšanos Krievijai, vērstu pret Prūsiju, tad nu iestājās poļu pusē. Franciju vismaz vārdos atbalstīja arī Lielbritānija un Austrija. Taču Krievijas imperators Aleksandrs~II Francijas sūtnim Krievijā L.~de Montebello paziņoja, ka viņš principā piekrīt Francijas, Lielbritānijas un Austrijas priekšlikumiem, bet nespēj tos realizēt praksē: ``Ja varētu atgriezt Polijas karalistei tās neatkarību, tad mans tēvs to izdarītu un es arī rīkotos tāpat. Taču [tās] neatkarība ir praktiski neiespējama: Polija nespēj dzīvot savās robežās, un paši poļi to neslēpj. Viņi vēlas iegūt savas 1772.~gada robežas, citiem vārdiem, vēlas sadalīt Krieviju''.

Lielbritānijai faktiski nerūpēja poļu intereses. Polijas valsts atjaunošana stiprinātu britu galveno sāncensi kontinentā~--- Franciju un vājinātu sabiedroto~--- Prūsiju. Gan Lielbritānijas valdība, gan parlaments, gan prese izteica līdzjūtību poļu tautai, taču tai pat laikā pauda viedokli, ka nekādu palīdzību poļu kustībai viņu valsts nesniegs. Lielbritānija tikai diplomātiski atbalstīja sacēlušos. Taču, kā norādījuši poļu vēsturnieki J.~Klačko un H.~Vereščuckis, par līdzsvara saglabāšanu Eiropā norūpējušās Lielbritānijas acīs neatkarīga Polija būtu pastāvīgs nemiera izraisītāja Eiropas centrā, jo Prūsija un Krievija censtos atgūt savus agrākos valdījumus, nestabilitāte traucētu brīvo tirdzniecību, kurā Lielbritānija bija ieinteresēta. Bez tam Polijas atrašanās Krievijas sastāvā Lielbritānijai bija izdevīga, jo tā pastāvīgi izraisīja Krievijas rūpes un kalpoja tās kā Lielbritānijas sāncenses vājināšanai. Tāpēc Lielbritānijas ārlietu ministrs Dž.~Rossels piedāvājumu atjaunot Poliju, iekļaujot tajā arī Pozeni un Galīciju, nosauca par ``neprātu''.

Kad Polijas karalistē 1863.~gadā sākās atklāta sacelšanās, A.~Hercens savā pirmajā tai veltītajā rakstā uzdeva jautājumu: ``Ko darīs Eiropa? Vai viņa pieļaus [poļu sacelšanās apspiešanu]? Vai viņa vēlreiz pieļaus?'' un pats atbildēja: ``Pieļaus''. Tā tas arī notika. Tikai atsevišķi slāvu un Rietumeiropas zemju demokrātisko aprindu pārstāvji kā brīvprātīgie devās cīnīties sacēlušos rindās. Īpaši spēcīgu atbalsi poļu sacelšanās sastapa Itālijā un Ungārijā. Ungāru revolucionāri vēlējās atdalīties no Austrijas. Itāļu patrioti cerēja atbrīvot no Austrijas kundzības Venēciju. Abu tautu revolucionāriem galvenais ienaidnieks bija Austrija. Pret to viņi vēlējās noslēgt savienību ar poļiem, lai kopīgiem spēkiem izraisītu sacelšanās pret Austriju Galīcijā, Ungārijā, Čehijā un Balkānos. Poļu vadītāji turpretī nevēlējās izvērst cīņu pret Austriju, cerot, ka tā netraucēs sacēlušos vienību organizācijai Galīcijā.

Austrija un Prūsija, kā valstis, kuras kopīgi ar Krieviju bija sadalījušas Poliju, bija pret piekāpšanos poļiem Polijas karalistē, kas varētu stimulēt poļu nacionālo kustību arī kaimiņvalstīs. Prūsija, kā jau minēts, 1863.~gada 8.~februārī (pēc vecā stila~--- 27.~janvārī) pat noslēdza ar Krieviju t.s. Alvenslēbena konvenciju, kura atļāva Krievijas karaspēkam vajāt poļu sacēlušos arī Prūsijas teritorijā. Austrijas valdība gan nepievienojās Prūsijas un Krievijas 8.~februāra konvencijai, bet pavēlēja atbruņot poļu sacēlušos vienības, kas šķērsoja valsts robežu, paziņoja aizliegumu Galīcijas teritorijā vervēt brīvprātīgos sacēlušos vienībās un izvirzīja papildus sava karaspēka daļas pie Krievijas robežas. Austrijas salīdzinoši ``mīkstā'' politika pret poļu sacelšanos, tās dalībnieku vidū un arī Francijā radīja zināmas cerības par Austrijas gatavību palīdzēt poļu lietai, taču tās nepiepildījās.

Tomēr poļu sacēlušos vadītāji utopiski cerēja uz Rietumu atbalstu. 1863.~gada 30.~jūnijā britu avīze ``\entxti{Morning Standart}'' rakstīja: ``Poļu dumpis beigtos pats no sevis, ja tā vadītāji necerētu uz Rietumvalstu militāru iejaukšanos''. Tika domāts par poļu spēku taupīšanu, lai izmantotu tos vēlāk, kad palīgā Polijai nāks ``visa Eiropa''. Kad dažviet vietējie militārie priekšnieki veica pasākumus, lai mobilizētu vīriešus vecumā no 18 līdz 30~gadiem, sacēlušos valdība izdeva dekrētu, kas atļāva pieņemt tikai tādu skaitu brīvprātīgo, kādam pietika ieroču. Atsevišķi sacēlušos vienību komandieri vairījās iesaistīties kaujā pret cara armijas daļām, tā vietā veicot nogurdinošus pārgājienus, kuri veda pie to dalībnieku spēku izsīkuma, bada, dezertēšanas. Atbilde uz pārmetumiem par tādu taktiku bija: ``Kaut ar mazu vienību, bet sagaidīt [Rietumu] intervenci.'' Sadursmes notika galvenokārt tad, kad cara karaspēks panāca kādu sacēlušos vienību un tai nācās vest kauju bez iepriekšējas sagatavošanās.

Tāpat tika uzskatīts, ka nevajag steigties ar ārzemju revolucionāru palīdzības pieņemšanu, jo tā varēja pasliktināt Rietumvalstu valdību attieksmi pret Poliju. Maijā notā, adresētā Francijas valdībai, Nacionālā valdība rakstīja, ka uzskata Rietumvalstu karu pret Krieviju par ``neizbēgamu'', tā kā tikai ar spēku iespējams piespiest Krieviju piekāpties; ka Polija ārvalstu armiju aizsardzībā ir spējīga organizēt 150~000 vīru lielu savu armiju un ir gatava samaksāt visus kara izdevumus. Vienlaikus izskanēja arī brīdinājums, ka palīdzības aizkavēšanās ``piespiedīs meklēt palīdzību revolucionāro partiju rindās''.

Tiesa, Francijas un Lielbritānijas valdības poļu sacelšanās sakarā sūtīja notas uz Pēterburgu, prasot ievērot Vīnes kongresa (1815) lēmumus. (Austrija tās nevarēja atbalstīt, jo pati 1846.~gadā bija tos pārkāpusi, likvidējot Krakovas republiku). Ar grūtībām Lielbritānijas, Francijas un Austrijas valdības saskaņoja un jūnijā nosūtīja Krievijai savas prasības: 1)~pasludināt pilnīgu amnestiju sacelšanās dalībniekiem, 2)~radīt nacionālu poļu valdību, 3)~pieļaut plašu poļu dalību administratīvajā pārvaldē, 4)~ieviest apziņas un katoliskās reliģijas ritu izpildes brīvību, 5)~poļu valodu atzīt par oficiālu visās iestādēs, 6)~noteikt likumīgu kārtību iesaukšanai armijā. Bez tam Krievijai tika piedāvāts noslēgt pamieru ar sacelšanās dalībniekiem un sasaukt lielvalstu, kuras bija parakstījušas Vīnes līgumu, konferenci, lai noteiktu kārtību, kā norādītās prasības jāīsteno dzīvē. Austrijas valdība, piedaloties šo prasību izstrādē, centās panākt, lai tās dotu karalistes iedzīvotājiem ne vairāk tiesību kā poļiem to bija Galīcijā. Prasības neatbilda sacēlušos vēlmēm iegūt pilnīgu savas tautas neatkarību. Sacēlušos valdība bija vīlusies, sagaidīja priekšlikumus ar ``sāpēm un izbrīnu'', vērtēja tos kā ``vislielāko pakalpojumu'' cara režīmam. Tā bija sarūgtināta, ka pamieram bija jāattiecas tikai uz Polijas karalisti, neskarot Krievijas rietumu guberņas (kaut bija zināms, ka Rietumu lielvalstis nekad pat nedomāja atzīt šo zemju piederību poļiem).

Cara valdība, pārliecinājusies, ka vismaz Austrija un Lielbritānija negatavojas iesaistīties militārā konfliktā, triju valstu prasības noraidīja, vienlaikus piekrītot sarīkot triju Poliju sadalījušo valstu konferencei, lai apspriestu stāvokli šo valstu poļu valdījumos un saskaņotu to atbilstoši ``laikmeta prasībām un apstākļiem''. Rietumu sabiedrībā, īpaši Francijā, Krievijas atbilde izsauca sašutumu, taču viena pati karot Francija nespēja, bet tās partnervalstis karu nevēlējās. Lielbritānija ar Polijas sacelšanos savu mērķi jau bija panākusi~--- nepieļāvusi Krievijas un Francijas tuvināšanos un vājinājusi Krievijas pozīcijas pasaulē. Tagad tā varēja arī piekāpties.

Jūlijā Nacionālā valdība publicēja aicinājumu ``Eiropas tautām un valdībām'', kurā sauca tās pārtraukt jebkādas attiecības ar carisko Krieviju un atzīt, ka tai nav nekādu tiesību valdīt Polijā.

Taču ``Eiropas valdības'' atteicās atzīt Poliju par karojošo pusi, kas varētu atvērt tai iespēju legāli iepirkt ārzemēs ieročus un vervēt brīvprātīgos. Sacēlušies tā arī palika dumpinieku statusā.

ASV Ziemeļu štati tai laikā karoja pret vergturu intereses atbalstošo Dienvidu štatu Konfederāciju. Lielbritānija vēlējās izmantot konfliktu savas ietekmes stiprināšanai Amerikas kontinentā. Krievija, kura 1861.~gadā bija atcēlusi dzimtbūšanu, ieņēma draudzīgu pozīciju pret ziemeļniekiem un pat nosūtīja divas eskadras uz Amerikas krastiem. Kā raksta krievu kara vēstures speciālists A.~Širokorads, Vidusjūrā uz Lielbritānijas jūras komunikācijām izgāja fregate ``\rutxti{Олег}'' un korvete ``\rutxti{Сокол}'' (``Vanags''). Vēl pirms tam Orenburgas ģenerālgubernators A.~Bezaks uzsāka ekspedīcijas korpusa formēšanu karagājienam uz Afganistānu un Indiju. Tas gan notika slepus, bet sekoja informācijas noplūde britu presei. Rietumu biržās sākās panika, kuģniecības kompānijas paaugstināja pārvadājumu cenas, apdrošinātāji mainīja apdrošināšanas noteikumus. Krievijas militāri-diplomātiskie soļi izrādījās pareizi. Galu galā Lielbritānija atteicās no sava nodoma iejaukties Amerikas Pilsoņu karā konfederātu pusē, ziemeļnieki turpināja dienvidnieku piekrastes blokādi, Dienvidiem katastrofāli sāka trūkt munīcijas un viņi karu zaudēja. ASV atbalstīja Krieviju Polijas jautājumā.

Galu galā iespējamo Krievijas pretinieku intereses bija pārāk atšķirīgas, lai tās nopietni iejauktos notikumos. Tā neviena valdība neatsaucās aicinājumam. Neviena valsts nevēlējās uzsākt karu ar Krieviju lai panāktu poļu brīvību. Rietumu lielvalstu politika vienīgi radīja veltas ilūzijas sacēlušos vidū, bet to palīdzības izpalikšana izsauca poļos lielu vilšanos.

1863.~gadā starptautiskā situācija pakāpeniski kļuva poļiem nelabvēlīga. Prūsija un Austrija tuvinājās kopīgai cīņai pret Dāniju. Drīz sacēlušies zaudēja savu atbalsta bāzi Galīcijā. 1864.~gada februārī Austrija tur ieveda kara stāvokli, arestēja vairākus tūkstošus cilvēku, likvidēja organizācijas, kuras palīdzēja cīnītājiem Polijas karalistē. Labvēlīgas izturēšanās pret poļu sacelšanos maska ar to tika nomesta. Poļu patriotu cerības uz Austrijas palīdzību galīgi izplēnēja.

Gan iekšējo, gan starptautisko notikumu pavērsiena rezultātā ``baltie'' no sacelšanās vadības tika izstumti un no 1863.~gada septembra vadošo vietu atkal ieguva ``sarkanie'', taču vienotības viņu rindās nebija. Oktobrī par diktatoru kļuva R.~Trauguts. Viņš kā virsnieks bija dienējis Krievijas armijā, 1862.~gadā atstājis dienestu, no 1863.~gada aprīļa vadījis sacēlušos vienību Grodņas un Minskas guberņās. R.~Trauguts sevi nepieskaitīja nevienai revolucionārajai partijai, pieturējās mēreniem uzskatiem, bet noteikti atbalstīja 22.~janvāra manifesta un agrāro dekrētu principus. Viņš mēģināja piesaistīt sacelšanās kustībai zemniekus. Taču neejot tālāk par 22.~janvāra dekrētiem, nesolot zemniekiem piešķirt vismaz daļu muižniekiem atsavinātas zemes, nedodot zemi bezzemniekiem, uz ko R.~Trauguts un viņa palīgi nebija spējīgi, zemnieku ietekmēšanai nebija citu reālu līdzekļu. Mēģinājumi aktivizēt sacelšanos izrādījās neveiksmīgi. Atsevišķi darbojošās sacēlušos vienības cieta sakāvi pēc sakāves. Decembrī tika reorganizēti sacēlušos bruņotie spēki. No atsevišķu vojevodistu vienībām tika veidota regulāra armija, sastāvoša no korpusiem, divīzijām, pulkiem, bataljoniem. Ar to tika stiprināta disciplīna, savstarpējā sadarbība. Taču tad pat~--- rudens un ziemas mēnešos skaidri iezīmējās poļu mantīgo slāņu, tai skaitā arī Pozenē un Galīcijā, atiešana no sacelšanās. Daudzi muižnieku un buržuāzijas pārstāvji atteicās maksāt nacionālo nodokli, devās uz ārzemēm.

Sacelšanās dalībnieku darbību traucēja arī bargie laika apstākļi. Tūkstoši cilvēku no sacēlušos vienībām tās pameta un vai nu bēga uz ārzemēm vai ieradās pie varas iestādēm nožēlot nodarīto un lūgt žēlastību. Ziemā aktīvo sacelšanās dalībnieku skaits sasniedza vien dažus tūkstošus. Tie formāli dalījās četros korpusos, no kuriem divi tā arī faktiski netika izveidoti. No 1863.~gada septembra līdz 1864.~gada aprīlim Polijas karalistē notika 392~bruņotas sadursmes, lielākoties gan visai nelielas.

Krievu sabiedrības lielākā daļa poļu sacelšanos uztvēra klaji negatīvi. Carisms, izmantojot revolucionārās kustības kritumu Krievijā pēc dzimtbūšanas atcelšanas, varēja koncentrēt spēkus cīņai pret Polijas atbrīvošanās kustību. Biedrība ``Zeme un brīvība'' gan mēģināja organizēt sacelšanos Pievolgā un Urālu apkaimē, bet neveiksmīgi. Arī poļu revolucionārs H.~Keņevičs ar vairāku Kazaņas garnizona virsnieku palīdzību, kuru bruņojumā bija daži revolveri, mēģināja organizēt zemnieku sacelšanos, izplatot Tambovas, Penzas un Saratovas guberņās viltotus it kā cara izdotus manifestus. Visi grupas dalībnieki tika arestēti un sodīti ar nāvi 1864.~gada jūnijā. Ir gan zināmi gadījumi, kad sacēlušos pusē pārgāja krievu virsnieki un karavīri, bet tā savu gatavību aizstāvēt citas tautas brīvību izpauda tikai atsevišķi cilvēki. Šādos apstākļos Krievijā veikt revolucionāro darbu kļuva ļoti grūti. Pakāpeniski novājinājās ``Zemes un brīvības'' organizācija. Drīz Krievijā vairs nebija revolucionāra centra. ``Acīmredzot,~--- rakstīja F.~Engelss K.~Marksam 1863.~gada jūnijā,~--- poļu sacelšanās šai ziņā radīja nelabvēlīgu ietekmi''. Secinājums atbilda patiesībai.

Liela daļa izglītotās krievu sabiedrības ieguva pārliecību, ka poļu sacelšanos izraisīja vienīgi Krievijas valdības pārliekā piekāpība uz nacionāliem sapņiem tendēto poļu priekšā. Muižnieku sapulces, pilsētu domes, universitātes, konservatori un liberāļi, baznīcas kalpotāji sūtīja Krievijas imperatoram savus vēstījumus, kuros atbalstīja soda ekspedīciju darbību Polijā. Kā atzina A.~Hercens, ``muižniecība, liberāļi, literāti, zinātnieki un pat skolnieki ir visi inficēti; viņu asinīs un audos ir iesūcies patriotiskais sifiliss''. Jau pēc dažiem mēnešiem pēc tam, kā A.~Hercens bija atklāti nostājies sacēlušos poļu pusē un paziņojis, ka ``neklusēs veselas tautas noslepkavošanas priekšā'', žurnāla ``\rutxti{Колокол}'', kurš pirms tam ar carisko ierēdņi patvaļas atmaskošanu Krievijā bija iekarojis milzu popularitāti, tirāža kritās piecas reizes. Poļu sacelšanos atbalstošie A.~Hercens un M.~Bakuņins strauji zaudēja popularitāti. Konservatīvā krievu prese A.~Hercenu un viņa piekritējus nesauca citādi kā par nodevējiem un poļu kustības aģentiem. (Pēc dažiem gadiem~--- 1867.~gada 29.~aprīlī A.~Hercens atzina, ka žurnāla ``\rutxti{Колокол}'' atbalsts poļiem 1860.--1863.~gadā bija ``kolosāla kļūda''. Tagad viņš bija pārliecināts ``par pilnīgu šīs drosmīgās un dumjās nācijas nespēju, bīstamību un trulumu'', kuru dēļ viņš un N.~Ogarevs pazudināja savu reputāciju.) Iespējams, nesekmīgā poļu sacelšanās aizstāvība ietekmēja A.~Hercena veselību, viņš šķīrās no dzīves, būdams salauzts, draugu pusaizmirsts un ienaidnieku apmelots. Galu galā arī pats viņš savos memuāros ``\rutxti{Былое и думы}'' (``Pagājība un apceres'') ar rūgtumu rakstīja: ``Poļos katolicisms ir attīstījis to mistisko ekzaltāciju, kura pastāvīgi viņus notur rēgu pasaulē \citespace{} Mesiānisms ir sajaucis galvu simtiem poļu un pašam Mickevičam''.

Smagu triecienu sacēlušies saņēma, kad 1864.~gada 19.~februārī (2.~martā) Krievijas imperators Polijas karalistē izdeva dekrētu par zemes piešķiršanu zemniekiem, un visas pēc 1846.~gada viņiem muižnieku atņemtās zemes atdošanu tiem atpakaļ. Ja 1863.~gadā cara valdība nežēlīgi vajāja tos, kuri sludināja un īstenoja Centrālās Nacionālās Komitejas 22.~janvāra agrāros dekrētus, tagad tā zemnieku reformu Polijas karalistē ņēma savās rokās.

Faktiski 1863.--1864.~gada sacelšanās bija pagrieziena punkts carisma agrārajā politikā poļu zemēs, kura rezultātā poļu muižniekiem nācās šķirties no cerības saglabāt visu zemi savās rokās. Jauno Krievijas kursu noteica vairāki faktori. Pirmkārt, valdība Pēterburgā nevarēja nerēķināties ar sacēlušos valdības 1863.~gada 22.~janvāra dekrētiem, kuri paredzēja zemnieku apsaimniekotās zemes nodošanu viņu rokās pret izpirkšanas maksu, carismam bija jākonkurē ar sacēlušos šļahtu un jānotur poļu zemnieki savā pusē. Otrkārt, neraugoties uz zemnieku kustības karalistē visumā mierīgo raksturu, 60.~gadu sākumā tā tomēr signalizēja par zemnieku neapmierinātību, lika carismam ar sevi rēķināties. Treškārt, sacelšanās sākums pārsvītroja carisma mēģinājumus izlīgt ar poļu muižniekiem, tāpēc tagad ar to interesēm varēja nerēķināties.

Īstenojot reformu, zemnieki tika ``izspēlēti'' pret šļahtu. Viņi tagad saņēma visu savu agrāk apstrādāto zemi, kā arī ēkas, mājlopus, darba rīkus un sēklu privātīpašumā, par ko maksāja zemes nodokli 2/3 apmērā no agrākā činša. Muižnieki saņēma kompensāciju par zaudējumiem no valsts fonda, kuru veidoja zemnieku maksājamais zemes nodoklis. Arī bezzemnieki ieguva nelielus zemes gabalus no valsts zemēm. Zemnieku zemes platība pieauga par 28\%, rezultātā zemnieku saimniecību skaits jau 1864.~gadā sasniedza 514~tūkstošus, kad 1859.~gadā to bija 424~tūkstoši. Pēc citiem datiem ap 200~000 bezzemnieku ģimeņu ieguva nelielus zemes gabaliņus. 40.~gados pēc reformas muižu platība karalistē samazinājās par 14\%, taču vēl joprojām 61,4\% visas zemes atradās muižnieku rokās. Kaut zemnieku ieguvums bija neliels, vairs nebija cerību zemi saņēmušos iesaistīt cīņā pret lielvaru.

Svarīgi, ka Polijas karalistē zemnieki zemi ieguva privātīpašumā, bet nevis kopienas īpašumā, kā tas notika Krievijā. Pie tam poļu zemnieki ieguva visu savu agrāk lietoto zemi, bez kādiem ``nogriezumiem'' kā Krievijā, viņi kļuva par tās īpašniekiem tūlīt, bez ``pagaidu atkarības stāvokļa''. Zemnieki tika atbrīvoti arī no muižnieku aizbildniecības, to pašvaldības bija brīvas no šļahtas ietekmes, bet bija atkarīgas no apriņķa varas iestādēm. Protams, tāds Krievijas varas ``radikālisms'' bija izskaidrojams ar centieniem apturēt zemnieku dalību sacelšanās kustībā. Jāuzsver, ka tieši poļu nacionālās atbrīvošanās kustība imperatoru, tā valdību virzīja uz konsekventāku, demokrātiskāku zemnieku jautājuma risināšanu, nekā uz to bija spējīga šļahta. Zemniekos tika stiprinātas monarhiskās ilūzijas, cerības uz valdības pasākumiem zemnieku aizsardzībai pret muižnieku patvaļu. Cara reformai poļu zemnieku acīs bija arī tā priekšrocību, ka to realizēja ``likumīga'' vara, kuras rokās bija visi līdzekļi tās realizēšanai, nevis ``dumpinieki'', kuru pašu vidū agrārajā jautājumā nebija vienotības. Zemnieku sacelšanos vairs nebija. Zemnieki, norūpējušies par pēc iespējas izdevīgāku sev cara valdības pasludinātās agrārās reformas īstenošanu, atteicās atbalstīt sacēlušos, dažreiz pat izdeva tos cara varas iestādēm. Tiesa, vairākums no jaunajām saimniecībām bija līdz 1,5~ha lielas (pareizāk~--- mazas). Norisa intensīvs zemnieku saimniecību platību sarukšanas process, galvenokārt mantošanas rezultātā. No 1864. līdz 1889.~gadam sīko saimniecību (no 1,5 līdz 7,5~ha) skaits ātri pieauga. Šo pusproletārisko saimniecību īpašnieki bija spiesti meklēt darbu ārpus saimniecības, muižnieku laukos. Tā tika radīts lēts darbaspēks, kas bija svarīgs nosacījums muižnieku saimniecību pārkārtošanai uz kapitālistiskiem pamatiem.

Reforma neatrisināja servitūtu jautājumu par zemnieku tiesībām lietot ganības un mežu, kur tie varēja iegūt būvmateriālus un kurināmo. Pēc reformas servitūti kļuva gandrīz vai par galveno gminas (\pltxti{gmina}~--- no vācu \detxti{Gemeinde}, kopiena, arī administratīvi-teritoriāla vienība, līdzīga pagastam) un muižnieku pretrunu punktu.

Arī blakus Polijai esošajā Lietuvā, Baltkrievijā un Labā krasta Ukrainā 60.~gados agrārā reforma tika pasteidzināta uz zemniekiem izdevīgākiem nosacījumiem.

Sacēlušos vadība neatrada pārliecinošu atbildi uz cara valdības reformu. Agrāro reformu ``visa poļu sabiedrība'' vērtēja kā demagoģisku pasākumu, un tiešu uzbrukumu svētajam šļahtas privātīpašumam. R.~Trauguts izdeva uzsaukumu, kurā pārmeta cara valdībai liekuļošanu, paziņojot, ka zemnieku pašvaldības neesot iespējams realizēt viņu analfabētisma dēļ, un, visbeidzot, draudēja, ka visus ievēlētos kopienu vecākos Nacionālā valdība uzlūkos kā ``brāļu slepkavas'' un vajās. Protams, tāds uzsaukums nesastapa labvēlīgu attieksmi zemnieku vidē.

Polijas karalistes dienvidos un Lietuvā sacēlušies vēl turpināja karot arī 1863./1864.~gada ziemā, bet domstarpības to vidū palīdzēja carismam uzvarēt. 1864.~gada aprīlī R.~Trauguts tika arestēts, zemākie komandieri bēga uz ārzemēm. Faktiski 1864.~gada maijā sacelšanās jau bija apspiesta. Aprīlī vēl notika 34~sadursmes, maijā~--- 14, jūnijā~--- 4, jūlijā~--- 1. Vasaras beigās likvidēti tika arī pēdējie pretošanās centri Polijas karalistes dienvidos. R.~Traugutam un nelegālās valdības locekļiem tiesa piesprieda nāves sodu un augustā viņi tika pakārti. Pēdējā nemiernieku vienība ksendza S.~Bžuzkas vadībā bēguļoja pa mežiem līdz 1865.~gada aprīlim, kad viņš tika saņemts gūstā un pakārts.

Sacelšanās ilga gadu un četrus mēnešus. Bruņotā cīņā no poļu puses piedalījās ap 50~000 cilvēku. Pavisam notika 1~229 bruņotas sadursmes, Polija karalistē 956, Lietuvā un Baltkrievijā 237, Ukrainā 35. Sacēlušos zaudējumi sasniedza 20~000 (ir pat dati, ka ap 30~000) cilvēku. Brīvprātīgi padevās ap 15~000, tikpat tika ieslodzīti un izvesti uz Sibīriju. Pēc sacelšanās sagrāves uz Rietumiem bēga ap 7~tūkstoši tās dalībnieku. Krievu armijas zaudējumi bija 3~343 cilvēki, no tiem 2~169 ievainoto. Īpaši skarbi cara varas iestādes izrēķinājās ar sacelšanās dalībniekiem~--- bijušajiem Krievijas armijas karavīriem. No 183 to rokās nonākušajiem krieviem 89 tika sodīti ar nāvi. Pēdējais nāves sods tika izpildīts 1866.~gada 22.~novembrī (4.~decembrī). Krievija, apspiežot sacelšanos rīkojās nežēlīgi, taču tikpat nežēlīgi, ja ne vēl briesmīgāk, tai pat laikā rīkojās rietumnieciskā Lielbritānija, apspiežot sipaju sacelšanos Indijā (\entxti{India's First War of Independence, the Great Rebellion}, 1857--1859).

1863.--1864.~gada sacelšanās, tāpat kā iepriekšējās poļu bruņotās uzstāšanās, neaptvēra visas poļu apdzīvotās teritorijas, taču visām tām bija vispārnacionāls raksturs tai ziņā, ka tās izsauca poļu nacionālās jūtas arī citos apgabalos, saņēma no turienes palīdzību, veicināja nacionālo ideju izplatību arī ārpus sacelšanās aptvertās teritorijas.

Daži vēsturnieki 1863.--1864.~gada sacelšanos vērtē galvenokārt kā poļu muižniecības (dažkārt literatūrā poļus arī šai periodā vēl sauc par ``šļahtiču nāciju'') kārtējo mēģinājumu sasniegt Žečpospolitas atjaunošanu ar ārvalstu palīdzību, kurš kārtējo reizi noveda zemi vēl lielākā nelaimē.

Turpretī revolucionāri, īpaši marksisti, augstu vērtēja poļu tautas pūliņus sasniegt brīvību un demokrātiju, jo tās panākumi būtu visas Eiropas demokrātijas interesēs, varētu vājināt Krievijas carismu. Sacelšanos sakāve izsauca dziļu F.~Engelsa nožēlu. Viņš rakstīja: ``Paies daudz laika, līdz Polija atkal varēs pacelties, pat ar citu palīdzību, bet Polija mums ir pilnīgi nepieciešama.'' Acīmredzot ar ``mums'' bija domātas visu Eiropas revolucionāru intereses.

1874.~gadā F.~Engelss uzsvēra, ka Polijas neatkarība un revolūcija Krievijā bija savstarpēji saistītas. Abu uzvara nozīmētu to, ka Vācijas valdība un buržuāzija turpmāk varētu paļauties tikai uz saviem spēkiem, ar kuriem laika gaitā vācu strādnieki paši tiktu galā. Jāatzīst gan, ka marksisma klasiķis pārāk romantiski vērtēja revolūcijas perspektīvas gan Krievijā, gan Vācijā.

Padomju vēsturnieki, kuri bija spiesti visur pārspīlēt šķiru cīņas nozīmi vēsturē, tomēr dažādi vērtēja tās raksturu. Vieni uzskatīja, ka tai piemita nacionālās atbrīvošanās cīņas un antifeodāls raksturs. Sacelšanās pēc būtības bijusi buržuāziska revolūcija. Citi uzsvēra, ka 1863.--1864.~gada sacelšanās nekļuva par buržuāziski--demokrātisku revolūciju, bet Polijas vēsturē iegāja kā neizdevusies buržuāziska revolūcija. Pēc šī darba autora domām sacelšanās spilgtākā komponente bija nacionālā, kura sevī saturēja arī ievērojamu demokrātisma lādiņu, sacelšanās uzvara varētu pavērt ceļu arī demokrātiskiem procesiem (ja vien poļu nacionālisti nesāktu konsekventi apspiest citas tautas). Taču tā nenotika ne tikai Krievijas militārā pārspēka, Rietumeiropas valstu neieinteresētības, bet arī poļu šļahtas nacionālā un sociālā egoisma rezultātā.

Pēc sacelšanās izskanēja interesants secinājums. To izteica poļu marķīzs A.~Veļepoļskis, vietvalža lielkņaza Konstantīna palīgs: ``poļu labā var vēl ko izdarīt, taču kopā ar poļiem~--- nekad.'' Tiesa, vēlāk A.~Veļepoļskis precizēja, ka tie nebija viņa vārdi, viņš tikai pievērsis tiem tautiešu uzmanību. Iespējams arī, ka dusmu brīdī tos bija pateicis pats vietvaldis. Taču neatkarīgi no autora personības zināmu patiesības graudu tie saturēja.

Pastāv pamatots viedoklis, ka 1863.~gada sacelšanās varēja uzvarēt tikai tad, ja tā būtu nacionāla revolūcija ar radikālas buržuāziskas agrārās reformas programmu, ja to atbalstītu visaptveroša sociāla zemnieku revolūcija.

Baltkrievu publicists A.~Tarass izdalījis četrus galvenos \strong{sacelšanās sakāves cēloņus}:

\begin{enumerate}

\item Reālas palīdzības trūkums no Rietumeiropās valstu puses. Bet sacelšanās tika gatavota tieši cerot uz Rietumu iejaukšanos!

\item Sacelšanās vadītājiem trūka pārdomāta kaujas darbību plāna.

\item Carismam izdevās piesaistīt sev poļu, ukraiņu, baltkrievu un lietuviešu zemniecību. (Autoram jāpiebilst, ka carismam tas neizdotos, ja sacēlušies būtu mērķtiecīgi aizstāvējuši zemniecības intereses.)

\item Nežēlīgās M.~Muravjova un F.~Berga represijas. (Atkal jāpiebilst, ka pret pretiniekiem nežēlīgas represijas pielietoja arī sacēlušies. Ir kontrproduktīvi skaidrot, kura puse pirmā lietoja represijās, kura bija nežēlīgāka. Kara gaitā tādus rēķinus neviens nestādīja. Sacēlušos vadītāji, ja viņi nebija garīgi atpalikuši, bet viņi tādi nebija, nevarēja ar šādām sekām nerēķināties, tomēr uzņēmās atbildību par kustības izraisīšanu).

\end{enumerate}

% page 161


Sacelšanās Polijā un 1864.~gada agrārā reforma nosacīti bija arī robežšķirtne starp diviem laikmetiem Polijā, kas nodalīja feodālismu un kapitālismu. Jau minētie padomju vēsturnieki V.~Djakovs un I.~Millers gan precizēja, ka ne visas poļu zemes šo šķirtni pārvarēja vienlaikus. Teritorijās, kura bija pakļautas Prūsijai un Austrijai, tā tika sasniegta jau 1848.~gadā, bet Krievijai pakļautajās~--- 1864.~gadā. Vairākums vēsturnieku gan par tādu nosacīti pieņem otro gadskaitli. Tomēr, kā norāda poļu autors V.~Kula, nenākas fetišizēt vienu datumu. Robeža starp feodālismu un kapitālismu visas Polijas mērogā veidojās XIX gadsimta vidū, kad Pozenes un Galīcijas apgabalos tiesiski zemniecība jau bija izcīnījusi atbrīvošanos no ārpusekonomiskajiem spaidiem. T.s. ``kongresa Polijā'' agrārā reforma norisa kā pēdējā no trijām poļu zemēm.

Pati 1863.--1864.~gada sacelšanās bija vismasveidīgākais un ilgstošākais poļu tautas atbrīvošanās mēģinājums no svešinieku kundzības. Taču ar tās sakāvi uz gadu desmitiem cīņa par ``brīvību un neatkarību'' (\pltxti{wolność i niepodległość}) tika atvirzīta citu, ikdienišķāku uzdevumu priekšā.

Krievijas impērijā pēc 1863.--1864.~gada tika forsēts kurss uz nacionālo nomāļu integrāciju. Viens no cēloņiem tam bija valdības vilšanās cerībās uz mierīgas un liberālas nacionālās politikas realizācijas sekmību. Pārsvaru guva doma, ka tikai stingra nacionālā politika palīdzēs saglabāt un nostiprināt impērijas vienotību.

1863.--1864.~gada sacelšanās, neraugoties uz neveiksmi, veicināja feodālo attiecību likvidāciju Polijas karalistē, piespiežot carismu īstenot tajā agrāro reformu uz zemniekiem labvēlīgākiem nosacījumiem nekā pašā Krievijā. Pēc 1864.~gada zemnieku reformai sekoja citas. Tai pat demokrātiskajā garā bija sastādīts 1864.~gada likums par gminu pašpārvaldi. Likuma uzdevums bija atbrīvot zemniekus no vietējo muižnieku ietekmes. (Pēc krievu vēsturnieka A.~Pogodina vērtējuma tas gan veda pie pretējā~--- nodibinājās kopējas vietējo iedzīvotāju materiālās un garīgās (nacionālās) intereses. Vietējie muižnieki kļuva tuvāki gminām un saskarsmē ar zemniecību ietekmēja zemnieku pasaules uzskata veidošanos. Zemnieki tāpat kā šļahtiči un garīdznieki sajuta, ka nacionālajai pastāvēšanai draud briesmas un mērķtiecīgi uzsāka tās aizstāvēšanu. Tomēr autoram šķiet, ka izklāstīto vērtējumu būtu grūti pamatot, jo vienas pretrunas~--- feodālās tika nomainītas ar citām~--- kapitālistiskām).

Poļu revolucionārs B.~Švarce, kurš gan tika arestēts un ieslodzīts neilgi pirms sacelšanās sākuma, bet pēc atbrīvošanas pievērsās sacelšanās izpētei un izvērtēšanai, savos darbos enerģiski apgalvoja, ka visas pozitīvās pārmaiņas, kuras notika Polijas karalistē pēc 1863.~gada, bet īpaši agrārā reforma, bija sacelšanās, tās carismam iedvesto baiļu rezultāts. Poļu vēsturnieks, sociologs, politiķis (sociālists) B.~Ļimanovskis uzskatīja, ka carisma realizētā agrārā reforma bija gandrīz vai norakstīta no poļu Centrālās Nacionālā komitejas agrārajiem dekrētiem. Pēc šī darba autora domām, starp carisma 1864.~gada agrāro reformu Polijas karalistē un sacelšanos bija vistiešākais sakars, tā kalpoja sacelšanās apspiešanai. A.~Tarass raksta, ka likteņa ironijas rezultātā poļu šļahta [sacelšanās gaitā] bija nolikusi savas galvas par to, lai uzlabotu stāvokli zemniekiem, kurus tās vairākums dziļi nicināja.

Daudzas citas pozitīvas pārmaiņas lauza sev ceļu, pārvarot carisma pretestību.

Pēc 1863.~gada poļu sacelšanās Krievijas valdība zaudēja cerības mierīgā liberālā ceļā panākt cittautiešu asimilāciju. Krievijas valdošie slāņi secināja, ka piekāpšanās, mēģinājumi pārvaldīt novadu ar liberālām metodēm poļu vidū izsauc tikai jaunas prasības, ved pie sacelšanās. Krievijas valdošajās aprindās radās sava veida ``poļu sindroms''~--- pārliecība par to, ka liberāls kurss ved pie dumpjiem, bet stingrs~--- pie nomierināšanās. Šis secinājums kļuva par pārliecinošu argumentu stingrāka kursa attaisnošanai un pārejai uz aktīvāku Polijas integrāciju impērijas sastāvā. Poļu nacionālisma vilnis izsauca krievu nacionālisma atbildes vilni. Tika cerēts, ka stingra nacionālā politika ļaus nostiprināt valsts vienotību.

Izplatījās panslāvisma ideoloģija, kuras pamatā bija ideja par visu slāvu tautu politisku apvienošanos uz kopīgiem etniskajiem, kultūras un valodas pamatiem, dzimusi XVIII gadsimta beigās~--- XIX gadsimta pirmajā pusē (terminu 1826.~gadā pirmo reizi lietoja slovāku jurists un rakstnieks J.~Herkels, XIX gadsimta 40.~gados to plaši izmantoja vācu un ungāru publicisti lai apzīmētu domājamos Krievijas draudus) pašā Krievijā sākotnēji nespēlēja nekādu lielo lomu, jo krieviem atšķirībā no citām slāvu tautām jau bija sava valsts. Situācija mainījās tikai pēc zaudētā Krimas kara, kad slāvu tautas tika uzlūkotas kā vēlams sabiedrotais cīņā par Krievijas ietekmi. Tomēr Krievijas impērijas ārpolitikā panslāvisma loma nebija necik nozīmīga, kaut tās politiskie pretinieki apgalvoja pretējo. Toties iekšpolitikā krievu augšslāņi panslāvismu centās realizēt, to lielākoties saprotot kā pankrievismu jeb vienkārši krievu nacionālismu, pārejošu šovinismā. Krievu panslāvisti pat mēģināja noliegt poļu piederību slāviem, apgalvot, ka tie piederot Rietumu sabiedrībai, jo nodevuši slāvu ideālus. Ar to klusējot tika atzīts, ka lielkrievu ideāli ir arī visu slāvu ideāli. No tā varēja secināt, ka poļiem ir būtiski jāmainās, lai tos atzītu par piederīgiem slāviem.

Krievijas administrācijas kurss uz nacionālo nomaļu integrāciju kļuva vispārējs un tika forsēts. Tam pievienojās unifikācija valodas jomā rusifikācijas formā. Jaunais kurss iezīmēja mērķi valodas un kultūras ziņā unificēt ne tikai Krievijas Rietumu zemes, bet arī Baltijas guberņas, Somiju, Aizkauāzu. Integrācijas politika izpaudās kā skolu denacionalizācija, preses, grāmatu dzimtajā valodā izdošanas, iestāšanās universitātēs un ģimnāzijās (sevišķi ebrejiem) ierobežojumi. Carisms poļu sacelšanos izmantoja kā ieganstu, lai pastiprinātu nacionālo spiedienu pret citām Krievijas mazākumtautām.

\strong{Cariskās Krievijas rīcība pēc} 1863.--1864.~gada \strong{sacelšanās sakāves} bija vēl krasāka nekā pēc 1830.--1831.~gada sacelšanās. Poļu pretošanās apspiešanas metodes Aleksandra II~--- cara-reformatora laikā izrādījās krietni skarbākas nekā viņa konservatīvā tēva Nikolaja I laikā. Nākamajās desmitgadēs Krievijas politika Polijā stādīja mērķi galīgi atrisināt tās jautājumu ar represiju un vardarbīgas integrācijas palīdzību. Taču tas nenozīmēja, kā to bez pamatojuma apgalvo baltkrievu publicists A.~Tarass, ka ``krievi kā mērķi stādīja poļu kultūras, poļu valodas un paša poļu dumpīgā gara likvidēšanu''. Polijas 1863.~gada sacelšanās apspiešana, poļu bargā sodīšana par nacionālās brīvības centieniem, parādīja pārējām slāvu tautām, ko tās var sagaidīt no Krievijas impērijas pakļaušanās un ko pretošanās gadījumā.

Ievērojamais krievu filozofs V.~Solovjovs pat izteica domu: ``Ja Vīnes kongresā toreizējais patvaldnieciskais imperators Aleksandrs I vairāk domātu par krieviem, nekā par poļu interesēm'', tad ``pamatiedzīvotāju apdzīvoto Poliju būtu atgriezis Prūsijai'', kuras sastāvā tā bija pēc Žečpospolitas dalīšanām. ``Tādā gadījumā,~--- sprieda krievu domātājs,~--- mums, iespējams, nevajadzētu daudz runāt par Poliju'', poļu jautājumu atrisinātu ``neizbēgama ģermanizācija'', no kuras ``Krievija 1814.~gadā izglāba Poliju''. Bijušajā Pozenes lielhercogistē, kuru arī Vīnes kongress izveidoja bijušajās Žečpospolitas zemēs, bet pievienoja Prūsijai, visi runāja vāciski, jo tur ``poļu elements nespēj pretoties vāciešiem un arvien vairāk un vairāk tiek to absorbēts.'' V.~Solovjovam nebija šaubu par to, ``kas būtu noticis, ja Prūsijas vācieši būtu saimnieki Polijas galvenajā daļā''. Tomēr, neraugoties uz savu atzinumu, ka ``Polijas ķermeni saglabāja un uzaudzināja Krievija'', krievu filozofs bija spiests konstatēt, ka ``poļu patrioti drīzāk piekristu noslīkt vācu jūrā, nekā patiesi samierināties ar Krieviju''. No tā izrietēja secinājums: ``Naidam ir dziļāks, garīgs iemesls.'' Tas ir ``mūžīgā strīda starp Rietumiem un Austrumiem izpausme, un Polijas jautājums ir tikai lielā Austrumu jautājuma fāze''. Un, ja tas tā ir, tad ``saprasties ar poļiem nav iespējams ne uz sociāliem, ne valstiskiem pamatiem.'' V.~Solovjova spriedumus XX gadsimtā faktiski apstiprināja poļu vēsturnieks un publicists A.~Bočeņskis, kurš nebūt nebija rusofīls, sapņoja par PSRS sabrukumu. Tūlīt pēc Otrā pasaules kara publicētajā grāmatā ``\pltxti{Dzieje głupoty w Polsce. Pamflety dziejopisarskie}'' (``Muļķības vēsture Polijā. Vēsturiski pamfleti. 1--4, Varšava, 1947.) viņš secināja, ka attiecībā uz poļiem, sākot ar Katrīnu Lielo, ``visi cari sekoja dinastiskās, nevis valstiskās un jo īpaši nacionālās absorbcijas līnijai''. Tomēr šī līnija ``vairakkārt tika pārtraukta, un gandrīz vienmēr to izdarīja Polijas puse'', ko ``vienmēr pavadīja neracionāls naids pret Maskavu, kas tika cītīgi kultivēts un uzpūsts līdz augstākajām robežām''.

Turpmāk Polijas karalistes sabiedriski--politiskajā iekārtā salīdzinājumā ar Krievijas citiem novadiem tika ieviests mazāk pārkārtojumu, kas ļautu piemēroties tirgus ražošanai. Tas nespēja apturēt ātro kapitālistiskās saimniekošanas veida ieviešanu Polijas karalistē, tomēr bremzēja to.

Kamēr Polija nebija pilnībā kļuvusi par Krievijas impērijas sastāvdaļu, vienmēr varēja gaidīt Eiropas valstu iejaukšanos Krievijas attiecībās ar to. Tāpēc likās nepieciešams likvidēt Polijas jautājumu kā starptautisku, kā pamatojošos uz 1815.~gada Vīnes traktātu. Visizdevīgāk to bija izdarīt 1864.--1865.~gadā, kad kļuva skaidrs, ka Polijas dēļ neviens negatavojas karot ar Krieviju, kad Prūsija un Austrija bija gatavas poļu sacelšanās laikā aizmirst Vīnes traktātu. Polijas karaliste tika pilnībā inkorporēta Krievijas sastāvā. Pasta un satiksmes ceļu pārvalde tika pakļauta tieši Pēterburgas ministrijām. Lai pastiprinātu administratīvo varu, bijušo piecu guberņu vietā izveidoja desmit. Pēc F.~Berga nāves 1874.~gadā tika likvidēts vietvalža institūts, kas pēc būtības jau bija zaudējis savu nozīmi. Viņa varas pārmantotājs P.~Kocebu tika nozīmēts par Varšavas ģenerālgubernatoru. Faktiski tika likvidēta Polijas ekonomiskā autonomija. Polijas karaliste zaudēja savu budžetu un patstāvīgu finanšu pārvaldes sistēmu. Valsts padomi un Pārvaldes padomi atlaida. Krievu administrācija Polijas karalistes nosaukumu nomainīja ar Vislas novada (krievu \rutxti{Привислинский край}, poļu \pltxti{Kraj Przywiślański, Kraj Nadwiślański}) vai Vislas guberņu (\rutxti{Привисленские губернии}) nosaukumu. Tas tika lietots jau 70.~gados (ir dati, ka no 1867.~gada), bet 1888.~gadā parādījās arī likumdošanas aktos, saprotot ar to 10 bijušās Polijas karalistes guberņas. Vislas novads bija arī viena no Krievijas ģenerālgubernatūrām. Kaut Polijas karalistes vārds gan turpināja vēl eksistēt, šeit netika ieviestas zemstes, kā tas notika Krievijā. Kaut carisms centās unificēt poļu zemes ar pārējām impērijas teritorijām, šeit netika ievestas Krievijā realizētās liberālās reformas (tiesu, cenzūras u.c.)

Kara tiesu represēto statistiskie aprēķini aptver vairāk nekā 20~tūkstošus sacelšanās dalībniekus un ar tiem sadarbojušos personu. No sodītajiem sacelšanās dalībniekiem 30,85\% bija zemnieki, 17,31\%~--- sīkpilsoņi, 36,42\%~--- tie šļahtiči, kuri bija pierādījuši savu ``augstdzimušo'' statusu, 10,21\%~--- tie šļahtiči, kuri šo statusu bija zaudējuši. Tādejādi gandrīz puse sodīto oficiāli vai neoficiāli piederēja šļahtiču kārtai. ``Sadarbojušos'' vidū šļahtiču bija vairākums. (Padomju vēsturnieks V.~Zaicevs, izanalizējot nepilnīgos Krievijas Kara ministrijas Auditoru departamenta datus par ap 8 tūkstošiem represētajiem sacelšanās dalībniekiem, gan devis nedaudz atšķirīgus skaitļus~--- šļahtiču šo represēto vidū bija līdz 30\%. Tomēr ievērojot, ka Polijas karalistes iedzīvotāju vidū pēc Krievijas senatora, valsts sekretāra Polijas karalistes lietās N.~Miļutina vadībā 1864.~gadā veiktā pētījuma, viņi kopā ar garīdzniekiem sastādīja tikai 5,27\%, arī tad šļahtiču īpatsvars represēto vidū jāvērtē kā visai augsts.) Var secināt, ka tieši viņi bija aktīvākie sacelšanās dalībnieki. Tikai pēc tam aktivitātes ziņā sekoja sīkpilsoņi un zemnieki. Pēc V.~Zaiceva datiem, jo tālāk uz Polijas karalistes austrumiem un dienvid-austrumiem, jo samazinājās nodokļus maksājošo un palielinājās priviliģēto slāņu represēto pārstāvju skaits.

Par dalību 1863.--1864.~gada sacelšanās ar nāvi sodīja ap 400~cilvēkus, ap 18~000 nosūtīja katorgā un izsūtījumā uz Sibīriju. Taču pēc poļu vēsturnieces V.~Sļivovskas savāktajām ziņām viņu dzīve bija tāla no mīta par ``pie ķerrām piekaltajiem katordzniekiem''. Par Usoļjes (\rutxti{Усолье-Сибирскоe}, pilsēta netālu no Irkutskas) katorgu kāds poļu memuārists rakstījis, ka nekad savā dzīvē viņš nav sastapis tik daudz interesantu cilvēku un dzirdējis tādas lekcijas un diskusijas kā šajā vietā. Protams, tā nebija visur. Dažu, īpaši garīdznieku liktenis bija visai smags.

Pati traģiskākā lappuse bija t.s. Aizbaikala sacelšanās (\rutxti{Забайкальское восстание}) 1866.~gadā. To organizēja 1863.~gada sacelšanās dalībnieki, kuri bija izsūtīti uz Sibīriju, kur tai laikā atradās ap 18~tūkstošu notiesāto un izsūtīto poļu revolucionāru. Krasnojarskā un Kanskā viņi kopā ar krievu revolucionāriem gatavoja sacelšanos. Poļu revolucionāri bija nozīmēti trakta (ceļa) celtniecībā no Irkutskas līdz Verhņeudinskai (\rutxti{Верхнеудинск}, tagad Ulanude (\rutxti{Улан-Удэ})). Ap 700~poļu 50 līdz 150 cilvēku grupās strādāja dažādos ceļa celtniecības posmos 200~km garumā. Aresti 1866.~gada sākumā izjauca sākotnējos plānus, sacelšanās ar mērķi izrauties uz Ķīnu sākās pienācīgi nesagatavota. Steidzīgi atsūtītā karaspēka priekšā sacēlušies atkāpās taigā, cenšoties sasniegt Krievijas robežu. Slikti bruņotos sacelšanās dalībniekus sakāva kazaku vienības, pēc divām nedēļām viņi bija spiesti padoties. Sacelšanās vadītājus nošāva Irkutskā 1866.~gada novembrī, simtus tās dalībnieku notiesāja mūža katorgā.

Pamatīgu triecienu saņēma šļahtiču kārta. Tika veikta tās ``tīrīšana'', ap 200~000 šļahtiču zaudēja savu statusu. Polijā muižniecība necieta vēl vairāk tikai tāpēc, ka to nevarēja aizvietot ar krievu vai lietuviešu inteliģenci. Turpretī ārpus Polijas karalistes poļu muižniecību šie represīvie pasākumi gandrīz neskāra, jo tur arī nebija plaša mēroga sacelšanās. Tūkstošus šļahtiču ģimeņu pārcēla uz Krieviju, ar to faktiski iznīka daudzi sīkie poļu ciemi. Pēc dažiem datiem uz Krieviju deportēja ap 70~000 cilvēku, pēc citiem~--- pat vairāk nekā 92~000. Sodītie poļi tika izmantoti mazapdzīvotu teritoriju kolonizācijai. Šim mērķim viņi saņēma aizdevumus un labus zemes gabalus. Izsūtīto vidū bija labi pelnoši ārsti, skolotāji, kantoru darbinieki. Tiesa, poļu pieņemšanu darbā dažādu nozaru iestādēs regulēja nevis likums, bet daudzas slepenas, nekad nepublicētas instrukcijas, cirkulāri, noteikumi, dažkārt arī mutiski izteikti aizliegumi, par kuru saturu var spriest tikai pēc faktiski pastāvošās situācijas.

Izsūtīto poļu vidū bija arī ievērojamais 1863.~gada sacelšanās dalībnieks, tās valdības komisārs Baltkrievijā un Lietuvā ģeogrāfs, zoologs, sabiedriskais darbinieks B.~Dibovskis. Viņam tika piespriests nāves sods pakarot, taču pateicoties vācu zoologu aizstāvībai un O.~f.~Bismarka vidutājībai, sods tika mīkstināts un viņš tika izsūtīts uz 12~gadiem Sibīrijā. Tur viņš pētīja Baikāla ezeru, Amūras upi, bagātināja zinātni ar virkni atklājumu zooloģijā, 1877.~gadā atgriezās dzimtenē, turpināja zinātnisko darbību. Dzīvi B.~Dibovskis beidza jau neatkarīgajā Polijā, būdams pasaulē atzīts zinātnieks, arī PSRS Zinātņu akadēmijas korespondētājloceklis.

Sibīrijas izpētē daudz devuši arī divi citi nedaudz vēlāk izsūtīti poļi: etnogrāfs, literāts, politiķis V.~Seroševskis un etnogrāfs, lingvists E.~Pekarskis. Pirmais par dalību strādnieku kustībā un pretošanos policijai tika izsūtīts uz Jakutiju, kur vāca etnogrāfiskus materiālus. 1896. un 1900.~gadā krievu valodā iznāca viņa darbs par jakutiem. Pēc trimdas viņš turpināja ražīgu zinātnisko, literāro un sabiedrisko darbību, 1933.--1939.~gadā bija Polijas Literatūras akadēmijas prezidents. Otrais 1880.~gadā tika arestēts par piederību sociālistu revolucionāru partijai un nelegālas literatūras glabāšanu notiesāts uz 15~gadiem katorgā. Šeit viņš nodarbojās ar dārzeņu audzēšanu, medībām, zveju. Apguvis jakutu valodu, viņš aizstāvēja vietējo iedzīvotāju intereses presē, rakstīja lūgumrakstus, uzsāka jakutu~--- krievu valodas vārdnīcas sastādīšanu. 1891.~gadā ieguvis tiesības atgriezties Krievijas Eiropas daļa, viņš tās neizmantoja. Tikai XX gadsimta sākumā E.~Pekarskis pārcēlās uz Pēterburgu, sākot tur strādāt Zinātņu akadēmijas Antropoliģijas un etnogrāfijas muzejā. 1907.~gadā iznāca viņa sastādītās jakutu valodas vārdnīcas pirmais laidiens. Līdz 1917.~gada oktobrim tika publicēti pieci vārdnīcas laidieni. 1930.~gadā iznāca tā pēdējais~--- 13.~laidiens. Bez vārdnīcas E.~Pekarskis uzrakstīja arī vairākus etnogrāfiskus darbus. 1927.~gadā viņš kļuva par PSRS Zinātņu akadēmijas korespondētājlocekli, 1931.~gadā~--- tās goda locekli.

Tā kā Baltkrievijā un Lietuvā katoļi nevarēja ieņemt valsts ierēdņu, arī skolotāju, amatus, daudzi no to pretendentiem bija spiesti paši meklēt darbu Iekškrievijas guberņās. Šādu izsūtīto un pārceļotāju pēcteči bija izcilais komponists D.~Šostakovičs, par kura vecvectēva poļu vai baltkrievu saknēm gan var strīdēties, un rakstnieks A.~Grīns. Poļu inteliģence deva lielu ieguldījumu Krievijas nomaļu attīstībā. Kad 1897.~gadā avīzē ``\rutxti{Новоe время}'' (``Jaunais Laiks'') tika publicēti dati par poļu speciālistu koncentrāciju dzelzceļa maģistrāļu celtniecībā impērijas austrumos, par to tika ziņots arī imperatoram. Nikolajs II tikai noplātīja rokas, krievu kadru bija pārāk maz, lai tie varētu nodrošināt milzīgās valsts industriālo attīstību. Piemēram, 1898.~gadā žurnāls ``\rutxti{Русский вестник}'' (``Krievu ziņotājs'') rakstīja par to, ka Permas-Kotlasas dzelzceļa maģistrāles celtniecība pilnībā atrodas poļu rokās, un tajā nodarbinātie krievi sūdzas par spaidiem no poļu puses. Tiesa, Krievijas valdības un sabiedrības zināmas aprindas arī pozitīvi vērtēja poļu darbu impērijas nomalēs. Tā sabiedriskais darbinieks un publicists F.~Umaņecs rakstīja: ``Krievijas pavalstnieki no ārzemniekiem Austrumos ir tādi pat eiropeiski-kristīgās idejas nesēji kā mēs \citespace{}, viņi šai stāvoklī vienmēr meklēs atbalstu tais pat valstiskuma pamatos, uz kuriem balstāmies arī mēs \citespace{} Rezultātā ir jāizdara praktisks secinājums, ka vācieši, zviedri, poļi, norvēģi utt. Usūrijā, pie Amūras, Turkestānā utt. ir tāds pat vēlams kolonizācijas elements, kā dzimušie Tveras, Maskavas vai Kurskas guberņās.''

Represijām pakļāva Polijas katoļu garīdzniecību. Šādam pavērsienam tā bija devusi iemeslu sacelšanās laikā, kad daudzi garīdznieki to atbalstīja. Laikabiedrs prūšu majors E.~Knorrs aprakstīja baznīcas kalpotāju izturēšanos sacelšanās laikā: ``Gandrīz visur klosteri un baznīcas kalpo kā ieroču un pārtikas noliktavas, kā slēptuves atsevišķiem insurgentiem [sacēlušamies] un veselām bandām''. Tāpēc ar 1864.~gada 8.~novembra dekrētu tika noteikta baznīcas īpašumu konfiskācija un klosteru, kā arī Varšavas garīgās akadēmijas slēgšana. No 197 klosteriem ar 2~187 garīgo ordeņu brāļiem turpināja darboties 35 klosteri ar 470~ordeņbrāļiem. Tiem bija aizliegts uzņemt jaunus brāļus. 1867.~gadā poļu katoļu baznīca tika pakļauta Romas katoļu kolēģijas Pēterburgā uzraudzībai. Virkne bīskapu, tostarp arī Varšavas, tika izsūtīti uz Krieviju. Romas pāvests ar to visu nevarēja samierināties, ar Krieviju tika pārtrauktas attiecības un beidza darboties 1847.~gada konkordāts. 1870.~gadā no 15 diocēzēm 12 nebija bīskapu. 1875.~gadā Krievijas pareizticīgās baznīcas Sinode (\rutxti{Святейший Правительствующий Синод}) nolēma likvidēt ūniju. 1879.~gadā tika oficiāli likvidēta uniātu baznīca, piespiedu kārtā apvienojot to ar pareizticīgo. Taču baznīca turpināja darboties pagrīdē. Tikai 80.~gadu sākumā notika daļējs Vatikāna un Krievijas izlīgums. Iedzīvotāji pasākumus pret katoļu garīdzniecību uztvēra kā vērstus pret katoļu reliģiju.

Lietvedība administratīvajās un tiesu iestādēs ``krievu'' Polijā kopš 1868.~gada oficiāli norisa krievu valodā. No 1875.~gada tā kļuva par obligātu tiesās. Varšavas Galveno skolu 1869.~gadā pārvērta par universitāti, kur mācības notika krievu valodā un darbojās Viskrievijas universitāšu statūti, kas ievērojami ierobežoja tās autonomiju. No 1872.~gada mācības visās vidējās izglītības iestādēs notika krievu valodā, nākamajā gadā sākās sākumskolu pāreja uz krievu valodu. (Varšavas apgabala un 9 Krievijas rietumu guberņu augstākajās un vidējās mācību iestādēs gan poļu valodu varēja pasniegt kā mācību priekšmetu.) Ar 1885.~gadu sākumskolas tika pakļautas pilnīgai rusifikācijai. Tikai ticības mācību katoļiem varēja pasniegt poļu valodā. Poļu zemnieki izrādīja pasīvu pretošanos skolu rusifikācijai, centās izvairīties no nodevām sākumskolu uzturēšanai. Arī poļu virsslāņi bija neapmierināti ar rusifikāciju, vienmēr uzsvēra carisma atbildību par kultūras attīstības bremzēšanu poļu zemēs, taču pēc oficiālajiem datiem paši maz ko darīja tautas izglītības līmeņa celšanai. Polijas karalistes lauku skolu budžetā privātziedojumu bija tikai 2,8~procenti, tas bija vismazākais īpatsvars starp visām Krievijas impērijas guberņām un apgabaliem. Diemžēl autora rīcībā nav datu, cik lielu atbalstu zemnieki un muižnieki sniedza slepenai poļu valodas mācīšanai, kura bija izplatīta ciematos.

Rusifikācijas politika augstāko punktu sasniedza 1883.--1894.~gados, kad Varšavas ģenerālgubernatora pienākumus pildīja ģenerālis J.~Gurko. Rusifikācijas izglītības jomā faktiski bija stiprākais līdzeklis kā modināt poļu zemnieku masu nacionālo pašapziņu, kad lielai daļai inteliģences nebija vairs spēka protestēt un bērnus tā sūtīja krievu skolās. Rusifikācijas rezultātā bija vērojama skolu sniegtā izglītības līmeņa pazemināšanās (skolas beidzēji nebija ieguvuši izglītību dzimtajā valodā un nebija apguvuši arī krievu valodu) un skolu tīkla sašaurināšanās. Ik gadu pieauga skolas neapmeklējušo bērnu skaits, kā rezultātā auga arī lasīt un rakstīt nepratēju skaits. Piemēram, Varšavā 1882.~gadā tādu bija 48,6\%, bet 1897.~gadā~--- 46,5\% no visiem pilsētas iedzīvotājiem.

Krievijas valdītājiem bija skaidrs, ka savu kundzību, novada pārvaldi nevar bāzēt uz poļu ierēdniecību, liela daļa no kuras sacelšanās laikā izrādījās nelojāla. To bija jānomaina ar krievu ierēdniecību. Krievijas pareizticīgās baznīcas Sinodes virsprokurors K.~Pobedonoscevs 1983.~gadā vēstulē imperatoram Aleksandram III rakstīja; ``nekādā gadījumā nedrīkst ar uzticību paļauties uz katoļiem-poļiem\dots{}'' Tā kā pat daudzi cariskās administrācijas ierēdņi krievi nevēlējās piedalīties patvaldības politikas realizācijā Vislas novadā, valdība radīja virkni privilēģiju tiem ierēdņiem, kuri dienēja Polijā. Tāpēc šurp devās dažādi karjeristi, pašlabuma meklētāji. Ierēdņu un vietējo iedzīvotāju attiecības pasliktinājās. Deklasētajiem poļu šļahtičiem bija slēgts ceļš uz valsts, sabiedriskā, kultūras darba virsotnēm, viņi varēja vienīgi iestāties darbā valsts iestādēs zemākajos amatos bieži pārlieku pašpārliecinātu un aprobežotu krievu ierēdņu uzraudzībā. Šādi šļahtiči saglabāja romantisku jūsmu par Polijas pagātni un naidu pret carisko patvaldību. Diemžēl naidu pret carismu viņi bieži pārnesa arī uz visu krievu tautu.

Šļahtiču nesamierināmību ar carismu uzskatāmi demonstrēja nabadzīgs šļahtičs A.~Berezovskis, mūzikas skolotāja dēls, kurš pretēji tēva gribai bija piedalījies 1863.~gada sacelšanās, pēc tās apspiešanas emigrējis un dzīvoja Parīzē, strādājot atslēdznieka darbnīcā. 1867.~gadā Parīzē notika Vispasaules izstāde, kuru apmeklēja arī imperators Aleksandrs II. 6.~jūnijā Buloņas mežā A.~Berezovskis šāva uz Alekasndru II, kurš brauca ekipāžā ar diviem saviem dēliem un Francijas imperatoru Napoleonu III. Pistoles lādiņš bija pārāk stiprs, tā uzsprāga, ievainojot rokā pašu A.~Berezovski, bet lode trāpīja kāda imperatora pavadoņa zirgu. Pūlis tūlīt sagrāba šāvēju. Viņam piesprieda mūža katorgu, bet 1906.~gadā amnestēja.

Taču, kā norādījis latviešu vēsturnieks Aldis Miņins, impērijas attieksme pret poļiem nebija totāli represīva. Pat pēc 1863.~gada poļu sacelšanās apspiešanas 6\% impērijas augstākajā ierēdniecībā veidoja poļu muižnieki, bet Polijas karalistes pārvaldē etniskie poļi vienmēr veidoja vairākumu, attiecīgi XIX gadsimta 60.~gados~--- 80\%, bet 90.~gados~--- nedaudz virs 50\% no visiem ierēdņiem. Tiesa, poļu ierēdniecības procentuālā īpatsvara kritums Polijas karalistē pēc sacelšanās apspiešanas bija iespaidīgs~--- gandrīz 30\%.

Sakāves sajūta un represijas, kuras Krievijas vara vērsa kā pret sacelšanās dalībniekiem, tā arī visu poļu sabiedrību, tikai padziļināja un pastiprināja poļos pretkrievisko noskaņojumu. Vairākuma prātos nostiprinājās pārliecība, ka Krievija ir bijusi un būs mūžīgi naidīga Polijai. Tautas vairākumam pēc 1863.~gada sacelšanās sakāves spēku gan pietika tikai pasīvai pretošanās demonstrēšanai.

Polijas karalistē un blakus esošajās Krievijas rietumu guberņās izplatījās lapiņas ar aicinājumiem cīnīties pret nacionālo apspiestību, draudu vēstules cara administratoriem. Bieži bija ``viņa majestātes'' (imperatora) apvainošanas gadījumi, aizlūgumi baznīcās atbrīvošanās cīņu dalībnieku piemiņai, to portretu izplatīšana, demonstratīva sēru drānu un atceres zīmju nēsāšana datumos, saistītos ar 1863.~gada sacelšanos. Manifestācijās iesaistījās diezgan daudz patriotisko spēku pārstāvju. Pilsētu un lauku apakšslāņu pārstāvji centās izvairīties no dienesta armijā. Tikai 1867.--1870.~gadā Varšavas apgabaltiesa piesprieda sodus 1504~poļu rekrūšiem par atteikšanos nodot militāro zvērestu vai bēgšanu uz ārzemēm.

Veidojās slepeni pulciņi. Tajos tika izstrādāti jauni bruņotas cīņas plāni. Nacionālās cīņas aktīvisti īpaši aktivizējās saasinoties starptautiskajai situācijai un kaut gan Francijas sakāve 1870.~gada karā pret Vāciju (padomju literatūrā to parasti sauca par franču-prūšu karu, kaut tajā piedalījās arī citas vācu valstis) nobīdīja poļu jautājumu no starptautiskās dienas kārtības, poļi tomēr loloja cerības uz tā atkārtotu izvirzīšanos.

60.~gados arī Krievijas pilsētās pastāvēja daudz dažādu virzienu nelegālu poļu pulciņu. Tie uzturēja sakarus ar poļu izsūtītajiem, emigrantiem ārzemēs. Vairāki poļu revolucionāri aktīvi piedalījās krievu narodņiku teroristiskajā darbībā. Tā tieši poļa I.~Griņevicka mestā bumba ņēma dzīvību gan imperatoram Aleksandram II, gan viņam pašam. L.~Mirskis 1879.~gadā neveiksmīgi šāva uz Krievijas žandarmērijas šefu A.~Drentelnu. L.~Kobiļanskis piedalījās nāvējošā atentātā pret Harkovas gubernatoru kņazu D.~Kropotkinu. Vēl daudzi citi poļi tika tiesāti par dalību terorismā, kā arī revolucionārā propagandā.

Toties poļiem juta līdzi ``visa'' ``progresīvā'' Eiropas sabiedrība. Tās acīs poļu sacelšanos apspiešana veidoja Krievijas kā despotiskas monarhijas tēlu, kura apspiež visas brīvības izpausmes, bremzē progresu, simbolizē reakciju un atpalicību. Kā norādījis poļu--krievu attiecību pētnieks E.~Gorizontovs, ``Poļu rusofobija būtiski ietekmēja Krievijas tēlu Rietumu acīs un atstāja iespaidu arī uz ukraiņu un baltkrievu kustību ideoloģiju.'' Turpretī par poļu tautu Eiropā nostiprinājās priekštats, ka tai ir īpaša loma cīņā ``par mūsu un jūsu brīvību''. 1867.~gadā K.~Markss rakstīja: ``\dots{}Eiropai ir atlicis izvēlēties vienu no diviem. Vai nu aziātiskais barbarisms ar moskaļiem priekšgalā kā lavīna nobruks pār to, vai tai nāksies atjaunot Poliju, ar to norobežojoties no Āzijas ar divdesmit miljoniem varoņu un izmantojot brīvās minūtes jaunu sabiedrisku pārveidojumu veikšanai''. Tiesa, Eiropas sabiedrisko un nacionālo atbrīvošanās kustību atbalsts nekādi nepalīdzēja Polijai, taču atstāja savu iespaidu tautas politiskajā apziņā.

Dažus gadus vēlāk būtiski mainījās situācija Eiropā, radās spēcīga, apvienota Vācijas impērija. Jau 70.~gados ``triju imperatoru savienība'' (tā dēvēja Krievijas, Vācijas un Austroungārijas imperatoru savienību, nostiprinātu ar 1873., 1881., un 1884.~gada līgumiem) apstiprināja Eiropas politiskās kartes nemainību un Polijas jautājums pārstāja būt par starptautisku un kļuva par Krievijas, Vācijas un Austroungārijas savstarpējo attiecību jautājumu. Poļiem zuda cerības bez nopietna starptautiska konflikta sekmīgi cīnīties pret apspiedējvalstīm Krieviju, Vāciju un Austroungāriju.

Pēc 1863.~gada sacelšanās sakāves veco emigrāciju papildināja jauni poļu pulki. Atšķirībā no vecās, galvenokārt šļahtiču veidotās, emigrācijas, šī jaunā emigrācija pārsvarā sastāvēja no inteliģences un amatniekiem. Taču tā vairs nebija vecā, ``lielā'' emigrācija. Ja pēc 1831.~gada sacelšanās emigrantu vidū pārsvarā valdīja optimisms un cīņas spars, ar viņiem sakaros stājās Rietumu valdības, tad pēc 1864.~gada, kā atzinis L.~Vasiļevskis, emigrantu vidū valdīja bezcerība, starp viņiem nebija Rietumos atzītu izcilu personību. Rietumvalstu valdības pārtrauca sakarus ar viņiem. Jaunie emigranti pēc sociālā sastāva un uzskatiem bija ievērojami demokrātiskāki nekā iepriekšējā viņu paaudze. Tas izsauca arvien dziļākas pretrunas starp poļu emigrantu grupām jautājumā par Polijas neatkarības atjaunošanas ceļiem.

Eiropas valstīs strādnieku vidū pastāvēja solidaritātes kustība ar 1863--1864.~gada poļu sacelšanos. 1863.~gada 23.~jūlijā pēc solidaritātes mītiņa Londonā tikās angļu, franču un poļu strādnieku delegācijas. Tika nolemts sagatavot uzsaukumu franču strādniecībai ar aicinājumiem radīt starptautisku strādnieku organizāciju un ``uzstāties kopīgi Polijas atbrīvošanas labā''. Kopīgajai poļu tautas aizstāvībai bija jākļūst par pirmo starptautiskās strādnieku savienības rīcības aktu. Franču strādnieku pozitīvā atbilde tika nolasīta 1864.~gada 28.~septembrī Londonā lielā mītiņā, kurā arī tika pasludināta Starptautiskās Strādnieku asociācijas (vēlāk sauktas par I Internacionāli, 1864--1872) dibināšana. Tāpēc arī K.~Markss rakstīja, ka 1863.~gada poļu sacelšanās, kura izsauca angļu un franču strādnieku protestu pret savu valdību starptautiskajām ļaundarībām, kalpoja par Internacionāles sākumpunktu.

Vēlāk I Internacionāles vadītāji pastāvīgi uzsvēra, ka Polijas atbrīvošanās jautājums ir viens no svarīgākajiem straptautiskajam proletariātam, ka Polijas atbrīvošana nozīmētu visu valstu, bet pirmkārt jau Vācijas un Krievijas, strādnieku atbrīvošanu, ka pats Polijas stāvoklis revolucionarizē to. Jautājums par ``Maskavas draudiem Eiropai un neatkarīgas un vienotas Polijas atjaunošanu'' tika iekļauts Starptautiskās Strādnieku asociācijas pirmās konferences darba kārtībā. Kaut K.~Markss un F.~Engelss vienmēr uzsvēra, ka tieši poļu valdošo slāņu egoisms ir pazudinājis Poliju, 1865.~gadā K.~Markss tomēr izteica cerību, ka pēdējās sacelšanās mācības šļahtai nav bijušas gluži veltas. (Jāsaka gan, ka no revolucionārās strādniecības viedokļa šāds tās politisko vadītāju revolucionārs romantisms kopumā neattaisnojās.)

1865.~gadā Londonā tika izveidota I Internacionāles Poļu sekcija, kuras sastāvā ietilpa vairāki desmiti emigrantu. Sekcijā iestājās gan Londonas poļu radikālo organizāciju, kam bija sociālistiska ievirze, dalībnieki, gan ``sarkano'' kreisā spārna pārstāvji~--- bijušie 1863.--1864.~gada sacelšanās dalībnieki, nonākušie emigrācijā. Vairāki poļi tika ievēlēti Internacionāles Ģenerālajā padomē. Poļu revolucionāri saņēma arī materiālu palīdzību no Internacionāles. Policijas ziņojumos bija pausta trauksme par Internacionāles ideju iespiešanos Polijā. Taču teiktais nenozīmēja, ka visi poļu emigranti pieņēma marksisma ideoloģiju. Daudzus piesaistīja nevis sociālistiskie ideāli, bet Internacionāles pozīcija Polijas jautājumā. Daļa poļu demokrātu, arī ``sarkano'' mērenā spārna pārstāvji pret Internacionāli izturējās noraidoši, pat naidīgi. Kopumā visas poļu emigrācijas rindās I Internacionāles piekritēju grupa veidoja nenozīmīgu mazākumu.

Daudzi emigrācijā nonākušie poļu revolucionāri atbalstīja nacionālās un sociālās atbrīvošanās cīņu citās Eiropas valstīs. Viņi piedalījās itāļu tautas cīņā pret austriešu apspiedējiem. Vācijas un Francijas kara (1870.--1871.) otrajā posmā simti poļu revolucionāru un tai skaitā arī J.~Dombrovskis cīnījās Francijas Republikas pusē, cerot, ka tās uzvara atvieglos Polijas ``atjaunošanu''. Vēlāk ap pieci simti poļu sociālistu un demokrātu darbojās Parīzes komūnā. Starp augstākajiem Komūnas militārajiem komandieriem apmēram 20 bija poļi. Kā atzina F.~Engelss, poļi Parīzes komūnas aizstāvju rindās bija vienīgie uzticamie un spējīgie karavadoņi. J.~Dombrovskis tika ievainots un no ievainojuma nomira. Versaļas valdības izveidotās kara tiesas gūstā kritušos poļus parasti lika nošaut. Piemēram, nāves spriedums neklātienē tika piespriests vienam no Komūnas spēku vadītājiem V.~Vrubļevskim. Viņš nonāca Londonā, izveidoja tur draudzīgas attiecības ar K.~Marksu un F.~Engelsu, pēc viņu priekšlikuma kļuva par I Internacionāles Ģenerālpadomes locekli, līdz mūža beigām turpināja simpatizēt sociālismam. 1875.~gada decembrī starptautiskā strādnieku mītiņā Londonā V.~Vrubļevskis paziņoja, ka īstais Polijas karogs ir sociāldemokrātiskais karogs, zem kura jāvienojas visiem poļu tautas progresīvajiem spēkiem cīņā par dzimtenes neatkarību un sociālo progresu. Mērķa sasniegšanas iespēju viņš saskatīja vienīgi nodibinot ciešu poļu tautas savienību ar rietumvalstu un īpaši Krievijas darbaļaudīm. ``Mums ir jāgatavojas uz sacelšanos,~--- teica V.~Vrubļevskis,~--- kopā ar krievu sociālistiem, tāpēc poļiem ir jāpalīdz krieviem Krievijā, tāpat kā krieviem jāpalīdz poļiem Polijā.'' K.~Markss augstu vērtēja poļu starptautisko darbību: ``Polija lēja savas asinis karā par Amerikas neatkarību, viņas leģioni cīnījās zem pirmās Francijas Republikas karogiem; 1830.~gadā tā ar savu revolūciju novērsa iebrukumu Francijā, kuru toreiz nolēma veikt Polijas dalīšanas dalībnieces: 1846.~gadā Krakovā Polija pirmā Eiropā pacēla sociālās revolūcijas karogu; 1848.~gadā tās dēli ieņēma izcilu vietu revolucionārajās cīņās Ungārijā, Vācijā un Itālijā; beidzot, 1871.~gadā tā deva Parīzes komūnai labākos ģenerāļus un pašus varonīgākos kareivjus''. Ievērojot Polijas kā nemiera un nestabilitātes perēkļa lomu Eiropā, K.~Marksa revolucionāro jūsmu var saprast. Taču, kā atzīmējusi krievu vēsturniece S.~Falkoviča, piedaloties notikumos Eiropā, cīnoties par citu tautu interesēm, poļi gandrīz vienmēr uzstājās arī par Poliju, meklēja kādu labumu poļu nacionālajai lietai. Tā, krievu-turku karā (1877--1878) pret Krieviju karoja arī poļi, kopā 145 cilvēki. Tomēr neizdevās turku mēģinājums, solot naudu, izraisīt poļu sacelšanos.

Pēc 1863.--1864.~gada sacelšanās Polijas karalistē \strong{īpaši cieta apkārtējie apgabali}. Jau 1863.~gada maijā par Krievijas Ziemeļrietumu novada (Lietuvas un Baltkrievijas) ģenerālgubernatoru tika nozīmēts ģenerālis M.~Muravjovs. (Revolucionāru vidū viņš tika iesaukts par ``Muravjovu--kārēju'', jo, kad 1831.~gadā M.~Muravjovu iecēla par Grodņas gubernatoru, viņam pajautāja, vai viņš nav pakārtā dekabrista S.~Muravjova-Apostola radinieks. M.~Muravjovs uz to atbildēja, ka ir cēlies nevis no tiem Muravjoviem, kurus kar, bet no tiem Muravjoviem, kuri paši kar.) Krievijas valdošajās aprindās pastāvēja gan birokrātu grupa naidīga Ziemeļrietumu novada ģenerālgubernatoram M.~Muravjovam. Tā, iekšlietu ministrs grāfs P.~Valujevs uzskatīja, ka ģenerālgubernators M.~Muravjovs vēlas pilnībā atstumt centrālās varas iestādes no Ziemeļrietumu novada pārvaldes. Tomēr galu galā arī viņš atzina vietvalža darbības rezultātus: ``Viņš rīkojas kā diktators, bet rīkojas!'' M.~Muravjovs uzskatīja, ka 1863.~gada sacelšanās gaitai bija jāpārliecina valdību, ka līdz tam realizētā piekāpšanās politika, jebkuras reformas poļu muižniecība uztver kā vājuma apliecinājumu un tikai pastiprina tās dumpīguma garu. Viņš rakstīja, ka poļu muižnieki ``\dots{}ar ``tēvzemes atjaunošanas'' ieganstu vēlas tikai \citespace{} iegūt agrākās patvarīgās tiesības pār zemniekiem un pārējiem nodokļu maksātājiem, kādas viņi neierobežoti izmantoja Žečpospolitas laikā''. M.~Muravjova valdīšanas laikā Krievijas Ziemeļrietumu guberņās represijām tika pakļauti ap 9~000 cilvēku, tai skaitā ar nāvi sodīti 128, katorgā nosūtīti ap tūkstoti, pārējie nodoti karadienestā vai arestantu rotās.

Krievijas kara ministrijas augsts ierēdnis I.~Jākobsons 1863.~gada 12.~jūnijā savā dienagrāmatā atstāja ierakstu par M.~Muravjova rīcību; ``Viņš pakāra Viļņā kādu ksendzu un insurgentu [sacelšanās dalībnieku] un atstāja viņus uz karātavām apgānīšanai veselu dienu. Kad viņu līķus apraka zemē, saradās daudz cilvēku, lai paklanītos viņu kapavietām, tika turēti aizlūgumi, tika ņemta zeme no kapiem par piemiņu, un tie tika dažādi godināti. Policijmeistars par to ziņoja Muravjovam un lūdza norīkot karaspēku, lai izdzenātu cilvēkus. Muravjovs tam nepiekrita, bet lika apliet kapus ar ekskrementiem un šai nolūkā atvest vairākus vezumus no slimnīcu atejām. Pavēle tika izpildīta, un tik tiešām vairāk neviens netuvojās kapiem. Taču kāds iespaids tika radīts fanātiskajiem katoļiem?''

Tautā plaši bija izplatīts M.~Muravjova teiciens: ``Ko nepabeidza krievu durklis, to pabeigs krievu skola''. Atbilstoši tam notika skolu rusifikācija. Skolās aizliedza poļu valodu, 1865.~gadā krievu valodu ieviesa kā vienīgo pārvaldes valodu. Par augstākajiem ierēdņiem nozīmēja tikai krievus. No ``nomierināšanas'' viedokļa M.~Muravjovs darbojās tik efektīvi, ka pēc amata atstāšanas saņēma grāfa titulu ar piebildi ``Viļņas'' (``\rutxti{Виленский''}) un Krievijas impērijas augstāko~--- Aleksandra Pirmdzimtā (\rutxti{Александра Первозванного}) ordeni. Aleksandrs II vēlāk izteicās par M.~Muravjovu, ka viņš esot bijis vienīgais ierēdnis, kurš pratis poļus ``turēt rokās''. Arī mūsdienu Krievijas historiogrāfijā tiek uzsvērts, ka M.~Muravjovs lielas pūles veltīja baltkrievu zemnieku emancipācijai un viņu atkarības no poļu muižniecības vājināšanai un viņa darbība pierādīja, ka krievu skola, krievu garīdznieki un ierēdņi izrādījās efektīvāki nekā krievu durkļi. Tiek pausts arī viedoklis, ka tieši viņš kā Krievijas impērijas augstākā amatpersona ielika pamatus mūsdienu baltkrievu nācijai.

Poļu muižnieku kārta Krievijas Ziemeļrietumu novadā pēc 1863.~gada sacelšanās sakāves piedzīvoja bēdu dienas. Kā norādījis poļu vēsturnieks V.~Smoļenskis, viena no galvenajām M.~Muravjova aktivitātēm bija sava veida kontribūcijas uzlikšana 10\% apmērā no ienākumiem visiem poļu zemes īpašumiem. Kaut arī sacelšanās beidzās, bet sods par to poļiem turpinājās. Tiesa, 1869.~gadā dekrēts samazināja M.~Muravjova noteiktās iemaksas, bet aizstāja to ar pastāvīgu nodokli ar 5\% nodevas nosaukumu no poļu izcelsmes personām. Šis nodoklis, kas lietuviešu apdzīvotajās guberņās ik gadu ienesa valsts kasē apmēram 2 miljonus rubļu, pakāpeniski izputināja poļu zemes īpašniekus.

Ja Polijas karalistē poļiem tika konfiscētas 1~660 muižas, tad Krievijas rietumu rajonos pat vairāk~--- 1~760. Dzimtbūšana, muižnieku privilēģijas bija atceltas. Daudzi muižnieki bija aizbēguši vai izsūtīti, saimniekot vajadzēja viņu sievām, kuras to neprata, vēl jo vairāk tāpēc, ka tagad vajadzēja izmantot algotu darbaspēku. Situāciju izmantoja bagātākie ebreji, kuri piedāvājās muižu saimniecēm, kā arī palikušajiem, bet saimniekot nespējīgajiem poļu šļahtičiem iznomāt viņu īpašumu. Īpašnieki saņēma pieklājīgu nomas maksu un tiem nebija jārūpējas par saimniecību. Protams, arī muižu apsaimniekotāji ebreji nepalika zaudētājos.

XIX gadsimta otrajā pusē blakus Polijas karalistei esošajā Grodņas guberņas Kameņecas (\rutxti{Каменец-Литовск}) pilsētiņā dzīvojušais ebreju tirgotājs E.~Kotiks savās atmiņās krāsaini aprakstījis vietējo iedzīvotāju, īpaši ebreju, attiecības ar poļu muižniekiem, kuras līdzīgas bija arī Polijas karalistē.

Pēc E.~Kotika liecības ap 65\% Grodņas guberņas muižu apmetās saimniekot ebreji. Muižnieki ar viņu saimniekošanu bija apmierināti. Atgriežoties mājup pēc izsūtījuma, muižnieki nepazinuši savas labiekārtotās saimniecības. Daži mēģinājuši turpināt saimniekot paši, bez ebreju nomniekiem, tomēr pēc diviem-trim gadiem muižas atkal panīkušas un muižnieki atkal tās nodevuši nomā ebrejiem. Taču, ja sākumā muižnieki savas muižas iznomāja par zemu cenu, tad turpmāk viņi to pamatīgi pacēla un pēc gadiem desmit muižu apsaimniekošana vairs ebrejiem nebijusi ``nemaz tik izdevīga''. (Tiesa, atmiņu autors nestāsta, kā ebreju saimniekošanas veidu vērtēja vietējais darbaspēks~--- zemnieki-baltkrievi. Acīmredzot viņos auga nepatika ne tikai pret poļu muižniekiem, bet arī nomniekiem--ebrejiem, kuri saimniekoja jau ar kapitālistiskām metodēm.) Ap pilsētu dzīvoja pāris simtu muižnieku, kam katram piederēja ap divi simti vai vairāk dzimtcilvēku. Katram muižniekam miestā bija viens vai divi uzticami ebreji-veikalnieki, ar kuriem viņš kārtoja savus darījumus. Ja ap muižnieku ``grozījās'' divi ebreji, tad viens no tiem parasti bija cienījams tirgotājs, bet otrs ``sīkāks'' kā tirdzniecības, tā tikumu ziņā. Pirmais tika izmantots galvenokārt kā padomdevējs, otrs~--- kā īpašu uzdevumu veicējs. Tomēr abi vienādi baidījās no muižnieka, kaut bieži dzīvoja uz tā rēķina. Pēc pirmās iegribas muižnieks varēja arī nopērt ``savu'' ebreju, pie tam vēl piebilstot: ``Turēsi muti~--- paliksi man kalpot, ja nē~--- paņemšu citu ebreju, bet tu man tik un tā neko nepadarīsi, jo gan policijas iecirkņa uzraugs, gan apriņķa policijas priekšnieks ir manās rokās''. Kopumā tā arī bija. Ebreji klusēja, domājot: ``Lai arī muižnieks mani var nopērt, taču es pie viņa nopelnīšu savu maizes gabalu. Arī kad nomiršu, mans dēls no viņa gūs peļņu''. Ja ebrejs nomira, muižnieks parasti pieņēma darbā sev bijušā izpalīga dēlu vai citu radinieku. Bez šī ``tirdzniecības lietu kārtotāja'' katram muižniekam miestā bija arī savi amatnieki: kurpnieks, drēbnieks, skārdnieks u.c. Tikai šim savam amatniekam tika piešķirti muižnieka pasūtījumi. Amatniekiem nopelnīt sev iztiku bija daudz grūtāk nekā tirgotājiem. Muižnieka rīcībā varēja būt arī savs ebrejs, saukts par faktoru, kurš par komisijas naudu kārtoja viņa darījumus, dzīvojot nevis pilsētā, bet muižā, kā arī ebrejs-nomnieks, kurš pārvaldīja viņam iznomāto to vai citu muižu. Ja muižniekam piederēja vairākas muižas, tad visās tajās nereti dzīvoja šādi faktori un nomnieki. Protams, ebreji augsti vērtēja savu pietuvinātību muižniekam, nevēlējās to zaudēt. Visu, kas muižniekam bija nepieciešams, viņš sagādāja ar sava ebreja starpniecību, uzskatot to par gudru, viltīgu, tomēr godīgu radījumu, jo ``savs, paša'' ebrejs taču neuzdrošināsies apkrāpt savu labvēli! Visus pārējos ebrejus muižnieki uzskatīja par negodīgiem. Ja pilsētas ebrejs atbrauca pie muižnieka, viņš vispirms gaidīja, kad kāds no vietējiem zemniekiem aizvedīs viņu pie faktora, jo pieeju muižas ēkai sargāja suņu bars. Tikai pēc tam kāds no mājas kalpotājiem atbraukušo pavadīja pie muižnieka. Ja muižnieks bija labvēlīgs, atpakaļ līdz vārtiem ebreju pavadīja sulainis, ja nē~--- viņam nācās iet vienam, mēģinot atkauties no suņiem. Kā rakstīja E.~Kotiks, to mocību apraksts, ko šādā gadījumā cieta ebreji no muižnieku suņiem, varētu saturēt daudz lappušu.

Līdz ar poļiem nācās ciest arī to sacelšanos neatbalstījušajiem citu tautu pārstāvjiem, arī latgaliešiem Vitebskas guberņā. Pret polonizāciju vērstais latīņu drukas aizliegums tika attiecināts arī uz Latgali, skarot arī latviešu valodā izdodamos darbus. 1864.~gadā izsludinātais un līdz 1904.~gadam spēkā esošais latīņu iespieddarbu aizliegums liedza iespēju Latgalē drukāt un izplatīt latviešu grāmatas. Latgaliešiem, tāpat kā poļiem, bija aizliegts iegādāties nekustāmo īpašumu bez gubernatora atļaujas utt.

Tiesa, attīstoties lietuviešu, baltkrievu un ukraiņu nacionālajām kustībām, kas pastiprināja tradicionālās reģionālās atšķirības, daļa Krievijas rietumu guberņu poļu izcelsmes šļahtas zaudēja savu poļu identitāti. Literatūrā var atrast piemērus, kur tuvi radinieki, dzimuši Krievijas Ziemeļrietumu novadā, bet kā personības veidojušies dažādos apstākļos: Viļņā, Varšavā un Pēterburgā, apzinājās sevi gan kā lietuvieši, gan poļi, gan baltkrievi.

Ukrainā jeb Krievijas Dienvidrietumu novadā (Kijevas, Podolijas un Volīnijas guberņās), kur arī dzīvoja poļi, stāvoklis tiem bija ievērojami labvēlīgāks, jo šeit kā sacelšanās kustībai bija mazāks vēriens, tā arī kā ģenerālgubernators saimniekoja daudz mērenākais ģenerālis N.~Anņenkovs.

\strong{XIX gs. pēdējā trešdaļā saimniecība Polijas karalistē} attīstījās samērā sekmīgi. 1807.~gadā pasludinātā dzimtcilvēku personīgā atbrīvošana, plašu mazzemnieku un bezzemnieku masu pastāvēšana, daļas muižnieku--šļahtiču tendence jau gadsimta pirmajā pusē pielietot kapitālistiskas saimniekošanas formas, muižu pastāvēšana, kurās jau tika pārsvarā pielietots algots darbs, radīja labvēlīgus priekšnoteikumus 1864.~gada agrārās reformas norisei.

Pirmie gadu desmiti pēc reformas sakrita ar samērā augstu labības cenu periodu pasaules tirgū. Auga arī labības pieprasījums iekšējā tirgū. Šie abi apstākļi noteica muižnieku saimniecību attīstības profilu. Tās turpināja galvenokārt ražot graudus. Muižnieku saimniecības uz kapitālistiskās saimniekošanas ceļu pārgāja pakāpeniski, bez straujām pārmaiņām ražošanas organizācijā un agrotehnikā.

Poļu slānis, kurš guva vislielāko labumu no Krievijas impērijas politikas pēc 1863.~gada sacelšanās apspiešanas, bija zemnieki. Pavisam ap 2 miljoni zemnieku (ar ģimenes locekļiem) ieguva zemes no muižnieku īpašumiem, vēl 110~000~--- no valsts un sekularizētajām zemēm. Zemnieku saimniecību skaits no ap 0,5~miljoniem 1864.~gadā pieauga līdz 0,6~miljoniem 1873.~gadā un līdz 0,78~miljoniem 1904.~gadā. Šai ziņā Polijā situācija bija daudz labāka nekā pārējā Krievijā pēc 1861.~gada reformas. Zemnieku zeme 1872.~gadā sasniedza 46\% no lauksaimniecībā izmantojamās zemes, tās īpatsvars auga arī turpmāk. Ar 1888.~gadu Zemnieku banka (\rutxti{Крестьянский поземельный банк}) izplatīja savu darbību arī Polijas karalistē, izsniedzot zemniekiem hipotekārus kredītus. Pēc L.~Vasiļevska datiem 42\% no visas zemnieku iepirktās zemes ieguva bezzemnieki. Tiesa, vidējā nopirktā zemes gabala lielums bija tikai ap 5 desetīnām (1 desetīna~--- 1,0925~ha). Trešdaļā zemnieku sētu nebija zirga, tomēr radās arī spēcīgs vidējās zemniecības slānis. Ciemos, lai mazinātu muižniecības ietekmi, tika organizētas gminas (kopienas). Ar to, lai cik paradoksāli arī tas skanētu, faktiski Krievijas apspiešanas politika palīdzēja veidoties modernajai poļu nācijai. 1899.~gadā Mehovas (\pltxti{Miechów}) apriņķī tika nodibināts pirmais lauksaimniecības artelis. Tas kļuva par paraugu citām lauksaimniecības biedrībām Polijas karalistē.

Taču pastāvēja arī virkne negatīvu parādību. Lēni auga lauksaimnieciskās tehnikas pielietojums. Piemēram, Varšavas apgabalā 1895.~gadā tikai 4\% saimniecību bija kuļmašīnas, kad Pozenes provincē Prūsijā ar tām strādāja jau 52\% saimniecību. Tikai pasaules lauksaimniecības krīze 80.--90.~gados, saistīta ar lētu graudu ievedumu no aizokeāna valstīm, paātrināja pārmaiņas, lielsaimniecību pārvēršanos par kapitālistiskiem uzņēmumiem. Taču tieši šai laikā muižnieki arī samazināja graudaugu sējumu platību un daļu muižu, kuras nedeva regulārus ienākumus, pārdeva, tai skaitā arī zemniekiem.

Lauku iedzīvotāju skaits auga tik strauji, ka pat esošo lielsaimniecību parcelācija (sadalīšana sīkākās) nevarēja novērst bezzemnieku skaita pieaugumu. 1870.~gadā bija 220~000~bezzemnieku, bet 1891.~gadā~--- jau 849~000. XX gs sākumā viņu bija ap 1,2 miljoni.

Kaut carisma ekonomiskā politika pret Polijas karalisti bija daļa no tā vispārējā kursa uz ``centra'' atbalstīšanu par ļaunu ``nomalēm'', \strong{Polijas karalistes rūpniecības attīstībā} vislielākā loma bija tam, ka tā varēja realizēt saražoto produkciju Viskrievijas tirgū. Pēc 1861.~gada dzimtbūšanas atcelšanas Krievijā strauji veidojās iekšējais tirgus, kas labvēlīgi ietekmēja arī Polijas karalisti, tās rūpniecības rajonus.

Liela loma Polijas karalistes attīstībā joprojām bija dzelzceļu celtniecībai. T.s. Lodzas fabriku dzelzceļa līnijas \rutxti{(Лодзинская фабричная железная дорога}) celtniecība (1867) zināmā mērā kompensēja to, ka agrāk uzceltā Varšavas--Vīnes līnija atstāja malā Lodzas rūpniecības rajonu. Pēc 70.~gados notikušās Terespoles (\pltxti{Terespol}) dzelzceļa savienošanas ar Maskavas-Brestas un Kijevas-Brestas dzelzceļiem Polijas karaliste ieguva sakarus ar Krievijas impērijas centrālajiem un dienvidu rajoniem. 80.~gados dzelzceļu tīkls Polijas karalistē bija visattīstītākais Krievijā. Ja 1870.~gadā Polijas karalistē visu dzelzceļu garums sastādīja 960~km, tad 1910.~gadā~--- jau 3~810~km. Sevišķi svarīga bija novada sasaiste ar Vīni (1845--1848), Pēterburgu (1862), Berlīni (1862) un Maskavu (1870). Varšavas dzelzceļa mezgls atpalika tikai no Maskavas mezgla.

XIX gs. 60--70.~gados Polijas karalistē norisa \strong{rūpniecības apvērsums}. (Metalurģijā un ogļrūpniecībā tas noslēdzās 80.~gados.) Ja 1850.~gadā Lodzā vēl strādāja tikai četras tvaika mašīnas ar 150 zirgspēku jaudu, tad jau 60.~gadu sākumā tur pastāvēja 10~fabrikas, kurās darbojās tvaika un mehāniskie dzinēji. Pēc tam, kad 60.~gados aušana tika mehanizēta, tekstilrūpniecības peļņa pieauga no 60~miljoniem rubļu 1880.~gadā uz 341~miljonu rubļu 1910.~gadā. 80.~gados ap 3/4 kokvilnas rūpniecības ražojumu izveda uz Krieviju vai caur to uz austrumu tirgiem. 1883.~gadā tekstilrūpniecību gan skāra krīze, bet tā nebija ilgstoša. 1885.~gadā tekstilrūpniecība un citas vieglās rūpniecības nozares deva vairāk nekā 50\% no visas karalistē saražotās rūpniecības produkcijas vērtības.

XIX gadsimta pēdējā desmitgadē karalistes rūpniecība mehanizācijas ziņā bija līdere Viskrievijas mērogā, bet 1867.~gadā no Varšavas, Kališas un Kelces guberņu daļām izveidotā Petrakovas guberņa (\pltxti{Gubernia Piotrkowska}) 1892.~gadā šai rādītājā ieņēma pirmo vietu, apsteidzot Pēterburgas, Jekaterinoslavas un Maskavas guberņas.

Auga rūpnieciskās ražošanas koncentrācija lielos uzņēmumos. XIX gadsimta beigās lielās rūpnīcas (kuras nodarbināja vairāk nekā 50~strādniekiem) sastādīja 31\% no visu rūpnīcu skaita, taču ražoja 88\% visas rūpnieciskās produkcijas, tajās bija koncentrēti 86\% visu strādnieku.

Kapitālistiskās saimniecības sistēmas attīstība prasīja modernu kredīta sistēmu, radās jauna tipa bankas~--- akciju sabiedrības. 1870.~gadā Varšavā tika dibināta Tirdzniecības banka, kurā tika iesaistīti lielo zemes īpašnieku kapitāli. Banka veica starpniecību graudu un cukura tirdzniecībā, piedalījās investīcijās rūpniecībā. 70.~gados radās banka Lodzā, kura apkalpoja tekstilrūpniecību un tās ražojumu eksportu uz Krieviju, arī citas.

Lai veiktu jaunas investīcijas rūpniecībā, bija nepieciešami lieli kapitāli. Tāpēc auga akciju sabiedrību veidoti uzņēmumi. XIX/XX gadsimta mijā akciju sabiedrību īpašumā esoši uzņēmumi sastādīja tikai 9\% no kopējā uzņēmumu skaita, taču ražoja vairāk nekā 50\% visas rūpniecības produkcijas. To īpatsvars saražotās produkcijas kopējā vērtībā kalnrūpniecības un metalurģiskajā rūpniecībā sastādīja 92,1\%, metālapstrādes rūpniecībā 67,8\%, cukurrūpniecībā 53,1\%.

Sakarā ar Krievijas realizēto protekcionisma politiku, raksturīgu ar augstām muitas nodevām uz no ārzemēm ievedamajām precēm, kuras citādi varētu izkonkurēt iekšzemes ražojumus, Polijas karaliste, tāpat kā Krievija kopumā, kļuva arī par ārzemju kapitāla iespiešanās objektu. XX gadsimta sākumā franču, beļģu un vācu kapitālam piederēja apmēram 1/4 no visiem karalistes metalurģiskās, ķīmiskās un tekstilrūpniecības uzņēmumiem, taču tas nodarbināja vairāk nekā 69\% no kopējā strādnieku skaita un radīja vairāk nekā 60\% rūpnieciskās produkcijas. Tā 1901./1902.~gadā ārzemju kapitālam Polijas karalistē piederēja 669~fabrikas, tas nodarbināja 37,6~tūkstošus strādnieku, radīja produkciju 272~miljonu rubļu vērtībā. Nacionālajam kapitālam piederēja 1971~fabrika, tas nodarbināja 84~tūkstošus strādnieku un ražoja produkciju 181~miljona rubļu vērtībā. Minētie dati ļauj konstatēt, ka viens ārzemju kapitālam piederošs uzņēmums vidēji ražoja produkciju 406~tūkstošu rubļu vērtībā, nodarbinot vidēji 56~strādniekus, kad vietējam kapitālam piederošs uzņēmums, nodarbinot vidēji 43~strādniekus, ražoja produkciju tikai 92~tūkstošu vērtībā. Pēc dažiem datiem ārzemju kapitāls sastādīja 39\% no karalistes rūpnieciskā kapitāla.

Tirdzniecība ar Krieviju un ārzemju, galvenokārt franču, beļģu un vācu kapitāli, investēti Polijas karalistes rūpniecībā, bija tās tautsaimniecības pamats. Tiesa, kā norādījis poļu ekonomikas vēsturnieks V.~Kula, ne visus ar vāciskiem uzvārdiem saistītos kapitālieguldījumus Polijā varēja uzskatīt par ārzemju kapitāliem. Tā, Lodzas tekstilrūpnīcu īpašnieki vācieši L.~Geiers un K.~Šeiblers savus līdzekļus bija ieguldījuši Polijā, paši tos pārvaldīja uz vietas, peļņu izmantoja ražošanas paplašināšanai, nevis izveda uz ārzemēm.

Šai sakarā jāpiemin, ka XIX gadsimta beigās~--- XX gadsimta sākumā zināmu izplatību ieguva t.s. kapitālisma Polijā attīstības ``koloniālā teorija'', kura kapitālismu uzskatīja nevis par iekšēju sociāli-ekonomisko pārmaiņu rezultātā radušos, bet svešu, ārpus Polijas dzimušu un tās augsnē ``mākslīgi'' ``pārstādītu'' parādību. To raksturojot ārzemju kapitāli, noieta tirgi ārzemēs, ārzemju izcelsmes buržuāzija un arī lielā daudzumā no ārzemēm iebraukuši strādnieki. Kā uzskatīja V.~Kula, minētajā posmā un īpaši pēc Polijas valstiskās neatkarības atjaunošanas 1918.~gadā šai teorijai bija uzdevums pierādīt, ka poļu rūpniecība ir radusies tikai pateicoties ``austrumu tirgiem'' un poļiem ir nacionāli vienoti jācīnās gan pret no ārzemēm nākušo lielburžuāziju, gan reizē par šo tirgu saglabāšanu. Pats V.~Kula marksistiski vērtēja kapitālisma veidošanos kā nesaraujami vienotu vēsturisku procesu, ko izsauc ražošanas spēku attīstība un progress sabiedriskā darba dalīšanā, iedzīvotāju noslāņošanās, jaunu kapitālismam raksturīgu šķiru~--- buržuāzijas un proletariāta rašanās, bagātības akumulācija vienā sabiedrības polā un nabadzības otrā, kapitāla uzkrāšana vienu rokās un otru pāreja algotu darbinieku kategorijā. Tomēr, atzīstot, ka komplekss tirgus ekonomikas izveides skaidrojums ir tuvāks īstenībai, iekšējie procesi bija noteicošie, nedrīkst arī vienkāršot situāciju un noliegt, ka gan poļu zemju kopumā, gan Polijas karalistes tautsaimniecība atsevišķi, nebija izolēta no apkārtējo valstu, īpaši Poliju sadalījušo, ekonomiskās attīstības, ka starptautiskās norises, arī kapitāla imports, ietekmēja poļu zemju ekonomisko attīstību.

No otras puses, Polijas karalistes rūpniecības arvien lielāka piemērošanās tirgiem austrumos padarīja to pārlieku atkarīgu no tiem. Dārgo izejvielu imports no Rietumiem palielināja poļu produkcijas pašizmaksu un neļāva tai sekmīgi konkurēt rietumu tirgos. Karalistes rūpniecība tāpēc kļuva īpaši jūtīga pret konjunktūras pasliktināšanos vai politisko attiecību izmaiņām.

Joprojām pirmo vietu rūpniecības attīstības ziņā Polijas karalistē ieņēma Lodzas (\pltxti{Łόdz}) rajons. Tajā ražoja 80\% visu kokvilnas audumu. Interesanti, ka jau 60.~gados Lodzā bija vērojams plašs katolisko poļu strādnieku pieplūdums. (XIX gadsimta beigās tajā bija nodarbināti 95 tūkstoši strādnieku.) Turpretī uzņēmēju vidū poļu bija maz. Rūpnieki galvenokārt bija vācieši. Poļi uzņēmēju Lodzā vēl pirms Pirmā pasaules kara sastādīja tikai 6,5\% un arī tie pārsvarā bija sīkīpašnieki. Vācieši šeit, balstoties uz saimniecisko spēku, varēja droši aizsargāt savu nacionālo identitāti. Kontakti dažādu tautību uzņēmēju starpā bija drīzāk izņēmums, nevis parasta parādība. Salīdzinājumā, piemēram, ar Varšavu vācu uzņēmēji Lodzā dzīvoja izolācijā. No viņiem tikai dažas ģimenes bija polonizētas. XIX gs. beigās starp Lodzas tekstiluzņēmējiem vācieši bija 49\%, ebreji~--- 45,7\%. XX gs. sākumā proporcija pat mainījās ebreju uzņēmējiem par labu. (Lodzā, tāpat kā Varšavā, radās asimilētas ebreju pilsonības slānis.) 1913.~gadā Lodzā ebreju uzņēmēju vidū bija jau 47,1\%, vāciešu 44\%. Tomēr lieluzņēmēju vidū vāciešu bija vairāk un viņi saglabāja tekstilrūpniecībā vadošo vietu. Tikai neliels skaits vāciešu ieradās Polijā jau ar vērā ņemamu kapitālu. Sekmīgai viņu darbības izvēršanai ļoti liela nozīme bija no valsts piešķirtajiem kredītiem. Krīžu laikā, kad kredīti aptrūka, tas veda pie lielām grūtībām. Tomēr XIX gs. beigās daļa uzņēmēju darbojās jau ar savu kapitālu, kļuva neatkarīgi no valsts kredītiem. Liela daļa vācu uzņēmēju tehniskās zināšanas bija jau apguvuši dzimtenē. Viņi arī Polijā uzturēja sakarus ar Vāciju, ar to Rietumu sasniegumus ienesa Polijā. Kā rakstīja poļu literatūras klasiķis V.~Reimonts, radās t.s. ``Lodzas cilvēks''~--- ``auksts, gudrs, lietišķs cilvēks, kurš bija gatavs uz visu''.

Tekstilrūpniecības attīstība, protams, bija saistīta ar metālizstrādājumu ražošanu~--- metalurģiju un mašīnrūpniecību, jo bija vajadzīgi darbgaldi, stelles u.c. iekārtas. Lodzā līdz ar tekstilrūpniecības attīstību, pilsētas iedzīvotāju skaita pieaugumu attīstījās arī pārtikas rūpniecība, papīra ražošana, tipogrāfiju darbība. Pirms Pirmā pasaules kara Lodzā bija jau ap 20~multimiljonāru ģimeņu. Vācu uzņēmēji Polijas karalistē izturējās ļoti ekspansīvi, ieguldīja savus kapitālus ne tikai savu pamatinterešu, bet arī citās saimniecības nozarēs tuvākajos Krievijas rietumu apgabalos. Taču rajonos, kur stingri dominēja krievu kapitāls~--- Aizkaukāzā, Centrālajā Āzijā, vācieši darboties negribēja, jo jau tā Lodzas uzņēmējiem nācās sīvi konkurēt ar Maskavas tekstilrūpniekiem.

Otrais pēc nozīmes Polijas karalistē bija Varšavas rūpniecības rajons, kur attīstījās dažādas rūpniecības nozares un pastāvēja gan lieli uzņēmumi, gan liels skaits sīku amatnieku darbnīcu. Tā 1895.~gadā šeit bija 44~000 rūpniecības strādnieku un 55~000 amatnieku.

Pakāpeniski arvien lielāku lomu ieguva Dombrovas ogļu baseins (mūsdienās ietilpst Silēzijas vojevodistē). XIX gadsimtā tas, atšķirībā no Augšsilēzijas, kura atradās Prūsijas sastāvā, ietilpa Polijas karalistē. Līdz pat Pirmajam pasaules karam akmeņogles tur ieguva galvenokārt daudzās sīkās, slikti aprīkotās šahtās. Tomēr, ja 1870.~gadā šeit ieguva 300~000 tonnu ogļu, tad 1890.~gadā~--- jau 3~000~000 t. XIX gs. beigās baseinā ieguva ap 4 miljonus tonnu ogļu gadā. Pirms Pirmā pasaules kara Dombrovas ogļu baseins deva jau 19\% no visas Krievijas impērijas akmeņoglēm. Kopumā Polijas karalistes ogļu ieguves rūpniecībā 19 šahtas piederēja 13 firmām, tai skaitā piecām lielām. Starp tām četras, kā arī divas mazākas pārstāvēja ārzemju kapitālu. Poļu kapitālam piederēja tikai divas firmas. 1897.~gadā sešas ogļu ieguves sabiedrības Dombrovas baseinā noslēdza sadarbības līgumu. Tām piederošajās šahtās ieguva 91\% no visām akmeņoglēm.

Neapturami samazinājās amatniecības pienesums rūpnieciskās produkcijas ražošanā. No 1855. līdz 1866.~gadam ap 25~000, no 1866. līdz 1889.~gadam vēl ap 22~000~amatnieku kļuva par rūpniecības strādniekiem. Ja 50.~gadu beigās amatnieki saražoja preces gandrīz par tādu pat summu kā manufaktūras un fabrikas, tad jau 1877.~gadā rūpnīcu un fabriku ražojumu vērtība pārsniedza amatniecības izstrādājumu vērtību vairāk nekā četras reizes.

Kaut arī agrākā feodālā cunfšu reglamentācija visur bija likvidēta, amatniecībā gan vēl saglabājās cunfšu organizāciju struktūra: māceklis~--- zellis~--- meistars. Zeļļi centās kļūt par meistariem, iekārtot savu darbnīcu, taču veiksmes gadījumi bija rets izņēmums. No 1871. līdz 1880.~gadam tikai kokvilnas audumu ražošanā iznīka ap 7~000~sīko darbnīcu. Dažas amatnieku specialitātes~--- audēju, vērpēju, kariešu izgatavotāju u.c., neizturot konkurenci ar fabrikām, vispār izzuda. Ja 1870.~gadā karalistē bija mazāk nekā 70~tūkstošu, tad 1880.~gadā Polijas karalistē strādāja jau 150~000 rūpniecības strādnieku (33\% no tiem~--- tekstilrūpniecībā). Salīdzinājumam~--- Vācijai un Austroungārijai piederošajos poļu apdzīvotajos apgabalos bija tikai 30~000 strādnieku. 1895.~gadā Polijas karalistē bija jau ap 250~tūkstošu strādnieku. Lielākā daļa viņu nāca no bezzemnieku un trūcīgo zemnieku vidus. 1897.--1901.~gadā rūpniecības strādnieku vidū gandrīz 1/4 sastādīja sievietes. Tekstilrūpniecībā viņu bija vairāk par pusi.

1890.~gadu sākumā Polijas karalistē saražoja 19,5\% visas Krievijas produkcijas (ir dati, ka pat 1/4), kaut iedzīvotāju tajā bija tikai 7\% no visiem impērijā dzīvojošajiem. Uz vienu cilvēku ražojamās produkcijas ziņā karaliste atpalika tikai no Pēterburgas un Maskavas. Ogļu un tērauda ražošanā Polija bija otrajā vietā aiz Doņeckas baseina. Polijā no visiem Krievijā ražotajiem izstrādājumiem tika izgatavots 42\% kokvilnas auduma, 26,9\% vilnas auduma, 77\% dzijas, 76\% adījumu. Lielākā daļa ražojumu (ap 70\%) tika pārdoti Krievijā. Laikā no 1880. līdz 1910.~gadam karalistes tirdzniecība ar Rietumu valstīm palielinājās divreiz, bet ar Krieviju~--- 11~reizes. Galvenokārt turp tika izvestas tekstilpreces, dzija, mašīnas, ogles. Šāds stāvoklis izraisīja lielu neapmierinātību krievu rūpniekos. Taču Polijas karaliste ne tikai izveda preces uz Krieviju, bet arī pati bija Krievijas noieta tirgus. No Krievijas karaliste importēja graudus, miltus, putraimus, 60\% nepieciešamās kokvilnas, 50\% vilnas, lielu daudzumu zīda, linu, dzijas, dzelzs rūdu no Krivoirogas (krievu \rutxti{Кривой Рог}, ukr. \uktxti{Кривий Ріг}).

Īpaši asa konkurence bija tekstilrūpniecībā un ogļrūpniecībā. Var teikt, ka risinājās Maskavas cīņa ar Lodzu. Maskaviešiem lētāk izmaksāja izejvielu piegāde, toties dārgāk kurināmais. Poļu ražotājiem bija pieejami tikai dārgāki kredīti kā krievu fabrikantiem, toties viņiem bija jāmaksā gandrīz divreiz mazāki nodokļi. Strādnieku algas Polijas fabrikās bija ievērojami augstākas nekā Krievijas centrā, kas ļāva maskaviešiem ietaupīt ievērojamus līdzekļus. Turklāt Maskavas fabrikās darba dienas ilgums bija 14 stundas, bet Polijā 12--13~stundas. Tiesa, Polijā gadā fabrikas strādāja 7~dienas vairāk. Kopumā ražošanas apstākļi bija aptuveni vienādi. No otras puses, pēc dažiem vērtējumiem, Maskavas fabrikas ievērojami atpalika darba organizācijā. Nav pamata apgalvot, ka Polijas rūpniecība varētu nest ievērojamus zaudējumus krievu ražotājiem, Polijas tirdzniecības balanss ar Krieviju pastāvīgi bija pasīvs, taču krievu publicisti un rūpnieki bieži pārspīlēja Polijas ekonomiskās izaugsmes rādītājus.

Jāuzsver, ka starp Polijas karalistes rūpniekiem, finansistiem un Rietumueiropas, kā arī Krievijas kolēģiem pastāvēja plaši sakari. Polijas karalistes kapitāls centās veikt investīcijas arī Krievijas tirgos. Piemēram, tas finansēja dzelzs ražošanu Krivoirogā un Kramatorskā (ukraiņu \uktxti{Краматорськ}~--- pilsēta Doņeckas apgabalā Ukrainā). Polijas karalistes teritorijā darbojās lielo Krievijas banku filiāles.

Lai nu kā, Polijas karaliste 19.gadsimta beigās stingri ieņēma trešo vietu ražošanas apjoma ziņā starp Krievijas rūpniecības rajoniem aiz Maskavas un Baltijas rajona. Asā konkurences cīņa ar Krievijas rūpniekiem atstāja arī jūtamu iespaidu uz poļu buržuāzijas politiskās platformas izveidi.

Salīdzinājumā ar poļu zemēm dienvidos un arī rietumos (izņemot Silēziju) Polijas karaliste bija vairāk rūpnieciski attīstīta, taču salīdzinājumā ar Rietumeiropas pirmrindas valstīm tā ievērojami atpalika.

Sekmīgā tautsaimniecības attīstība nodrošināja salīdzinoši labvēlīgu demogrāfisko situāciju. Līdz XIX~gadsimta beigām Polijas karalistē strauji pieauga iedzīvotāju skaits. Ja 1863.~gadā tajā bija 5 miljoni iedzīvotāju, tad 1897~--- 9,5 miljoni, sasniedzot 75 cilvēkus uz km$^{2}$. Ap 3/4 iedzīvotāju bija poļi, 14\% ebreji, 4,7\% krievi un ukraiņi, (Pēc 1897.~gada skaitīšanas datiem no Polijas iedzīvotājiem tikai 2,8\% bija krievi. No citiem Krievijas rajoniem viņu vēl mazāk bija tikai Somijā~--- 0,23\%) 4,4\%~--- vāciešu, 3,3\%~--- lietuviešu. Pēc konfesionālās piederības katoļu bija 75\%, judaistu~--- 14\%, protestantu~--- 6\% un pareizticīgo 5\%. Visā karalistē līdz 2/3 iedzīvotāju nodarbojās ar zemkopību, taču jau 27\% pelnīja iztiku ar tirdzniecību, amatniecību un rūpniecību. Šļahtiči sastādīja vairs tikai 1,5\% iedzīvotāju).

Tā kā 1870.--1890.~gados rūpniecība Polijā intensīvi attīstījās, strauji pieauga pilsētu iedzīvotāju skaits. Ja XIX gadsimta vidū Polijas karalistē to bija ap 16\%, tad gadsimta beigās jau 35\%. Varšavā 1858.~gadā bija 160~000, bet 1897.~gadā~--- jau 638~000 iedzīvotāju, taču Lodzā pieauguma tempi bija 2~reizes straujāki: no 40~000 līdz 315~000, kas izskaidrojams ar tās rūpniecisko attīstību.

1897.~gadā Polijas karalistē bija ap 250~000 rūpniecības strādnieku, vairāk nekā 67~000 dzelzceļnieku un ceļu būvētāju, apkalpojošā sfērā bija nodarbināti vairāk nekā 270~000 cilvēku. Varšavas metālistu un Dombrovas kalnraču vidū augsts bija kvalificēto strādnieku īpatsvars. Var teikt, ka tas viss liecināja par salīdzinoši viduvēju ekonomiskās attīstības līmeni. Strādnieku pamatmasa Polijas karalistē bija poļu tautības. Pārējo daļu veidoja ebreju un nelielā skaitā arī vācu strādnieki. Ebreju strādnieku liela daļa koncentrējās Lodzā un Belostokā, vāciešu~--- Lodzā un Dombrovas-Sosnoveckas rajonā. Toties augsts bija vāciešu procents visas karalistes rūpniecības vidējā tehniskā personāla vidū.

Turpinājās tālāka \strong{poļu nacionālās apziņas veidošanās}. Gan nepārtrauktās sakāves, ko piedzīvoja poļu nacionālisti, gan pārmaiņas Polijas karalistes sociāli-ekonomiskajos apstākļos veicināja jaunu idejisku procesu attīstību. Jāatzīmē, ka Polijas karalistes tautsaimniecības attīstībai labvēlīgā cariskās Krievijas ekonomiskā politika bija ne tikai priekšnoteikums tās ekonomiskajam uzplaukumam, bet arī poļu nacionālās pašapziņas pieaugumam, sabiedriskās aktivitātes kāpumam. Šļahtas loma mazinājās. Liela inteliģences daļa gan vēl nāca no šļahtas, bet, kā vēlāk atzina viens no poļu sociālistu (PPS) ideologiem, L.~Vasiļevskis, izteica jau rūpnieciskās buržuāzijas intereses.

Tieši ar kapitālisma attīstību poļu zemēs bija saistīta liberālā virziena attīstība. Liberāļi prasīja likvidēt feodālisma paliekas un šļahtiskās tradīcijas, aicināja visus slāņus apvienoties cīņā par nacionālajām interesēm. Taču Polijā liberālisms nespēja attīstīties tādā varenā strāvojumā, kāds tas bija valstīs, kuras izvirzījās priekšgalā kapitālisma attīstībā. Salīdzinoši vēlākā poļu zemju nostāšanās uz kapitālistiskās attīstības ceļa noteica dažādu sociālo un politisko procesu pārbīdi un sakrišanu laikā. Polijā salīdzinoši ātri attīstījās t.s. ``trešās šķiras'' (sīkpilsonības) diferenciācija, kas laupīja liberālismam to sociālo pamatu, ko šī ``trešā šķira'' varēja radīt, uzstājoties pret feodālismu kā vienots vesels. Salīdzinoši ātri norisa strādniecības un zemniecības politiskā pašnoteikšanās.

Pēc 1863.--1864.~gadu sacelšanās sakāves jaunākā poļu nacionālistu paaudze nereti deva priekšroku praktiskam darbam savas tautas labā, nevis cīņai par nacionālas valsts izveidi, kurai, kā toreiz likās, nav perspektīvu. Poļu aristokrāti Vislas novadā augstu vērtēja savas krievu aristokrātijai pielīdzinātās privilēģijas, arī poļu lielburžuāzija parasti bija konservatīva, jo ar carismu to saistīja milzīgā Krievijas tirgus izmantošanas intereses.

1876.--1881.~gadu periodu var uzskatīt par zināmu ``atkusni'' carisma antipoliskajā politikā. Krievijas varas iestādes šai laikā vairāk bija nodarbinātas ar ārpolitiskām akcijām un cīņu pret revolucionāro kustību pašā Krievijā. Poļu sabiedrībā auga cerības uz pašvaldības tiesību piešķiršanu karalistei, plaši tika apspriesta iespēja panākt samierināšanos ar cara valdību.

XIX gadsimta 70.--80.~gados mēreno poļu politiķu vidū pilsoniskās jeb buržuāziskās kustības ietvaros radās ``pozitīvisma'' (šeit~--- politiskās un tiesiskās domas virziens XIX gs. otrajā pusē, kurš par vienīgajām atzina valsts izveidotās tiesību normas) jeb tā saucamās ``organiskā darba'' (\pltxti{praca organiczna}) skolas socioloģijā piekritēji. (Šī skola bija radusies XIX gs. 40.~gados Vācijā un pēc tās uzskatiem sabiedrība ir organisms un attīstās pēc bioloģiskām likumsakarībām.) Viņu atbalstītā ``darba pie pamatiem'' (\pltxti{praca u podstaw}) un ``organiskā darba'' teorija virzīja uz tautas ekonomisko un kultūras sasniegumu tālāku attīstību, atstājot politiskās prasības tikai pašvaldību līmenī, neprasot valstiskuma atjaunošanu. Šie mērenie politiķi uzsvēra, ka konspiratīvā darbība un bruņota cīņa par nacionālo neatkarību ir kaitīga. ``Organiskā darba'' skolas programma paredzēja koncentrēšanos ekonomisko, sociālo un kultūras jautājumu risināšanai, propagandēja rūpniecības un tirdzniecības, zinātnes, tehnikas attīstību, noraidot ticību poļu mesiānismam un cerībām ko panākt ar ieročiem.

Šai sakarā var pieminēt vēsturisku anekdoti. Kad Francijas valstsvīram A.~Tjēram kāds poļu sabiedriskais darbinieks pajautājis, kas poļiem jādara, lai atkal tiktu uz augšu, viņš atbildējis: ``\frtxti{Enrichissez vous'}!'' (Kļūstiet bagāti!)

Par ļoti svarīgu tika uzskatīta apzināta darbība sabiedrības attīstībā. Liberāļi-pozitīvisti pasludināja, ka ``lai arī kādā stāvoklī būtu sabiedrība, vienmēr atradīsies sfēra, kurā tā var strādāt savā labā''. Liela loma tika piešķirta izglītībai.

Krievijā šo virzienu šai laikā propagandēja viens no narodņiku līderiem N.~Mihailovskis ar savu ``mazo darbu'' teoriju. XIX gs. 80.~gadu sākumā valstu~--- Polijas sadalītāju savstarpējās nesaskaņas, arī zināma Krievijas politikas liberalizācija veicināja domu par kompromisa ar Krieviju atrašanas nepieciešamību.

Polijā viens no pirmajiem ``organiskā darba'' sludināšanu uzsāka žurnāls ``\pltxti{Przegląd Tygodniowy}'' (''Iknedēļas Apskats''). Tas daudz vērības veltīja zemes ekonomiskajai attīstībai, propagandēja rūpniecības un tirdzniecības, zinātnes un tehnikas attīstību, uzsvēra, ka sabiedrības uzdevums ir materiālo vērtību uzkrāšana. ``Organiskā darba'' teorija noraidīja uz vardarbību virzītu ideoloģiju, atzina mierīgu, reformistisku darbību. Politikā pozitivisma ietekmē arī radās t.s. samierināšanās (\pltxti{ugoda}) programma, no šejienes šī grupējuma pārstāvjus arī sāka saukt par ugodoviešiem. Viņi centās pārliecināt cara valdību par poļu uzticamību un nepieciešamību tāpēc ieviest Polijas karalistē zemstes, pilsētu pašvaldības un zvērināto piesēdētāju tiesas, veikt citas reformas. Tiesa, nacionāli-patriotisko tradīciju aizstāvji nereti ugodoviešus uzskatīja par nodevējiem, renegātiem.

Kā atzīmējis E.~Meijers, pūliņiem veicināt visu slāņu, arī zemnieku, labklājību bija panākumi, bet tie veicināja arī lietuviešu un ukraiņu nacionālo apziņu, to, ka šīs tautas par savu ienaidnieku uzskatīja ne tik daudz Krievijas impēriju, bet gan poļu augšslāni, iepriekšējās kopības starp poļiem, lietuviešiem un ukraiņiem izjukšanu. Pēc autora domām par šādas kopības pastāvēšanu līdz XIX gadsimta beigām var runāt tikai nosacīti. Tā gan darbojās kopēju cīņu laikā pret ārējiem iebrucējiem, bet sociālajām sadursmēm starp apspiestajiem un apspiedējiem jau iepriekšējos gadsimtos bija arī nacionāla nokrāsa, jo apspiedēji lielākoties bija poļi. Toties jāpiekrīt minētā vēsturnieka konstatējumam, ka jēdziens ``Polija'' šai laikā sašaurinājās līdz poļu apdzīvotajām teritorijām. Krievijas rietumguberņu iedzīvotāji uz savām teritorijām to vairs neattiecināja.

Uz Vislas novadu lielu iespaidu atstāja izmaiņas Krievijā valdošajā dinastijā. 1881.~gada 1.~(13.)~martā, krievu revolucionārās organizācijas ``\pltxti{Нaродная воля}'' (Tautas griba) poļu-baltkrievu izcelsmes biedra I.~Griņevicka mestās bumbas sprādziena rezultātā gāja bojā Aleksandrs II. (Baltkrievu publicists A.~Tarass, padevies asinsatriebības garam un ignorējot to, ka atentātā gāja bojā 3 nevainīgi cilvēki un vēl 17 tika ievainoti, šai sakarā rakstīja, ka Aleksandrs II ``beidzot, ar savu dzīvību samaksāja par Kaļinovska, Serakovska, Trauguta un tūkstošu citu poļu un baltkrievu revolucionāru dzīvībām''.) Tronī nonāca Aleksandrs III, vara spēja pārdzīvot narodņiku izraisītos satricinājumus, tā atkal pārgāja uzbrukumā pret tiem, ko uzskatīja par valstij naidīgiem spēkiem. Krievijas valdības attieksme pret poļiem atkal pasliktinājās. 80.~gadu vidū kļuva skaidrs, ka ``organiskais darbs'' nav tik radikāls līdzeklis kā panākt uzlabojumus, kā agrāk tas šķita pozitīvistiem.

Aleksandrs III gan valdīja samērā neilgi, tomēr no revolucionāru-marksistu viedokļa raugoties, Krievijas stāvoklis, un arī viņu attieksme pret to būtiski mainījās. Ja vēl 1893.~gadā K.~Markss rakstīja, ka ``Krievijā var būt tikai tāds vai šāds dumpis, pie tam cietīs vācu tērpos ģērbtie, bet revolūcijas nekādas un nekad nebūs'', tad 1881.~gadā viņš paziņoja, ka ``Krievija ir revolucionārās kustības priekšpulks Eiropā''. Šāds vērtējums balstījās uz Aleksandra II nogalināšanas faktu, krievu teroristu skaļo darbošanos. Tika cerēts, ka cara patvaldība, kura viņaprāt bremzēja revolūcijas attīstību Eiropā, tiks iznīcināta. Mūsdienās skaidri redzams, ka secinājums, pamatots nevis ar strādniecības augšanu, tās ``idejisko briedumu'', kura svarīgumu tik ļoti uzsvēra marksisti, bet gan ar teroristu organizācijas īslaicīgu veiksmi, bija utopisks. Taču to uztvēra K.~Marksa sekotāji Krievijā, kuri sāka runāt par XIX gadsimta beigās notikušo ``pasaules revolucionārās kustības centra pārvietošanos uz Krieviju''. Līdz ar revolucionāro pretrunu nobriešanu Krievijā Polija nevarēja nezaudēt savu izcilo vietu Eiropas revolucionārajā kustībā.

Imperatora Nikolaja II nākšana Krievijas un Polijas tronī (1894) izsauca poļu cerību atdzimšanu uz valdošā režīma izmaiņām. Šai pat gadā notikusī vecā Varšavas ģenerālgubernatora J.~Gurko nomaiņa arī pastiprināja samierinātāju cerības. Daudziem šķita, ka rusifikācijas smagums jau ir aiz muguras. Tika organizēta virkne deputāciju uz Pēterburgu, lai panāktu pārmaiņas Polijas karalistē. Viena no lielākajām savu pavalstniecisko jūtu demonstrēšanas akcijām notika 1897.~gadā, kad Nikolajs II apmeklēja Varšavu. Rusofīlie elementi organizēja krāšņu viņa sagaidīšanu. Nikolajs II ziedoja 1 miljonu rubļu Varšavas tehniskās augstskolas dibināšanai, atļāva uzcelt pieminekli nacionālajam dzejniekam Ā.~Mickēvičam. Krievija uzņēma 30~gadus atpakaļ pārtrauktos kontaktus ar Vatikānu. Taču poļu ugodoviešu cerības neattaisnojās, jau 1898.~gadā atkal tika atjaunota stingra izturēšanās pret ``polonismu''.

Tā poļu historiogrāfija, kas balstās tikai uz aktīvo cīņas formu pret apspiedējiem slavināšanu, līdz šai dienai pozitīvisma teorijas valdīšanas periodu sabiedrībā raksturo kā ``sociālo marasmu''. Arī padomju vēsturnieki uzsvēra, ka Polijā, kur kapitālisma attīstība bija sākusies vēlāk nekā Rietumos, toties tā attīstība notika straujāk, tāpēc arī šķiru cīņa stimulēja buržuāzijas ātrāku nobriešanu un tā, nepaguvusi līdz galam izveidoties, neieguva to revolucionāro potenciālu, kāds bija pilsonībai Rietumos buržuāzisko revolūciju periodā. Tāpēc arī poļu pilsonība nevarējusi vadīt buržuāziski-demokrātisko kustību Polijā. Tā, latviešu vēsturnieks Indulis Ronis rakstīja, ka pozitīvisma programma ``atspoguļoja valdošo šķiru centienus pilnā mērā izmantot tās ekonomiskās izmaiņas un iespējas, ko pavēra kapitālisma attīstība, un ar reformu palīdzību buržuāziski pārveidot sabiedrību. Tādēļ, no vienas puses, pozitīvisti kritizēja agrāko šļahtas ideoloģiju, tās feodālās privilēģijas, bet, no otras~--- bija klaji naidīgi politiskajai, bet it īpaši revolucionārajai cīņai''. Mūsdienās nākas konstatēt, ka šī pieeja bija dogmatiska, mākslīgi nodalot revolucionāro cīņu no cīņas par reformām, cīņu par pilsonības ekonomiskajām interesēm no cīņas par demokrātiju. Atkarībā no vēsturiskajiem apstākļiem arī poļu buržuāzijas interešu priekšplānā izvirzījās dažādi uzdevumi. Kā rādīja vēlākie XX~gadsimta notikumi, tā spēja uzstāties arī kā demokrātisks spēks.

XIX gadsimta pēdējā trešdaļā pieauga \strong{sasprindzinājums attiecībās starp Vislas novada iedzīvotāju vairākumu~--- poļiem un nacionālajiem mazākumiem}. Poļu zemēs dzīvojošajās nepoļu tautu pārstāvji sāka paust savu nacionālismu

Polijas karalistē dzīvojošie \strong{vācu} zemnieki jau no XIX gadsimta 30.~gadiem izrādīja vēlmi izceļot no Polijas uz blakus esošo Krievijas Volīniju (ukraiņu \uktxti{Волинь}, poļu \pltxti{Wołyń}, mūsdienās novads aptver vairākus apgabalus Ukrainā un Ļubļinas vojevodistē Polijā), jo apkārtējie poļi izturējās pret tiem naidīgi. Kad Vācijā 1871.~gadā nodibinājās vienota impērija un tās kanclers O~.f.~Bismarks tajā realizēja pret poļiem naidīgu ģermanizācijas politiku Vācijas austrumu provincēs, pasliktinājās poļu attieksme pret vietējiem vāciešiem arī t.s. ``kongresa Polijā''. Vācu ieceļotāji tika uzlūkoti kā zemei naidīgi. Tika runāts par vācu tieksmi iespiesties austrumos, realizēt ``\detxti{Drang nach Osten}'' (dziņu uz austrumiem). 1880.~gada kāds poļu žurnālists rakstīja: ``Viss jau būtu labi, ja vācieši dzīvotu poļu ciemos kā vieninieki un drīz, kā tas notiek pilsētās, kļūtu par poļiem. Taču viņi apmetas noslēgtos savos ciematos, tāpēc ir no tiem jābrīdina''. Poļi vāciešus sauca par ``kartupeļu rijējiem'', švābiem, prūšiem. Galu galā vācu vidū radās sakāmvārds ``\detxti{Solange die Welt eine Welt wird sein, wird der Deutsche nicht des Polen Bruder sein}'' (Kamēr pasaule pastāvēs, vācietis polim nekad nebūs brālis). Vācieši, acīmredzot augstākas izglītības, darba tikuma dēļ, bija veiksmīgāki. Poļi tāpēc viņus uztvēra kā visu nelaimju cēloni. Tas atspoguļojās arī literatūrā, piemēram, B.~Prusa garstāstā ``Priekšpostenis'' (``\pltxti{Placowka}'', 1885.~gads), kurā rakstnieks stāstīja par poļu sadursmēm ar vāciešiem. Tomēr daļa poļu inteliģences spēja pacelties pāri nacionālajiem aizspriedumiem. 1901.~gadā B.~Pruss avīzē ``\pltxti{Gazeta Polska}'' (``Poļu Avīze'') rakstīja: ``Ar vācu tautu mums vienmēr bijušas labākās attiecības. No viņiem pārņēmām mēs gotisko stilu būvniecībā, kokgriešanu, lielu daudzumu dažādu iekārtu, trauku, amatnieku instrumentu, lielu daudzumu zinātnes atziņu, amatu, tirdzniecību, daudzas parašas, organizācijas formas \citespace{} Nekautrēsimies no patiesības: šai cēlajai tautai mums jāpateicas par lielāko mūsu civilizācijas daļu. Kā pretdevumu tam mūsu vidū apmetušies vācu iedzīvotāji izmanto īpašas privilēģijas un laucinieki lielus atvieglojumus. Kā viņi jūtas mūsu vidū, rāda tas, ka simti tūkstošu vāciešu brīvprātīgi bez kādas piespiešanas pieņēmuši mūsu tautību~--- un pateiksim to skaļi~--- dāvinājuši mums labākos strādniekus un cienījamākos pilsoņus. Mūsu zeme ir kļuvusi viņiem laba māte un viņi~--- tai labi dēli.'' Tiesa gan, jāatzīst, ka rakstnieks vairāk rādīja ideālu, kādām būtu jābūt savstarpējām poļu un vācu attiecībām, bet dzīvē tās parasti nebūt tādas nebija.

Polijas karalistē poļu vairākuma vidū dzīvojošie ukraiņi un baltkrievi izrādīja tendenci pārorientēties no poļu vērtību sistēmas, ko viņi bija uztvēruši Žečpospolitas laikā, uz krievu kultūru, taču pilnībā to izdarīt nespēja. Uz poļu nacionālajiem uzdevumiem orientētās politiskās organizācijas un darbinieki neuzskatīja mazākumtautas par potenciāliem sabiedrotajiem, ignorēja tās, un uz lietuviešu, ukraiņu un baltkrievu nacionālo kustību skatījās kā uz ``krievu'' vai ``vācu'' iecerētu pret poļiem vērstu intrigu. Lietuviešu, ukraiņu, baltkrievu un ebreju nacionālismam bija katram sava pozīcija, kas kopīgu cīņu kā pret Krievijas impērijas, tā poļu muižniecības apspiešanu padarīja neiespējamu.

Ar Polijas dalīšanu Krievijas sastāvā nonāca daudz \strong{ebreju}. Krievijā ebreji ilgi palika izolēta sabiedriska grupa un, kā uzskatīja pazīstamais ebreju vēstures pētnieks S.~Dubnovs, par Krievijas ebrejiem var runāt tikai no XIX gs. 70.~gadiem. Līdz tam tos pareizāk bija saukt par ``Polijas ebrejiem'', jo vēsturiskās saknes viņiem saistījās ar Poliju.

Tā kā jūdaisma normas noliedza dzimstības regulēšanu un salīdzinājumā ar citiem apkārtējiem iedzīvotājiem ebrejiem bija augstākas higiēnas normas, viņu skaits strauji pieauga. 1914.~gadā Krievijas impērijā dzīvoja apmēram 5,250~miljonu ebreju (puse no visiem ebrejiem pasaulē).

Agrākajā Polijas--Lietuvas valstī ebreji bija kapitālistisko tendenču nesēji, tieši tāpēc, lai viņi veicinātu saimniecības attīstību, viņus uz Poliju aicināja karaļi. Ebreji parasti dzīvoja savos \emph{geto} (no itāļu \ittxti{ghetto}~--- pilsētas daļa, kurā viduslaikos nometināja pie svešas reliģijas piederīgus iedzīvotājus, galvenokārt ebrejus), runāja apkārtējiem nesaprotamā valodā~--- \emph{jidišā} (no vācu \detxti{jüdisch}~--- ģermāņu valodu grupas valoda, ko stipri ietekmējušas citas valodas un kurā runāja ebreji Centrālajā un Austrumu Eiropā), valkāja savu tradicionālo apģērbu, kurš faktiski bija ap XVI gadsimtu lietotais poļu apģērbs, tikai poļi to nomainīja, ebreji~--- nē. Ebreju vīrieši nēsāja bārdas un vaigu bārdas, sievietes turpretī skuva galvas un valkāja parūkas. Loti aktīvi ebreji bija tirdzniecībā~--- gan vairumā, gan mazumā, uzstājās kā starpnieki. Viņi bieži darbojās kā muižu pārvaldnieki, pilnībā aizvietojot šļahtičus, nomāja lauksaimniecības objektus, vietējos uzņēmumus, īpaši krogus, faktiski kontrolēja sīko un vidējo kredītu izsniegšanas jomu~--- t.i.~--- nodarbojās ar augļošanu. Nav brīnums, ka šāda viņu darbība izsauca apkārtējo kristīto iedzīvotāju sašutumu. Taču līdz Polijas dalīšanas laikam ebreju ietekme saimniecībā bija jūtami mazinājusies. Īpaši finanšu jomā ebreji nespēja izturēt bagāto muižnieku un klosteru konkurenci. Tomēr priekšstati par ebrejiem kā apkārtējo kristīto, sevišķi zemnieku, ekspluatatoriem izrādījās visai noturīgi.

Krievijas sastāvā ebreju stāvoklis ievērojami mainījās. Katrīna II aizliedza kā aizskarošu lietot vārdu ``žīds'', izdeva likumus, kas pielīdzināja ebrejus tiesībās citiem pilsētu iedzīvotājiem, taču realizēt tās ebrejiem neizdevās, jo tam pretojās vietējie iedzīvotāji, īpaši poļu muižnieki. Nereti viņus vienkārši nepielaida pašvaldību vēlēšanās. Varas iestādes nespēja panākt likuma izpildi. Tā pati Katrīna II, krievu tirgotāju, kuri nevēlējās ciest ebreju konkurenci, spiediena rezultātā 1791.~gadā noteica Krievijā ``ebreju apmešanās zonu''. Ebrejiem, kas kļuva par Krievijas impērijas pavalstniekiem, bija atļauts apmesties tikai viņiem atvēlētajā nometinājuma zonā gar valsts rietumu robežu. Vēlākajos gados tā tika gan paplašināta, gan sašaurināta. XIX gs. beigās tajā ietilpa 15 guberņas: Besarābijas, Viļņas, Vitebskas, Volīnijas, Grodņas, Jekaterinoslavas, Kauņas, Minskas, Mogiļevas, Podoļskas, Poltavas, Taurijas, Hersonas, Čerņigovas un Kijevas. Taču likumi bieži tika pārkāpti un lielākajās Krievijas pilsētās sāka veidoties ebreju kopienas, it sevišķi Sanktpēterburgā, Maskavā, Kijevā un Odesā. XIX otrajā pusē ebreju migrācijas apjoms ievērojami pieauga. Zināmā mērā to var izskaidrot ar pieaugošo demogrāfisko spiedienu, kā arī pastāvīgajiem modernizācijas un urbanizācijas procesiem. Uz Krievijas iekšējiem rajoniem devās arī ebreji no Polijas.

Krievijas varas iestāžu realizētajā politikā apkārtējo iedzīvotāju ``aizsardzība'' no ebreju ekspluatācijas kļuva par vienu no stūrakmeņiem. Ebreji bija ``grēkāzis'', uz ko novelt vainu par grūto dzīvi. Bieži tika izdoti rīkojumi ar aizliegumiem viņiem apmesties pilsētās un miestos, par to izsūtīšanu no laukiem, ierobežojumiem nodarboties ar tādu vai citu saimniecisko darbību. Ebreji atbilstoši Krievijas likumiem nevarēja iegādāties zemi ārpus pilsētas robežām. Tā kā ebreju skaits strauji auga, valdības politikas rezultātā drīz viņu pamatmasa nonāca nabadzībā. XX gadsimta sākumā ebreju amatnieka ienākumi vidēji bija 1,5 līdz 2~reizes zemāki kā zemnieka ienākumi (attiecīgi 150--300 un 400--500 rbļ.).

Bez tam Krievijas varas iestādes par savu uzdevumu uzskatīja cīnīties pret ``ebreju fanātismu'', kurš izpaužoties sevi uzskatot par Dieva izredzētu tautu, nicinot apkārt dzīvojošos kristiešus, jo ebreji dzīvojot noslēgti no tiem, neparādot lojalitāti pret pastāvošo valsts varu, savu reliģisko normu izpildi uzskatot par svarīgāku nekā valsts likumu ievērošanu! (Tiešām, draudzības saites ebrejiem ar kristiešiem veidojās reti. Piemēram, ebrejs nevarēja doties viesos pie kristieša, jo tā mājā ēdienu gatavoja, neievērojot ebreju reliģiskās prasības.) Savus uzplūdus cīņa pret ``ebreju fanātismu'' sasniedza Nikolaja I valdīšanas laikā. Krievijā no 1827.~gadā uz ebrejiem tika attiecināts arī rekrūšu došanas pienākums. Pie tam viņiem iesaukšanas normas bija noteiktas trīs reizes lielākas nekā no citām iedzīvotāju grupām. Varas iestādes karaklausības ieviešanu ebrejiem un viņu dienestu armijā izmantoja, lai piespiestu ebrejus kristīties. Tikai 1874.~gadā Krievijā tika ieviesta vispārējā kara klausība.

Krievu liberāļi uzskatīja, ka ebrejiem vispirms ir jādod tādas pat tiesības kā visiem citiem, tad viņi ``labosies'', turpretī konservatori domāja, ka vispirms ir jāpanāk ebreju ``labošanās'' un tikai tad varēs viņiem dot vienlīdzīgas tiesības. Asimilācijas tendences XIX gadsimtā aptvēra arī ebrejus, kā rezultātā radās ebreju izcelsmes, bet kristīti ārsti un advokāti, žurnālisti, augstskolu profesori, aktieri un režisori. Tas savukārt saasināja citu tautību nepatiku pret viņiem. Ja agrāk krieviem, poļiem u.c. nācās konkurēt ar ebreju veikalniekiem, krodzniekiem vai rentniekiem, tagad jutās apdraudēti arī brīvo profesiju pārstāvji. Tāpēc tika aizmirsts, ka juridiski tie vairs nav ebreji, bet kristieši.

Ebrejiem Krievijā bija piešķirtas zināmas pašpārvaldes tiesības, kopiena jeb \emph{kagals} (\rutxti{кахал}), tās vēlētā pārstāvniecība ievāca kopīgus nodokļus, t.s. kastes ievākumu (\rutxti{коробочный сбор}), jo savāktos līdzekļus glabāja kastē, ko uzraudzīja kopienas pilnvarots cilvēks), regulēja saimniecisko darbību, lai ierobežotu savstarpējo konkurenci, centās ierobežot patēriņu, greznību.

Varas iestādes centās ierobežot ebreju izglītošanās iespējas, taču par spīti visiem šķēršļiem, viņi pēc tās tiecās arvien vairāk. Aleksandra II laikā (60.--70.~gados) tika izdoti vairāki likumi, ar kuriem ārpus apmešanās zonas atļāva dzīvot jūdu ticības iedzīvotājiem ar augstāko izglītību, kā arī dantistiem, farmaceitiem, feldšeriem, pēc tam amatniekiem, bijušajiem karavīriem. Ebreji ieguva tiesības iestāties valsts dienestā, piedalīties pilsētu un zemstu pašvaldību darbā. Tomēr nelīdztiesība, smagie dzīves apstākļi noveda pie tā, ka jau 70.~gados arvien vairāk ebreju iesaistījās revolucionārajā kustībā. Ja 1871.--1873.~par politiskām lietām izmeklēšanā nonākošo ebreju skaits sastādīja 4--5\%, kas atbilda viņu skaitam iedzīvotāju vidū, tad XIX gs. 80.~gadu beigās viņi jau sastādīja 35--40\% visu revolucionāru.

Lūzuma moments iestājās 1881.~gadā, kad sakarā ar cara Aleksandra II nogalināšanu (1881.~gada 1.~martā) Krievijā izveidojās nestabila politiskā situācija. Izplatījās baumas, ka cara slepkavas esot ebreji un varas iestādes devušas netiešus norādījumus visur sarīkot ebreju grautiņus. Ebreju iedzīvotāju nelīdztiesība, antisemītiski noskaņojumi un daudzu iedzīvotāju neapmierinātība ar ebreju konkurenci veicināja šādu baumu plašu izplatību. Vietējās administrācijas, policijas un karaspēka pasivitāte it kā apliecināja baumu pamatotību. Pirmais tāds grautiņš notika Jeļizavetgradā (tagad Kropivņicka (\uktxti{Кропивницький})) Ukrainas dienvidos 15-16.~aprīlī, pēc tam turpinājās citur Krievijas impērijas rietumu un Dienvidrietumu apgabalos. 13.~decembrī notika ebreju grautiņš Varšavā, ko bija inspirējušas Krievijas varas iestādes. Grautiņā piedalījās arī poļu sabiedrības apakšslāņi. Inteliģence to nosodīja, katoļu garīdznieki devās ielās, cenšoties pierunāt grautiņu dalībniekus izklīst. Tikai trešajā dienā karaspēks apspieda nekārtības. Bija sagrauts ap pusotra tūkstoša ebreju dzīvokļu, citu telpu, ievainoti 24 cilvēki. Atsevišķas vietās ebreju grautiņi turpinājās līdz pat 1884.~gadam. Valdība gan mēģināja tos novērst, gan arī reizē centās atbildību par tiem uzkraut pašiem ebrejiem.

1887.~gadā tika ieviesta t.s. ``procentu norma'' viņu iestājai augstākajās un vidējās mācību iestādēs~--- 10\% ebreju apmešanās zonā, 5\%~--- ārpus tās, 3\%~--- galvaspilsētās. Atsevišķās vietās attieksmi pret ebrejiem noteica augstākās varas personas. Piemēram, kad par Maskavas ģenerālgubernatoru 1891.~gadā kļuva lielkņazs Sergejs Aleksandrovčs, no pilsētas un guberņas izsūtīja ap 20~tūkstošu ebreju amatnieku un pat Nikolaja I laika karavīru.

Kā raksta poļu vēsturnieks J.~Tazbirs, poļu--ebreju attiecības saasinājās XIX gadsimta beigās, jo poļu apdzīvotajās teritorijās ieceļoja daudz t.s. ļitvaku (no jidiša \hetxti{ליטװאַק}, lietuviešu~--- \lttxti{litvakai}~--- ebreji, kuri dzīvoja Baltkrievijas, Lietuvas, Latvijas, dažos Krievijas un Polijas rajonos). Viņi tikuši izsūtīti no Krievijas iekšienes, slikti zinājuši poļu valodu un atradušies krievu kultūras lokā. Poļi tos uztvēruši kā rusifikācijas īstenotājus, kuri bremzējuši asimilācijas procesus, kas norisinājušies Polijas ebreju vidū. Tas novedis pie tā, ka ne tikai poļu tirgotāju un amatnieku, bet arī inteliģentu vidū atbalsi guvuši antisemītiski lozungi. Domājams, ka pret šo vērtējumu jāizturas kritiski. Vajadzēja pacensties, lai ebrejus, cietušus no antisemītisma izpausmēm Krievijā, attēlotu kā rusifikācijas īstenotājus. Acīmredzot darbojās tā pati vien konkurence starp poļu un ebreju amatniekiem, tirgotājiem, brīvo profesiju pārstāvjiem. Protams, poļu antisemīti to nevarēja atzīt.

Nabadzības, nevienlīdzības un arī grautiņu rezultātā sākās plašs ebreju emigrācijas vilnis, galvenokārt uz ASV. Līdz Pirmajam pasaules karam no Krievijas impērijas emigrēja vairāk nekā 2~miljoni ebreju. Grautiņu radītais šoks, ierobežojumi izsauca ideju par ebreju atgriešanos Palestīnā. Gan visā Krievijas impērijā, gan Polijas karalistē atbalstu guva cionisma kustība, kas bija vērsta uz ebreju valsts Izraēlas izveidi un nostiprināšanu un ebreju nācijas attīstību.

Daži teikumi jāpasaka arī par \strong{ārpus etniskās Polijas} palikušajiem bijušajiem Polijas-Lietuvas apgabaliem.

\strong{Lietuviešu} apdzīvotās teritorijas pēc Polijas trešās dalīšanas nonāca Krievijas impērijas sastāvā. Vārdu ``Lietuva'' tagadējās Lietuvas valsts teritorijas apzīmēšanai sāka lietot tikai XIX gadsimta otrajā pusē. Pēc nošķiršanas no Žečpospolitas lietuvieši daudz mazākā mērā tika pakļauti polonizācijai, viņos sāka veidoties sava nacionālā apziņa. Arī \strong{ukraiņu} nacionālajā inteliģencē Krievijā XIX gadsimta vidū veidojās sava nacionālā pašapziņa. Vēsturnieks N.~Kostomarovs, kurš daudz nodarbojās ar Ukrainas vēsturi, publicēja rakstu ``Divas krievu tautības'' (``\rutxti{Две русские народности}''), kura nosaukums jau izsaka tā saturu. 1905.~gadā Krievijas Zinātņu akadēmija oficiāli atzina ukraiņu valodas esamību. Trešās ``krievu tautības''~--- \strong{baltkrievu} nacionālās pašapziņas veidošanās un viņu kā atsevišķas ``tautības'' atzīšana norisa krietni lēnāk. Baltkrievu iedzīvotāju vairākums neapzinājās savu nacionālo identitāti, uz jautājumiem par to deva sekojošas atbildes: ``tautība~--- ``\pltxti{tutejšij}'', ticība~--- ``\pltxti{viera polska}'', valoda~--- ``\pltxti{jazik prostij}''. Vietējās baltkrievu apdzīvoto apgabalu inteliģences prātos cīkstējās poļu un t.s. ``rietum-krieviskuma'' (``\rutxti{западно-руссизм}'') pašapziņa. Saskaņā ar otro baltkrievi bija tikai viena no krievu tautas etnogrāfiskajām grupām. Baltkrievu tautas patstāvības ideja ieguva skanējumu tikai XIX gadsimta 80.~gados Pēterburgā, kur bija studenti no baltkrievu novadiem. Taču vēl vairākus gadu desmitus šīs idejas aizstāvjiem nācās pierādīt tās tiesības uz eksistenci. Un tomēr jēdziens ``baltkrievi'', lai arī vēl tīri etnogrāfiskā nozīmē, pakāpeniski iespiedās masu apziņā. 1897.~gada Krievijas iedzīvotāju skaitīšanā 74\% no tās Ziemeļ-Rietumu apgabala iedzīvotājiem par savu dzimto atzina baltkrievu valodu.

Tādejādi tajos bijušajos Žečpospolitas, bet vēlāk Krievijas apgabalos, kuri neietilpa Polijas karalistē, kaut dažādiem tempiem, bet tomēr attīstījās agrāk poļiem pakļauto tautu pašapziņa, kas arvien vairāk apdraudēja poļu cerības apvienot šīs teritorijas atdzimušā Žežpospolitā. Ja XVIII gadsimta beigās t.s. [Dņepras] Labā krasta Ukrainā poļi sastādīja valdošā slāņa vairākumu, tad XIX gadsimtā, īpaši pēc abu poļu sacelšanos sakāvēm, viņi bija zaudējuši savas pozīcijas, ar nostaļģiju atceroties Žečpospolitas ``labos laikus''. Taču kādu nopietnu cerību tos atgriezt viņiem nebija. Ārpus Polijas karalistes esošajos baltkrievu apgabalos dzīvojošo poļu vidū pastāvēja divas tendences. Vieni pirmajā vietā izvirzīja visa novada un tajā dzīvojošo tautu intereses, etniskās nācijas vietā izvirzot nācijas kā politiskas vienības kategoriju, uzsverot lietuviešu un baltkrievu apdzīvoto zemju atšķirību no etniskās Polijas. Otrie uzsvēra poļu nacionālo uzdevumu prioritāti.

XIX gadsimta beigās Polijā sāka veidoties \strong{politiskās partijas} mūsdienu izpratnē (ar vairāk vai mazāk izteiktām partijas pazīmēm~--- savu programmu, statūtiem, vadošajām iestādēm un vietējām organizācijām, savu propagandas aparātu un piekritēju kustību).

Līdz ar kapitālisma attīstību Polijā iezīmējās galvenie politiskās attīstības virzieni, atspoguļojoši ne tikai šķirisko diferenciāciju, raksturīgu tirgus saimniecībai, bet arī sabiedrības uzbūves sarežģītību, daudzslāņainību, kurā saglabājās vēl pirmskapitālistiskās struktūras. Neizdzēšamu iespaidu uz Polijā radušos politisko partiju un sabiedrisko strāvojumu veidolu atstāja nacionālā jautājuma specifika, kā arī zemi sadalījušo monarhiju iekšējā kārtība, kura pilnībā vai daļēji liedza poļu partijām likumdošanas un citas parlamentārās un pat vispār legālās darbības iespējas.

Pārmaiņas ekonomiskajā un sabiedriskajā dzīvē, galvenokārt no Rietumiem nākošu ideju izplatība, tai skaitā augstāk izklāstītās ``pozitīvisma'' un ``organiskā darba'' idejas, ietekmēja Polijā divu politisku strāvojumu veidošanos, kam, tāpat kā citās Eiropas valstīs, bija visievērojamākā loma mūsdienu vēsturē. Tie bija nacionālisms un sociālisms. Abi aptvēra visus trīs poļu reģionus dažādās valstīs, saglabājot savstarpējos kā intelektuālos, tā zināmā mērā arī organizatoriskos sakarus.

60.~gadu beigās~--- 70.~gadu sākumā notika pirmās poļu \strong{strādnieku šķiras} uzstāšanās. Rūpniecības centri Lodza, Varšava kļuva par strādnieku streiku arēnu. Polijas agrīnās strādnieku kustības (kā plebejiski-nacionālistiskās, tā arī revolucionāri-sociālistiskās, jeb t.s. ``proletāriskās'') kodolu veidoja bijušie amatnieki un no izputējušajiem šļahtičiem nākušie strādnieki. Polijā taču dzimtbūšana bija atcelta ātrāk nekā Krievijā un kapitālisms attīstījās straujāk. Jau XIX gs. vidū eksistēja ievērojams bezzemnieku slānis, kuri, dodoties uz pilsētu, papildināja fabriku strādnieku skaitu. No šļahtičiem nākušie strādnieki saglabāja kādreiz līdz smieklīgumam nonākošu šļahtičiem raksturīgu savas pašapziņas sajūtu, taču tā bieži arī veda viņus pie varonīgas rīcības. Šie strādnieki nevēlējās ciest pazemojumus no ekspluatatoriem, bija gatavi dumpoties. Poļu strādniekos deklasētie šļahtiči ienesa iepriekšējā posma atbrīvošanās cīņas tradīcijas, it īpaši nacionālajā jautājumā.

\strong{Poļu sociālistiskais virziens} veidojās ciešā saistībā ar krievu narodņikiem, iegūstot piekritējus poļu studentu vidū Krievijas augstskolās. Tai pat laikā Varšavas studentu rokās nonāca sociālistiskās brošūras no Rietumiem. Poļu sociālistiskā kustība izauga uz Rietumu un pirmkārt Vācijas sociālistiskās kustības pieredzes pamata. Šis sociālisms jau bija orientēts uz strādniekiem. Pirmo poļu sociālistu darbība izvērsās augsnē, ko jau bija sagatavojuši dažādi poļu sabiedriskās kustības virzieni 1830.--1831.~gada un 1863.--1864.~gada nacionālo sacelšanos laikmetā. Kā uzsver poļu vēsturnieki, poļu sociālisti smēla iedvesmu arī ``pozitīvismā''.

Strādnieku kustība Polijā attīstījās jau no XIX gs. 70.~gadiem, kad plaši streiki notika Varšavā, Lodzā u.c. pilsētās. Pirmais poļu studentu sociālistiskais pulciņš izveidojās ārpus Polijas~--- Pēterburgā 1874.~gadā. 1876.~gadā arī Varšavā tika organizēts pirmais ``tīri sociālistiskais'' pulciņš. 1877.~gadā tika izdota pirmā sociālistiskā brošūra poļu valodā, paredzēta izplatīšanai strādnieku vidū. 1878.~gadā notika pirmās strādnieku sapulces, kas lika pamatus vēlāk dibinātajām strādnieku organizācijām, sākās organizācijas statūtu un programmas projekta izstrāde. Krievijas narodņiku ietekmē pirmajos poļu sociālistiskajos pulciņos bija izplatīti teroristiski noskaņojumi. Tika gatavots atentāts pret ģenerālgubernatoru grāfu P.~Kocebu, organizācijas vadītājiem ar grūtībām izdevās atturēt aktīvistus no šī soļa. 1880.~gadā strādnieki nogalināja organizācijā iekļuvušu provokatoru. Pielietoja arī ``papīra'' (ekonomisko) teroru, kad rūpniecības uzņēmumu īpašniekiem nosūtīja prasības paaugstināt darba algu, uzlabot izturēšanos pret strādniekiem, pretējā gadījumā draudot iznīcināt kā uzņēmumus, tā pašus tā īpašniekus. Tomēr izskanēja arī marksistiski secinājumi. Tā, 1880.~gada 29.~novembrī starptautiskā mītiņā Ženēvā L.~Variņskis kā pirmais poļu sociālists atzīmēja, ka ``Krievija vairs nav reakcijas balsts'', ka sociālistu priekšā stāv nevis slāvu federācijas ideja vai jautājums par poļu valsts robežām, bet cīņa par ``visas pasaules proletariātam kopīgu ekspluatējamo solidaritāti cīņā pret apspiedējiem.'' 1881.~gada oktobrī starptautiskajā sociālistu kongresā Kurā (\svtxti{Kur}, Šveice) L.~Variņskis paziņoja, ka poļu proletariāta uzdevums ir nevis nacionālas sacelšanās gatavošana, bet sociāla apvērsuma sagatavošanā, kura gaitā tiks likvidēta arī nacionālā apspiestība. Tiesa, L.~Variņskis uzskatīja, ka sociālismam vispār ``nav nekā kopīga ar nacionālo jautājumu\dots{}''

1881.~gadā, protestējot pret bezdarbu, Varšavas strādnieki izgāja pirmajā manifestācijā. 1882.~gadā L.~Variņskis nodibināja pirmo Starptautisko sociāli revolucionāro partiju \strong{``Proletariāts''} (\pltxti{Międzynarodowa Socjalno-Rewolucyjna Partia ``Proletariat''}). Partijai izveidojās piekritēju pulciņi arī Pozenes un Krakovas novados, tai izdevās izdot pirmo nelegālo avīze, kas tā arī saucās~--- ``\pltxti{Proletariat}'', arī virkni brošūru un uzsaukumu. Izveidojās sakari starp partiju un krievu revolucionāru~--- narodovoļcu (``\rutxti{Народная воля}''~--- ``Tautas griba'') organizāciju. Partijas programmā tika atzīmēta strādnieku un zemnieku interešu sakritība, uzsvērta nepieciešamība solidarizēties ar visiem apspiestajiem, neatkarīgi no viņu tautības. Nacionālais jautājums programmā bija skatīts visai šauri~--- tikai kā visu tautību un konfesiju piederīgo tiesību nolīdzināšana.

1883--1885.~gadā notika plaši aresti, četri revolucionāri 1886.~gadā tika pakārti, L.~Variņski ieslodzīja Šliselburgas cietoksnī, kur viņš nomira. Partija zaudēja nozīmību. 1888.~gadā to mēģināja atjaunot. Jaunā partija iegāja vēsturē ar nosaukumu \strong{``Proletariāts II''} (\pltxti{Socjalno-Rewolucyjna Partia ``Proletariat'' II)}. Tam pievienojās arī poļu sociālistu grupa Rīgā. Pilsētā nelegālā tipogrāfijā tika drukātas sociālistiska satura brošūras, kuras izmatoja propagandai ne tikai Polijas karalistē, bet arī citur Krievijā. Taču atšķirībā no L.~Variņska paaudzes sociālistiem ``Proletariāts II'' darbinieku vidū valdošais bija uzskats, ka reiz jau patvaldība Krievijā ir tikusi galā ar revolucionāro spēku spiedienu un pārvarējusi pagaidu krīzi, poļu sociālistu uzdevums ir nevis tieša sociālistiskās revolūcijas sagatavošana, bet cīņa par Konstitūciju Krievijā un Polijas karalistes autonomiju Krievijas valsts sastāvā. Par visiedarbīgāko politiskās cīņas līdzekli ``Proletariāts II'' uzskatīja teroru, jo neredzēja citu iespēju kā izsaukt masu kustību valdošā despotisma apstākļos. Terora koncepcija bija pārņemta no krievu narodovoļciem. Terora lomas izcelšana neļāva pareizi novērtēt ekonomiskās cīņas nepieciešamību, lai iesaistītu strādniekus masu kustībā. Kaut kopumā šai laikā sociālistu idejiski-politiskā ietekme uz Varšavas sabiedrisko dzīvi bija neliela, ``Proletariāts II'' tomēr nodrošināja poļu sociālistiskās kustības nepārtrauktību gados pēc ``Proletariāta'' sagrāves.

1889.~gadā krāsotājs J.~Marhļevskis, žurnālists Ā.~Varskis, virkne Varšavas studentu un strādnieku nodibināja Poļu strādnieku savienību (\pltxti{Związku Robotników Polskich}). Par savas darbības paraugu tā uzskatīja Vācijas sociāldemokrātiju. No ``Proletariāta II'' savienība atšķīrās ar uzsvērti negatīvu attieksmi pret terora taktiku visās tā izpausmēs. 1890.~gadā Varšavas strādnieki pirmie Krievijas impērijā atzīmēja darbaļaužu starptautiskās solidaritātes dienu 1.~maiju, 1892.~gadā Lodzā notika pirmais Polijā vispārējais politiskais streiks. Tomēr tikai 1892.--1893.~gadā radās sociālistiskas politiskas organizācijas šā vārda pilnā nozīmē.

Sociāli politisko procesu specifika 90.~gados, liberālā un demokrātiskā virziena vājums poļu pilsoniskajās organizācijās, nacionālā jautājuma asums ietekmēja sociālistiskās kustības attīstību. Tajā izveidojās divas strāvas. Viena pirmajā vietā izvirzīja šķiriskos, otrā~--- nacionālos uzdevumus.

1893.~gadā, apvienojoties ``Proletariātam II'' un Poļu strādnieku savienībai, nodibinājās Polijas karalistes sociāldemokrātija (\pltxti{Socjaldemokracja Królestwa Polskiego}), kura 1900.~gadā tika pārdēvēta par \strong{Polijas karalistes un Lietuvas sociāldemokrātiju} (PKunLSD, \pltxti{Socialdemokracja Krolestwa Polskiego i Litwy}). Tās vadītāji bija jau minētais J.~Marhļevskis un R.~Luksemburga. J.~Marhļevskis 1896.~gadā beidza Cīrihes universitāti, aizstāvēja disertāciju ``Fiziokrātisms vecajā Polijā'', kurā pirmo reizi poļu historiogrāfijā demonstrēja marksistisku pieeju tā laika poļu sabiedrības problēmām. R.~Luksemburga arī studēja Cīrihē, aizstāvēja disertāciju ``Rūpniecības attīstība Polijā''. Vēl partijā darbojās tādi pazīstami revolucionāri kā J.~Tiška, F.~Dzeržinskis u.c.

Minētie sociālisti stadijā, kad tiem vēl nebija plašas vispārdemokrātiskas platformas, nespēja izstrādāt arī pozitīvu programmu nacionālajā jautājumā. Partija pirmajā vietā stādīja sociālos uzdevumus, bet nacionālas valsts izveidi uzskatīja par šķērsli ceļā uz sociālismu. Partijas rindās pārsvaru guva uzskati, ka atsevišķas Polijas daļas jau ir tik cieši saistītās ar Poliju sadalījušajām valstīm, ka vienotas Polijas valsts izveide nav vairs iespējama. Idejiski partija lielā mērā atradās R.~Luksemburgas ietekmē, kura 90.~gados noformulēja teoriju par Polijas ekonomisko attīstību kā poļu sociāldemokrātu politiskās programmas pamatu. Viņa aizstāvēja uzskatu, ka rūpniecība Polijas karalistē pastāv un attīstās tikai pateicoties politiskajai piederībai Krievijai, kapitālisma attīstības tendence ir arvien ciešāka dažādu teritoriju (šai gadījumā Polijas karalistes un Krievijas) ekonomiska saplūšana, bet nacionālu valstu (piemēram, Polijas) izveide ir pretrunā ar kapitālistiskās sabiedrības attīstības likumiem. ``Polijas neatkarība,~--- rakstīja R.~Luksemburga,~--- bija līdzkapitālistiskā, šļahtiskā, naturālās saimniecības perioda ideāls''. Poļu strādnieki, pēc viņas domām, nedrīkstēja iekļaut savās prasībās Polijas neatkarību, jo tad viņi kļūtu par ekonomiskā progresa pretiniekiem. Tāpat kā kapitālisti, arī strādnieki varēja apvienoties ārpus valstu un nāciju robežām, strādnieki bija tendēti uz šķiru, nevis nacionālo atbrīvošanas kustību. Nacionālo problēmu loku viņa ierobežoja tikai ar kultūras sfēru. Tā šaura, nedialektiska pieeja ekonomiskās attīstības likumu traktēšanai noveda šo marksisma teorētiķi pie nacionālā jautājuma nepamatotas ignorēšanas. R.~Luksemburga un viņas sekotāji, vedot principiālu cīņu pret nacionālismu, pieļāva kļūdu, uzstājoties pret neatkarības lozungu, kurš Polijas politiskās dzīves konkrētajos apstākļos XIX/XX gadsimta mijā viņai likās par poļu-krievu revolucionāras savienības, tautu brīvības un sociālisma pretstatu. Kļūdaina bija prasība, lai krievu sociāldemokrāti atteiktos no prasības par nāciju pašnoteikšanās tiesībām, arī poļu tiesībām uz valstisku neatkarību. Aizraušanās ar cīņu pret nacionālismu noveda pie strādnieku šķiras cīņas internacionālā un nacionālā aspektu pareiza samēra zaudēšanas. R.~Luksemburgas principiāla kļūda bija arī Polijas karalistes rūpniecības pastāvēšanas ārpus Krievijas iespēju noliegšana.

Kā pirmais R.~Luksemburgas kļūdas vēsturiskajos uzskatos kritizēja jau bijušais partijas ``Proletariāts'' darbinieks lietuviešu revolucionārs L.~Janovičs. Citi marksisti, tai skaitā V.~Ļeņins, šai laikā atzina, ka labākos apstākļus kapitālisma attīstībai rada tieši nacionāla valsts. Carisma ekonomiskā politika virknē gadījumu traucēja Polijas rūpniecības attīstību, arī savu laiku pārdzīvojusī politiskā iekārta bremzēja poļu tautas progresu. Taču PKuLSD atteicās no nacionālās neatkarības lozunga izvirzīšanas, uzskatot, ka poļu proletariātam ir jāapvieno savi spēki ar citu zemju proletariātu un šī kopīgi realizētā sociālā revolūcija atrisinās visas strādnieku šķiras problēmas. Ar to tika noliegts nāciju pašnoteikšanās demokrātiskais saturs. Galvenokārt jautājumā par nāciju pašnoteikšanās tiesībām radās domstarpības ar krievu sociāldemokrātiem, tāpēc arī PKunLSD atteicās apvienoties ar Krievijas Sociāldemokrātisko strādnieku partiju. 1893.--1895.~gadā Polijas karalistes sociāldemokrātija cieta smagus zaudējumus arestos.

Kā pretspēks internacionāli orientētām sociālistu organizācijām no dažādām sociālistiskām grupām 1892.~gadā Parīzē nodibinājās Poļu sociālistu ārzemju savienība (\pltxti{Związek Zagraniczny Socjalistów Polskich}), kura pasludināja uzticību 1863.~gada sacelšanās tradīcijām, stādīja uzdevumu organizēt sacelšanos pret Krieviju. Attiecībā pret Vāciju un Austroungāriju tāds uzdevums netika stādīts. No savienības 1893.~gadā jau noformējās \strong{Polijas Sociālistiskā partija} (\pltxti{Polska Partia Socjalistyczna}) jeb \strong{PPS}. Tā lielu uzmanību veltīja nacionālajam jautājumam. Rietumu sociālistu izvirzītie uzdevumi partijas programmā bija apvienoti ar tieksmi pēc nacionālās neatkarības. Partijas radītāji ticēja, ka kapitālisma progress radīs spēcīgu strādnieku šķiru, kura varēs cīnīties kā par privātīpašuma likvidāciju, tā par nacionālo neatkarību. PPS nepieciešamo sociālo reformu īstenošanu saistīja ar neatkarīgas demokrātiskas Polijas Republikas izveidi. Pēc PPS ideologu uzskatiem par neatkarīgu Polijas Republiku bija jācīnās vēl līdz sociālistiskajai revolūcijai. Neatkarīgajai valstij bija jārada poļu strādnieku šķirai labvēlīgi apstākļi nākotnes cīņai par sociālistiskas iekārtas nodibināšanu. Polijas neatkarības idejas aizstāvji balstījās uz pesimistisku Krievijas revolucionārās kustības perspektīvu novērtējumu. Viņu žurnāls ``\pltxti{Przedswit}'' (``Rītausma'') 1894.~gadā rakstīja, ka ``nopietnai revolucionārai kustībai Krievijā izredžu nav''. Ja krievu sociāldemokrāti dažādu tautu nacionālajā kustībā saskatīja palīgspēku savu sociālistisko mērķu labā, šie poļu sociālisti Krievijas revolucionārajā kustībā redzēja līdzekli, izmantojamu savu nacionālo mērķu sasniegšanai.

1896.~gadā PPS uzstāja, lai II Internacionāles IV kongress Londonā iekļautu Polijas neatkarības prasību savā programmā. PPS atsaucās uz marksisma pamatlicēju atbalstu Polijas neatkarībai. Tomēr kongress šai prasībai nepakļāvās, reizē atzīstot visu nāciju pašnoteikšanās tiesības, bet arī aicinot visu nāciju proletariātu uz internacionālu vienotību. Taču PPS turpināja aizstāvēt Polijas izņēmuma stāvokli, 1897.~gadā savā IV kongresā izstrādāja līguma noteikumus ar Krievijas sociālistisko partiju (kuras nodibināšana bija paredzama nākotnē), kuru būtība bija tā, ka Krievijas sociālistiem bija jāizvirza Polijas neatkarības prasība un aktīvi jāaģitē krievu vidū par šādas prasības izpildi. Faktiski partijas pozīcijā bija guvuši izpausmi jau novecojušie uzskati par poļu jautājuma izcilo, izņēmuma lomu starptautiskajā revolucionārajā kustībā. Kad nākamajā~--- 1898.~gadā notika Krievijas Sociāldemokrātiskās Strādnieku partijas dibināšanas kongress, PSS pārstāvji uz to netika lūgti, jo krievu sociāldemokrātu grupas, kas organizēja kongresu, PPS neuzskatīja par sociāldemokrātisku.

Der sīkāk pateikt par vienu no partijas vadītājiem \strong{J.~Pilsudski}. J.~Pilsudskis bija dzimis senā, bet nabadzīgā šļahtiču (viņa mātes pūrā gan bija vairāk nekā 10~000 ha zemes, bet tēva novatoriskie saimnieciskie pasākumi praksē visi beidzās neveiksmīgi un par iztikas avotu kļuva banku kredīti) ģimenē Zalavā (toreiz Krievijā, tagad Lietuvā), netālu no Viļņas. Ģimenē valdīja poļu sacelšanos varoņu kults. Jau tajos gados jaunā šļahtiča apziņā notika carisma pielīdzināšana krievu tautai, kurai tika piestādīts rēķins par patvaldības pāridarījumiem poļu tautai. Vēlāk viņš rakstīja: ``Tai laikā visi mani sapņi koncentrējās ap sacelšanos un bruņotu cīņu pret moskaļiem, kurus es nīdu ar visu dvēseli, uzskatot katru no viņiem par nelieti un zagli.'' Ar laiku antikrieviskums kļuva par vienu no viņa politiskās programmas galvenajiem postulātiem. 1885.~gadā Juzefs beidza Viļņas ģimnāziju, uzsāka medicīnas studijas Harkovas universitātē, piedalījās kādā studentu politiskā demonstrācijā, par ko dažas dienas pavadīja arestā. 1887.~gadā, 20 gadu vecumā, saistībā ar cara Aleksandra III slepkavības mēģinājumu (Juzefs sava vecākā brāļa Broņislava \emph{uzdevumā pārnesa gatavojamās bumbas sastāvdaļas, pats nezinot par to}) jaunieti arestēja un izsūtīja uz 5 gadiem uz Austrumsibīriju. (Starp citu krievu politologs S.~Čerņahovskis ir pievērsis uzmanību tam, ka gan V.~Ļeņina brālis, gan J.~Pilsudska brālis piedalījās atentātā pret Aleksandru III, bet neviens nepievēršot uzmanīgu tam, ka V.~Ļeņina brālis Aleksandrs tika sodīts ar nāvi, bet J.~Pilsudska brālis tikai izsūtīts. Papildinot teikto, gan jāpasaka, ka Juzefa Pilsudska brālim Broņislavam, tāpat kā A.~Uļjanovam un virknei citu, tika piespriests nāves sods, bet imperators, apstiprinot spriedumu, nomainīja Broņislavam nāves sodu ar 15 gadiem katorgas darbos Sahalīnā.)

Tuvāka iepazīšanās ar krievu tautas pārstāvjiem neizsauca J.~Pilsudskī cieņas un labvēlības jūtas pret tiem. Lūk, viņa vārdi: ``Visi viņi ir vairāk vai mazāk slēpti imperiālisti. Viņu vidū ir daudz anarhistu, taču, dīvaina lieta, republikāņus viņu vidū es nemaz nesastapu.'' Spriežot pēc šī izteiciena, J.~Pilsudska sociāli politiskā orientācija bija visai miglaina. Grūti saprast kā var būt par anarhistu, kurš neatzīst kādu personību, grupu vai šķiru varu pār tautu, tai pat laikā esot imperiālistam. Pēc atgriešanās Viļņā 1892.~gadā viņš loloja plānus patstāvīgi nokārtot universitātes gala pārbaudījumus juridiskajās zinātnēs (Tādi precedenti bija. Piemēram, 1891.~gadā pašmācības ceļā jurista licenci ieguva V.~Uļjanovs-Ļeņins, taču J.~Pilsudskim pietrūka uzņēmības savus plānus īstenot.) Šajā laikā J.~Pilsudskis iepazinās ar savu nākamo dzīves biedri M.~Juškeviču, par kuras labvēlību viņam nācās cīnīties arī pret R.~Dmovski, vēlāko nacionāldemokrātiskās partijas līderi. Sāncensība, kura vēlāk apauga leģendām, iespējams, ietekmēja arī abu vēlākās politiskās attiecības.

Veidojoties PPS, tajā iekļāvās arī J.~Pilsudskis. Publicistikā ir izteikts viedoklis, ka notika tas nevis viņa pasaules uzskata attīstības, bet gan aizraušanās ar savu nākamo dzīves biedri rezultātā. (Var atzīmēt, ka ļoti daudzi revolucionāri par tādiem ir kļuvuši sabiedrisku un personīgu cēloņu mijiedarbības rezultātā.) 1899.~gadā viņš salaulājās ar M.~Juškeviču. Tā kā viņa bija šķirtene, laulības nevarēja notikt katoļu baznīcā. Abi nolēma pieņemt protestantismu. Tikai 1916.~gadā J.~Pilsudskis atgriezās katoļu baznīcā. (Kad viņš jau neatkarīgajā Polijā ieņēma augstākos valsts amatus, par šo atteikšanās faktu no katoļu reliģijas bija nolaists klusēšanas plīvurs, jo katoliskajā poļu sabiedrībā tas nevarēja stiprināt politiķa popularitāti. Tikai 1934.~gadā fakts kļuva zināms atklātībai.) 1900.~gadā J.~Pilsudskis atkal tika arestēts, ieslodzīts Varšavas Citadelē. Viņš simulēja garīgi slimo, tika pārvests uz psihiatrisko slimnīcu Pēterburgā, no kuras izbēga ar ārsta, PPS biedra V.~Mazurkeviča palīdzību.

Jau no darbības partijā sākotnes J.~Pilsudskis aktīvi propagandēja savus uzskatus. Tā 1893.~gadā žurnālā ``\pltxti{Przedswit}'' viņš pārmeta krievu sociālistiem to, ka viņi neuzskatīja Lietuvu, Baltkrieviju un Ukrainu par Polijas daļu, apgalvoja, ka krievu sociālisti nevēlas vairāk attīstītās Polijas atdalīšanos no Krievijas, jo ``tas nobremzētu sociālistiskās kustības attīstību Krievijā. Tika apgalvots, ka tur, kur darbojas PPS grupas (bet par savu darbības sfēru PPS uzskatīja visu XVIII gadsimta Žečpospolitas teritoriju), krievu revolucionāri drīkst darboties tikai ar poļu grupu piekrišanu, un ``to kontrolē''.

J.~Pilsudskis aktīvi propagandēja cīņu pret carisko Krieviju, daudz spēku veltot nelegāla laikraksta ``\pltxti{Robotnik}'' (``Strādnieks'') izdošanai. 1895.~gadā viņš izteica pārliecību, ka ``poļu strādnieku šķira, kura jau tagad sagādā carismam ne mazums raižu, vedīs aiz sevis cīņā citu paverdzināto tautu strādājošo masas un, gūstot atbalstu no pašas Krievijas revolucionārās kustības, vedīs tās uz uzvaru, kas nodrošinās brīvību un atbrīvošanos ne tikai poļiem, bet visiem pārējiem, kuri cietuši no carisma.'' Turpmāk gan viņa uzskatu sistēmā uzsvari pārvietojās. Poļu strādnieku šķiras intereses tika atbīdītas nacionālo interešu vārdā, proletariāts viņam bija tikai līdzeklis poļu nācijas atbrīvošanai. Lai sasniegtu šo mērķi, gūtu tam sabiedrības atbalstu, J.~Pilsudskis un viņa sabiedrotie bija spiesti izvirzīt sociālistiskus mērķus, kaut paši nopietni centās panākt tikai nacionālas buržuāziski demokrātiskas republikas izveidi. Krievijas revolucionārā kustība vairs netika vērtēta kā uzticams ``citu paverdzināto tautu strādājošo masu'' sabiedrotais. J.~Pilsudska rakstos PPS presē ne vienreiz vien atkārtojās tēze, ka, kamēr Polija ietilpst Krievijas impērijā, tā nevar normāli attīstīties, jo krievu tautas stāv uz ievērojami zemākas kulturālās attīstības pakāpes nekā poļi. Krievijas dzīves apstākļi vienmēr traucēs gan pašas krievu tautas, gan it īpaši poļu virzību uz priekšu. Kā primārais uzdevums tika izvirzīta poļu nacionālās atbrīvošanās cīņa. Sabiedriskiem pārkārtojumiem bija jānoris vēlāk, jau neatkarīgajā Polijā. Ar strādnieku šķiru J.~Pilsudski saistīja nevis sociālo tiesību izcīnīšanas programma vai vēlme mainīt sabiedrisko iekārtu, bet cerība, ka proletariāta pieaugošo spēku izdosies izmantot cīņā par Polijas neatkarību. Saviem draugiem sociālistiem J.~Pilsudskis esot pateicis: ``Mēs kādu laiku varam braukt vienā vilcienā. Taču pieturvietā ``Polijas neatkarība'' es izkāpšu''. Kā K.~Markss strādnieku šķirā saskatīja līdzekli, ar ko panākt sociālisma ideālu īstenošanu, tā J.~Pilsudskis tajā redzēja līdzekli poļu nacionālo interešu sasniegšanai. Kā atzīmējis padomju autors A.~Manusevičs, 90.~gadu strādnieku streiku izvēršanās, marksisma plašas izplatīšanās apstākļos nevarēja pat domāt par strādnieku šķiras iesaistīšanu cīņā par Polijas neatkarību, nesavienojot nacionālās neatkarības ideju ar cīņu par demokrātiju un sociālismu. Proletariāta tieksmi izcīnīt sociālismu J.~Pilsudskis u.c. PPS līderi vēlējās izmantot carisma jūga gāšanai, agrākajai Žečpospolitai līdzīga valstiskuma radīšanai ``brīvas federācijas'' veidā. Šis mērķis jau sākotnēji bija pretrunīgs, tā kā saistīja taisnīgo poļu tautas cīņu par neatkarību ar lietuviešu, baltkrievu, ukraiņu apdzīvoto zemju iekļaušanu nākamās Polijas sastāvā kaut vai federācijas veidā. Ar to Polijas brīvība tika pretstatīta Lietuvas, Baltkrievijas, Ukrainas tautu tiesībām uz pašnoteikšanos, sava valstiskuma iedibināšanu.

Viens no J.~Pilsudska līdzgaitniekiem V.~Jodko-Narkevičs žurnālā ``\pltxti{Przedswit}'' rakstā ``Etapi'' centās pierādīt, ka, ja agrāk~--- pirmajā poļu strādnieku kustības etapā partija ``Proletariāts'' centās panākt vienīgi revolūcijas uzvaru pār carismu, tad tagad poļu sociālisti iet tālāk, cenšoties apvienot visas poļu zemes. Taču runa negāja par poļu apdzīvoto un Krievijas, Vācijas un Austroungārijas pakļauto teritoriju apvienošanu, bet gan par Krievijā ietilpstošo, kādreiz Žečpospolitas sastāvā esošo lietuviešu, baltkrievu, ukraiņu u.c. tautu apdzīvoto zemju iekļaušanu Polijā. Mērķa sasniegšanai tika ieteikts censties vienoties ar sociālistiskajām organizācijām tais tautās, ``kuras sastāda ar poļiem vienu zemi''. Poļu sociālistu uzdevums bija panākt dažādo Krievijas tautu opozicionāro grupu vidū ``separātisku tendenču'' attīstību. Žurnāls ``\pltxti{Przedswit}'' rakstīja, ka lietuvieši un latvieši, tāpat kā ukraiņi ``tikai savienībā ar poļu sociālistiem un zem viņu programmas karoga varētu kļūt par ievērojamu spēku''.

XX~gadsimta sākumā J.~Pilsudskis jau mēģināja arī latviešu revolucionārus pārliecināt par nepieciešamību cīnīties ne tikai par sociālismu un autonomiju Krievijas sastāvā, bet arī par pilnīgu neatkarību. Piemēram, Miķelis Valters atzina poļu sociālistu un paša J.~Pilsudska ietekmi uz savu uzskatu veidošanos.

XIX gadsimta beigās Vācijas sociāldemokrātu teorētiskajā žurnālā ``\detxti{Die Neue Zeit}'' (``Jaunais laiks'') iedegās diskusija par Polijas jautājuma vietu starptautiskā proletariāta cīņā. Pārstāvēti bija trīs uzskati. PPS aizstāvēja viedokli, ka Polijas jautājums XIX gadsimta otrajā pusē nav mainījies salīdzinājumā ar iepriekšējo laikmetu un starptautiskās strādnieku kustības gandrīz vai pats svarīgākais uzdevums ir cīņa par Polijas neatkarību. Polijas karalistes sociāldemokrātijas pārstāvji krita citā galējībā un, sekojot R.~Luksemburgai, apgalvoja, ka Polijas neatkarības prasība ir utopiska, neizpildāma, ir pretrunā ar proletariāta sociālistiskajām interesēm un poļu zemju ekonomiskās attīstības interesēm. Visbeidzot, savu viedokli pauda arī Vācijas sociāldemokrātu atzītais teorētiķis K.~Kautskis. Viņš uzskatīja, ka sakarā ar spēcīgas revolucionārās kustības attīstību Krievijā Polijas jautājums neapšaubāmi ir zaudējis savu izcilo starptautisko nozīmi, taču tas nenozīmē, ka poļu sociāldemokrāti nedrīkst izvirzīt Polijas neatkarības prasību. Šim viedoklim piekrita arī toreiz viens no jaunākajiem krievu sociāldemokrātu līderiem V.~Leņins.

Poļu sociālisti cerēja arī uz starptautiskiem satricinājumiem savu nacionālo mērķu īstenošanai. PPS vadība ar J.~Pilsudski priekšgalā uzskatīja, ka Eiropu aptveroša konflikta rezultātā var rasties labvēlīgi apstākļi kārtējās poļu sacelšanās realizācijai.

Zīmīga bija PPS pozīcija jautājumā par Krētas iedzīvotāju sacelšanos pret Turcijas jūgu (1896--1897), kurai juta līdzi lielākā daļa eiropiešu. Arī Polijas karalistes sociāldemokrātija izteica atbalstu sacelšanās dalībniekiem. Turpretī PPS uzskatīja, ka Turcijas novājināšanās var būt izdevīga Krievijai, uzstājās par atbalstu Turcijai cīņā pret Krētas tautas nacionālās atbrīvošanās cīnītājiem. Žurnāls ``\pltxti{Przedswit}'' centās pārliecināt lasītājus, ka ziņas par turku zvērībām Krētā, Armēnijā, slāvu zemēs esot pārspīlētas, ka musulmaņus raksturojot liels iecietīgums, bet sultāna valsts kristieši ``esot zemākā attīstības līmenī'' nekā turku iedzīvotāji, ka jebkāda palīdzība krētiešu sacelšanās kustībai esot palīdzība Krievijai. ``Turcija ne ar ko neapdraud demokrātiju Eiropā, bet Krievija ir tās ienaidnieks un sliktākais no visiem ienaidniekiem'',~--- apgalvoja ``\pltxti{Przedswit}''.

PPS ārpolitisko raksturoja tās attieksme pret angļu-būru karu (1899—1902). Vārdos deklarējot, ka PPS piekrīt vispārējai līdzjūtībai pret būriem, žurnāls ``\pltxti{Przedswit}'' tālāk paziņoja, ka angļu strādnieku dzīves līmenis tik tālu pārsniedz Transvālas darbaļaužu dzīves līmeni, ka to interesēs būtu savas zemes pārvēršana par angļu koloniju. Taču poļiem nav svarīgi tas, cik slikti vai labi dzīvo cilvēki Dienvidāfrikā, jo ``mūsu galvenais un nesamierināmais ienaidnieks ir krievu carisms \citespace{} viss kas to novājina, atvieglo mūsu cīņu \citespace{} Kaitējumam Krievijai ir jābūt par revolucionārā proletariāta starptautiskās politikas izejas punktu''. Pēc žurnāla domām PPS nevar ``vēlēties tāda nemazsvarīga Krievijas pretinieka, kā Anglija, novājināšanu \citespace{} un mēs nekādā gadījumā nevaram pievienoties skarbajam Anglijas ienaidnieku un būru draugu korim''. Acīmredzot šādu pozīciju jāvērtē kā aklu nacionālu egoismu.

PPS mazākums, neatsakoties no Polijas neatkarības idejas, vairāk cerēja uz Eiropas mēroga sociālo revolūciju, kā arī uz liberalizāciju Poliju sadalījušajās valstīs. Kā norāda krievu pētnieki I.~Jažborovska un N.~Buharins, jau no 1900.~gada PPS rindās kristalizējās kreisas grupas, kuras gan arvien tika no partijas izspiestas.

Kā jau minēts, Polijas karalistes poļu sociālistu organizatoriskajam piemēram sekoja arī Prūsijas un Galīcijas sociālistiski orientētie poļi, nodibinot Galīcijas un Cešinas Silēzijas poļu sociāldemokrātisko partiju (\pltxti{Polska Partia Socjalno-Demokratyczna Galicji i Śląska Cieszyńskiego}) un Poļu sociālistisko partiju Prūsijā (poļu \pltxti{Polska Partia Socjalistyczna Zaboru Pruskiego}, vācu \detxti{Polnische Sozialistische Partei in Preußen}). Tās abas izvirzīja lozungus par strādnieku stāvokļa uzlabošanu, demokrātiskām reformām, apvienotu un neatkarīgu Poliju, taču abas partijas bija diezgan vājas. Abas bija daudz tuvākas PPS nekā Polijas karalistes sociāldemokrātijai.

Īpaši jāatzīmē, ka arī Polijas ebreju vidū izveidojās sava sociālistiska partija. Ebreju inteliģenci, sīkpilsonību, strādniecību blakus cionismam saistīja arī sociālima mācība. 1897.~gadā izveidojās Vispārējā Ebreju strādnieku savienība Lietuvā, Polijā un Krievijā (jidišā \hetxti{אַלגעמיין ייִדיש לייבער יוניאַן פון ליטע, פוילן און רוסלאַנד}, poļu \pltxti{Powszechny Żydowski Związek Robotniczy na Litwie, w Polsce i Rosji}, krievu \rutxti{Всеобщий еврейский рабочий союз в Литве, Польше и России}), ko saīsināti sauca par Bundu (savienību). Partija iestājās par demokrātiju un ražošanas līdzekļu sabiedriskošanu. Marksismu Bunda biedri traktēja, savienojot to ar tradicionālajiem priekšstatiem par ebreju tautas īpašo misiju, kaismīgi aizstāvot ideju par ebreju proletāriešu, kā vairāk par citiem apspiestiem, īpašajām vajadzībām. Tāpēc arī tam bija jādibina sava nacionāla partija, nevis jāapvienojas kopā ar citu tautību pārstāvjiem. Nacionālajā jautājumā Bunds aizstāvēja nacionāli kulturālo autonomiju Austrumeiropas ebrejiem, laicīgu izglītību jidiša valodā. Bunda biedri ticēja, ka tādā veidā ebreji varēs saglabāt savu identitāti, nevis asimilēsies. Bundisti arī bija antireliģiski noskaņoti un noliedza nepieciešamību ebrejiem doties uz Palestīnu, ko aizstāvēja cionisti. 1898.~gadā Bunds piedalījās Krievijas Sociāldemokrātiskās strādnieku partijas I kongresa sagatavošanā un iegāja šai partijā kā organizācija autonoma jautājumos, kas skar ebreju proletariātu.

90.~gados blakus iepriekšējai poļu emigrācijai Eiropā, parādījās jauna~--- sociāldemokrātiska.

1889.~gadā uz ārzemēm bēga R.~Luksemburga, nākamajā gadā J.~Tiška. 1892.~gadā poļu sociāldemokrātu grupā, kas grupējās ap R.~Luksemburgu, iekļāvās A.~Varskis. 1893.~gadā Cīrihē ieradās J.~Marhļevskis. Grupa sāka uzstāties ar programmatiskiem paziņojumiem, polemizēja ar PPS ideologiem. Reizē katorgā un izsūtījumā Sibīrijā dažādos gados atradās tūkstoši poļu revolucionāru. Tā, no 1897. līdz 1912.~gadam F.~Dzeržinskis tika arestēts sešas reizes. Vienpadsmit gadu viņš atradās cietumā un katorgā, trīs reizes bija izsūtīts, tai skaitā uz Sibīriju. B.~Veselovskis no 46 savas dzīves gadiem ap 20 pavadīja cietumos un izsūtījumā. No 1900. līdz 1917.~gadam septiņas reizes arestēts bija J.~Unšlihts. Četras reizes arestēts, bet 1906.~gadā izsūtīts uz Turuhanskas novadu bija E.~Pruhņaks. Ap 20~gadu Sibīrijā sabija turp izsūtītais F.~Kons. Jāsaka gan, ka mūsu priekšstati par dzīvi izsūtījumā bieži neatbilst patiesībai. Piemēram, atrodoties izsūtījumā Jeņisejas guberņas Šušenskas ciemā, nākamais krievu boļševiku vadonis V.~Ļeņins ar bisi rokās dienām ilgi pastaigājās pa mežiem un laukiem kopā ar poļu revolucionāru J.~Prominski.

Strādnieku un sociālistiskās kustības panākumus tieši Polijas karalistē salīdzinājumā ar poļu zemēm Vācijā un Austroungārijā padomju vēsturnieki saistīja ne tikai ar karalistes straujāku ekonomisko attīstību, bet arī ar jau minēto marksistisko ideju par XIX gadsimta beigās notikušo ``pasaules revolucionārās kustības centra pārvietošanos uz Krieviju''. Līdz ar to poļu strādnieku kustība ieguvusi varenu sabiedroto~--- krievu strādnieku kustību, kura radījusi varenu marksistisku partiju ar revolucionāru teoriju. Nenoliedzami, ka savijoties Krievijas politiskajai atpalicībai un straujajai ekonomiskajai attīstībai XIX~gs. beigās, pretrunas tajā izpaudās asāk nekā Vācijā un Austroungārijā. Tas ietekmēja situāciju arī Polijas karalistē, strādnieku kustību tajā. Taču nevajadzētu arī aizmirst to, ka tieši Polijas karalistē, salīdzinājumā ar Vācijas un Austroungārijas valdījumā esošajām poļu zemēm, bija vismazāk demokrātijas, arī nacionālā apspiestība sasniedza visaugstāko līmeni. Tāpēc cīņa par nacionālajām interesēm Polijas karalistē norisa ar vislielāko sasprindzinājumu, prasīja vislielākos upurus. 1909.~gada rudenī V.~Ļeņins kategoriski rakstīja, ka ``Polijas brīvība nav iespējama bez Krievijas brīvības''. Ja raugās tikai uz Polijas karalistes XIX gadsimta vēsturi, secinājums atbilda patiesībai, taču plašākā skatījumā visas Polijas brīvība nebija iespējama bez visu triju to sadalījušo monarhiju atbrīvošanās no feodālisma paliekām un politiskas demokratizācijas.

\strong{Nacionālistiskais virziens} mūsdienu politiskas kustības izpratnē poļu zemēs sāka formēties vēlāk nekā sociālistiskais. Acīmredzot tas izskaidrojams ar nacionālo interešu aizstāvju labāko materiālo stāvokli, kas kavēja to politiskās aktivitātes izvēršanos, ar draudiem turīgajiem cīņas gaitā par saviem mērķiem zaudēt ne tikai brīvību (vai pat dzīvību), bet arī mantu, kuras parasti nebija cīnītājiem par sociālo taisnīgumu.

Nacionālisms kļuva par poļu buržuāzijas opozīcijas formu svešzemju valsts varai un galveno tās idejiski politisko virzienu. Tas guva atbalsi arī citos sabiedrības slāņos kā reakcija uz nacionālajiem spaidiem.

Daži poļu vēsturnieki uzsvēruši, ka arī pēc poļu sacelšanās apspiešanas, neraugoties uz grūtībām, ko izsauca īpašumu konfiskācijas un sāktās reformas, šļahtiči nebija salauzti nedz ekonomiski, nedz arī politiski. Viņi saglabāja savu ietekmi uz zemniekiem. Turklāt tas, ka zemnieki saņēma zemi, sekmēja ekonomiskos un sociālos procesus, pastiprinājās lauku iedzīvotāju pieplūdums pilsētās, paātrinājās dzelzceļu celtniecība, kas vienoja lauksaimnieciskos novadus. Zemnieku lietas tagad tika skatītas krieviskajās iestādēs un tiesās, kur bez krievu valodas nevarēja iztikt. Zemnieki sāka saprast, ka viņu nacionālā piederība viņus nostāda otrās šķirās Krievijas pavalstnieku stāvoklī. Tātad, pateicoties privātīpašuma ieguvei, pastiprinājās poļu zemnieku kopējās intereses ar šļahtu.

Pilsētnieku vairākums piederēja mazizglītotajiem un mazturīgajiem iedzīvotāju slāņiem. Strādnieku slānis no agrāko gadsimtu pilsētu plebsa mantoja nacionālās jūtas. Pilsētu apdzīvotības blīvums un to iedzīvotāju kulturālā aktivitāte veicināja strādnieku nacionālās pašapziņas veidošanos, kaut arī, kā raksta poļu vēsturnieki, intensīvais papildinājuma pieplūdums no laukiem pazemināja šīs nacionālās pašapziņas līmeni. (Pēc grāmatas autora domām, šis jautājums vēl ir pētāms, jo bijušie nesenie zemnieki, kuri vēl nebija iesaistījušies šķiru cīņā, nesa sev līdzi šļahtiču ietekmi un varēja stiprināt pilsētnieku nacionālo apziņu pretstatā šķiriskajai.)

Blakus ugodoviešu (samierināšanās ar carismu piekritēju) grupējumam izveidojās radikālāks strāvojums, kurš uzstājās par Žečpospolitas atjaunošanu uz nacionāli-liberālu ideju pamata. Ievērojamākās personības tajā bija J.~Poplavskis un S.~Balickis. J.~Poplavskis sāka izstrādāt virziena idejisko platformu, kura paredzēja tautas iesaistīšanu ``nacionālās politikas ietvaros'', ar to liekot idejiskos pamatus nacionāl-demokrātiskajai kustībai.

1887./1888.~gadā Šveicē nodibinājās nelegāla liberāla virziena grupa~--- \pltxti{Liga Polska} (``Poļu līga''), kura aicināja pārtraukt lojalitātes politiku pret carismu, cīnīties par Polijas neatkarības atjaunošanu uz federālisma pamata. Organizācijas vadītājs sākotnēji bija Z.~Milkovskis. Viņš aizstāvēja konsekventas cīņas pret Polijas sadalītājvalstīm nepieciešamību, uzstājās par Polijas atjaunošanu tās 1772.~gada robežās. Līga sludināja gatavību ``uz aktīvu nacionālu uzstāšanos''. Pēc vairākiem darbības gadiem tajā nostiprinājās tradicionālās pretkrievu pozīcijas, balstoties uz Austrijas un Vācijas savienību. Par lietderīgu tika atzīts viss, kas varēja vājināt Romanovu impēriju. Ar laiku Līgas priekšgalā izvirzījās J.~Poplavskis, T.~Valigorskis, \strong{R.~Dmovskis}.

Pēdējais bija Līgas biedrs no 1889.~gada. R.~Dmovskis arī piedalījās nelegālas studentu organizācijas ``Polijas jaunatnes savienība ``\pltxti{Zet}'' (``\pltxti{Związek Młodzieży Polskiej ``Zet''}) darbībā. 1890.~gadā, kad Varšavas studenti apsprieda jautājumu vai viņiem atbalstīt Krievijā sākušās studentu uzstāšanās pret valdību, viņš neslēpa savu antikrievisko nostāju: ``Polijas revolucionārās kustības uzdevums ir izkarot Polijas neatkarību no Krievijas, neatkarīgi no tā, kas to pārvalda un pārvaldīs, patvaldniecisks cars vai parlaments, kaut pats liberālākais. Sabiedrībā ir jāstiprina neatkarības gars, visur jāuzsver mūsu mērķu atšķirība no krievu mērķiem un centieniem''. 1791.~gada 3.~maija Polijas Konstitūcijas pieņemšanas simtajā gadadienā R.~Dmovskis organizēja studentu demonstrāciju, tāpēc tika arestēts un izsūtīts uz Mītavu (tagad Jelgava). 1893.~gadā R.~Dmovskis izvirzījās par līgas idejisko vadītāju, bet vēlāk kļuva redzamākais vēlākā Polijas valsts vadītāja J.~Pilsudska konkurents XX gadsimta Polijas sabiedrības vēsturē.

Kopš 1893.~gada grupa veica pagriezienu izteikta nacionālisma virzienā, tika pārdēvēta par ``Nacionālo līgu'' (\pltxti{Liga Narodowa}). Kopš 1895.~gada Lembergā (tagad~--- Ļvova), pēc tam Krakovā iznāca Līgas preses orgāns ``\pltxti{Przegląd Wszechpolski}'' (``Vispolijas Apskats''). ``Nacionālās līgas'' šūniņas pastāvēja arī Prūsijā un Austrijā. No 1895.~gada uz Galīciju devās tie Līgas vadītāji, kuriem draudēja arests Krievijā. Prūsijā Līgas darbība attīstījās sarežģītāk, šeit pirmos biedrus tā ieguva tikai 1899.~gadā. Līga pirmkārt centās ietekmēt poļu zemniekus, no 1896.~gada izdeva tiem domātu žurnālu ``\pltxti{Polak}'', (``Polis'') taču nevairījās uzrunāt arī strādniekus. Nacionālā līga izstrādāja cīņas programmu gan pret krieviem, gan vāciešiem.

S.~Balickis un R.~Dmovskis arī centās izlabot ``poļu raksturu'', viņu nacionālo psihi. Viņi vadījās no uzskata, ka nācijai kā ``organizācijai'' sabiedriskai ir jātiek vadītai un ir jāpastāv īpašiem vadības instrumentiem. ``Normālos'' apstākļos tādi instrumenti ir tiesību normas, morāle, tradīcijas. Sadalīšanas un iekļaušanas citos valstiskos organismos rezultātā poļu nācija ieguvusi ``svešas'' tiesību normas, ``sliktas'' tradīcijas un ``sabojātu'' morāli. Tāpēc nāciju vajadzēja ``garīgi atveseļot''. No nācijas bioloģiskās teorijas, attieksmes pret to kā dzīvu organismu izauga pārliecība par nācijas ``labošanas'' iespēju, piemēram, neveselīgo elementu amputācijas ceļā. Pēc S.~Balicka poļu nācijas ``nenormālo'' stāvokli noteica tas, ka tā bija spiesta dzīvot blakus ar ``naidīgām nacionalitātēm'', tai skaitā tādām, kuras agrāk bija ietilpušas Žežpospolitas sastāvā. No šejienes viņš izdarīja secinājumu par poļu kā nācijas, kura atradās ``ārkārtējā stāvoklī'' īpašām tiesībām un uzdevumiem. Tā kā poļu dzīves apstākļi prasīja dzelzs raksturu, stingru gribu, karavīra sīkstumu un spēku, pilsoņu tiesības šādos apstākļos varēja saņemt tikai karotājs. Tāda karotāja-pilsoņa pienākums bija aizstāvēt etniskās, valodas, reliģiskās un psiholoģiskās atšķirības. ``Nacionālās līgas'' idejisko pamatu tad arī veidoja ideja par organizācijas biedru īpašo misiju, viņu tiesībām uz morālu un politisku diktatūru.

1897.~gadā no Nacionālās līgas attīstījās \strong{Nacionāldemokrātiskā partija} (\pltxti{Stronnictwo Narodowa Democratyczne} jeb \pltxti{Narodowa Demokracja}, saīsināti \pltxti{Endecja}), sākotnēji kā slepena organizācija, bet pēc 1905.~gada revolūcijas (Austroungārijai piederošajā Galīcijā~--- jau kopš 1903.~gada) tā varēja darboties arī legāli. Taču turpināja pastāvēt arī nelegāla Nacionālā līga, gan sastāva ziņā, gan idejiski cieši saistīta ar Nacionāldemokrātisko partiju. 1905.~gadā Nacionālajā līgā bija 585~biedri. Tikai 1928.~gadā Nacionālā līga tika likvidēta un iekļauta Nacionāldemokrātiskās partijas struktūrā. (Visu šo organizāciju biedrus tautā saīsināti sauca par endekiem.)

Diemžēl kā ikviens šāda veida virziens, arī poļu nacionālisms dzīvo nacionālo pašapziņu pārvērta par abstraktu principu, kurš savas nācijas intereses izvirzīja kā noteiktu pretējību citām tautām. No tā izrietēja naidīgums ne tikai pret tiešajiem apspiedējiem, bet arī pret veselām tautām, kā arī etniskām grupām, kuras dzīvoja poļu zemēs, atvirzot otrajā plānā kopējās sociālās problēmas un šķiru pretišķības. Nacionāldemokrāti pretendēja uz visas poļu sabiedrības politisko vadību, neraugoties uz tās sociālo neviendabību. R.~Dmovskis atklāti aicināja polonizēt ukraiņus, baltkrievus un lietuviešus. Par galveno nacionālistu uzbrukuma objektu kļuva ebreji, tika prasīta viņu izraidīšana no Polijas. Ja agrāk poļiem svešā ebreju kultūra, īpaši ticība vienkārši izsauca negatīvas emocijas, tad ar tirdzniecības un rūpniecības attīstību, kur ebreji spēlēja aktīvu lomu, radās arvien vairāk pretrunu, kuras savā propagandā centās izmantot poļu nacionālisti. 1903.~gadā iznāca R.~Dmovska darbs ``\pltxti{Mysli nowoczesnnego Polaka}'' (``Mūsdienu poļa pārdomas''), kurā viņš, gan atzīmējot nacionālisma nehumānās izpausmes, tomēr nāciju cīņu par savu izdzīvošanu uzskatīja par sabiedrības attīstības likumu, velkot paralēles ar Č.~Darvina teoriju par dabisko izlasi. Pakāpeniski notika nacionāldemokrātu mērķu korekcija. Ja 1897.~gadā parijas kongress par tās galveno uzdevumu stādīja ``nacionālu sacelšanos un dzimtenes atbrīvošanu ar spēku'', bet 1899.~gadā avīze ``\pltxti{Przegląd Wszechpolski}'' paziņoja, ka atbrīvošanos no krievu jūga nāksies panākt ar ``uguni un dzelzi'', tad drīzumā pozīcija mainījās, tuvinoties ar ugodoviešiem. Gadsimtu mijā, atbildot uz nacionālistiskās propagandas pieaugumu, uzbrukumiem slāvu tautām Vācijā, arī poļu nacionālisti bija spiesti pārorientēties, strauji pāriet no pārsvarā pretkrieviskas propagandas uz pārsvarā pretvācisku. Glābiņu no ``\detxti{Drang nach Osten}'' R.~Dmovskis piedāvāja meklēt poļu vienotībā ar visiem slāviem uz jauniem, antivāciskiem pamatiem.

Runājot par poļu politisko darbību tomēr noteikti jāievēro, ka varas iestāžu noliedzošā attieksme pret pavalstnieku politisko darbību un pārstāvniecības iestāžu ierobežotā loma, to izveidošanas nedemokrātiskais veids (ne tikai Polijas karalistē, bet arī citās Polijas daļās), atstāja iedzīvotāju lielāko vairumu ārpus politisko partiju ietekmes, kas bija raksturīgi autoritārai sabiedriskai kārtībai. Tautas masas lielākoties saglabāja konservatīvu politisku vienaldzību.

\section{Polija XX~gadsimta sākumā}

\epigraph
{Vēloties mainīt valsts iekārtu, ir jāņem vērā pavalstnieku gatavība tam.}
{Tits Līvijs (\latxti{Titus Livius})}

\epigraph
{Naids pret buržuāziju~--- tikumības sākums.}
{Gistavs Flobērs (\frtxti{Gustave Flaubert})}

\epigraph
{Tā tauta, kas domā, ka cilvēkus godīgus padara ticība, bet nevis labi likumi, man šķiet visai atpalikusi.}
{Denī Didro (\frtxti{Denis Diderot})}

\epigraph
{Revolūcijas periodos politisko karjeru visbiežāk taisa vismazāk tikumīgie un sociāli atbildīgie darbinieki, kuri vienmēr ir gatavi sacīt: ``Ir tāda partija!''}
{Igors Kons (\rutxti{Игорь Семёнович Кон})}

\epigraph
{Galvenā cilvēku kļūda ir tā, ka no šodienas nelaimēm viņi baidās vairāk nekā no rītdienas nelaimēm.}
{Kārlis fon Klauzevics (\detxti{Carl Philipp Gottlieb von Clausewitz})}

\epigraph
{Lielvalstis vienmēr ir uzvedušās kā bandīti, bet mazās valstis kā prostitūtas.}
{Stenlijs ~Kubriks (\entxti{Stanley Kubrick})}

\epigraph
{Pirmais pasaules karš bija divdesmitā gadsimta galvenā katastrofa, jo visas pārējās vēlākās katastrofas ir tikai tā sekas.}
{Pēteris Krupņikovs}

\subsection{1905.~gada revolūcija poļu teritorijās}

% page 195


Poļu zemes XX gadsimta sākumā bija ievērojami pavirzījušās uz priekšu savā ekonomiskajā attīstībā. Ja XIX gadsimta sākumā tajās ar lauksaimniecību nodarbināto iedzīvotāju īpatsvars sasniedza 85\%, tad XX gadsimta sākumā tas jau bija samazinājies līdz 56\%. Poļu zemes no pārsvarā lauksaimnieciskām pārvērtās par agrāri-rūpnieciskām. Braucieni uz ārzemēm darba meklējumos un lauku iedzīvotāju migrācija uz pilsētām aptvēra jau ap 9 miljoniem cilvēku. Norisa industrializācija un urbanizācija, kas bija saistītas ar iedzīvotāju proletarizāciju. Dažādu valstu valdījumos šie procesi gan risinājās ar dažādu intensitāti. Gadsimtu mijā izveidojās trīs sociāli-ekonomiskās struktūras tipi. Galīcijā tas bija agrārs, kur līdz 80\% iedzīvotāju iztikas avots bija lauksaimniecība, Polijas karalistē un t.s. Lielpolijā ap Vartas upi, Piejūrā agrāri-rūpniecisks (ap 60\% iedzīvotāju), Silēzijā~--- rūpniecisks (mazāk par 40\%). Visos apgabalos lauksaimniecībā valdošais stāvoklis joprojām palika lielīpašnieku rokās. Polijas karalistē tiem piederēja ap 56,6\%, Prūsijas poļu apgabalos~--- ap 50\%, Galīcijā~--- ap 42\% zemes. Taču politiski poļu sabiedrība savā attīstībā ievērojami atpalika no rietumos esošajiem paraugiem. Sava valstiskuma trūkums, nacionālā apspiestība, poļu buržuāzijas vājās ekonomiskās pozīcijas bremzēja tās politisko un organizatorisko konsolidāciju.

XX gadsimta sākumā \strong{Polijas karaliste} bija kļuvusi par nozīmīgu Krievijas ekonomisku rajonu, kurā bija koncentrēti ap 15\% visu impērijas rūpniecības strādnieku. Taču 1900.--1903.~gadā Polijā, tāpat kā visā pasaulē, izpaudās ekonomiskā krīze. Dažādas saimniecības nozares tā skāra dažādā laikā un nevienādā mērā. 1904.~gada janvārī ar Japānas jūras kara spēku uzbrukumu Krievijas flotei Portarturas bāzē sākās krievu--japāņu karš. Karš traucēja saimniecības attīstību ne tikai pašā Krievijā, bet arī Polijas karalistē. Kopumā XX gadsimta sakumā Polijas karalistē rūpniecības produkcijas ražošana saruka par 35\%. 1904.~gadā Polijā bija zema raža. Samazinājās rūpniecības uzņēmumu skaits. Kritās algas rūpniecībā, auga bezdarbnieku skaits. Kritās strādājošo dzīves līmenis. Pilnīgs vai daļējs bezdarbs Varšavā aptvēra 60\%, bet Lodzā 70\% strādnieku, darba alga samazinājās par 20--25\%, ievērojami pieaugot pirmās nepieciešamības preču cenām.

Tas viss kopā radīja neapmierinātības noskaņojumu tautā. Sabiedrības uzmanību piesaistīja poļu sociāldemokrātiskās kustības veterāna M.~Kaspšaka rīcība 1904.~gada 27.~aprīlī Varšavā. Viņa ierīkoto nelegālās literatūras spiestuvi atklāja policija. M.~Kaspšaks nevēlējās padoties, atšaudoties nogalināja četrus un ievainoja vēl vienu policistu. (Pēc dažiem mēnešiem viņam piesprieda nāves sodu, ko nomainīja ar mūža ieslodzījumu, taču 1905.~gada septembrī viņš Varšavas citadelē pakārās.) Jau 1904.~gada otrajā pusē politiskās aktivitātes Polijas karalistē ievērojami pieauga. Vietējie politiskie spēki uz to reaģēja dažādi.

Konservatīvie poļu politiķi (ugodovieši) centās demonstrēt cara impērijai atbalstu grūtā stundā, lai pēc kara saņemtu piekāpšanos nacionālajā sfērā. Viņi organizēja katoļu hospitāli frontei, piedalījās pieminekļa atklāšanā Katrīnei II Viļņā. 1905.~gadā šī kustība noformējās par \strong{Reālās politikas partiju} (\pltxti{Stronnictwo Polityki Realne}j), kuras biedrus tautas valodā sauca arī par ``reālistiem'' jeb ``pozitīvistiem''. Partiju vadīja grāfs Z.~Veļepoļskis un publicists E.~Piļcs. Tās programma ietvēra poļu vienlīdzības prasību, kas praksē nozīmēja municipālo pašvaldību~--- zemstu ieviešanu, zvērināto piesēdētāju tiesas, prasības pēc poļu pielaides administratīvajiem amatiem, vienādiem nodokļiem, reliģiskās iecietības un poļu valodas pielietošanas izglītībā, kas būtu pirmais solis uz Polijas karalistes autonomiju. Faktiski programma negāja tālāk par vēl 1815.~gadā Vīnē pieņemto Polijas sadales noteikumu atjaunošanu. Partija neguva plašu atbalstu Polijas karalistes iedzīvotājos, tai bija maz biedru. Taču karalistē tai bija diezgan liela ietekme, jo tās vadoņiem bija labi sakari ar valdošajām aprindām Pēterburgā.

Salīdzinājumā ar XIX gadsimta 90.~gadiem būtiski no pretkrieviskās orientācijas uz pretvācisku bija mainījusies \strong{Nacionāldemokrātiskās partijas} (\pltxti{Stronnictwo Narodowo-Demokratyczne}) nostāja. \strong{R.~Dmovskis}, izanalizējis Vācijas un Austroungārijas vadošo politiķu attiecības pret poļu mazākumiem savās valstīs, nonāca pie konstatējuma, ka Vācija veic spiedienu uz Habsburgu impēriju, virzītu uz tās pakļaušanu Vācijas ekspansionistiskajiem plāniem. No tā izrietēja tālāks secinājums: poļu orientācijai uz Austroungāriju nav perspektīvu, ir jāmeklē citas politiskās alternatīvas.

Sadarbības ar Krieviju izdevīgums kļuva tik acīm redzams, ka Nacionāldemokrātiskās partijas kongress 1903.~gadā savā rezolūcijā deklarēja: ``Neatsakoties no sava Polijas atjaunošanas ideāla, dotajā vēsturiskajā situācijā mēs uzskatām par nepieciešamu censties izcīnīt tai pēc iespējas lielāku patstāvību Krievijas ietvaros''. R.~Dmovskis neatbalstīja poļu sacelšanās plānus, pamatoti uzskatot, ka tai nebūs izredžu uz panākumiem, taču izteicās par nacionālo apziņu stiprinošām manifestācijām. 1904.~gadā Nacionāldemokrātiskā partija pilnībā atteicās no nacionālās neatkarības prasības un aicināja cīnīties par visu poļu zemju (ietilpstošu Krievijā, Vācijā un Austroungārijā) apvienošanu vienotā autonomā Polijā Krievijas impērijas sastāvā. Politiskās un kulturālās autonomijas ieguve bija paredzēta sadarbības un sarunu ceļā ar Krieviju. Par poļu nācijas galvenajiem ienaidniekiem nacionāldemokrāti uzskatīja vāciešus un ebrejus. Partijas programma pamatojās politiskajā pragmatismā, paredzēja ``organisku darbu''~--- tautsaimniecības attīstību, tautas izglītošanu.

Tomēr savās prasībās pret Krieviju Nacionāldemokrātiskā partija ar R.~Dmovski priekšgalā gāja tālāk kā konservatīvie poļu politiķi. Attiecībā pret Krieviju partijas programma izvirzīja prasību pēc kulturālas un politiskas autonomijas, ko vajadzētu sasniegt sadarbības un sarunu ceļā. Nacionāldemokrātija prasīja poļu valodas ieviešanu ne tikai skolās, bet arī administrācijā. Kopumā R.~Dmovskis bija pārliecināts, ka Krievijas impērija pārvarēs krīzi un tāpēc uzstājās pret jebkādām radikālām prasībām. Nacionāldemokrāti jeb tautas valodā endeki tikai sākotnēji atbalstīja streikus un demonstrācijas, bet turpmāk bija gatavi pret tiem cīnīties, ja Polijai tiktu piešķirta autonomija. Vēlāk R.~Dmovskis rakstīja: ``Nacionāldemokrātiskā nometne jau no paša sākuma noteikti uzstājās pret visiem sacelšanās vai revolucionāras kustības [izvēršanas] mēģinājumiem.'' R.~Dmovskim 1905.~gada sākumā ierodoties Varšavā, tika veikti pasākumi partijas organizatoriskai nostiprināšanai, hierarhiskas struktūras izveidei. 1905.~gada jūnijā pie Nacionāldemokrātiskās partijas izveidojās arī Nacionālā strādnieku savienība (\pltxti{Narodowy Zwiazek Robotniczy}). Nacionāldemokrātiskajai partijai nebija savas kaujinieku organizācijas, taču tāda darbojās Nacionālajā strādnieku savienībā. Īslaicīgi darbojās arī dažas citas endeku ietekmē esošas kaujas grupas.

``Reālisti'' un ``endeki'' veda sarunas ar krievu liberāļiem (nākamajiem kadetiem), kuri principā piekrita Polijas autonomijai, bet nekādas konkrētas norunas negribēja slēgt. Nacionāldemokrātu un ``reālistu'' prasība pēc autonomijas, ar kuras izpildi, iespējams, varētu sekmīgi apstādināt revolucionāro kustību, tā arī palika varas iestāžu nesadzirdēta.

Principiāli citu pozīciju uzturēja \strong{Polijas Sociālistiskā partija} (\pltxti{Polska Partia Socijalistyczna}) \strong{(PPS)}. Tajā t.s. ``jaunie'' sākotnēji ticēja, ka drīzumā var sākties revolūcija Vācijā, sacelšanās Polijā un progresīvās Eiropas karš pret reakcijas citadeli~--- carisko Krieviju. Taču PPS rindās uzvarēja t.s. ``vecie'' ar \strong{J.~Pilsudski} priekšgalā, kuru līnija veda uz kara un revolūcijas Krievijā izmantošanu, lai atbrīvotu Polijas karalisti no tās. J.~Pilsudskis uzskatīja, ka labvēlīgs brīdis kārtējai bruņotai sacelšanās par Polijas brīvību jau ir pienācis vai arī tūlīt pienāks. ``Vecie'' paļāvās ne uz Krievijas revolucionāro kustību, bet uz citām tautām, kuras apspieda carisms. Šī grupa arvien vairāk attālinājās no marksistiskās ideoloģijas. PPS vadītāji nolēma sadarboties ar Japānas valdību. J.~Pilsudskis 1904.~gada jūlijā ieradās Tokijā, lai organizētu poļu leģionus no gūstā saņemtajiem Krievijas armijas poļiem cīņai par Polijas neatkarību. Japānas imperatora ierēdņiem viņš arī piedāvāja finansēt PPS pretkrieviskās akcijas, apgādāt to ar ieročiem, solot traucēt armijas mobilizācijai Krievijā, veikt izlūkošanu un sabotāžu un pie labvēlīgiem nosacījumiem sarīkot sacelšanos kopā ar citām Krievijas impērijas apspiestajām tautām.

J.~Pilsudska uzskati bija koncentrēti izklāstīti memorandā, kuru viņš 1904.~gada 13.~jūlijā iesniedza Ārlietu ministrijas ierēdnim Tokijā. Pēc poļu politiķa domām Krievijas impērijas galvenais vājuma avots, tās ``Ahileja papēdis'', kurā visiem tās ienaidniekiem ir jācenšas trāpīt, ir tās vienotības trūkums. Ne visām nācijām impērijā esot vienāda vērtība. J.~Pilsudskis līdzīgi vācu nacionālistiem visas nācijas dalīja ``vēsturiskajās'' un ``nevēsturiskajās''. Pie pirmo kategorijas viņš pieskaitīja poļus, somus, gruzīnus, zināmā mērā'' armēņus, pie otrajām~--- lietuviešus, baltkrievus un latviešus. Viņš gandrīz pilnībā noliedza iespēju, ka patstāvīgi politiski varētu uzstāties rusīni (rutēņi), tātad ukraiņi. Poļiem viņa memorandā bija veltīta īpaša uzmanība. Ja pārējie nacionāļi patstāvīgi varēja tikai radīt politisku opozīciju valdībai un sabotēt tās darbību, tad ``poļi ir spējīgi novest notikumus līdz atklātai cīņai un piesaistīt sev pārējās nācijas uz bruņotu cīņu par atbrīvošanos no krievu paverdzinātāja''. Revolūcija pašā Krievijā tika vērtēta kā maz iespējama.

Taču Japānā J.~Pilsudskis sastapās ar R.~Dmovski, kurš bija atbraucis pārliecināt japāņus, ka nevajag provocēt poļu sacelšanos. R.~Dmovski pieņēma Japānas Ģenerālštāba priekšnieks, pēc kura lūguma viņa sastādīja divus plašus ziņojumus: par stāvokli Krievijā un par Polijas jautājumu. R.~Dmovskis centās pierādīt, ka par nacionāla mēroga sacelšanos Polijas nevar būt runas, bet lokālas uzstāšanās nesīs ļaunumu gan Polijai, gan Japānai, jo Krievija, baidoties no nemieriem, pie Vislas tur veselu armiju (tā mazinot savu kaujas potenciālu Tālajos Austrumos), kura noslīcinās asins jūrā jebkuru dumpi. Ja Polijā sāktos sacelšanās, Vācija labprāt palīdzētu Krievijai to apspiest. Jebkura ārēja palīdzību poļu separātistiem tikai veicinātu Krievijas sabiedrības saliedēšanos.

Japānas valdošās aprindas ar neuzticību raudzījās uz J.~Pilsudska plāniem, radīt leģionu netaisījās, taču pētīja iespējas izmantot poļu nacionālistus sev derīgas informācijas ieguvei un pēc dažām ziņām piešķīra šim nolūkam 20 tūkstošus (daži poļu vēsturnieki uzskata, ka pat ap 30 tūkstošu) sterliņu mārciņu. Lauvas tiesa no šīs naudas tika izmantota ieroču iegādei un PPS darbības finansēšanai.

Atgriezies Polijā, J.~Pilsudskis nodibināja kaujas organizāciju (\pltxti{bojòwka}). Jau 1904.~gada aprīlī PPS kaujas družīnas sāka t.s. ``spiegu'' (faktiski~--- savu politisko pretinieku, turētu aizdomās par sadarbību ar varas iestādēm) ``sodīšanu''~--- slepkavošanu. Kaujinieki arī nodarbojās ar ekspropriācijām~--- banku un pasta vilcienu aplaupīšanām partijas finansiālām vajadzībām. Tiesa, ilgu laiku J.~Pilsudskis pats personīgi šajos bīstamajos pasākumos nepiedalījās, aprobežojoties ar dziļi konspirēta vadītāja lomu, kas, kā rakstīja žurnāliste un sabiedriska darbiniece I.~Pannenkova, kura 1922.~gadā ar pseidonīmu \emph{Jan Lipecki} publicēja darbu ``\pltxti{Legenda Piłsudskiego}'' (``Pilsudska leģenda''), viņa nelabvēļu vidū izsauca runas par bailīgumu.

1904.~gada 13.~novembrī PPS organizēja Varšavā grandiozu protesta demonstrāciju pret poļu karavīru iesaistīšanu krievu-japāņu karā. Pret policiju, kas centās traucēt demonstrācijas norisi, darbojās poļu kaujinieku grupa. Policistu vidū bija viens nogalinātais un vairāki ievainotie. Sākās sadursmes ar policiju arī citās Polijas pilsētās. PPS popularitāte, sevišķi strādnieku vidū strauji auga, 1905.~gadā tajā bija ap 50~000 biedru.

J.~Pilsudskis un viņa atbalstītāji PPS uzskatīja teroru par līdzekli kā uzkurināt revolūciju Polijas karalistē. Kaujas organizācijas tika uzskatītas par kaujas armijas iedīgli, kurai bija jāizraisa karš pret carismu. T.s. ``jaunie'' PPS sastāvā, kuri noraidīja bruņotu cīņu, jau 1905.~gada martā panāca iepriekšējās partijas vadības atlaišanu, taču bruņošanās tolaik tikai pastiprinājās.

Turpretī \strong{Polijas karalistes} un Lietuvas sociāldemokrātija uzskatīja, ka revolūcijā, lai gāztu carisko režīmu, ir jāsadarbojas ar Krievijas revolucionāriem. PKunLSD bija cieši saistīta ar Krievijas sociāldemokrātisko strādnieku partiju (KSDSP). Taču, kā jau iepriekš teikts, poļu sociāldemokrātiem bija savi patstāvīgi uzskati. 1903.~gadā PKunLSD savā IV kongresā nolēma, ievērojot noteiktas prasības, apvienoties ar KSDSP. Saskaņā ar to KSDSP II kongress 1903.~gada jūlijā uzaicināja tajā ar padomdevēju balsīm piedalīties arī PKunLSD pārstāvjus. Šie pārstāvi: J.~Gaņeckis un A.~Varskis uzstājās pret KSDSP programmas projektā ietverto prasību pēc nāciju pašnoteikšanās tiesībām, domājot, ka tas nāks par labu poļu nacionālistiem no PPS. Kad projekts ar minēto prasību tika pieņemts, viņi atstāja kongresu. Abu partiju apvienošanās notika jau revolūcijas gaitā KSDSP IV kongresā 1906.~gada pavasarī, kad poļu sociāldemokrāti atteicās no savas nostājas nacionālajā jautājumā un vairs neprasīja lai krievu sociāldemokrāti pārskata savu nostāju par nāciju pašnoteikšanās tiesību atzīšanu. Šai kongresā PKunLSD pārstāvji jau sadarbojās ar krievu boļševikiem. A.~Varskis un F.~Dzeržinskis kļuva par KSDSP CK locekļiem.

1904.~gada beigās PKunLSD cara patvaldības gāšanas, politisko brīvību izcīnīšanas lozungus papildināja ar demokrātiskas republikas, satversmes sapulces sasaukšanas un poļu zemju autonomijas lozungiem. Partija, kura līdz 1905.~gadam noliedza teroru, sākoties revolūcijai, pieņēma to kā ``taktisku'' cīņas līdzekli. Sociāldemokrātu kaujas organizācijas uzdevums bija aizstāvēt strādniekus pret cara varas iestāžu patvaļu un vēlāk arī pret endeku kaujiniekiem.

1905.~gada 9.~(22.)~janvāris~--- ``asiņainā svētdiena'' arī Polijā izsauca plašu rezonansi. Praktiski uzreiz kā Poliju sasniedza ziņas par notikumiem Pēterburgā, visas politiskās partijas griezās ar paziņojumiem pie tautas. Dažas no tām, kā PPS, uzsvēra nacionālo jautājumu. Tagad grūti pateikt vai tautas vairākums ticēja iespējai panākt nacionālo autonomiju, vai vienkārši cerēja uz liberālām pārmaiņām valsts iekārtā. Faktiski gan vienu, gan otru mērķi varēja panākt tikai cīņā pret cara monarhiju, ko pirmajā vietā izvirzīja sociāldemokrāti. Atsaucoties uz ``asiņaino svētdienu'' Pēterburgā, daudzās Polijas karalistes pilsētās uzliesmoja streiki, kuri ilga trīs nedēļas, streikoja ap 93\% visu strādnieku. Viņu ekonomiskās un daļa politisko prasību (par strādnieku komiteju atzīšanu) tika apmierinātas, taču Varšavā, Lodzā u.c. norisa sadursmes ar policiju, lija asinis.

Līdz februāra vidum oficiāli jau bija uzskaitīti 19 vardarbīgas nāves gadījumi, to faktiskais skaits droši vien bija lielāks. 1905.~gada jūnijā, kad, tika apšauta tekstilrūpniecības strādnieku demonstrācija Lodzā (bojā gāja 25~cilvēki), pilsētu pārklāja barikādes. To skaits bija ap 100. Nākamajās četrās dienās bojā gāja jau ap 200~cilvēku. Faktiski tā bija kā nacionālās, tā sociālās cīņas izpausme, stihiska bruņota uzstāšanās, pirmā pēc 1863.~gada sacelšanās.

Notikumi Polijā bija Krievijas revolūcijas sastāvdaļa. To vēriens, iznākums bija atkarīgs no revolucionārās cīņas vēriena un iznākuma visā Krievijā. Polijas karaliste kļuva par vienu no svarīgākajiem revolucionārās cīņas centriem Krievijā. Notikumiem Polijas karalistē bija pat lielāka nozīme nekā citās, impērijas krievvalodīgajās provincēs. Pēc dažu pētnieku datiem Polijas karalistē šai laikā dzīvoja 13,5\% no visiem Krievijas fabriku un rūpnīcu strādnieku, taču streikotāju skaits šeit sasniedza 29\% no visiem streiku dalībniekiem Krievijā. Pēc citiem datiem starp visiem Krievijas impērijas streikotājiem Polijas karalistes strādnieki 1905.~gadā sastādīja 34\%, 1906.~gadā~--- 46\%, bet 1907.~gadā~--- 14\%. Ievērojami pieauga revolucionāro partiju biedru skaits. Sociāldemokrātu skaits no 2~000 izauga gandrīz līdz 40~000, PPS rindās skaitījās 55~000~--- 60~000~cilvēku.

1905.~gada pavasarī aktivizējās muižu laukstrādnieku kustība, kuri prasīja darba apmaksas paaugstināšanu. Zemnieki necentās sadarboties ar laukstrādniekiem, taču savukārt ap 50~apriņķos zemnieki izvērsa cīņu par servitūtiem. Drīzumā savās sapulcēs viņi prasīja poļu valodas atjaunošanu zemākajās iestādēs gminās, tiesās un skolās. Tika panākts, ka kopā ar krievu valodu gminās varēja lietot arī poļu valodu. 1905.~gadā Polijas karalistē uzliesmoja ``skolu revolūcija''. Streikoja ne tikai Varšavas universitātes studenti, bet ari vidējo un pat sākumskolu skolnieki. Šo streiku 1905.~gada 6.~februārī (jaunais stils) atbalstīja arī vecāku mītiņi. 1905.~gada pirmajā pusē imperators Nikolajs II pēc Ministru komitejas ierosinājuma Polijas karalistes sākumskolās atļāva virkni mācību priekšmetu mācīt dzimtajā valodā. Tika atcelti ierobežojumi poļu ekonomiskajai darbībai Krievijas Rietumu guberņās. Tai skaitā viņiem atļāva šeit iepirkt nelielus zemes gabalus izmantošanai rūpnieciskām vajadzībām. Tomēr 1905~.~gada 1.~maijā imperatora apstiprinātajā Ministru komitejas žurnālā tika ietverts slepens punkts, ar kuru tika aizliegts deviņās rietumu guberņās valsts amatus ieņemt ``poļu izcelsmes personām''. Tai skaitā bija arī krievu valodas, vēstures, ģeogrāfijas un pedagoģijas skolotāju amati. Ar 1905.~gada 1.~oktobra cara dekrētu tika atļauts dibināt privātskolas ar poļu mācību valodu. Taču šo skolu absolventiem nebija tiesību iestāties Krievijas impērijas augstākajās mācību iestādēs. Praksē tas noveda pie tā, ka viņi turpināja izglītību ārzemēs.

Pēc cara Nikolaja II 1905.~gada 17.~oktobra manifesta, kurš solīja pavalstniekiem dot vienkāršākās pilsoniskās tiesības, legalizējot politiskās partijas, ieviešot ko līdzīgu vispārējām vēlēšanu tiesībām un sasaucot Valsts Domi kā centrālo likumdevēju iestādi, Polijā sākās ``brīvības dienas'' (\pltxti{dni Wolności}), notika plaši mītiņi un demonstrācijas. Tautas masas nesamierinājās ar carisma piekāpšanās apjomu. Varšavas ģenerālgubernators G.~Skalons 26.~oktobrī Iekšlietu ministrijai Pēterburgā ziņoja: ``Revolucionārā kustība man uzticētajā novadā redzami un ātri pastiprinās, aptverot arvien lielākus ļaužu lokus un nokļūstot pat zemnieku masās. Situācija ir visai nopietna, un es redzu tikai vienu izeju~--- tūlītēju kara stāvokļa ievešanu visā Polijas karalistē. Jebkuru vilcināšanos es uzskatu par bīstamu''. 28.~oktobrī Nikolajs II parakstīja dekrētu par kara stāvokļa ievešanu Polijas karalistē. Uzreiz sākās masu aresti, tika slēgtas revolucionāro avīžu redakcijas, tika izdota pavēle šaut uz demonstrantiem. Taču daudzās Krievijas pilsētās izvērsās solidaritātes kustība. Uz notikumiem Polijā reaģēja ne tikai revolucionāri, bet arī Krievijas liberālā kustība. Konstitucionālo demokrātu (kadetu) partijas I kongress 1905.~gada 12.--18.~oktobrī jau bija pieņēmis programmu, kura atzina, ka Polijas karalistei ir jādod autonomijas tiesības, saglabājot valsts vienotību. Tajā bija paredzēts, ka robeža starp Polijas karalisti un kaimiņu guberņām var tikt labota saskaņā ar tautību sastāvu un vietējo iedzīvotāju vēlmēm, pie kam Polijas karalistē ir jādarbojas visas valsts garantētām pilsoniskajām brīvībām un tautību tiesībām uz kulturālo autonomiju, jābūt nodrošinātām mazākumu tiesībām. Pēc kara stāvokļa ievešanas Polijas karalistē liberāļi organizēja lielus mītiņus, kuru rezolūcijas prasīja atcelt kara stāvokli Polijas karalistē un piešķirt tai autonomiju. Prasību atbalstīja arī Krievijas zemstu darbinieku kongress 6.--13.~novembrī. Pēc krievu liberāļu uzskatiem autonomijas piešķiršana likvidētu ``karstu punktu'' Krievijas valstī, ko nespēja varas iestādes ar represiju politiku. Ievērojamais krievu sabiedriskais un politiskai darbinieks V.~Maklakovs rakstīja: ``Poļu autonomijas ideja bija tam laikam visvalstiskākā no idejām, kuras bija mūsu sabiedrības rīcībā.'' Tiesa, krievu oktobristi (``oktobristi'' jeb ``17.~oktobra savienība'' 1905--1917.~gadā bija labēja politiska partija Krievijā) uzstājās pret autonomijas piešķiršanu Polijas karalistei, norādot, ka vēsturiski tā pati ir vainīga tās zaudēšanā.

Drīzumā Krievijas Ministru padomes priekšsēdētājs S.~Vitte griezās pie ģenerālgubernatora G.~Skalona ar jautājumu, vai nevajag atcelt kara stāvokli. Pēdējais iesniedza šādu priekšlikumu. Valdība paziņoja, ka kara stāvokļa ieviešana ``atskurbinājusi'' Poliju un tas pēc ģenerālgubernatora lūguma tiek atcelts. Tomēr Polijas karalistē tika izvietoti ap 300~000 krievu karavīru. Kad 1905.~gada nogalē centrālajā Krievijā notikumi sasniedza kulmināciju, Polijā streiku skaits jau samazinājās. Tomēr līdz ar vispārēja politiska streika, kas pārauga sacelšanās, sākšanos Maskavā, baidoties no līdzīgas notikumu attīstības Polijā, ģenerālgubernators 21.~decembrī novadā atkal ieviesa kara stāvokli. Tomēr apstākļi, lai sekmīgi izvērstu sacelšanos, karaspēka appludinātajā Polijā nebija un streiki neieguva tādu apjomu kā 1905.~gada janvārī--februārī un arī oktobrī--novembrī. Tomēr 1905.~gadā strādnieku dalības streikos koeficients Polijas karalistē bija 2,1~reizes augstāks nekā visā Krievijas impērijā kopā.

Saasinājās arī pretrunas pašā poļu sabiedrībā. 1905.~gada oktobrī Lodzā jau poļi šāva uz poļiem, jo endeki centās neitralizēt PPS, kura pēc viņu domām veda poļus uz veltu asins izliešanu. PPS iestājās par konfrontāciju, nacionāldemokrāti~--- par kooperāciju ar varas iestādēm. Starp nacionālistiem un sociālistiem arī vēl 1906./1907.~gada ziemā Lodzā notika asiņainas cīņas. Tas radīja savstarpēju rūgtu naidu, kurš pastāvēja arī vēlāk~--- jau patstāvīgajā Polijas valstī.

Krievu vēsturnieks N.~Postņikovs, kurš pētījis poļu teroristu darbību 1905.~gada revolūcijā, aprakstījis poļu kaujinieku darbību. Visu partiju kaujas organizāciju struktūra bija apmēram līdzīga: Partijas vadībai pakļāvās kaujas nodaļa, kura vadīja guberņu kaujas organizācijas, kas savukārt dalījās ``desmitos'' vai ``pieciniekos''. Šo organizāciju darbība bija stingri konspirēta. Kaujas grupās pastāvēja stingra disciplīna, notika militārā apmācība pēc armijas parauga. J.~Pilsudska vadībā tika gatavoti vadītāji bruņotām cīņām. Par kaujiniekiem kļuva dažādu slāņu pārstāvji, taču parasti viņi bija paši aktīvākie savu partiju biedri. Skaitliski lielākā un kaujas spējīgākā bija PPS Kaujas organizācija, kurā 1905.--1907.~gadā sastāvēja 7~631 cilvēks. Endekiem pakļautajās, īpaši Nacionālas strādnieku savienības kaujas organizācijā, bija ap 1~000 cilvēku. PKunLSD kaujinieku skaits svārstījās no 1~000 līdz 1~500 cilvēkiem.

Kaujinieku realizētie terora akti bija dažādi. Tos parasti veica viena ``piecnieka'' biedri. Piemēram, uz ielas viņi piegāja pie izraudzītā upura (kāda policista, virsnieka vai ierēdņa, kuri visi nēsāja mundierus), tiešā tēmējumā viņu nošāva un nozuda no nozieguma vietas. Cits plaši izplatīts teroristisku uzbrukumu veids bija bumbu sviešana valsts iestādēs (policijas iecirkņos, karaspēka daļu sardzes telpās). Bumbu meta viens kaujinieks, pārējie viņu piesedza.

Pavisam 1905.--1907.~gadā Polijas karalistes teritorijā tika veikti 3~166 kaujinieku uzbrukumi (1905.~gadā~--- 1~441, 1906.~gadā~--- 1~090, 1907.~gadā~--- 635) 1~108 apdzīvotās vietās. 1905.~gadā un četros pirmajos 1906.~gada mēnešos bija nogalināti 127 policisti, žandarmi, militāristi, ievainoto skaits sasniedza 207. Kopā 1905.--1906.~gadā karalistē tika nogalināti 790~varas pārstāvji, vairāk nekā 860~ievainoti. No 1906.~gada terora akti tika vērsti arī pret fabrikantiem, streiklaužiem, visiem, kurus uzskatīja par ``nodevējiem''. Par streiklaužiem uzskatīja veikalu, kafejnīcu, restorānu īpašniekus, kuri streiku laikā nepārtrauca savu uzņēmumu darbu, lauciniekus, kuri pieveda pilsētām produktus u.tml. Bez tam karalistes teritorijā tika veiktas 443~ekspropriācijas, sagrābjot 600~143 rubļus.

Īpaši PPS Kaujas organizācijas akcijas poļu atmiņā atdzīvināja nacionālo sacelšanos romantiskās tradīcijas. 1905.--1907.~gadu laikā tā veica 70,2\% visu Polijas karalistē īstenoto kaujas akciju. Tikai 1906.~gadā uzskaitīti J.~Pilsudska vadīto kaujinieku veikti 678~atentāti, kuriem par upuri krita 336~krievu militāristi vai ierēdņi. 1906.~gada augustā šī darbība sasniedz augstāko punktu, notika vairāki uzbrukumi pasta vilcieniem. ``Asiņainajā trešdienā'' (\pltxti{Krwawa środa})~--- 1906.~gada 15.~augustā vienlaikus notika sekmīgi uzbrukumi vairākiem policijas iecirkņiem, kuros tika nogalināti 72 policisti un žandarmi, bet 18.~augustā~--- 19~gadus vecā V.~Krahaļska veica nesekmīgu atentātu pret ģenerālgubernatoru G.~Skalonu. Endeku kaujinieki pavisam veica 214 akciju (desmit reižu mazāk nekā PPS kaujinieki), apmēram tikpat kaujas akciju bija uz visu pārējo Polijas karalistē darbojošos partiju kaujinieku rēķina.

1906.~gada beigās sāka mainīties kaujinieku sastāvs. Ja agrāk par tiem cilvēki kļuva politisku motīvu vadīti, tad tagad viņu vidū iesaistījās arvien vairāk puskriminālu elementu, kuri izmantoja ieročus krimināliem mērķiem. Nacionālās strādnieku savienības kaujinieku sadursmes ar sociālistiem jau atgādināja atklātu pilsoņu karu un ar to izsauca neapmierinātību sabiedriskajā domā. ``Kaujinieku'' uzbrukumi parastiem iedzīvotājiem izsauca dziļi negatīvu attieksmi.

Valsts reakcija pret teroristu darbību bija nekavējoša. Polijas karalistes cietumi bija pārpildīti. Tika piespriesti un izpildīti nāves sodi. 280 PPS Kaujas organizācijas dalībniekiem tika piespriests nāves sods, 747~--- katorgas ieslodzījums. Pavisam 1904.--1907.~gadā tika apcietināti 2~739~kaujinieki. Taču pēc Polijas valstiskuma atjaunošanas virkne PPS Kaujas organizācijas dalībnieku (arī pēc nāves) par saviem nopelniem tika apbalvoti ar ordeņiem.

1906.~gada vasarā atkal saasinājās cīņa laukos. Muižu algādži sāka iznīcināt muižniekiem piederošos lopus, zemnieki sagrāba gan muižniekiem, gan valstij piederošās zemes platības, dedzināja muižu ēkas, iesaistījās sadursmēs ar viņus ``nomierināt'' atsūtīto karaspēku.

1906.~gada septembrī J.~Pilsudskis mēģināja nodibināt sadarbību ar Austroungārijas izlūkdienestu un tā pārstāvim stādījās priekšā kā 70~000 cilvēku lielas bruņotas partijas priekšstāvis. Taču abu lielvalstu: Austroungārijas un Krievijas militārs konflikts bija vēl tālu priekšā un austrieši nesteidzās ar atbildi.

Šī J.~Pilsudska darbība izsauca asas nesaskaņas starp PPS spārniem. Tās saasināja PPS Kaujas organizācijas 48~kaujinieku uzbrukums karavīru apsargātam pasta vagonam pie Rogovas (\pltxti{Rogów}, Lodzas tuvumā), kad tika nogalināts 1 policists, ievainoti trīs pasta darbinieki, divi slepenpolicisti un 13~karavīri. Uzbrucēju laupījums bija ap 30~000 rubļu. PPS vadība uzskatīja, ka šādas akcijas nedrīkst novest pie upuriem karavīru vidū, jo tas varēja traucēt revolucionārajai propagandai armijā un nolēma apturēt kaujas organizācijas darbību. Uz to atbildot šīs organizācijas vadība noturēja savu konferenci, kura pieprasīja, lai partijas tuvākajā kongresā tiktu pieņemts lēmums par gatavošanos uz bruņotu sacelšanos un bruņotas milicijas radīšanu kaujas organizācijas vadībā.

PPS IX kongresā 1906.~gada novembrī kaujas organizācijas darbības apspriešana kļuva par galveno jautājumu. Delegātu vairākums pieņēma rezolūciju, ka kaujas organizācija ``realizēja politiku un taktiku, kura atradās klajā pretrunā ar partijas līniju'', kas veda ``pie revolucionārās kustības Polijā nošķiršanās no šīs kustības visā Krievijas valstī''. Pēc tam 7 delegāti atstāja kongresu. Viņi noturēja savu konferenci, kurā nolēma norobežoties atsevišķā grupā. Tā radās PPS~--- ``Revolucionārā frakcija'' (\pltxti{Polska Partia Socjalistyczna}~--- \pltxti{Frakcja Rewolucyjna}) jeb ``fraki'', kā to parasti sauca, ar J.~Pilsudski priekšgalā. Tajā no 46~tūkstošiem PPS biedru pārgāja ap 14~tūkstošu. Pēdējos atbalstīja arī Galīcijas sociāldemokrāti. ``Revolucionārā frakcija'' aizstāvēja bruņotas cīņas un nacionālās sacelšanās organizācijas idejas. PPS-frakcijas īpatnība bija tā, ka sociāli-politiskās programmas ierobežotais raksturs tajā apvienojās ar dumpīgumu, radikālismu, kas bija raksturīgi iepriekšējo sacelšanos tradīcijām, šļahtičiem raksturīgu ``varoņu'' kultu. Pēc būtības ``revolucionārajā frakcija'' kļuva par sīkburžuāzisku militāri-politisku organizāciju ar uz nacionālās atbrīvošanās cīņu orientētu virzību. Vadoties no tās darbības, V.~Ļeņins rakstīja: ``Fraki'' ir nevis proletāriska, sociālistiska, bet sīkburžuāziska, nacionālistiska partija''. Jāatzīmē, ka no ``fraku'' kaujas vienību dalībnieku vidus nāca daudzi vēlākie augstākie Polijas valsts darbinieki. Tomēr ``fraku'' ietekme masās ievērojami kritās. Ja 1907.~gadā to rindās bija ap 14~tūkstošu cilvēku, tad 1909.~gadā~--- vairs tikai ap 3~tūkstoši. 1911.~gadā ``fraku'' vidū izveidojās opozīcija, tomēr, sākoties karam 1914.~gadā, tā atgriezās ``revolucionārās frakcijas'' rindās.

PPS IX kongresa vairākuma pozīcijas stāvošie pieņēma nosaukumu Polijas Sociālistiskā partija~--- ļevica (\pltxti{Polska Partia Socjalistyczna~--- Ļevica}), kura stāvēja tuvu Polijas karalistes un Lietuvas sociāldemokrātijai (daļa no PPS-ļevicas biedriem 1909.~gadā arī iestājās PKunLSD). No PPS uz PPS-ļevicu pārgāja vairākums biedru. PPS-ļevica nevēlējās atbalstīt bruņotu teroristisku cīņu, turpināja darboties patstāvīgi galvenokārt arodbiedrībās, dažādās kultūrizglītības biedrībās un tamlīdzīgās organizācijās, vēlāk~--- jau Pirmā pasaules kara laikā ieturēja internacionālistisku pozīciju, 1918.~gada decembrī, apvienojoties ar PKuLSD, izveidoja Polijas Komunistisko strādnieku partiju (\pltxti{Komunistyczna Partia Robotnicza Polski}), pie tam nacionālā jautājuma atrisināšanu atvirzot otrajā plānā.

Strādnieku kustība zem sociālistiskiem lozungiem, plašais terora izvirdums nobiedēja konservatorus. ``Reālisti'' atklāti vērsās pie Krievijas varas iestādēm, aicinot apspiest revolucionāro kustību. Līdz tam viņi nebija tik tālu gājuši vēlmē sadarboties ar impērijas varu. Viņus atbalstīja arī katoļu baznīca, kura baidījās no sociālistu vidū izplatītās ateistiskās domāšanas.

R.~Dmovskis griezās pie Krievijas valdības vadītāja S.~Vittes ar priekšlikumu piešķirt Polijas karalistei autonomiju, brīdinot, ka citādi neizdosies pārvarēt anarhiju. Taču Krievijas varas iestādes labi atcerējās XIX gadsimta 60.~gadu sākumu, kad zināmas piekāpšanās noveda pie nacionālistiskās kustības eskalācijas.

No 1906.~gada vidus par strādnieku šķiru cīņa arēnu atkal kļuva Lodza. Šeit notika virkne ekonomisku un politisku streiku, kā arī no 1906.~gada decembra līdz 1907.~gada martam norisa masu lokauts (uzņēmumu slēgšana un strādnieku masveida atlaišana, lai uzņēmēji varētu uzspiest tiem savus noteikumus.) Fabrikanti izvēlējās ziemu, kad strādnieki visvairāk cieta no sala un bada, lai piespiestu tos padoties. Ieganstu viņi atrada kādas strādnieku grupas konfliktā ar inženieri vienā no pilsētas rūpnīcām. Tās direkcija slēdza rūpnīcu un darba atjaunošanai uzstādīja strādniekiem jaunus darba noteikumus un paziņoja par 98~aktīvistu atlaišanu. Kad strādnieki atteicās piekrist, 17.~decembrī direkcija pasludināja lokautu, bet 29.~decembrī to pieteica visas lielākās Lodzas firmas. Bez darba izrādījās ap 30~000 strādnieku. Viņu biedri visā Polijas karalistē vāca līdzekļus to atbalstam. Tomēr pēc triju mēnešu cīņas, strādnieku spēku izsīkumu rezultātā lokauts beidzās ar fabrikantu uzvaru, strādnieki pieņēma viņu noteikumus. Tomēr kopumā poļu strādnieki streiku cīņu turpināja.

Krievijas varas iestādes no 1905.~gada sākuma gatavoja \strong{Krievijas Valsts domes} sasaukšanu. Izstrādājot tās vēlēšanu noteikumus, etniskajā jautājumā sadūrās divas nostājas. Viena~--- vai nu pielietot vienotas (vai gandrīz vienotas) normas visos pēc etno-konfesionālajām un kultūr-ekonomiskajām pazīmēm dažādajos reģionos vai, otrādi, katrā reģionā ieviest īpašus noteikumus. Notikumiem attīstīties varas iestādēs nostiprinājās viedoklis, ka dažādās nekrievu tautas, to pārstāvniecības Domē būs opozīcijas pusē. Pats Nikolajs II prasīja, lai, īstenojot reformu, ``tiktu nodrošināts, lai katrs krievs visā impērijā varētu justies savās mājās'' un uzsvēra nepieciešamību izdalīt ``krievu nacionālo kodolu'' (t.~i.~--- dot tam privilēģijas). Sākotnēji bija paredzēts, ka par Domes deputātiem nevarēs kļūt cilvēki, kuri ``nezin krievu valodu'' un ``neprot lasīt un rakstīt krieviski'', taču, ņemot vērā Iekškrievijas guberņu zemnieku analfabētismu, no otrās prasības nācās atteikties. Īpaši tika skatīts jautājums par vēlēšanu tiesību piešķiršanu ebrejiem. Bija cilvēki, kuri prasīja ``kaitīgās ebreju nācijas nepielaišanu Domē'', tomēr pēc S.~Vittes priekšlikuma ebrejiem piešķīra gan pasīvās, gan aktīvās vēlēšanu tiesības, jo tika cerēts, ka Domē iekļūs tikai pāris ebreju, kas nevarēs ietekmēt pārējo deputātu masu. (Patiesībā I Valsts domē bija 13 ebreju deputātu, bet II Domē~--- 6, tikai pēc vēlēšanu likuma izmaiņas III un IV Domē bija pa trijiem ebreju deputātiem.) Poļu apdzīvotajām guberņām tika nolemts piešķirt tādu pat pārstāvniecību kā Eiropas Krievijas guberņām, jo tajās bija augsts izglītības un saimnieciskās attīstības līmenis. Polijas karalistei tika piešķirtas 36 deputātu vietas. Eiropas Krievijā viens deputāts bija vidēji uz 227~800, Polijas karalistē~--- uz 254~100 iedzīvotājiem. Pēc 17.~oktobra manifesta izdošanas Polijas karaliste ieguva vēl 2 deputātu vietas. (Pavisam Domē bija 524 vietas). Nacionāldemokrātiskā partija cara 17.~oktobra manifestu un Domes vēlēšanas uztvēra kā izšķirošu panākumu, nosodīja sociālistus par streiku turpināšanu. Daudzi poļu pilsonības pārstāvji uzskatīja, ka Krievijas un Polijas attiecībās sākas jauns posms. Arī krievu vēsturnieks A.~Pogodins 1915.~gadā rakstīja, ka pati dzīve pierādījusi, ka pēc 1863.--1864.~gada poļu sacelšanās Krievijas valdības izdarītie secinājumi bijuši ``nemērķtiecīgi un politiski nepareizi'', bet 1905.~gada 17.~oktobra manifests iezīmējis jaunu robežlīniju, ``lai arī cik nemanāma būtu pāreja''.

1906.~gada martā notika vēlēšanas uz \strong{I Valsts domi}. Sociālisti (PPS, KSDSP, PKunLSD u.c.) tās boikotēja (boikotu atbalstīja tikai lielāko rūpniecības centru proletārieši un tas nespēja traucēt vēlēšanas). I~Dome izrādījās zemnieciska (vairāk kā 50\% deputātu bija tuvi zemniecībai) un daudznacionāla (vairāk nekā 40\% bija nekrievi). Nacionāldemokrāti ar līderiem R.~Dmovski un V.~Grabski priekšgalā izcīnīja 34 no iespējamajiem 38 mandātiem. Vēl 19 poļu tautības deputātus ievēlēja no Lietuvas, Baltkrievijas un Ukrainas.

Par poļu frakcijas (\pltxti{kolo}) priekšsēdētāju Domē kļuva R.~Dmovskis. Frakcija aizstāvēja Polijas autonomiju Krievijas impērijas sastāvā, kas paredzēja poļu valodu kā oficiālo, ticības brīvību un katoļu baznīcas tiesības, no Polijas iesaukto tiesības miera laikā dienēt savā zemē, patstāvīgu karalistes budžetu. Endeki piedāvāja bijušās Polijas-Lietuvas valsts teritorijā dzīvojušajiem ``leišiem un rusīniem'' atbalstīt endeku programmu, citādi solot izvērst pret tiem politisku cīņu. Viņi draudēja arī izspiest ebrejus no sabiedriskās dzīves. Kā uzsver krievu vēsturnieki, ksenofobijas un antisemītisma elementi endeku programmā ieņēma ievērojamu vietu. R.~Dmovskis un viņa piekritēji gribēja radīt tādu Polijas valsti, kurā etniskie poļi būtu pārsvarā, bet nacionālie mazākumi būtu pietiekami vāji, lai tos varētu asimilēt. Tāpēc, pēc endeku viedokļa, poļiem nevajadzēja cīnīties par Polijas atjaunošanu 1772.~gada robežās, jo tad šie nacionālie mazākumi būtu pārāk lieli. R.~Dmovskis 1906.~gadā pasludināja ``bezierunu'' atbalstu ``neoslāvismam'', cenšoties nostiprināt slāvu ietekmi Austroungārijā un tuvināt to ar uz konstitucionālisma ceļa nostājušos Krieviju. R.~Dmovskis mēģināja nodibināt savienību ar krievu kadetiem, bet reizē viņš arī nevēlējās, lai Krievijas varas iestādes poļu nacionāldemokrātus identificētu ar krievu liberālo opozīciju.

Poļu deputāti I Valsts domē varēja jau izmantot pieredzi, kuru bija guvušas poļu frakcijas Vācijas Reihstāgā un Austrijas Reihsrātā. Poļu pārstāvji Domē veidoja Autonomistu savienības kodolu. Savienībā bija 63~deputāti, bet kopā ar tādiem, kuri vienlaikus ietilpa citās frakcijās un arī autonomistu grupā, viņu bija ap 120~cilvēku. Savienības galvenais mērķis bija panākt Krievijas nacionālo reģionu pārvaldes decentralizāciju. No nacionālajiem jautājumiem kā pirmais Domē tika izvirzīts tieši poļu jautājums. 1906.~gada 30.~aprīlī 27~poļu deputāti iesniedza Domei dokumentu, kurā, atsaucoties uz 1815.~gada Vīnes kongresa aktu un Valsts Pamatlikumiem, argumentēja nepieciešamību atjaunot Polijas karalistes tiesības un autonomiju.

Jāuzsver, runa gāja tikai par autonomiju, bet nevis neatkarību. Kaut Krievijā pastāvošais administratīvais nacionālisms nekrievu tautās izsauca negatīvu reakciju, ātrā rūpniecības, tirdzniecības, satiksmes ceļu attīstība valstī tās attīstītāko nomaļu~--- Polijas, Baltijas, Somijas atdalīšanos no Krievijas padarīja ekonomiski gauži neizdevīgu. Poļu \pltxti{kolo} Domē iesniegtais likumprojekts pasludināja: ``Polijas karaliste kā neatņemama Krievijas valsts sastāvdaļa savā iekšējā dzīvē tiek pārvaldīta ar īpašiem nolikumiem uz īpašas likumdošanas pamata''. Prasot radīt Polijā Seimu ar likumdošanas tiesībām, īpašu valsts kasi un tiesu iekārtu, likumprojekts paredzēja visas impērijas kompetencē atstāt ``pareizticīgās baznīcas, ārlietu, armijas, flotes, monetāros jautājumus; likumdošanu muitas, akcīzes, pasta, dzelzceļu satiksmes, preču zīmju un privilēģiju, literārā un mākslas īpašuma, valsts aizņēmumu un saistību, dumpju pret centrālo varu un valsts nodevību lietās''. Šajā laikā Polijas nacionālie līderi bija gatavi aizmirst divu nacionālo sacelšanos rūgtumu, lai piedalītos sekmīgajā daudznacionālās Krievijas saimnieciskajā un pēc 1905.~gada revolūcijas arī cerīgajā politiskajā attīstībā.

Nacionālajiem reģioniem aktuāli bija arī agrārais jautājums, kadetu un nacionālo partiju ierosinājums atcelt nacionālos un reliģiskos ierobežojumus. Likumprojektu prezentējušais kadets F.~Kokoškins nosodīja daudzos ``poļu izcelsmes'' personu ierobežojumus, norādot, ka likums pat nepasaka to, kā var pierādīt, ka konkrētā persona ir tieši ``poļu izcelsmes''. Tādejādi I Dome sāka plaši iztirzāt cariskās valdības nacionālās neiecietības un ksenofobijas politiku. Taču jau 1906.~gada jūnijā I Valsts dome tika atlaista.

Jāatzīmē, ka I Domē Poļu \pltxti{kolo} neatbalstīja kadetu agrāro likumprojektu, pēc tās atlaišanas neparakstīja t.s. Viborgas uzsaukumu, kurš aicināja sabiedrību uz pilsonisko nepakļaušanos (\rutxti{гражданское неповиновение}) varai~--- nemaksāt nodokļus, nepakļauties iesaukumam armijā u.t.t., ar to iegūstot zināmā mērā labvēlīgu attieksmi valdības sfērās.

\strong{II Domes} (tā darbojās no 1907.~gada marta līdz jūlijam) vēlēšanās no poļu partijām nepiedalījās vienīgi PPS. PKunLSD uzskatīja par nepieciešamu izmantot vēlēšanu kampaņu aģitācijas un propagandas nolūkiem. Ievēlētajā 34~cilvēku lielajā II Domes poļu deputātu grupā dominēja nacionāldemokrāti. Viņi ieguva 29~mandātus. Kopā ar Lietuvas un Baltkrievijas Teritoriālo frakciju viņi kopā izveidoja 46~deputātu lielu grupu. Poļu kolo prezīdiju vadīja nacionāldemokrāts R.~Dmovskis. Jaunā Krievijas premjera P.~Stolipina valdība mēģināja piedāvāt II Domei līdzdarboties, īstenojot valdības sociāli ekonomisko kursu. Reizē P.~Stolipins savā deklarācijā pacentās izvairīties no nacionālā jautājuma atzīšanas Krievijā, ignorēja nacionālās pašvaldības un autonomijas jautājumus. Taču, uzsverot pareizticīgās baznīcas lomu, viņš norādīja, ka tai ir jāsaņem no valsts puses īpaša aizsardzība. Faktiski tas nozīmēja citu reliģiju, konfesiju noniecināšanu. Valdības vadītājs solīja Krievijas Rietumu apgabalā un Polijā ieviest pašpārvaldi ``ar dažām vietējām īpatnībām'', mainot dažas administratīvās robežas, izdalot īpašā vienībā tīri krieviskos iedzīvotājus, ``kuriem ir savas specifiskās intereses''. Šo programmu P.~Stolipins mēģināja realizēt kā II, tā arī III Valsts domē.

II Domē poļu deputātiem nācās karsti diskutēt ar pareizticīgās garīdzniecības pārstāvjiem. Tā apspriežot likumprojektu par sodu atcelšanu par ``slepenu apmācību'' (t.i.~--- nelegālu skolu uzturēšanu) bīskaps Jevlogijs apvainoja poļu skolas ``tīri krievisko iedzīvotāju polonizācijā''. R.~Dmovskis uz to norādīja, ka visas tautas vēlas, lai to bērni varētu mācīties dzimtajā valodā, bet Polijā šis jautājums atšķiras no citiem apgabaliem ar to, ka valdības uzraudzītās poļu sākuma, vidējās un augstākās skolas bija iznīcinātas. Tiesa, kā reakcija uz 1905.~gadā notikušajiem ``skolu streikiem'', bija atļauts dibināt privātskolas ar poļu mācību valodu. Nodibinājās Skolu savienība (\pltxti{Macierz Szkolna}), kurā bija ap 100~000 biedru un kura atvēra ap 800~sākumskolu. Asas debates izvērsās, apspriežot Izglītības ministrijas budžetu. Izrādījās, ka, kaut Krievijas Ziemeļrietumu novadā bija atļauta poļu un lietuviešu valodas mācīšana, taču par to bija jāmaksā skolēnu vecākiem. Kad P.~Stolipins mēģināja atcelt virkni ierobežojumu ebrejiem, Nikolajs II viņam to neatļāva darīt, norādot: ``Iekšējā balss man arvien noteiktāk saka, lai es neuzņemtos pieņemt šo lēmumu''. Daudzus likumprojektus, kas paredzēja likvidēt nacionālo nevienlīdzību, iesniedza opozīcijas partijas. Arī agrārās reformas nepieciešamības apspriešanas gaitā tika skarts Polijas jautājums. Kāds zemnieku deputāts [A.~Krakovskis] no Rietumu apgabala R.~Dmovska uzstāšanos ar prasību piešķirt pašvaldības tiesības Polijas karalistei iztēloja kā ``poļu muižnieku galveno trumpi, lai pakļautu savai ietekmei pareizticīgos zemniekus''. Viņš arī apgalvoja, ka ieguvuši kontroli pār vietējām skolām, ``poļi to izmantos, lai ``pavedinātu pareizticīgos katolicismā''. Tā II Domē skaidri parādījās dažādas nostājas Polijas jautājumos. Valdība vēlējās nacionālo jautājumu novirzīt reliģiski-konfesionālā gultnē, novēršot virkni nacionāli reliģisko problēmu. Domes opozīcija cīnījās par nacionālo un reliģisko vienlīdzību, paplašināt reģionālu vietējo pašpārvaldi.

Kopumā II Domē poļu deputāti izraisīja valdības neapmierinātību, jo viņu balsīm dažkārt izrādījās izšķiroša nozīme. Pēdējās II Domes pastāvēšanas dienās poļu deputāti iesniedza likumprojektu par Polijas autonomiju. Valdība kļuva pret viņiem atklāti naidīga.

Poļi bija pārstāvēti arī Krievijas Valsts padomē, kura sastāvēja kā no vēlētiem, tā cara nozīmētiem locekļiem. Vēlētos Valsts padomes locekļus ievēlēja uz 9 gadiem, ik pēc 3 gadiem notika viņu rotācija. Ar 1906.~gada 20.~februāra dekrētu Polijas muižnieki ieguva tiesības ievēlēt 6~Padomes locekļus, kas netika atļauts, piemēram Aizkaukāza muižniecībai. Kā vērtēja padomju vēsturnieks A.~Stepanskis, tas liecināja par carisma vēlmi izlīgt ar poļu muižniecību. Arī no deviņām Lietuvas, Baltkrievijas un Labā krasta Ukrainas guberņām, kurās pārsvarā bija poļu muižnieku zemes īpašums, Valsts padomē bija pārstāvēti pēdējie. Pavisam Padomē bija 18~poļu, gandrīz 10\% no visiem tās locekļiem, dažkārt viņu balsis varēja izšķirt balsojumu iznākumu. Ja I un II Valsts domē poļu deputātu vidū vairākums bija nacionāldemokrātiem, tad Valsts padomes poļu locekļi kopumā pārstāvēja t.s. ugodoviešu jeb ``reālistu'' virzienu. Kopumā šī grupa Valsts padomē uzstājās kā vienots vesels, sākumā parasti pieslējās oktobristiski-melnsimtnieciskajam (melnsimtnieki~--- Krievijas galēji labējo organizāciju pārstāvju kolektīvais nosaukums 1905.-1917.~gadā, kas uzstājās ar patvaldības un pareizticības stiprināšanas saukļiem) Padomes vairākumam. Tā poļu grupas līderis I.~Korvins-Miļevskis aizstāvēja oktobristu un melnsimtnieku tēzi par to, ka agrārā jautājuma risinājums jāmeklē ne muižnieku zemes atsavināšanā, bet zemnieku saimniecību intensifikācijā. Taču labējo polonofobija drīz piespieda poļus norobežoties no tiem. Poļi Valsts padomē atrada kopēju valodu ar t.~s. ``centru'', kaut arī tas noraidīja pat Polijas autonomijas lozungu. Kaut kādus demokrātiskus risinājumus no saviem Valsts padomes locekļiem poļu tautai nenācās gaidīt ne revolūcijas gados, ne arī pēc tās.

Var teikt, ka 1904.--1907.~gada notikumi Polijas karalistē kļuva par XX gadsimta sākuma poļu politiskās vēstures kvintesenci. Notika dabīga politisko ideju un organizāciju atlase, kuras izteica atsevišķu sociālo un nacionālo grupu intereses un centienus. Vēsturiskajā arēnā iznāca šķiras un partijas, kuras pusgadsimta garumā turpmāk noteica Polijas vēstures gaitu. Revolūcija atklāja un saasināja ideju konfliktus, atšķirības starp galvenajiem politiskajiem spēkiem.

Notikumu Polijas karalistē un visā Krievijas impērijā ēnā palika procesi, kas norisa šai laikā Pozenē un Galīcijā, kaut arī šeit XX gadsimta sākumā dzīvoja tikai nedaudz mazāk iedzīvotāju nekā Polijas karalistē. (1900.~gadā karalistē dzīvoja ap 10~miljonu, bet Austroungārijai pievienotajās poļu zemēs~--- 7,6~miljoni un Vācijas valdījumos~--- 7,3~miljoni.) Tomēr 1905.~gada revolūcija Krievijā guva atskaņas arī citās poļu zemēs.

Vācu iestādes \strong{Pozenes} provincē XX gs. sākumā turpināja agrāk aizsākto poļu ierobežošanas politiku. Kanclers B.~Bīlovs runāja par ``sociālistiskajām un poļu briesmām Prūsijai un impērijai''. Sarunā ar žurnālistiem B.~Bīlovs salīdzināja poļus ar trušiem, kuru ``vairošanos ir jāapstādina''. Pretpoļu nostāju atbalstīja arī ķeizars Vilhelms II.

Spilgtākais tās izpaudums bija 1901.~gadā notikušais t.s. ``Vrešenas (vācu \detxti{Wreschen}, poļu \pltxti{Wrzesnia}, apdzīvota vieta, kurā no 5~000~iedzīvotājiem 70\% bija poļi) incidents''(vācu \detxti{Wreschener Schulstreik}, poļu \pltxti{Strajk dzieci wrzesińskich}). Tika izdots rīkojums, ka poļu bērniem ticības (reliģijas) mācībā jāizmanto tikai vācu valodā iespiestas grāmatas un jāatbild vāciski. 118~skolnieki un skolnieces priestera J.~Laskovska vadībā atteicās pieņemt vācu valodā drukāto katķismu un atbildēt vāciski. Izglītības inspektors pavēlēja bērnus nopērt. Vecāki pieprasīja atcelt sodus. Nonāca līdz nemieriem. 18~Vrešenas pilsoņu--poļu par miera traucēšanu tika notiesāti uz mēnešiem ilgu apcietinājumu. Šis Vrešenas ``skolu streiks'' izsauca plašu sabiedrības rezonansi. Būtiski sabojājās attiecības vietējo iedzīvotāju~--- poļu un vāciešu starpā. Varas pārstāvji nevēlējās padoties. Prūsijas iekšlietu ministrs H.~f.~Hammeršteins 1904.~gadā paziņoja: ``Poļu jautājums mūs pārstās interesēt tikai tad, kad pie mums nebūs poļu, bet būs tikai prūši, kas runā poliski. Manuprāt, mūsu pacietība ir pārāk liela. Mums nav darīšana ar līdzīgu pretinieku, mūsu lieta~--- pavēlēt, viņu~--- paklausīt''.

1905.~gada pavasarī un vasarā notika virkne strādnieku streiku Pozenes pilsētās. Rudenī, kad Krievijas revolūcijā bija vērojams augstākais pacēlums, Vācijas varas iestādes poļu apdzīvotajās zemēs koncentrēja karaspēku. Situācija atkal saasinājās 1906.~gada pavasarī. Lejassilēzijas pilsētas Breslavas (vācu \detxti{Breslau}, poļu \pltxti{Wrocłow}) metalurģisko rūpnīcu īpašnieki pasludināja lokautu, atlaižot ap 10~000 strādnieku. Viņi organizēja plašu demonstrāciju, pret kuru tika raidītas policijas vienības. Simti strādnieku tika ievainoti, no tiem vairāki desmiti~--- smagi.

Kad 1906.~gadā vēl 200~tautskolas pārgāja uz reliģijas mācības pasniegšanu vācu valodā, 1906./1907.~gadā Pozenes provincē notika jauns liels ``skolu streiks''. Streiks šoreiz vairs nebija lokāli ierobežots, izplatījās ne tikai visā Pozenes provincē, kurā toreiz mācījās 370~000 skolēnu, (No tiem 118~000 atzina par dzimto vācu valodu, bet 240~000~--- poļu valodu, bet ap 20~000 nevarēja izšķirties par vienu no tām, jo bija no jauktajām ģimenēm), bet arī Rietumprūsijā un Augšsilēzijā, kur tika lietotas abas~--- poļu un vācu valodas. 755~Pozenes provinces skolās no ap 90~000 skolēnu streikoja gandrīz 47~000. No 1906.~gada rudens līdz 1907.~gada beigām Prūsijas tiesas poļu rietumu zemēs bija pārslogotas ar lietām, ierosinātām pret poļu nacionālās kustības dalībniekiem, kuri atbalstīja skolu streiku. Pasaules sabiedriskā doma stāvēja poļu pusē, jo visi saprata, ka sākumskolā ir jāmācās dzimtajā valodā. Poļu dzejniece M.~Konopņicka, kura bija aktīvi piedalījusies jau Vrešenas streika izraisītajos notikumos, 1908.~gadā uzrakstīja dzejoli ``\pltxti{Rota}'' (``Zvērests''), kura pirmās rindas skanēja:

\vspace{1.5em}

\noindent
\begin{minipage}{0.5\textwidth}
\pltxti{Nie rzucim ziemi skąd nasz ród,\\
Nie damy pogrześć mowy,\\
Polski naród, polski lud,\\
Królewski szczep piastowy.\\
Nie damy by nas zniemczył wróg!\\
Tak nam dopomóż Bóg!\\
Tak nam dopomóż Bóg!}
\end{minipage}
\hspace{1em}
\begin{minipage}{0.5\textwidth}
Neatdosim zemi, kurā dzimām,\\
Nemirs mūsu valoda,\\
Polijas mēs bērni, poļu tauta,\\
Karaliskā Pjastu celma.\\
Ienaidnieks mūs nepārvācos!\\
Dievs mums palīdzēs!\\
Dievs mums palīdzēs!
\end{minipage}

\vspace{1.5em}

% page 207

% ----- INFO PAGE -----
\newpage\thispagestyle{plain}
{\centering

% ----- LICENSE SIGNS BLOCK -----
\includegraphics[width=2.5em]{cc.pdf}~\includegraphics[width=2.5em]{by.pdf}
% ----- LICENSE SIGNS BLOCK -----

\vspace{0.2em}

% ----- LICENSE BLOCK -----
{\ml{$0$}{Данная книга распространяется под лицензией \textbf{CC~BY~4.0}.}{This book is distributed under the \textbf{CC~BY~4.0} license.}\par}
{\ml{$0$}{Подробнее о лицензии:}{Details:} \href{https://creativecommons.org/licenses/by/4.0/deed.ru}{creativecommons.org/licenses/by/4.0}\par}
% ----- LICENSE BLOCK -----

\vfill

% ----- AUTHOR AND TITLE -----
{\large\bookauthor\par}
\vspace{0.5em}
{\Huge\textbf{POLIJA}\par}
{\Large\textbf{XIX un XX gadsimtā}\par}
% ----- AUTHOR AND TITLE -----

\vfill

% ----- BOOK INFO -----
{\ml{$0$}{Начато:}{Started:} \textit{\bookstarted}\par}
{\ml{$0$}{Последняя редакция:}{Latest revision:} \textit{\bookfinished}\par}
{\ml{$0$}{Подробнее о книге:}{The book details:} \href{https://github.com/regnveig/tofa}{github.com/regnveig/tofa}\par}
% ----- BOOK INFO -----

\vfill

% ----- DOCUMENT INFO -----
{\ml{$0$}{Дизайн обложки: \textit{Э.\,Весна}}{The cover design by E.\,Viesn\'a}\par}
{\ml{$0$}{Компьютерная вёрстка: \textit{Э.\,Весна}}{The computer layout by E.\,Viesn\'a}\par}
{\ml{$0$}{Создано с помощью \XeLaTeX}{Created with \XeLaTeX}\par}
% ----- DOCUMENT INFO -----

\vspace{0.5em}

% ----- PUBLISHER INFO -----
{\textbf{\ml{$0$}{Свободное издательство <<Цунами>>}{\textsc{Tsunami}, an independent publisher}}\par}
{\ml{$0$}{Томск 2023}{Tomsk 2023}\par}
% ----- PUBLISHER INFO -----

\vfill

% ----- QR CODES -----
~
% ----- QR CODES -----

% }
% % ----- INFO PAGE -----
%
% % ----- EMPTY PAGE -----
% \thispagestyle{plain}~
% % ----- EMPTY PAGE -----
%
% % ----- BACK COVER -----
% \includepdf[pages={1}]{cover_back.pdf}
% % ----- BACK COVER -----

\end{document}
